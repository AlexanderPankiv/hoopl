\newif\ifpagetuning \pagetuningfalse  % adjust page breaks

\newif\ifnoauthornotes\noauthornotesfalse

\newif\ifgenkill\genkillfalse  % have a section on gen and kill
\genkilltrue


\newif\ifnotesinmargin \notesinmargintrue 
\IfFileExists{notesinline.tex}{\notesinmarginfalse}{\relax}

\documentclass[blockstyle,preprint,nocopyrightspace]{sigplanconf}


\usepackage{alltt}
\usepackage{array}
\newcommand\lbr{\char`\{}
\newcommand\rbr{\char`\}}
 
\clubpenalty=10000
\widowpenalty=10000

\usepackage{verbatim} % allows to define \begin{smallcode}
\newenvironment{smallcode}{\par\unskip\small\verbatim}{\endverbatim}

\newcommand\arrow{\rightarrow}

\newcommand\join{\sqcup}
\newcommand\slotof[1]{\ensuremath{s_{#1}}}
\newcommand\tempof[1]{\ensuremath{t_{#1}}}
\let\tempOf=\tempof
\let\slotOf=\slotof

\makeatletter
\newcommand{\nrmono}[1]{%
  {\@tempdima = \fontdimen2\font\relax
   \texttt{\spaceskip = 1.1\@tempdima #1}}}
\makeatother

\usepackage{times}  % denser fonts
\renewcommand{\ttdefault}{aett} % \texttt that goes better with times fonts
\usepackage{enumerate}
\usepackage{url}
\usepackage{graphicx}
\usepackage{natbib}
\bibpunct();A{},
\let\cite\citep
\let\citeyearnopar=\citeyear
\let\citeyear=\citeyearpar

\usepackage[ps2pdf,bookmarksopen,breaklinks,pdftitle=dataflow-made-simple]{hyperref}

\newcommand\naive{na\"\i ve}

\usepackage{amsfonts}
\newcommand\naturals{\ensuremath{\mathbb{N}}}
\newcommand\true{\ensuremath{\mathbf{true}}}
\newcommand\implies{\supseteq}  % could use \Rightarrow?

\newcommand\PAL{\mbox{C{\texttt{-{}-}}}}
\newcommand\high[1]{\mbox{\fboxsep=1pt \smash{\fbox{\vrule height 6pt
   depth 0pt width 0pt \leavevmode \kern 1pt #1}}}}

% Put figures in boxes
\usepackage{float}
\floatstyle{boxed}
\restylefloat{figure}



% Set \noauthornotestrue to suppress notes
%\newcommand{\qed}{QED}
\ifnotesinmargin
  \long\def\authornote#1{%
      \ifvmode
         \marginpar{\raggedright\hbadness=10000
         \def\baselinestretch{0.8}\tiny
         \it #1\par}%
      \else
          \unskip\raisebox{-3.5pt}{\rlap{$\scriptstyle\diamond$}}%
          \marginpar{\raggedright\hbadness=10000
         \def\baselinestretch{0.8}\tiny
         \it #1\par}%
      \fi}
\else
  % Simon: please set \notesinmargingfalse on the first line
  \newcommand{\authornote}[1]{{\em #1}}
\fi
\ifnoauthornotes
  \def\authornote#1{\unskip\relax}
\fi

\newcommand{\simon}[1]{\authornote{SLPJ: #1}}
\newcommand{\norman}[1]{\authornote{NR: #1}}
\let\remark\norman
\def\finalremark#1{\relax}
% \let \finalremark \remark % uncomment after submission
\newcommand{\john}[1]{\authornote{JD: #1}}
\newcommand{\todo}[1]{\textbf{To~do:} \emph{#1}}

\newcommand\secref[1]{Section~\ref{sec:#1}}
\newcommand\secreftwo[2]{Sections \ref{sec:#1}~and~\ref{sec:#2}}
\newcommand\seclabel[1]{\label{sec:#1}}

\newcommand\figref[1]{Figure~\ref{fig:#1}}
\newcommand\figlabel[1]{\label{fig:#1}}


\newcommand{\CPS}{\textbf{StkMan}}    % Not sure what to call it.


\usepackage{code}   % At-sign notation

\begin{document}
\title{Dataflow Optimization Made Simple}
%\subtitle{\today}

\authorinfo{Norman Ramsey}{Tufts University}{nr@cs.tufts.edu}
\authorinfo{Jo\~ao Dias}{Tufts University}{dias@cs.tufts.edu}
\authorinfo{Simon Peyton Jones}{Microsoft Research}{simonpj@microsoft.com}


\maketitle
 
\begin{abstract}
We present a design that makes it easy for compiler writers
to implement program transformations based on dataflow analyses.
The compiler writer must identify (a)~a family of logical assertions
on which the transformation will be based;
(b)~an \emph{approximate} representation of such assertions, which
must have a lattice structure such that every assertion can be increased at
most finitely many times;
(c)~transfer functions that approximate weakest preconditions or
strongest postconditions over the assertions; and
(d)~rewrite functions whose soundness is justified by the assertions.
To~guide the compiler writer,
we show how dataflow analyses are related to
seminal work on program 
correctness. 
Finally, we have implemented our design using the
algorithm of 
\citet{lerner-grove-chambers:2002}, which enables the compiler writer to
compose very simple analyses and transformations in a way that achieves
the same precision as complex, handwritten
``super-analyses.''
Program transformations based on dataflow do most of the work of a new
back end for the Glasgow Haskell Compiler.
\end{abstract}

\makeatactive   %  Enable @foo@ notation

\section{Introduction}

Anyone writing a compiler for an imperative language benefits
from over forty years' work on code improvement, also called
``optimization.''
But the work is typically presented
as a large collection of apparently unrelated analyses and
transformations, each with its own name---and not everyone can easily
remember the difference 
between
such names as
``copy propagation'' and ``constant propagation.''
Typical treatments obscure fundamental principles of
code improvement.

%The contribution of this paper is to elucidate a large body of work on code
%improvement; the body of work known as ``dataflow optimization.''
This paper makes two contributions:
\begin{itemize}
\item
We show that the ad-hoc ``optimization zoo'' consists mostly of special
cases of reasoning techniques that have long been understood and used
by semanticists and functional programmers:
assertions about states, assertions about continuations, and
substitution of equals for equals (\secref{next-700}).
What distinguishes dataflow optimization from classic formal reasoning
about programs is that in dataflow optimization, all assertions are
computed automatically, and they are
\emph{approximated}. 
\item
We embody our ideas in an implementation that makes it not just
possible but \emph{easy} to adapt imperative, dataflow-based
code-improvement techniques to a purely functional compiler.
We~write analyses and transformations so that
the code which is specific to a particular analysis is small, simple,
and easy to get right.
The complex part of dataflow optimization, which is hard to get right, is
written once and is shared by all analyses and transformations.
\end{itemize}

We consider code improvements over low-level
imperative codes, including intermediate languages and machine
languages.
As \citet{benitez-davidson:portable-optimizer} have shown, all the
classical scalar and loop optimizations can be performed over such
codes.
Moreover, any compiled functional program is
translated to such a code.


We introduce the subject by analyzing and transforming example code in
\secref{example:xforms},
thinking about and justifying classical optimizations using
Hoare logic and substitution of equals for equals.
To~support our claim that we make dataflow optimization easy, 
we spend most of the paper explaining how
a compiler writer creates new dataflow analyses or transformations
(\secref{making-simple}) and showing complete implementations of significant
analyses (\secref{example-analyses}) and transformations
(\secref{example-rewrites}) from the Glasgow Haskell compiler.
We spend less space on the shared implementation (\secref{engine}).


\section{Dataflow analysis {\&} transformation by \rlap{example}}

\seclabel{example:transforms}
\seclabel{example:xforms}

In dataflow optimization, code-improving transformations are justified
by assertions about the program;
such assertions are typically computed using
strongest postconditions or weakest liberal preconditions.
The most typical transformations are
insertion of assignments to unobserved variables,
substitution of equals for equals, 
and
removal of assignments to unobserved variables,
all of which preserve semantics.
Insertion and removal are often composed to achieve the effect called
``code motion,'' as done by \citet{knoop:lazy-code-motion}, for example.
The examples below express classical code
improvements by composing small analyses and transformations.



\subsection{Simple transformations}

\seclabel{constant-propagation}

Here is a sequence of three assignments separated by assertions.
We compute the assertions by starting with the weakest possible
assertion (@true@) and computing strongest postconditions.
% \footnote
{Variables do not alias.}
\begin{verbatim}
    { true }
  x = 7;
    { x == 7 }
  y = 8: 
    { x == 7 && y == 8 }
  z = x + y;
\end{verbatim}
In the assignment to~@z@, the assertion @x == 7@ justifies
substituting 7~for~@x@, leaving @z = 7 + y@.  
This transformation is traditionally called ``constant propagation.''
We may also substitute 8~for~@y@.
Finally, because @7 + 8 == 15@, we may again substitute equals for
equals, leaving the final assignment as
\begin{verbatim}
  z = 15;
\end{verbatim}
This transformation, although also an instance of substituting equals
for equals, has a different name: ``constant folding.''

\subsection{A complex transformation}
\simon{It's a pity that this transformation occupies nearly the entire second page, 
and then plays no subsequent role in the paper whatsoever.
One possibility: move it to ``the next 700'' section, as a substantiating example
to the claims made there.
But then we'd need another example here... well the sink/reload example of
Section 4 might be perfect.}

\seclabel{induction-var-elim}

\ifpagetuning \enlargethispage{-1\baselineskip} \fi

The loop optimization known as ``induction-variable elimination'' is
actually a collection of simple transformations.
We begin by showing a simple loop, FORTRAN style,
although the code is~C:
\begin{verbatim}
  struct pixel { double r, g, b; };
  double sum_r(struct pixel a[], int n) {
    double x = 0.0;
    int i;
    for (i = 0; i < n; i++)
      x += a[i].r;
    return x;
  }
\end{verbatim}
To explain the improvement we wish to make, we show the same
code at the machine level.
We write machine-level examples using our low-level compiler-target
language,~{\PAL} % tuning for line breaks
\cite{peyton-jones-ramsey:garbage-collection:inproceedings,peyton-jones-ramsey:exceptions}: 
\begin{verbatim}
  sum_r("address" bits32 a, bits32 n) {
       bits64 x; bits32 i;
       x = 0.0;
       i = 0;
   L1: if (i >= n) goto L2;
       x = %fadd(x, bits64[a+i*24]);
       i = i + 1;
       goto L1;
   L2: return x; 
  }
\end{verbatim}
The code improvement called ``induction-variable elimination''
replaces~@i@ with a new variable~@p@ so that we can avoid repeating
the computation @a+i*24@ on each iteration through the loop.
The new variable~@p@ is intended to satisfy the invariant
\begin{verbatim}
   { p == a + i * 24 }
\end{verbatim}
The variable @i@ is also used in the loop-termination test.
To rewrite that test, 
we introduce a new variable @lim@ satisfying
the invariant % line breaking
@lim == a + n * 24@,
so that @i >= n@ if and only if @p >= lim@.

We implement the code improvement as a sequence of transformations.
After each transformation, the observable behavior of the program is unchanged,
and it is possible to generate code (see \secref{vpoiso}).
%
Our first transformation declares @p@ and @lim@ and inserts suitable
assignments. 
New code is \high{boxed}\,.
\begin{alltt}
  sum_r("address" bits32 a, bits32 n) \lbr
       bits64 x; bits32 i; \high{bits32 p, lim;}
       x = 0.0;
       i = 0; \high{p = a; lim = a + n * 24;}
   L1: if (i >= n) goto L2;
       x = %fadd(x, bits64[a+i*24]);
       i = i + 1; \high{p = p + 24;}
       goto L1;
   L2: return x; 
  \rbr
\end{alltt}

As written, the assignments to @p@~and~@lim@ have no
effect on the program, but they enable the compiler to establish the assertions
@p == a + i * 24@ and @(i >= n) == (p >= lim)@.
On~the basis of these assertions, the compiler substitutes equals for
equals, resulting in the new code in boxes below:
\begin{alltt}
  sum_r("address" bits32 a, bits32 n) \lbr
       bits64 x; bits32 i; bits32 p, lim;
       x = 0.0;
       i = 0; p = a; lim = a + n * 24;
   L1: if (\high{p >= lim}) goto L2;
       x = %fadd(x, bits64[\kern1pt{}\high{p}\kern1pt{}]);
       i = i + 1; p = p + 24;
       goto L1;
   L2: return x; 
  \rbr
\end{alltt}

At this point, the compiler switches from reasoning about states to
reasoning about continuations.
In~particular, we reason about whether the value of a variable can be
used by a continuation; this reasoning is called ``liveness analysis.''
A~\naive\ analysis would show that although @i@~is not live at
label~@L2@, it is nevertheless live immediately after 
the assignment
@i = i + 1@ in the loop body,
because the value of~@i@ could be used by the next iteration of the
loop.
But we use the framework of
\citet{lerner-grove-chambers:2002} to \emph{interleave} liveness
analysis with 
``dead-assignment elimination.'' 
Dead-assignment elimination removes an assignment if the variable
assigned to is not live, that is, if it cannot be used by the
assignment's continuation.
As~explained in detail by Lerner, Grove, and Chambers, no sequential
composition of liveness analysis and dead-assignment elimination can
get rid of these assignments to~@i@, but interleaving analysis with
transformation does the trick.\footnote
{An experienced reader might be tempted to modify the
liveness analysis so that
@i = i + 1@ is not considered a ``use'' of~@i@ if @i@~is itself
dead.
This modification is tantamount to writing a single, combined dataflow
pass that understands \emph{both} liveness analysis and dead-code
elimination.
In~this particular example, writing a single, combined pass presents
few difficulties, but the approach does not scale:
most combined passes are more complicated than the examples shown here;
the cost of writing combined passes does not scale linearly with
the number of individual passes;
combined passes often cannot be composed;
and 
some combined passes require nonstandard, handwritten traversals of
the control-flow graph.
\citet{lerner-grove-chambers:2002} discuss these issues in detail;
\citet{click-cooper} present a handwritten combined pass of
significant complexity.}
\secref{dfengine} describes our implementation of their interleaving
method, which eliminates the boxed assignments to~@i@:
\begin{alltt}
sum_r("address" bits32 a, bits32 n) \lbr
     bits64 x; \high{bits32 i;} bits32 p, lim;
     x = 0.0;
     \high{i = 0;} p = a; lim = a + n * 24;
 L1: if ({p >= lim}) goto L2;
     x = %fadd(x, bits64[{p}]);
     \high{i = i + 1;} p = p + 24;
     goto L1;
 L2: return x; 
\rbr
\end{alltt}

After the insertion of assignments to @p@~and~@lim@, the substitution
of equals for equals, the removal of newly dead assignments
to~@i@, we have ``eliminated the induction variable:''
\begin{alltt}
sum_r("address" bits32 a, bits32 n) \lbr
     bits64 x; bits32 p, lim;
     x = 0.0;
     p = a; lim = a + n * 24;
 L1: if ({p >= lim}) goto L2;
     x = %fadd(x, bits64[{p}]);
     p = p + 24;
     goto L1;
 L2: return x; 
\rbr
\end{alltt}

  
\section {Making dataflow simple}

\seclabel{making-simple}

\seclabel{create-analysis}

The goal of dataflow optimization is to compute valid
assertions, then use those assertions to justify code-improving
transformations.
%
% only Don Knuth knows why, but the paragraph break gets us three
% bullets on this page where a single paragraph gets only two!
%
Assertions are written in a restricted
language of \emph{dataflow facts}.\john{Almost sounds like we're restricting the language}
\simon{Exactly!  That was one reason I rewrote this stuff on Weds.}
The connection between dataflow facts and
\ifpagetuning\else traditional \fi
program logic is
elucidated in \secref{next-700}, but the most salient points are
these:
\begin{itemize}
\item
% A dataflow fact is equivalent to an assertion about program state or
% about a continuation.
% For example, in \secref{constant-propagation}, @x == 7@ is a dataflow
% fact that describes the program state. 
There are two kinds of dataflow facts.
The first kind is an assertion about the paths from the procedure
entry to a program point;
these facts are computed by a forward dataflow analysis.
A common special case is an assertion about state at a program point,
such as the assertion @x == 7@ in \secref{constant-propagation}.
% Assertions about program state are usually sufficient to show that
% a transformation preserves semantics,
% but to decide whether a transformation will improve the code,
% we sometimes need an assertion that describes how the program
% state was established.

The second kind of dataflow fact is an assertion about paths from the program point
to the procedure exit;
these facts are computed by a backward dataflow analysis.
In the parlance of functional programmers, these dataflow facts are assertions on
\emph{continuations}.

Assertions about state are easy to formalize, but path properties are harder;
we describe path properties informally in our assertions.
\simon{Shall we mention that typically forward
analysis use the former and backwards the latter?}
\simon{Side remark (not for paper): I was wondering about the difference
between ``state'' and ``continuation'', why so different?  Actually it
is perhaps not so different.  In a backward analysis, an 
assertion tells you something about what will happen from this 
program point onwards; in a forward analysis the assertion tells you
something about what has happened up to this point.
In general, a forward analysis could compute more stuff than assertions
about the program state at this moment in time; for example, you could
approximate the number of times a variable x has been written so far, something
that would not normally be considered part of the ``state'' of the program.
Now the two look more dual.}
\john{Indeed, this was a problem. I think we've addressed the latter comment now.}
\simon{Yes, better.}
\item
Each analysis or transformation may use a different language of
dataflow facts.
\item
Each language of dataflow facts forms a lattice.
To ensure that analysis terminates,
it is enough if
no fact has more than finitely many distinct facts above it.
\end{itemize}


Assertions hold at program points, which we represent as edges in
a \emph{control-flow graph}.\footnote
{We discuss only intraprocedural optimization here;
interprocedural optimizations are the work of the GHC~inliner
\cite{peyton-jones:secrets-inliner}.} 
The edges connect nodes, each of which represents a label, an assignment, or
a control transfer.

To write a dataflow analysis, the compiler
writer must 
\begin{itemize}
\item
Choose a representation~$F$ of dataflow facts and a logical interpretation.
thereof
\item
Implement lattice operations over~$F$.
\item
Write \emph{transfer functions} that relate dataflow facts before and
after each type of node.
\end{itemize}

To write a transformation based on an analysis, the compiler writer
must create a \emph{rewrite function}, which is presented with a
flow-graph node and with the dataflow facts on the edges coming
into that node.
The function either suggests that the node should be replaced with a
fresh subgraph, or it leaves the node alone.
The rewrite function uses incoming facts to guarantee that
any proposed replacement preserves semantics.
For example, in \secref{constant-propagation} the fact @x == 7@ is
used to justify replacing @z = x + y@ with @z = 7 + y@.

Given lattice operations and transfer functions, our
polymorphic, reusable \emph{dataflow engine} computes a
set of dataflow facts, one for each edge in the control-flow graph 
(\secref{zdfSolveFwd}).
The set of facts is a fixed point of the transfer functions.
Given, in addition, a rewrite function for each type of node,
the dataflow engine implements a semantics-preserving transformation
(\secref{rewrites}). 

\begin{figure}
\begin{code}
data ChangeFlag = NoChange | SomeChange
data TxRes a    = TxRes ChangeFlag a
data DataflowLattice a = DataflowLattice
 {fact_bot        :: a,
  fact_add_to     :: a -> a -> TxRes a,
  fact_name       :: String } -- for debugging
\end{code}
\caption{Representation of a dataflow lattice} \figlabel{lattice-type} \figlabel{lattice}
\end{figure}


\subsection{Dataflow lattices}


As an example lattice of dataflow facts, 
we present a representation for facts about constant propagation.
At any program point, a standard constant-propagation analysis
computes exactly one of the following three
facts about a variable~$x$:
\remark{We're a bit careless about distinguishing
 ``fact'' and ``assertion'', but the general idea is that an
 ``assertion'' is God's truth in logic and a ``fact'' is an
 approximation or representation of an assertion.
Perhaps future revisions will clarify.}
\begin{itemize}
\item
The analysis shows that
$x = k$, where $k$~is a compile-time constant of type @Const@.
\item
The analysis shows that $x$~is \emph{not} a compile-time constant.
We~notate this fact as $x = \top$.
\item
The analysis shows nothing about~$x$, which we notate $x=\bot$.
\end{itemize}
The bottom element of the lattice is~$x=\bot$, and
the join operation corresponds to disjunction.
\emph{The logic implied by the
representation of dataflow facts cannot represent disjunction
exactly}; instead, a
disjunction of two inconsistent facts is represented by~$x=\top$.
\simon{At this point it's very unclear why one should focus on 
disjunction here.  Why not conjunction, or something else?
I tried to fix that in my rewrite.}
Here are some examples:
\begin{itemize}
\item
$i = 7 \lor i=\bot \equiv i=7$ (no loss of information)
\item
$i = 7 \lor i= 7 \equiv  i=7$ (no loss of information)
\item
$i = 7 \lor i = 8 \equiv i = \top$ (loss of information)\footnote
{The compiler writer gets to choose how much information is lost.
For example, if doing a similar analysis for a functional language,
a~compiler writer might wish to track whether a value~$v$ is known to
be the application of a value constructor~$C_i$.
In this analysis, there's no need to limit the representation to a
single constructor or to~$\top$;
the compiler writer could choose to represent such facts as ``$v$~is
the application of a constructor drawn from the set $\{C_1, C_2,
C_4\}$.''
Because every algebraic data structure has finitely many constructors,
there are finitely many sets and therefore finitely many facts in the
lattice, so a dataflow analysis over this lattice will always reach a
fixed point.
}
%%  The compiler writer gets to choose how much information is lost;
%%  in a different analysis, it might be useful to 
%%  choose a lattice which can say
%%  that the value of a variable is one of,
%%  say, at most three constants.
\simon{What is ``the compiler''?  The library or the client?
Also the ``the logic implied by the representation of the dataflow facts cannot represent
disjunction''.  So we have the compiler, the logic, and the representation all swirling
about in my head.  I don't think we can fix my problem by a few words here; so out it goes.}
\remark{We'll try to get our terminology sorted.
Meanwhile, please let me know if the footnote adds anything. }
\simon{It's an improvement.}

\end{itemize}

The lattice  used by the analysis is the Cartesian product of the
lattices for all the local variables.
To~represent this lattice, a good choice is a finite map from variable
to a value of type @Maybe Const@.
If~a variable $x$ is not in the domain of the map then $x=\bot$;
if $x$~maps to @Nothing@ then $x=\top$; if $x$~maps to $@Just@\;k$ then
$x=k$.

Because in any one program there are only finitely many variables,
only finitely many facts are computed at any program point, 
and any one fact can increase at most twice.
Together, these properties
guarantee that the dataflow engine will
reach a fixed point.


The dataflow engine uses the lattice join operation in a stylized way.
Joins occur at labels.
If~$f_{\mathit{id}}$ is the fact currently associated with the
label~$\mathit{id}$, 
and if a transfer function propagates a new fact~$f_{\mathit{new}}$
into the label~$\mathit{id}$, 
the dataflow engine replaces $f_{\mathit{id}}$ with
the join  $f_{\mathit{new}} \join f_{\mathit{id}}$.
Furthermore, the dataflow engine wants to know if
  $f_{\mathit{new}} \join f_{\mathit{id}} = f_{\mathit{id}}$,
because if not, it has not reached a fixed point.

When computing a join, 
it is typically cheap to learn if the join
is equal to one of the arguments.
We therefore use the nonstandard representation of lattice operations
shown in \figref{lattice}.
The join operation~$\join$ and equality test~$=$ are represented by a
single function called @fact_add_to@.
The term $@fact_add_to@\;f_{\mathit{new}}\;f_{\mathit{id}}$ is equal to
$@TxRes NoChange@\; f_{\mathit{id}}$ if $f_{\mathit{new}} \join f_{\mathit{id}} = f_{\mathit{id}}$
and is equal to
$@TxRes SomeChange@\; (f_{\mathit{new}} \join f_{\mathit{id}})$ otherwise.
The @fact_bot@ value is the bottom element, 
and @fact_name@  is used for debugging.

\begin{figure}
\begin{code}
newtype LastOuts a = LastOuts [(BlockId, a)]
data ForwardTransfers mid last a = ForwardTransfers
 {ft_first_out  :: BlockId -> a -> a,
  ft_middle_out :: mid     -> a -> a,
  ft_last_outs  :: last    -> a -> LastOuts a} 

data BackTransfers mid last a = BackTransfers
 {bt_first_in  :: BlockId -> a              -> a,
  bt_middle_in :: mid     -> a              -> a,
  bt_last_in   :: last    -> (BlockId -> a) -> a} 
\end{code}
\caption{Transfer functions for forward and backward analyses.}
\figlabel{transfers}
%
% elided: 
%    ft_exit_out   ::            a -> a
%
\end{figure}



\subsection{Transfer functions} \seclabel{tffuns}

A~transfer function is presented with dataflow facts on edges coming
into a node, and it computes the dataflow facts on outgoing edges.
To~understand transfer functions, we must first
understand how the nodes and edges of a control-flow graph are organized.

A~control-flow graph is a collection of labelled \emph{basic blocks}.
Each basic block is a sequence beginning with a \emph{first node},
containing zero or more \emph{middle nodes},
and ending in a \emph{last node}.
A~first node is always a label;
a~typical middle node assigns to a register or memory
location;
a~typical last node is a conditional, unconditional, or indirect branch.
\remark{Forward reference?}

First nodes are the only targets of control transfers;
middle nodes never perform control transfers;
and
last nodes always perform control transfers.
Hence, a~first node has arbitrarily many predecessors and exactly one
successor;
a~middle node has exactly one predecessor and one successor;
and a last node has exactly one predecessor and arbitrarily many
successors. 

These constraints on number of predecessors and successors determine
the signatures of 
transfer functions, 
which are shown in \figref{transfers}.
For each type of node (first, middle, last) and for each kind of
analysis (forward, backward), there is a distinct transfer function.
Functions are grouped by kind of analysis, and each group is
parameterized over a dataflow fact of type~@a@ and over the types
@mid@ and @last@ of middle and last nodes.  
The choice of  @mid@ and
@last@ types is up to the compiler writer; it is by means of this
choice that the compiler writer determines the intermediate
representation used for optimization.
\norman{We are having trouble understanding SLPJ's obsession with
distinguishing the ``client'' (an entity we wish not to name) from the
``library'' (another entity we wish not to name} 


The type @BlockId@ represents a label.
Because a fact in a forward analysis typically represents an assertion
about program state,
\remark{Somewhere we need an alert that a fact in a forward analysis
might also represent an assertion about all paths reaching a
 state---maybe a footnote here?}
 and because passing a label does not change
program state, the transfer function @ft_first_out@ is typically 
@flip const@---a variation on
the
identity function.\footnote
{A~case in which @ft\_first\_out@ is \emph{not} @flip const@ is an
 analysis we use to help convert {\PAL} to continuation-passing style
 \emph{without} 
 duplicating code that is common to multiple continuations.
This analysis tracks the continuation to which each basic block should belong.
}
\remark{Somewhere we should explain what a fact in a backward analysis
typically represents}
For a middle node, the transfer function @ft_middle_out@ is given a
node and a precondition and returns an approximation of the strongest
postcondition. 
For a last node, different postconditions may be propagated to
different successors; for example, the true and false successors of a
conditional branch may accumulate information implied by the truth or
falsehood of the condition.
A~collection of (successor, fact) pairs is represented by a value of
type @LastOuts a@.




In a forward analysis, the dataflow engine starts with the fact at the
beginning of a block and applies transfer functions to the nodes in
that block until eventually the transfer function for the last node
computes the facts that are propagated to the block's successors.
For example, in the basic block
\begin{verbatim}
  L1: x = 7;
      y = 8;
      z = x + y;
      goto L2;
\end{verbatim}
a forward analysis would propagate the fact 
$@x == 7@ \land @y == 8@$, which we will call $f_{\mathit{new}}$,
along the edge to~@L2@. 
%%  \remark{We've elected not to get to the level of detail where
%%  we show how propagating a fact~$f$ through \mbox{@x = 7;@} results in a new
%%  fact either $(f \setminus @x@) \land @x == 7@$.}
The dataflow engine then \emph{replaces} the current fact
at~@L2@~($f_{\mathtt{L2}}$) with the lattice join $f_{\mathit{new}}
\join f_{\mathtt{L2}}$. 
The dataflow engine iterates over the blocks repeatedly, creating new
facts~$f$ and joining them with facts $f_{\mathit{id}}$ until
\mbox{$f \join f_{\mathit{id}} = f_{\mathit{id}}$} at every label~$\mathit{id}$.
When the facts at labels stop changing, the dataflow
engine has reached a fixed point.
%\remark{Promissory note: compose this analysis with two
%transformations: constant propagation and constant folding}



\subsection{Running the dataflow engine}

\seclabel{zdfSolveFwd}

Given lattice operations of type @DataflowLattice a@
and transfer functions of type @ForwardTransfers m l a@,
the compiler writer can run the analysis by calling the
dataflow-engine function @zdfSolveFwd@:
\begin{code}
 zdfSolveFwd 
  :: LastNode l             -- Constrains last nodes
  => PassName               -- Name of this analysis
  -> DataflowLattice a      -- Lattice
  -> ForwardTransfers m l a -- Transfer functions
  -> a                      -- Input fact
  -> Graph m l              -- Control-flow graph
  -> ForwardFixedPoint m l a
\end{code}
The engine is polymorphic in the types of middle and last nodes
@m@~and~@l@ and in the type of the dataflow fact~@a@.
The first three arguments characterize the analysis.
The next argument is the dataflow fact that holds on entry to the
graph;
because a procedure's caller may establish some facts about
parameters or about the stack,
this fact
is not always~$\bot$.
The last argument to @zdfSolveFwd@ is the graph, and the result is a 
fixed point.\footnote
{The type of @zdfSolveFwd@ has pedagogical lies;
truth is told in~\secref{engine-truth}.}
%%  \remark{One of the lies is
%%  that I've deliberately conflated the three representations of graphs.}

The @ForwardFixedPoint@ data structure is a big bag
of information about the solution.
The most significant information is
a finite map from each block label to the dataflow fact that holds at
the label, which is extracted using function @zdfFpFacts@:
% Simon really wants these type signatures!
\begin{verbatim}
type BlockEnv a = Data.Map BlockId a
zdfFpFacts :: ForwardFixedPoint m l a -> BlockEnv a
\end{verbatim}






\iffalse
% never tell the whole truth!
\begin{code}
class DataflowSolverDirection
        transfers fixedpt where
  zdfSolveFrom :: (DebugNodes m l, Outputable a)
    => BlockEnv a        -- Init facts
    -> PassName          -- Analysis name
    -> DataflowLattice a -- Lattice
    -> transfers m l a   -- Transfers
    -> a                 -- Input fact
    -> Graph m l         -- CFG
    -> DFMonad (fixedpt m l a ())
\end{code}
\fi



%%  The main contribution of this paper is to present an interface which
%%  enables many powerful program transformations based on dataflow
%%  analysis while keeping the individual dataflow passes as simple as
%%  possible.
%%  We keep the \emph{concepts} simple by relating dataflow facts and
%%  transfer functions to classic work in program correctness, as
%%  discussed in \secref{next-700}.
%%  We keep the \emph{implementations of dataflow passes} simple by pushing
%%  as much work as
%%  possible into the dataflow engine, which is implemented just once.
%%  We have also made the dataflow engine and its interface polymorphic in
%%  the types of 
%%  the nodes that appear in the control-flow graph \secref{polymorphic-framework}.
%%  Parametricity ensures separation of concerns between the dataflow
%%  engine and the individual dataflow passes.





\section{Example analysis passes}

\seclabel{example-analyses}


\newcommand\T{\rule{0pt}{0.6ex}}
\newcommand\B{\rule[-0.05ex]{0pt}{0pt}}
\newcolumntype{C}{>{\begin{minipage}{5.35in}}l<{\end{minipage}}} % code
\newcolumntype{L}{>{\Large\bfseries}m{1.3in}<{\centering}}       % label
\newenvironment{codetable}{\begin{tabular}{CL}}{\end{tabular}}

\begin{figure*}
\begin{codetable}
\T\begin{code}
data AvailVars = UniverseMinus VarSet
               | AvailVars     VarSet
extendAvail  :: AvailVars -> LocalVar  -> AvailVars  -- add var to set
delFromAvail :: AvailVars -> LocalVar  -> AvailVars  -- remove var from set
elemAvail    :: AvailVars -> LocalVar  -> Bool       -- set membership
interAvail   :: AvailVars -> AvailVars -> AvailVars  -- set intersection
smallerAvail :: AvailVars -> AvailVars -> Bool       -- compare sizes
\end{code}\B
& Dataflow fact and operations\\
\hline

\T\begin{code}
availVarsLattice :: DataflowLattice AvailVars
availVarsLattice = DataflowLattice "reloaded registers" empty add False
    where empty = UniverseMinus emptyVarSet
          add new old = let join = interAvail new old in
                        if join `smallerAvail` old then aTx join else noTx join
\end{code}\B
& Lattice\\
\hline

%%  \T\begin{code}
%%  agen  :: UserOfLocalVars    a => a -> AvailVars -> AvailVars
%%  akill :: DefinerOfLocalVars a => a -> AvailVars -> AvailVars
%%  agen  a avail = foldVarsUsed extendAvail  avail a
%%  akill a avail = foldVarsDefd delFromAvail avail a
%%  \end{code}\B
%%  & Gen/Kill \mbox{functions}\\
%%  \hline
%%  
\T\begin{code}
avail_reloads_transfer :: ForwardTransfers Middle Last AvailVars
avail_reloads_transfer = ForwardTransfers (flip const) middleAvail lastAvail

middleAvail :: Middle -> AvailVars -> AvailVars
middleAvail (MidAssign (CmmLocal x) (CmmLoad l) avail
                 | l `isStackSlotOf` x = extendAvail avail x
middleAvail (MidAssign lhs _expr) avail = 
  foldVarsDefd delFromAvail avail lhs  -- remove variables defined in 'lhs'
middleAvail (MidStore l (CmmReg (CmmLocal r))) avail
                 | l `isStackSlotOf` r = avail
middleAvail (MidStore (CmmStackSlot (RegSlot r) _) _) avail = delFromAvail avail r
middleAvail (MidStore {}) avail = avail

lastAvail :: Last -> AvailVars -> LastOuts AvailVars
lastAvail (LastCall _ (Just k) _ _) _ = LastOuts [(k, AvailVars emptyVarSet)]
lastAvail l avail = LastOuts $ map (\id -> (id, avail)) $ succs l
\end{code}\B
& Transfer \mbox{functions}\\
\hline

\T\begin{code}
cmmAvailableReloads :: Graph Middle Last -> BlockEnv AvailVars
cmmAvailableReloads g = zdfFpFacts soln
  where soln = zdfSolveFwd "available reloads" availVarsLattice 
               avail_reloads_transfer (fact_bot availVarsLattice) g
\end{code}\B
& Available-reloads analysis\\
%\hline

\end{codetable}
% \caption{Available-variable analysis}
\caption{Dataflow analysis pass to compute available variables}
\figlabel{avail-all}
\figlabel{avail}
\figlabel{avail-lattice}
\figlabel{avail-gen-kill}
\figlabel{avail-transfers}
\figlabel{avail-running}
\end{figure*}
% Old captions' text:
% The dataflow fact for the available-reload analysis describes
%   the set of registers for which a reload is available.
%   We list the types of the functions that manipuate sets of available registers,
%   as well as the definition of the lattice.
% The standard gen and kill functions for available expressions
% The transfer functions for the available-reloads analysis.
% Running the available-reloads analysis and extracting the results with \texttt{zdfFpFacts}
% The rewrite functions to insert redundant reloads immediately before uses

% Probably no space for the implementations:
% interAvail (UniverseMinus s) (UniverseMinus s') =
%   UniverseMinus (s `plusVarSet`  s')
% interAvail (AvailVars     s) (AvailVars     s') =
%   AvailVars (s `timesVarSet` s')
% interAvail (AvailVars     s) (UniverseMinus s') =
%   AvailVars (s  `minusVarSet` s')
% interAvail (UniverseMinus s) (AvailVars     s') =
%   AvailVars (s' `minusVarSet` s )
% 
% smallerAvail (AvailVars _) (UniverseMinus _) = True
% smallerAvail (UniverseMinus _) (AvailVars _) = False
% smallerAvail (AvailVars     s) (AvailVars    s')  =
%   sizeVarSet s < sizeVarSet s'
% smallerAvail (UniverseMinus s) (UniverseMinus s') =
%   sizeVarSet s > sizeVarSet s'
% 
% extendAvail (UniverseMinus s) r =
%   UniverseMinus (deleteFromVarSet s r)
% extendAvail (AvailVars     s) r =
%   AvailVars (extendVarSet s r)
% 
% delFromAvail (UniverseMinus s) r =
%   UniverseMinus (extendVarSet s r)
% delFromAvail (AvailVars     s) r =
%   AvailVars (deleteFromVarSet s r)
% 
% elemAvail (UniverseMinus s) r =
%   not $ elemVarSet r s
% elemAvail (AvailVars     s) r =
%   elemVarSet r s


Our big claim is that using our data structures, algorithms, and
interfaces makes it easy to write compiler passes based on dataflow.
In~this section we provide evidence for that claim by showing
implementations of two analyses;
related program transformations appear in \secref{example-rewrites}
below. 
These analyses help solve a real problem in the Glasgow Haskell
Compiler:
because most calls are tail calls, GHC uses no 
callee-saves registers.
This means at each (rare) non-tail call, all live
variables must be spilled to the stack.
In order to reduce register pressure,
such variables are spilled as early as possible and reloaded as late as possible.

To illustrate the results of the analyses and their associated transformations,
here is a contrived example program in the style of \secref{example:xforms}:
\begin{alltt}
f (bits32 a) \lbr
  bits32 w, x, y, z;  // local variables
  x = a * a;
  w = a + a + a;
  y = g(w);
  z = y + y;
  if (y > 0) \lbr
    return z;
  \rbr else \lbr
    return z + x;
  \rbr
\rbr
\end{alltt}
Spills and reloads should be inserted as follows:%
\seclabel{spill-reload-example} % space is deliberate
\newcommand\bigstrut{%
  \leavevmode\vrule width 0pt height 11pt depth 6pt }
\begin{alltt}
f (bits32 a) \lbr
  bits32 w, x, y, z;  // local variables
  x = a * a;
  \high{SPILL x;}
  \bigstrut\high{w = a + a + a;} // no register pressure from x
  \high{y = g(w);} // call instruction
  z = y + y;
  if (y > 0) \lbr
    return z;
  \rbr else \lbr
    \high{RELOAD x;}
    return z + x;
  \rbr
\rbr
\end{alltt}
Although the @SPILL@ and @RELOAD@ operations are introduced because of
the call to @g(a)@, they are moved as far from the call as possible:
the variable~@x@ is spilled immediately after being assigned @a * a@,
and @x@ is reloaded not
immediately after the call to~@g@, but just before its use in the
expression @z + x@.
 On the control-flow path to @return z@, @x@~needn't be reloaded
at all.

Spills and reloads are inserted in the right places
by a sequence of three dataflow passes:
\begin{enumerate}
\item
\label{insert-spills}
A backward analysis computes liveness
to identify the variables that should be spilled at call sites.
An accompanying transformation inserts reloads immediately after each call
site and inserts spills not immediately before call sites, but
rather immediately after the appropriate reaching definitions.
\item
\label{reload-duplication}
A forward analysis finds uses of variables that have been reloaded
from the stack, and an accompanying transformation
inserts redundant reloads before the uses.
By keeping variables on the stack longer, this pass reduces register pressure.
% \simon{Why is the second pass a second pass?  The first
% pass added spills and reloads in; could the first pass not have 
% added the reloads immediately before the
% reloaded variables are used as well?  Maybe the text can say?
% I think the answer is that this is a \emph{forwards} analysis, since
% it is propagating forward the information about which variables currently
% have an up-to-date stack slot.}
\item
\label{remove-dead-reloads}
A backward analysis computes liveness,
and an accompanying transformation, dead-assignment elimination,
removes redundant reloads.
\end{enumerate}
Pass~\ref{insert-spills} is not shown in this paper.
Passes
\ref{reload-duplication}~and~\ref{remove-dead-reloads} cooperate to ``sink''
the reloads as far as possible from the call site.
The analyses used in
passes~\ref{reload-duplication}~and~\ref{remove-dead-reloads}
are described in \secreftwo{avail}{liveness} below;
the transformations are described in
\secreftwo{sink-reloads}{dead-code-elimination}.
%\simon{I would put the forward references
%to the sub-sections in the bullets themselves; they are more easy to
%find there.
%  No room, and not worthy of a bullet! ---NR}




\subsection{Available variables: a forward analysis} 

\seclabel{avail}


%%%%%%%%%%%%%%%%%%%%%%%%%%%%%%%%%%%%%%%%%%%%%%%%%%%%%%%%%%%%%%%%%%
%
%  desperately trying to get figures to emerge in a decent order
%
\begin{figure*}
\begin{codetable}
\T\begin{code}
type Live = VarSet
\end{code}\B
& Dataflow fact\\
\hline

\T\begin{code}
liveLattice :: DataflowLattice Live
liveLattice = DataflowLattice "live LocalReg's" emptyVarSet add False
  where add new old =
          let join = unionVarSets new old in
          (if sizeVarSet join > sizeVarSet old then aTx else noTx) join
\end{code}\B
& Lattice\\
\hline

%%  \T\begin{code}
%%  gen  :: UserOfLocalVars    a => a -> Live -> Live
%%  kill :: DefinerOfLocalVars a => a -> Live -> Live 
%%  gen  a live = foldVarsUsed extendVarSet  live a
%%  kill a live = foldVarsDefd delFromVarSet live a
%%  \end{code}\B
%%  & Gen/Kill \mbox{functions}\\
%%  \hline

\T\begin{code}
liveTransfers :: BackwardTransfers Middle Last Live
liveTransfers = BackwardTransfers (flip const) middleLiveness lastLiveness

middleLiveness :: Middle -> Live -> Live
lastLiveness   :: Last -> (BlockId -> Live) -> Live
middleLiveness m = addUsed m . remDefd m
lastLiveness   l = addUsed l . remDefd l . lastLiveOut l 

addUsed :: UserOfLocalVars    a => a -> Live -> Live
remDefd :: DefinerOfLocalVars a => a -> Live -> Live 
addUsed a live = foldVarsUsed extendVarSet  live a
remDefd a live = foldVarsDefd delFromVarSet live a

lastLiveOut :: Last -> (BlockId -> Live) -> Live
lastLiveOut l env = last l 
  where
    last (LastBranch id)        = env id 
    last (LastCondBranch _ t f) = unionVarSets (env t) (env f)
    last (LastSwitch _ tbl)     = unionManyVarSets $ map env (catMaybes tbl)
    last (LastCall { })         = emptyVarSet
\end{code}\B
& Transfer \mbox{functions}\\
\hline

\T\begin{code}
cmmLiveness :: Graph Middle Last -> BlockEnv Live
cmmLiveness g = zdfFpFacts soln
    where res = zdfSolveBwd "liveness analysis" liveLattice liveTransfers emptyVarSet g
\end{code}\B
& Liveness \mbox{analysis}\\
\end{codetable}
\caption{Liveness analysis}
\figlabel{liveness-all}
\figlabel{liveness}
\figlabel{live-lattice}
\figlabel{live-transfers}
\figlabel{live-running}
\end{figure*}
%
%
%%%%%%%%%%%%%%%%%%%%%%%%%%%%%%%%%%%%%%%%%%%%%%%%%%%%%%%%%%%%%%%%%%



To~understand the analysis, you must know that for each variable~$x$,
there is a stack slot~\slotof x that is used to save the value of~$x$.
%
\remark{Changing ``register'' to
``variable'' makes the example easier to understand and hides a
GHC-ism, but there's a slippery slope here.}
%
If the variable and the stack slot hold the same value,
that is if $x = \slotof x$,
then it is safe to insert a reload without changing the observable
behavior of the program.

To sink a reload of a variable $x$, we insert redundant reloads immediately
before a use of $x$.
We use an analysis that identifies not only when it is safe to insert
such a reload, but when the inserted reload will make an earlier reload redundant.
Specifically, we should insert a reload before a node that uses $x$ only if,
on every path to the node, the most recent definition of $x$ was a reload from
$\slotof x$; we assert this property with the informal notation
$\mathit{lastDef}(x) = x @=@ \slotof x$.

For a single variable, the logical assertion takes the form
\mbox{$(x = \slotof x) \wedge
       (\mathit{lastDef}(x) = x @=@ \slotof x)$}.
For multiple variables, the logical assertion is a conjunction of assertions
on single variables.
To represent this assertion, our analysis uses a 
set of variables~$X$ such that \mbox{$\forall x \in X \mathrel
: (x = \slotof x \wedge (\mathit{lastDef}(x) = x @=@ \slotof x))$}.
The lattice-join operation is set intersection,\footnote
{A variable $x = \slotof x$ at a label~@L@ if and
only if $x = \slotof x$ on every control-flow edge coming into~@L@.}
and the bottom element
is the universal set containing all variables.
\simon{I'm puzzled about why you are treating this example so differently
to constant-prop in Section 3.1.  It looks almost identical to me.  We could
keep a fact for every variable: $x=\bot$ means nothing is known; $x=s_x$ means
x's stack slot is up to date; $x=\top$ means x's stack slot is out of date.
Then keep a finite map as we do for constant prop.  If there is a difference
that drives the rep you have here, let's say so. If the difference is
purely accidental, we should eliminate it.  (Or maybe we don't have time, in
which case we should remark that there is no diff.)}
Because universal sets can be awkward to manipulate, we represent a
set using one of two alternatives
(see~\figref{avail-lattice}):
\begin{itemize}
\item \texttt{UniverseMinus~s}: All variables except those in the set~@s@
\item \texttt{AvailVars~~~~~s}: The variables in the set~@s@
\end{itemize}
The bottom element is @UniverseMinus emptyVarSet@.
It is optimistic to assume that a variable can be redundantly reloaded from the stack,
but this optimistic assumption is what enables us to improve the
code---and if the assumption is not justified, it will be corrected before
the analysis reaches a fixed point.




%%  Among many other uses, an available-expressions analysis can be~used
%%  in code-motion optimizations.
%%  For example, when making a function call, we insert
%%  spills and reloads to save and restore the values of local variables
%%  around the call site.
%%  It is easy to insert the reloads at the function-call return site,
%%  but to avoid register pressure, it would be better to leave the variable
%%  where it was spilled on the stack.
%%  Rather than complicating the code that inserts the spills and
%%  reloads around call sites,
%%  we write an analysis to insert a redundant reload immediately
%%  before a reloaded variable is used.
%%  Then, we rely on a dead-code elimination to~remove
%%  the early reload.

To manipulate sets of variables, we provide the functions shown at the
top of \figref{avail}.
\simon{Whoa!  We've moved from sets of \emph{assertions} to 
sets of \emph{variables}. I see what you are doing, but it's a bit slippery.
How about saying this: 
(a) the dataflow fact on an edge is the set of
variables whose stack slots are up to date; 
(b) the interpretation of such set S is that $x=s_x$ for each $x\in S$.
(I'm not sure the ``interpretation'' makes it clearer; I'd be happy with (a).)}
The implementations use analogous operations on @VarSet@s.

%%In~the lattice of available reloads (\figref{avail-lattice}),
%%the~join is computed by taking the intersection of two sets
%%of available reloads,
%%The join operation (@add@) returns a \emph{transaction} @aTx@
%%if the join has not yet reached a fixed point;
%%otherwise, it returns @noTx@ to indicate that the dataflow fact
%%did not change.
%%The bottom element of the lattice is the universe of all variables,
%%reflecting the initial assumption that all variables
%%have available reloads.
%%\john{Shouldn't it be @AvailRels@ instead of @AvailVars@?}



The most interesting part of the pass is the @middleAvail@ transfer
function in \figref{avail-transfers}.
\simon{Hang on!  We have not discussed the data type for middle or last nodes, 
yet they are essential to understand the code. This is a tough
one, because there is a fair amoutn to say.}
\begin{itemize}
\item
The first case \simon{``first case''/``second case'' is easier when writing than reading.  Could we
add @(Case 1)@ and @(Case 2)@ to the code we have an unambiguous URL?
\textbf{NR}: If we have time, we want line numbers in the code
figures.  John says it will be quick.} 
identifies an assignment that reloads local
variable~@x@ from its stack slot.\finalremark{I propose the compiler be
modified to use @isStackSlotOf@ as I've written. JD~approves.}
After such an assignment, $x = \slotof x$,
and the last definition of $x$ is a reload,
so @x@~is added to the set of available variables.
% No other variable is affected.
\item
In the next case, an assignment to a local variable means that the
variable is no longer necessarily equal to the value in its stack
slot, so if @lhs@ is a local variable, it is removed from the set of
available variables.
Function @foldVarsDefd@ is a function which, in this
instance, calls @delFromAvail@ if @lhs@ is a local variable and does
nothing if @lhs@ is a hardware register.
\remark{To Simon and John: I've removed ``overloaded but am a bit nervous
because @foldVarsDef@ shows up as overloaded later in the paper, and I
had thought to signpost it here.  Your thoughts?}
\item 
There are three cases for @MidStore@ nodes.
The first case matches a node that spills a variable $x$ to the stack.
After such a node, $x = \slotOf x$,
but there is no guarantee that $x$ was defined by a reload instruction,
so $x$ is not added to the set of available variables.
The second case matches a node that spills a variable to the stack,
which neither establishes nor invalidates an available variable.
The final case matches any other store to a stack slot $\slotof x$,
which may invalidate any assertion of the form $x = \slotof x$.



% \footnote
% {Although @MidStore@ may overwrite a stack slot \slotof x, GHC
% carefully arranges that all stores to \slotof x have the form
% $\slotof{\mathtt{x}}= @x@$.
% These stores could be used to extend the set of available variables,
% but it is not useful to do so.}
% \simon{Why not? asks the reader.  Perhaps
% because such saves immediately precede calls?}
% \remark{How do we know that @MidStore@ doesn't
% destroy a stack slot??  I've put in a footnote but it will probably be
% simpler to fix the code.}
% \simon{Ah, this harks backe to the start of this sub-section, where
% we say ``to understand the analysis, you must know that...''.  Another
% thing you must know is that $s_x$ is used exclusively for $x$.  That's
% all, I think.}
\end{itemize}
%% \john{I've expunged MidComment from our example} % well done ----NR


The transfer function for a last node checks to see if the node is a
function call; if so, the set of
available variables is empty on every outgoing control-flow edge.
Other last nodes do not change values of variables or stack slots, 
so the set of available variables remains unchanged.
%
A~first node has no effect on program state, so its transfer function
is @flip const@.

%%Using the @agen@ and @akill@ functions, we define the transfer functions
%%for the available-reloads analysis (see~\figref{avail-transfers}):
%%\begin{itemize}
%%\item \emph{First nodes}:
%%  A first node cannot define a variable,
%%  so the set of available reloads is the same before and after a first node.
%%  We return the set unchanged using @flip const@.
%%\item \emph{Middle nodes}:
%%  If a middle node reloads a value from a register's slot on the stack (@RegSlot@),
%%  then we add the register the register to the set of available reloads.
%%  For any other assignment to a register, we remove the register from the
%%  set of available reloads.
%%  Similarly, because a function call can overwrite the value of any local variable,
%%  the set of available reloads is empty after a~function call.
%%  Any other middle node leaves the set of available reloads unchanged.
%%\item \emph{Last nodes}:
%%  If the last node is a function call, the outgoing set of available reloads
%%  is empty for every successor basic block.
%%  Otherwise, the last node cannot modify any variables,
%%  so the set of available reloads remains unchanged.
%%\end{itemize}

Given the lattice and the transfer functions,
we perform the available-reloads analysis by calling
the dataflow engine in the form of function @zdfSolveFwd@
(\figref{avail-running}).
The function @zdfFpFacts@ returns 
a finite map from basic-block IDs to the set of available reloads
at the beginning of each block.
Except for the implementations of the set operations on @AvailVars@,
\figref{avail} shows the \emph{entire} analysis.

\subsection{Liveness: a backward analysis} 
\simon{To my mind 4.2 is easier to understand than 4.1. It's a totally
well-known analysis.  Can we re-order to put it first?
Fine by me ---NR}
\simon{Same question as in current 4.1: what is the connection to
passes 1,2,3?  Irrelevant: the analysis should stand on its own ---NR.
SLPJ responds: Not irrelevant to the reader!  I expect 4.2 to follow in some way
from 4.0 (the bit before 4.1).  
Indeed we say (start of 4) ``In~this section we provide evidence for that claim by showing
implementations of two analyses...that help solve a real problem in GHC''.
So presumably this analysis is somehow connected to 4.0.}

\seclabel{liveness}

The assertion computed by 
a backward dataflow analysis typically applies to a
\emph{continuation} at a program point, not to a state.\john{Why ``typically?''}
The classic example is liveness analysis;
the logical fact of interest is that at a particular program point,
the answer produced by the continuation does not depend on
the value of a particular variable~$x$.
If~so, $x$ is said to be \emph{dead} at that point.
If the answer produced by the continuation \emph{might} depend on the
value of~$x$, $x$~is \emph{live}.\footnote
{Liveness cannot be decided accurately; it reduces easily to the halting problem.
As usual, we approximate liveness by reachability.}

In a modern compiler, liveness analysis supports many program
transformations,
including
dead-assignment elimination,
which removes assigments to dead variables, 
and register allocation, which
ensures that if two variables are 
live at the same time, they are not assigned to the same register. 

The dataflow fact we use to represent liveness assertions is the set of
live variables.
The bottom element of the lattice is the empty set, and the join
operation is set union (\figref{live-lattice});
a~variable is deemed live if it is live on \emph{any} edge leaving a
node. \simon{You mean ``deemed live after the node if it is live on any edge...''.
You don't mean ``deemed live on entry to the node..'' (think of assignments).
It's a bit tricky to state this clearly I grant you.}

The transfer functions for liveness rely on two auxiliary functions
@addUsed@ and @remDefd@, which are shown in \figref{liveness}.
\simon{I suggest that we give more explicit
references into the code, mentioning the functions @middleLiveness@, and 
using @(1)@ etc as URLs. [NR: we want line numbers]  Even then it's a bit confusing becuase we
have last-nodes and liveness, 
both of which start with ``@l@''.  Oh well.}
A~transfer function is given a set of variables live on the edges
going out of the node, and it removes from that set any variable
defined by the node, then adds to the set any variable used by the
node.
The~idea is that if a node has the assignment
\begin{verbatim}
  i = n - 1;
\end{verbatim}
then @i@ is not live just before the node, since if we start the
program just before the assignment @i = 0@, the answer cannot 
depend on the value of~@i@, which is about to be overwritten.
However, the answer might well depend on the value of~@n@, so
@n@~is considered live before the assignment.
If a variable appears on both left- and right-hand sides of an
assignment, as in
\begin{verbatim}
  i = i + 1;
\end{verbatim}
then the answer might well depend on~@i@, so @i@~is considered live
before the assignment.
For this reason the transfer functions in \figref{liveness} must
remove defined variables \emph{before} adding used variables.

For a last node, function @lastLiveOut@ consults the solution in
progress to find out what variables are live at the successors of a
last node. 
For an unconditional branch, we look up the live set at the label
branched to;
for a conditional branch, we look at both true and false edges, and
for a switch (computed goto), we consider every possible target of the
branch.
The~remaining case is a call, 
and since a call destroys the values of all local variables, the
only variables live just before a call are ones that might be used to
compute the address of the procedure being called, which are added by
@addUsed@ in @lastLiveness@.

Given the lattice and the transfer functions,
we perform the liveness analysis by calling
the dataflow engine in the form of function @zdfSolveBwd@
The function @zdfFpFacts@ returns 
a finite map from basic-block IDs to the set of live variables
at the beginning of each block.
\figref{avail} shows the \emph{entire} analysis.
(The overloaded functions
@foldVarsUsed@ and @foldVarsDefd@ are used throughout the back end and
should not be considered part of liveness analysis.)
\simon{Please let us give their types somewhere!  Every function
 should have either a type signature, or a type and an implementation. Otherwise
the code becomes opaque.  \textbf{NR}: Where do you suggest we put the types?
\textbf{SLPJ} I don't think it matters much.  Perhpas in a Figure
that collects the signatures of otherwise less-important functions, or in Figure
3 itself (but it's so wide it wastes space to put a type sig in!}

%%  Each transfer function takes an instruction and
%%  the set of live registers in the instruction's continuation (the \liveout\ set);
%%  the result of the function is the set of live registers before the instruction.
%%  The standard procedure is to start with the set \liveout\,
%%  then remove each register defined by the instruction and
%%  add any register used by the instruction.
%%  In classic optimization literature, this procedure
%%  is~described in terms of @gen@ and @kill@ functions
%%  that add and remove registers from the live sets.
%%  
%%  Using the @gen@ and @kill@ functions, we define the 
%%  transfer functions for the liveness analysis as @liveTransfers@
%%  (see~\figref{live-transfers}):
%%  \begin{itemize}
%%  \item \emph{First nodes}:
%%    The set of live registers is the same before and after a first node,
%%    so we return the \liveout\ set unchanged using @flip const@.
%%  \item \emph{Middle nodes}:
%%    The set of live registers is determined by the @gen@ and @kill@
%%    functions, which are composed in @middleLiveness@.
%%  \item \emph{Last nodes}:
%%    Like a middle node, the set of live register is determined by
%%    the @gen@ and @kill@ sets, which are composed in @lastLiveness@.
%%    But because a last node may have multiple sucessors,
%%    we use the @lastLiveOut@ function to compute the set of registers that
%%    must be live in any continuation of the last node.
%%    For most last nodes, we take the union of the live sets from all of the successors.
%%    The exception is for call nodes: because a function call can overwrite any
%%    local register, there must not be any registers live out of a call site.
%%  \end{itemize}
%%  
%%  Finally, we can perform the liveness analysis by calling the dataflow framework's
%%  @zdfSolveFrom@ function with the lattice
%%  and transfer functions (see~\figref{live-running}).
%%  The function @zdfFpFacts@ returns the result of the analysis
%%  in the form of a map from basic-block IDs to the set of variables
%%  live at the beginning of each block.
%%  \john{Don't forget to go back and check consistency of names (e.g. s/CmmLive/Live/)}
%%  \john{I've dropped some sanity checking code.}


\ifgenkill
\section{Writing transfer functions using \texttt{gen} and \texttt{kill}}

\remark{I've tried to rewrite this section without the abstract guff;
Simon, can you let us know what you think?}


\seclabel{gen-kill}

If you pick up a compiler textbook, you might see dataflow
optimization explained in terms of functions \texttt{gen} and \texttt{kill},
which say what dataflow facts are ``generated'' and
``killed'' by each node.
Our design is compatible with explanations based on \texttt{gen} and
\texttt{kill};
first you define functions \texttt{gen} and \texttt{kill}, then you
use them to write the transfer functions.
%%  Using our design, you can stillIt's~possible to write transfer functions i thi
%%  In~a forward analysis, for example an assignment might establish an
%%  assertion (\texttt{gen}), or it might change state in such a way that an
%%  assertion no longer holds (@kill@), or both.
%%  If~you want to transliterate specifications that use @gen@ and @kill@,
%%  it's easy.
\remark{I tried to repair @smallcode@ so as not to affect
line spacing in the previous~\P.}
For example, the functions @addUsed@ and @remDefd@ in \figref{liveness}'s
liveness analysis correspond directly to the traditional @gen@ and
@kill@ functions.
In the available-variables analysis of \figref{avail}, @gen@ and
@kill@ can be defined as follows:
\begin{smallcode}
gen  :: UserOfLocalVars    a => a -> AvailVars -> AvailVars
kill :: DefinerOfLocalVars a => a -> AvailVars -> AvailVars
gen  a avail = foldVarsUsed extendAvail  avail a
kill a avail = foldVarsDefd delFromAvail avail a
\end{smallcode}
Using @gen@ and @kill@ entails no loss of efficiency;
for example, in @middleAvail@ in \figref{avail}, GHC~will identify and
inline the type-specific instance of @foldVarsUsed@, which for the
case of a local variable is the identity function, so @gen x avail@
reduces to @extendAvail avail x@.

\fi

\section{Using dataflow facts to rewrite graphs, with examples}

\seclabel{rewrites}

\remark{Orphaned text:
Analyzing a subgraph also uses a non-bottom entry fact.
For example, if a rewrite function proposes to replace 
@z = x + y@ with @z = 7 + y@, an analysis will be run on the one-node
graph @z = 7 + y@ with input fact
$@x == 7@ \land @y == 8@$.
}

\seclabel{example-rewrites}

\begin{figure}
\begin{code}
type Rewrite mid last = Maybe (Graph mid last)
data ForwardRewrites mid last a = ForwardRewrites
 {fr_first  :: BlockId -> a -> Rewrite mid last,
  fr_middle :: mid     -> a -> Rewrite mid last,
  fr_last   :: last    -> a -> Rewrite mid last} 

data BackRewrites mid last a = BackRewrites
 {br_first  :: BlockId  -> a  -> Rewrite mid last,
  br_middle :: mid      -> a  -> Rewrite mid last,
  br_last   :: last ->
               (BlockId -> a) -> Rewrite mid last} 
\end{code}
\caption{Rewrite functions for forward and backward transformations.}
\figlabel{rewrites}
\end{figure}


\begin{figure*}
\begin{codetable}
\T\begin{code}
availRewrites :: ForwardRewrites Middle Last AvailVars
availRewrites = ForwardRewrites first middle last
  where first _ _ = Nothing
        middle m avail = maybe_reload_before avail m (mkMiddle m)
        last   l avail = maybe_reload_before avail l (mkLast l)
        maybe_reload_before avail node tail =
            let used = filterVarsUsed (elemAvail avail) node
            in  if isEmptyVarSet used then Nothing
                else Just $ reloadTail used tail
        reloadTail vars t = foldl rel t $ varSetToList vars
          where rel t r = mkMiddle (reload r) <*> t
\end{code}\B
& Rewrite \mbox{functions}\\
\hline

\T\begin{code}
insertLateReloads :: (Graph Middle Last) -> DFMonad (Graph Middle Last)
insertLateReloads g = liftM zdfFpContents result
  where result = zdfRewriteFwd RewriteShallow "insert late reloads"
                               availVarsLattice avail_reloads_transfer
                               availRewrites (fact_bot availVarsLattice) g
\end{code}%
& Insert late reloads\\
\end{codetable}
\caption{Late-reload insertion, which relies on the analysis of \figref{avail}}
\figlabel{avail-rewrites}
\end{figure*}


We compute dataflow facts in order to enable code-improving
transformations.
When creating a transformation, the compiler writer uses
the assertions represented by dataflow facts to ensure that the
transformation preserves semantics.
We~write a transformation as a triple of
\emph{rewrite functions};
functions for both forward and backward transformations are shown in
\figref{rewrites}. 
%
A~rewrite function is given a dataflow fact and a node.
It~may choose to replace the node with a new graph~$g$, in which case it
returns $@Just@\;g$, or it may do nothing, in which case it returns @Nothing@.


To create a program transformation in our framework,
the compiler writer must 
\begin{itemize}
\item
Create a dataflow lattice and transfer functions for the supporting
analysis, as described in \secref{create-analysis}. 
\item
Create sound rewriting functions for first, middle, and last nodes;
it is sufficient that given the dataflow fact(s) coming into the node,
the behavior of 
any proposed replacement is not observably different from the original node.
\end{itemize}
Given the rewriting functions and the supporting analysis, the
compiler writer can then use the 
dataflow-engine function @zdfRewriteFwd@ to compute a solution:
\begin{code}
  zdfRewriteFwd 
    :: RewritingDepth         -- Rewrite recursively?
    -> PassName               -- Name of this pass
    -> DataflowLattice a      -- Lattice
    -> ForwardTransfers m l a -- Transfer functions
    -> ForwardRewrites m l a  -- Rewrite functions
    -> a                      -- Input fact
    -> Graph m l              -- Control-flow graph
    -> DFMonad (ForwardFixedPoint m l a (Graph m l))
\end{code}
%%  
%%  class DataflowSolverDirectiontransfers fixedpt =>
%%        DataflowDirection
%%          transfers fixedpt rewrites where
%%    zdfRewriteFrom :: (DebugNodes m l, Outputable a)
%%      => RewritingDepth    -- Recursive rewrites?
%%      -> BlockEnv a        -- Init facts
%%      -> PassName          -- Analysis name
%%      -> DataflowLattice a -- Lattice
%%      -> transfers m l a   -- Transfers
%%      -> rewrites m l a    -- Input fact
%%      -> a                 -- Input fact
%%      -> Graph m l         -- CFG
%%      -> DFMonad (fixedpt m l a (Graph m l))
%%
%%
As explained by \citet{lerner-grove-chambers:2002}, 
@zdfRewriteForward@ implements
the transformation
and its supporting
analysis in two phases:\seclabel{solver-phase}
\begin{itemize}
\item
In the first phase, when a rewrite function proposes to replace a
node, the replacement graph is analyzed recursively, and the results
of that analysis are used as the new dataflow
fact(s) flowing out of the original node.
But \emph{the original node is not replaced}; indeed, the replacement
graph is abandoned, and only the facts remain.
If,~during iteration, the original node is analyzed again, perhaps
with a more conservative input fact, the rewrite function may propose
a different replacement or even no replacement at all.
This phase is called the \emph{solver}.
The solver computes a fixed point of the dataflow analysis
\emph{as if} nodes were replaced, while avoiding ever replacing a node
unsafely.
\simon{The rewrite functions must presumably satisfy
some monotonicity property.  Something like: given a more informative
fact, the rewrite function will rewrite a node to a more informative graph
(in the fact lattice.).
\textbf{NR}: actually the only obligation of the rewrite function is
to preserve observable behavior.  There's no requirement that it be
monotonic or indeed that it do anything useful.  It just has to
preserve semantics (and be a pure function of course).
\textbf{SLPJ} In that case I think I could cook up a program that
would never reach a fixpoint. Imagine a liveness analysis with a loop;
x is initially unused anywhere.
At some assignment node inside the loop, the rewriter behaves as follows: 
if (and only if) x is dead downstream, 
make it alive by rewriting the assignment to mention x.
Now in each successive iteration x will go live/dead/live/dead etc.  I
maintain my claim that rewrite functions must satisfy some
monotonicity property.
\textbf{JD}: in the example you cite, monotonicity of facts at labels
means x cannot go live/dead/live/dead etc.  The only way we can think
of not to terminate is infinite ``deep rewriting.''
}
\item
Once the solver is complete, the resulting fixed point is sound,
and the facts in the fixed point are used by the second phase, in which
no dataflow facts change, but
each replacement proposed by a rewriting function is actually
performed.
This phase is called the \emph{rewriter}.
\end{itemize}


\emph{[When you analyze a written graph, you could get a fact that's lower,
but when you reach a join point, you take the join, and the fact
doesn't get smaller.  So things terminate.]}

\simon{But do we apply rewrites even before the analysis reaches a fixed point?
If so, what property do the rewrites have to satisfy to ensure soundness?
If not, even a single rewrite might destroy the fixed-point property of the
current facts.  Or perhaps we iterate the analysis to a fixpoint, and only \emph{then}
do rewriting? If so, do we need the transfer functions at that stage?

Also the fixed-point of the analysis relies on upward chains. What if
the rewrite pushed it downward?  Or is it the case that a rewrite must
change a node $n$ into a graph $g$ so 
that $\mathit{fwdtrans}(n) \leq \mathit{fwdtrans}(g)$?

Also the fixpoint calculation requires multiple passses; do the 
rewrites then apply multiple times?

I'm deliberately playing the role of the reader here, and not peeking at
the code.  I don't think it's enough to say ``go look at Chambers paper''; 
I suggest we say enough (half a column would do it) to address the obvious
questions and point to Chambers for details.


\textbf{NR}: Good questions, but let's have a forward reference to \secref{dfengine}}

The rewriting function is like the @zdfSolveFwd@ function in
\secref{zdfSolveFwd}, but it uses and produces extra
information.\seclabel{engine-truth} 
\begin{itemize}
\item
The @RewritingDepth@ parameter controls recursive rewriting;
if~a graph produced by a rewrite function cannot be further rewritten,
rewriting is \emph{shallow};
if~a graph produced by a rewrite function can be rewritten \emph{ad
infinitum}, rewriting is \emph{deep}.
Deep rewriting is essential to achieve the results of
\citet{lerner-grove-chambers:2002}, e.g., to remove the induction
variable from the loop in the example in \secref{induction-var-elim}.
\item
The rewriter requires rewriting functions as well as transfer
functions.
\item
The type constructor @ForwardFixedPoint@ has a fourth
type parameter,\footnote
{We lied to you earlier.}
which is a value contained in the fixed point.
The~value can be extracted using function @zdfFpContents@, which has
type @ForwardFixedPoint m l a b -> b@.
Here the type parameter~@b@ is instantiated to @Graph m l@: the fixed point
contains the rewritten graph.
In the solver, @b@~is instantiated with
the unit type~@()@. \simon{I'm afraid I do not understand this bullet
at all.}
\item
The computation is monadic in the \emph{dataflow monad}, written
@DFMonad@.
This monad provides a supply of fresh labels, and it
also helps implement the fault-isolation strategy invented by
\citet{whalley:isolation} (\secref{vpoiso}).
Because they share substantial implementation in common, the truth
about the solver is that it also returns a value of monadic type.
\end{itemize}

In the rest of this section we present two transformations:
\secref{sink-reloads} shows how to insert a reload instruction just
before the first use of each spilled variable, and
\secref{dead-code-elim} shows how to eliminate dead assignments.
When these two transformations are run in sequence, the effect is to
sink reloads and produce programs like the example shown in
\secref{spill-reload-example}. 





%%  After defining the lattice, the transfer functions, and the rewrite functions,
%%  the client runs the analysis by invoking the dataflow framework
%%  (see~\figref{framework-fns}).
%%  The function @zdfSolveFrom@ performs an analysis on an input control-flow graph,
%%  using a dataflow lattice and a set of transfer functions.
%%  The additional arguments to the function provide
%%  the name of the analysis,
%%  the initial set of dataflow facts (usually empty),
%%  and the initial fact (usually bottom)
%%  that flows into either the entry or exit of the graph,
%%  depending on whether the transfers define a forward or backward analysis.
%%  The result of the function is the fixed point of the analysis,
%%  which stores the dataflow fact on entry to each basic block.
%%  \john{Maybe we should export a simple version of these functions to clients?
%%    Do they always do the obvious things with the initial facts and in-fact?
%%    Initial facts can reuse results of a previous analysis, but then you lose
%%    interleaving.}
%%  
%%  To combine an analysis and a transformation,
%%  the client calls the @zdfRewriteFrom@ function,
%%  which takes the same arguments as @zdfSolveFrom@,
%%  with the addition of the set of rewrite functions
%%  and a parameter (@RewritingDepth@) that decides whether the result
%%  of a rewrite function should be considered for further rewriting.
%%  The result of the function is not only the fixed point of the
%%  analysis interleaved with the transformation,
%%  but also the transformed control-flow graph.
%%  



\begin{figure*}
\begin{codetable}
\T\begin{code}
deadRewrites = BackwardRewrites nothing middleRemoveDeads nothing Nothing
  where nothing _ _ = Nothing

middleRemoveDeads :: Middle -> CmmLive -> Maybe (Graph Middle Last)
middleRemoveDeads (MidAssign (CmmLocal x) _) live
    | not (x `elemVarSet` live) = Just emptyGraph
middleRemoveDeads _ _ = Nothing
\end{code}\B
& Rewrite \mbox{functions}\\
\hline

\T\begin{code}
removeDeadAssignments :: (Graph Middle Last) -> DFMonad (Graph Middle Last)
removeDeadAssignments g = liftM zdfFpContents result
     where result = zdfRewriteBwd RewriteDeep "dead-assignment elim"
                                  liveLattice liveTransfers rewrites emptyVarSet g
\end{code}%
& \mbox{Dead-code} elimination\\
\end{codetable}
\caption{Dead-assignment elimination, which relies on the analysis of
\figref{liveness}} 
\figlabel{dead-elim}
\end{figure*}


\subsection{Sinking reloads: a forward transformation}
\simon{Backward reference needed to ``Pass 2 of the algorithm described
in the preamble to Section 4''}

\simon{I believe that in fact we never make the call to @zdfSolveFwd@ given 
in Figure 3?  Instead we pass the transfer functions @avails\_reloads\_transfer@
to @zdfRewriteFwd@?  Whose call is not given in Figure 6. Perhpas we should give
the latter call, and point out that the call in Figure 3 is illustrative and
is not in fact used?}

\simon{Incidentally, I wonder if we should
use record notation when constructing @ForwardRewrites@?}

\seclabel{sink-reloads}

We can use the available-variables analysis of \secref{avail} to
insert reloads
immediately before uses of variables.
The transformation is implemented by the rewrite functions in the top
half of \figref{avail-rewrites}.
A~first node uses no variables and so is never rewritten.
For middle and last nodes, the @maybe_reload_before@ function
computes @used@, which is the set $\{ x \mid x = \slotof x\}$, i.e., 
the set of variables whose values are the same as the values in the
corresponding stack slots.
If that set is not empty, function
@reloadTail@ replaces @node@ with a new graph in which @node@ is
preceded by a (redundant) reload for each variable in the set~@used@.

To do the rewriting, the code uses several functions that create
graphs:
\begin{code}
mkMiddle :: m -> Graph m l
mkLast   :: l -> Graph m l
(<*>)    :: Graph m l -> Graph m l -> Graph m l
\end{code}
%% \simon{Aha!  Here is where we lie by pretending that @Graph@ = @AGraph@.  Ok.}
%% \simon{Does @mkLast@ really have a @LastNode l@ context? How does it use it?}
%%        Yes, it really does, but it's used only for prettyprinting
%%        the graph (for debugging).  So I've expunged it from the
%%        paper. ---NR
The infix @<*>@ function is graph concatenation.

The transformation is implemented by the call to @zdfRewriteFwd@
shown at the bottom of \figref{avail-rewrites}.
Rewriting is shallow, which is to say that a graph returned by
@maybe_reload_before@ is not itself rewritten.
(If~it \emph{were} rewritten, an nonempty @used@ set would make the
compiler insert an infinite sequence of reloads before @node@.)
Once the reloads are inserted, the original reloads immediately
following the call site are dead, and they can be eliminated by our
next transformation, dead-assignment elimination.

\subsection{Dead-assignment elimination: a backward \rlap{transformation}}
\simon{Backward reference needed to ``Pass 3 of the algorithm described
in the preamble to Section 4''}

\seclabel{dead-code-elimination}
\seclabel{dead-code-elim}

\seclabel{bwd-rewrite}


\def\liveout{$\mathit{live_{out}}$}

We use the liveness analysis of \secref{liveness} to identify
assignments which modify only variables that are not used.
Such \emph{dead assignments} can be removed without changing the
observable behavior of the program.
The removal is implemented by the rewrite functions in the top half of
\figref{dead-elim}. 
First and last nodes are not assignments and are therefore never
rewritten.
A~middle node is rewritten to the empty graph if and only if it is an
assignment to a dead variable.
In~the bottom half of \figref{dead-elim}, we call @zdfRewriteBwd@, and
that is all there is to it.
\john{Need to run this version of the code in anger.}
%
\remark{In this space we should have some guff about
composing transformations, which should refer to the example on
eliminating the induction variable.}




\begin{figure*}
\begin{code}
    forward_sol :: LastNode l => (forall a . Fuel -> Maybe a -> Maybe a) -> RewritingDepth -> PassName 
                -> BlockEnv a -> ForwardTransfers m l a -> ForwardRewrites m l a
                -> a -> Graph m l -> Fuel
                -> DFM a (ForwardFixedPoint m l a (), Fuel)
    forward_sol squash depth name start_facts transfers rewrites = ...
      where solve :: a -> Graph m l -> Fuel -> DFM a (a, Fuel)
            solve in_fact g fuel =
                  do { fuel <- run "forward" name set_successor_facts blocks fuel
                     ; b <- getExitFact
                     ; return (b, fuel) }
              where blocks = G.postorder_dfs g
                    set_successor_facts (Block id tail) fuel =
                      do { idfact <- getFact id
                         ; (last_outs, fuel) <-
                             ... solve_tail (ft_first_out transfers id idfact) tail fuel ...
                         ; set_or_save last_outs
                         ; return fuel }

            solve_tail in' (G.ZTail m t) fuel =
              case squash fuel $ fr_middle rewrites m in' of
                Nothing -> solve_tail (ft_middle_out transfers m in') t fuel
                Just g -> do { (a, fuel) <- subAnalysis' $
                                  case depth of
                                    RewriteDeep -> solve in' g (oneLessFuel fuel)
                                    RewriteShallow -> do { a <- anal_f getExitFact in' g
                                                         ; return (a, oneLessFuel fuel) }
                             ; solve_tail a t fuel
                             }
            ...
\end{code}
\caption{Excerpts from the forward solver}
\figlabel{solver-excerpts}
\end{figure*}



\section{The Dataflow Engine}
\seclabel{engine}
\seclabel{dfengine}

\simon{Need a linking sentence like ``So far we have discussed clients of 
our library.  Now we turn our attention to the implementation of the
library itself.''  Yes, you'll want to use different vocabulary!}

The dataflow engine is implemented in two layers.
The lower layer comprises the \emph{dataflow monad} and the \emph{fuel monad}.
The dataflow monad has type @DFM a@, where @a@~is the type of a
dataflow fact.
It keeps track of the values of dataflow facts as the
engine iterates.
The fuel monad contains
computations of type @Fuel -> (a, Fuel)@ (see \secref{vpoiso} below).
It provides a way to suppress
optimization selectively in order to isolate faults \cite{whalley:isolation}.

The upper layer of the dataflow engine is divided into four parts:
a forward solver, a forward rewriter,
a backward solver, and a backward rewriter.

Despite the implicit claim in the title of this paper,
the dataflow engine is not simple.
The benefit of our design is that the \emph{interface} to the dataflow
engine (lattice, transfer functions, rewriting functions) is simple,
and as shown above, compiler passes written \emph{using} the engine
are very simple indeed.
But~the engine itself is complex.

Most of the complexity in the dataflow engine arises because we
implement the ambitious algorithm of
\citet{lerner-grove-chambers:2002}, who write
\begin{quote}
\emph{Previous efforts to exploit [the mutually beneficial
interactions of dataflow analyses] either (1)~iteratively performed
each individual analysis until no further improvements are discovered
or (2)~developed [handwritten] ``super-analyses'' that manually
combine conceptually separate analyses. We have devised a new approach
that allows analyses to be defined independently while still enabling
them to be combined automatically and profitably. Our approach avoids
the loss of precision associated with iterating individual analyses
and the implementation difficulties of manually writing a
super-analysis.}
\end{quote}
Adapting this work to a purely functional setting results in an
implementation that is significantly simpler than the original.
We~sketch that implementation below.


%%  Note that the dataflow engine is the only part of the system that is
%%  hard to get right---this is where all the hair is.
%%  Prime benefit of our system is that once this is right, everything is
%%  easy (and indeed is just logic, strongest postcondition, or weakest
%%  precondition). 
%%  

\subsection{Throttling the dataflow engine using ``optimization
  fuel''}

\seclabel{vpoiso}

We have extended Lerner, Grove, and Chambers's algorithm with
Whalley's \citeyearpar{whalley:isolation} algorithm for isolating
faults.
Whalley's algorithm is used to test a faulty optimizer by automatically
finding the first program transformation that introduces a fault.
It works by giving the optimizer a finite supply of \emph{optimization
fuel}.
Each time a rewrite function proposes to replace a node, one unit of @Fuel@ is
consumed.
When the optimizer runs out of fuel, further rewrites are suppressed.
In~normal operation, the optimizer has unlimited fuel, but during
debugging, a fault can be isolated quickly by doing a binary search on
the size of the fuel supply.


\subsection{The forward solver}

\simon{By this time I'm getting tired.  This section has a lot of nitty
gritty detail but, becuase it is explaining a complex implementation,
it is necessarily incomplete. So the question in my mind is this: what
does the reader learn from the rest of Section 7?  What new insights
are gained, beyond a quick sketch of a particular implementation?
For example, is it notably simpler than the OCaml version or Chambers's version?
If so, what led to this simplicity?

Give the time, one possibility would be to trim Section 7 down a lot,
which would give more space (and time!) for clarifying remarks earlier.
(We are already up to page 14.)

There are quite a few Simon comments in what follows, but I gave it
much less time per line than earlier stuff.}

\remark{It's dramatically simpler than Chambers's stuff because it's
  purely functional and it's polymorphic.  For the OCaml stuff I'd
  have to compare more carefully.
I'd like John to weigh in here---I~understood him to say that if we
  don't deliver the implementation of the dataflow engine, we're
  omitting a critical part of our design.}
\simon{PS: currently the polymorphism is mentioned iter alia, but
is never the focus; yet it's one of our major claims, so it deserves
explicit treatment.  Not necessarily here.}

To avoid duplicating code,
\emph{the dataflow engine implements only composed
analysis and transformation}.
Pure analysis is as a special case in which no node is
ever rewritten.



\begin{figure*}
%%  forward_rew
%%          :: forall m l a . 
%%             (DebugNodes m l, LastNode l, Outputable a)
%%          => (forall a . Fuel -> Maybe a -> Maybe a)
%%          -> RewritingDepth
%%          -> BlockEnv a
%%          -> PassName
%%          -> ForwardTransfers m l a
%%          -> ForwardRewrites m l a
%%          -> a
%%          -> Graph m l
%%          -> Fuel
%%          -> DFM a (ForwardFixedPoint m l a (Graph m l), Fuel)
%%  forward_rew squash depth xstart_facts name transfers rewrites in_factx gx fuelx = fixed_pt_and_fuel
%%    where
%%      fixed_pt_and_fuel =
%%          do { (a, g, fuel) <- rewrite xstart_facts getExitFact in_factx gx fuelx
%%             ; facts <- getAllFacts
%%             ; let fp = ... facts ... g ...
%%             ; return (fp, fuel)
%%             }
%%  
\begin{code}
      ...
    rewrite :: BlockEnv a -> DFM a b -> a -> Graph m l -> Fuel -> DFM a (b, Graph m l, Fuel)
    rewrite start finish in_fact g fuel =
      let Graph eid blockenv = g
          blocks = G.postorder_dfs_from blockenv entry
      in do { forward_sol squash depth name start transfers rewrites in_fact g fuel
            ; (rewritten, fuel) <- rewrite_blocks blocks emptyBlockEnv fuel
            ; a <- finish
            ; return (a, Graph eid rewritten, fuel)
            }

      ...

    rewrite_blocks (Block id t : bs) rewritten fuel =
      ... rew_tail h (ft_first_out transfers id a) t rewritten fuel ...
    rewrite_blocks [] rewritten fuel = return (rewritten, fuel)

    rew_tail head in' (G.ZTail m t) rewritten fuel =
      case squash fuel $ fr_middle rewrites m in' of
        Nothing -> rew_tail (G.ZHead head m) (ft_middle_out transfers m in') t rewritten fuel
        Just g -> do { markGraphRewritten
                     ; (a, g, fuel) <- inner_rew getExitFact in' g fuel
                     ; let (blocks, h) = G.splice_head' head g
                     ; rew_tail h a t (blocks `plusBlockEnv` rewritten) fuel }
\end{code}
\caption{Excerpts from the forward rewriter}
\figlabel{rewriter-excerpts}
\end{figure*}

In a workshop paper \cite{ramsey-dias:applicative-flow-graph}, we
showed a backward solver and rewriter that kept dataflow facts in
mutable cells.
\emph{[New stuff is pure, polymorphic, and shows how to use
    fuel]}\remark{fix me}


\figref{solver-excerpts} shows excerpts from the forward solver 
@forward_sol@, which is the function used to implement
@zdfSolveFwd@.
As~it is also used to solve subgraphs and to help the rewriter, it
takes a few more parameters than @zdfSolveFwd@.
\begin{itemize}
\item
An analysis of a subgraph starts with known facts, not bottom facts.
These facts are stored in parameter
@start_facts@.
\item
In the solver phase described in \secref{solver-phase},
a node can be rewritten, the replacement graph is used to compute new
dataflow facts, and then the replacement graph is abandoned.
But sometimes we want to prevent a node from being rewritten at all,
either because
the solver is implementing a pure analysis, or because the optimizer
is out of fuel. 
We~do so using parameter @squash@.
\item
We track the amount of fuel remaining using
parameter @fuel@.
Parameter @fuel@ comes last because the fuel abstraction works with
functions of type @Fuel -> (a, Fuel)@.
\end{itemize}
A~fixed point is computed by initializing the facts using
@setAllFacts@, which is an operation in the dataflow monad.
\simon{I think it would help to list the operations of the monad. 
E.g. How should I think of @setAllFacts@ for example? Perhaps it's a state monad, whose
state includes a mapping from @BlockId@ to fact, and @setAllFacts@ initialises this
mapping?
\textbf{NR}:
I~had written a whole section on the monad, which was interminable
and seemed peripheral to the main point.  If we had the space I would
definitely go for a figure giving the operations mentioned here. 
I'm not sure we want to use our space that way, however.  It's a conundrum.
}
Function @solve@, also sketched in \figref{solver-excerpts}, finds a
fixed point and returns a pair containing the exit fact and the
remaining fuel supply.
This fact, plus other state extracted from the dataflow monad, are
used to create a fixed point.

\simon{@solve@ transforms the @Fuel@.  But isn't @Fuel@ part of the dataflow monad?
Is @DFMonad@ in the type of @zdfRewriteFwd@ the same as @DFM@ in this
figure?
\textbf{NR}:  Sadly not.  Is @DFMonad@ in the type of @zdfRewriteFwd@
is the composition of @DFM@ with @Fuel -> (a, Fuel)@.
I~fear we are trying to cram 10~pounds of information into a 5-pound sack.
}
 
Function @solve@ takes a graph
and applies a higher-order @run@ function,
not shown here,
% And don't you dare ask for its type.  The type of run would make you
% vomit.  ---NR
which repeatedly runs
@set_successor_facts@ each block in the graph, until the internal facts
stop changing.
Each block comes from  the list of reachable blocks in~@blocks@,
which is computed by postorder depth-first traversal of the graph.
Although any order will result in a correct fixed point,
postorder depth-first order hastens convergence of forward dataflow
analyses---often just a few iterations suffice.
(The @LastNode l@ constraint in the type of @forward_sol@
guarantees that we can
  find the successors of last nodes and so do the traversal.)
Once @run@ reaches a fixed point, @solve@ returns the exit fact and
the remaining fuel.
(Computation @getExitFact :: DFM a a@ is an operation in the dataflow
monad.)

To understand @set_successor_facts@, you must understand the
representation of basic blocks.
A~basic block is a sequence beginning with a first node, continuing
with zero or more middle nodes, and ending in a last node.
A~first node contains only a unique identifier of type @BlockId@; the
types of middle and last nodes are parameters.
A~sequence of middle nodes followed by a last node is reresented by a
value of type @ZTail m l@, where @m@~and~@l@
are the types of middle and last nodes:
\begin{code}
data ZTail m l = ZLast (ZLast l) | ZTail m (ZTail m l)
\end{code}
The type @ZLast l@ extends~@l@ with an additional case, @LastExit@,
which represents a subgraph that ends not in a control transfer but by
``falling off the end.''
A~block is the first node followed by the tail:
\begin{code}
data Block m l = Block BlockId (ZTail m l)
\end{code}

Function @set_successor_facts@ walks a basic block from first node to
last node, using and updating the state maintained by
the dataflow monad.
The initial fact is the fact associated with the block's identifier,
extracted using @getFact@, which has type @BlockId -> DFM a a@.
Function @solve_tail@ walks the block and computes @last_outs@, which
gives the new facts to be propagated to the successors of the blocks,
as defined in \figref{transfers}.
Function @set_or_save@ updates the facts in the dataflow monad,
and finally @set_successor_facts@ returns the remaining fuel.

The workhorse of the analysis is @solve_tail@, which handles three
cases: a middle node, a last node, and ``falling off the end.''
\figref{solver-excerpts} shows only the code for the middle node.
The first step is present the incoming fact~@in'@ and the middle
node~@m@ to the approriate rewrite function.
If~that function returns @Nothing@, or if the value is squashed to
@Nothing@,\footnote
{@squash fuel a@ returns @Nothing@ if @a@~is @Nothing@, if @fuel@~is
  exhausted, or if we are running @zdfSolveFwd@, which uses no rewrite
  functions.
%In~the last case, lazy evaluation guarantees that @rewrites@ is not
%evaluated. 
}
In~that case @solve_tail@ simply calls the transfer function
@ft_middle_out transfers@ and continues with the next node.



The interesting case occurs when the rewrite function 
@fr_middle rewrites@
proposes a replacement graph~@g@.
\remark{Orphaned text:
and
a list of facts that hold on edges leaving the graph, which is  
 extracted using function @zdfFpLastOuts@.
\iffalse %%% WRONG -- it's the @goto@ that does this...
In the example above, when the subgraph
@z = 7 + y@ is analyzed, @zdfFpLastOuts@ will contain
the pair $(@L2@, @x == 7@ \land @y == 8@)$.
\fi
}
\begin{enumerate}
\item
If we are doing \emph{deep} rewriting, @g@~may be further rewritten,
so we call @solve@ recursively after removing one unit of fuel.
Graph~@g@ replaces a middle node, so we expect @g@ to fall off the
end, and therefore we want @solve@ to extract the exit fact.
\item
If we are doing \emph{shallow} rewriting,  new graph~@g@ must not be
rewritten, but we must still find a fixed point of the transfer
equations.
We compute that fixed point using @anal_f@, which  recursively calls
@forward_sol@ using the @squash@ function
@\ _ _ -> Nothing@, 
thereby doing no rewriting and consuming no fuel.\remark{Line number
  to where squash is used}
\end{enumerate}
Whether rewriting is shallow or deep, the new graph~@g@ is analyzed as
a ``sub-analysis'' in the dataflow monad, which means it has fresh
state for tracking facts and deciding whether analysis has reached a
fixed point.
State changes made in a sub-analysis cannot be observed by the outer
analysis. 
In~fact, after the sub-analysis all the new state \emph{and the new
  graph} are discarded.
All~that is kept is the new dataflow fact~@a@ and the diminished
supply of fuel.
Function @solve_tail@ continues tail-recursively solving~@t@.
The only difference between rewriting \remark{line number} and not
rewriting \remark{line number} is the
\emph{fact} and the \emph{fuel} passed to the tail call.

The full implementations of @solve@ and @solve_tail@, which are
included in a supplement to this paper, contain similar code which may
rewrite first nodes and last nodes and sub-analyze the replacement
graphs. 


%%\begin{itemize}
%%\item
%%\item
%%The internal @solve@ function is higher-order in the parameter
%%@finish@, which extracts from the dataflow monad either the unique
%%exit fact or the set of @LastOuts@, depending on context.
%%\item
%%The function @set_or_save@ calls @setFact@ for @BlockId@s located
%%within graph~@g@ and calls @addLastOutFact@ for @BlockId@s located
%%outside graph~@g@.
%%\end{itemize}



\subsection{The forward rewriter}

The core of the rewriter is shown in \figref{rewriter-excerpts}.
A~graph~@g@ to be rewritten is reprsented by a finite map from  @BlockId@
  to basic block, together with the @BlockId@ of the entry block:
\begin{code}
data Graph m l = Graph BlockId (BlockEnv (Block m l))
\end{code}
The function @rewrite@ first calls @forward_sol@ to iterate to a fixed
point.
At~this point, the facts stored in the dataflow monad can safely be
used for rewriting.
The work is done by @rewrite_blocks@; the monadic action @finish@
extracts any value of interest; and that value is returned together
with the rewritten graph and the remaining fuel.

The actual rewriting is done by @rewrite_blocks@, whose second parameter is
an accumulating parameter which carries a finite map containing those
blocks which have been fully rewritten.
As~above, we show only the code relevant to rewriting middle codes;
code for first nodes and last nodes is similar.
The @rew_tail@ function uses exactly the same test as in the solver:
it scrutinizes
\begin{verbatim}
   squash fuel $ fr_middle rewrites m in'
\end{verbatim}
If there is no rewriting, the action taken is almost the same, but
there is an additional accumulating parameter, @head@, which stores
the fragment of the basic block leading up to the node~@m@ which is
a candidate for rewriting.
To~store this fragment, we use type @ZHead m@, which makes the
nearest node the easiest to access; we append~@m@ to the
fragment~@head@ by forming @ZHead head m@:
\begin{verbatim} 
data ZHead m = ZFirst BlockId | ZHead (ZHead m) m
\end{verbatim}

The interesting case is where the middle node~@m@ is rewritten into
the replacement graph~@g@. \remark{line number}
We~first mark the graph as rewritten, so that whatever called this
pass may know that the graph has changed.
As~before, we convert~@g@ by supplying it with fresh labels.
We then call @inner_rew@, which calls @forward_sol@ when rewriting
shallowly (i.e., not rewriting~@g@) and calls @rewrite@ recursively
when rewriting deeply.
Finally, the rewritten graph, newly rebound to the name~@g@, is
spliced onto the head of the block, producing a new head~@h@ and a set
of fully rewritten blocks~@blocks@.
The splicing algorithm is described by
\citet{ramsey-dias:applicative-control-flow}. 
The new head~@h@ replaces @head@, and @blocks@ is added to the
accumulating parameter @rewritten@.
Function @rew_tail@ continues with the next node by making a tail call
to rewrite~@t@.






%%  \subsection{The dataflow monad}
%%  
%%  The primary purpose of the dataflow monad is to keep track of 
%%  dataflow facts as the engine iterates.
%%  Dataflow facts are found in three places:
%%  \begin{itemize}
%%  \item
%%  There is a dataflow fact associated with every labelled basic block in
%%  the current graph;
%%  the dataflow monad maintains this association in a finite map.
%%  The functions @getFact@ and @setFact@ query and update this map.
%%  \item
%%  The current graph may be a subgraph of a larger graph, in which case a
%%  forward dataflow pass may produce dataflow facts that flow to labelled
%%  blocks that are outside the current graph.
%%  These facts must be retained and propagated even if the current graph
%%  is abandoned; such facts are added with @addLastOutFact@ and recovered
%%  with @bareLastOuts@.
%%  \item
%%  Finally, a foraward dataflow pass over a subgraph may propagate a fact forward by
%%  ``falling off the end;'' such a fact is set with @setExitFact@ and
%%  recovered with @getExitFact@.
%%  \end{itemize}
%%  In addition to keeping track of facts, 
%%  the dataflow monad provides a number of other facilities to manage
%%  changes in state as graphs are rewritten and facts climb the dataflow
%%  lattice:
%%  \begin{itemize}
%%  \item
%%  The monad keeps track of whether any fact has changed.
%%  \item
%%  It provides a @subanalysis@ function which makes it possible to
%%   analyze a subgraph using the current set of facts, then discard any
%%   changes in state that may have resulted from the analysis of the
%%   subgraph.
%%  \item
%%  It provides a supply of fresh @BlockId@s, which are available for use
%%  by rewrite functions.
%%  \item
%%  It tracks the supply of \emph{optimization fuel}.
%%  As~shown below, when fuel runs out, the dataflow engine stops
%%  calling rewriting functions, effectively halting optimization.
%%  Binary search on the size of the fuel supply enables the compiler to
%%  identify unsound rewrites quickly \cite{whalley:isolation}.
%%  \end{itemize}











\section{The next 700 dataflow analyses}

\seclabel{next-700}
\simon{I know that this section is Dear to the Ramsey Heart, but
I still can't figure out what it is trying to say to the reader.

Let me try one stab at what I think you might be trying to say.
In this paper we have described a re-usable, polymorphic 
framework (concretely, a library)
for the analysis and transformation of control flow graphs.
This is not a specialised tool; on the contrary we claim that
a huge variety of optimisations can be viewed through the lens of
dataflow optimisation.

If that's the thrust, then the generalities of ths section would
be stronger if they were supported with a list of dragon-book stuff
that are all special cases. 

But I may have misunderstood.}

\remark{Here's what I'm trying to say: Gentle Reader,
Bad people may have told you that dataflow analysis is new,
special, difficult, and intimidating.  But it's not really---it's
stuff you knew about all along.  Why, if you already understand the
classic material on program correctness, you know how to write a new
dataflow analysis---just invent an approximation of logic that has no
infinite ascending chains.

I could indeed go through the dragon book, but there's no time.}
\simon{Well that is certainly better. Why not say that?  It's much 
shorter too!  Evidence of the claim that data flow analysis is presented
as special difficult intimidating might be good.}


%%  \section{Logical view of optimization}
%%  
%%  Connection to Hoare logic:
%%   - facts derived by forward analysis as assertions on state
%%   - facts derived by backward analysis as assertions on continuation.
%%  
%%  Examples on board.

\seclabel{logic}


Dataflow analysis is closely related to the seminal work done by
\citet{floyd:assigning-meanings},
\citet{hoare:axiomatic-basis},
and
\citet{dijkstra:discipline}
on semantics of programming languages and
 formal methods of program construction.
%
We would like to elucidate the relationship between the formal methods a
person uses to 
\emph{construct} a new program with the formal methods a compiler uses
to \emph{analyze} a program that already exists.
Understanding the relationship is a bit tricky, because compilers and
people typically work in opposite directions.

The most highly developed
methodology for constructing a procedure (Dijkstra's) starts with a
postcondition~$Q$ that the procedure must establish and a
precondition~$P$ that the procedure may assume.
Using Dijkstra's method, a programmer 
works backwards, adding a statement, computing the weakest
precondition of that statement with respect to~$Q$,
and continuing with more statements until
eventually the weakest precondition of the procedure body is implied by~$P$.
A~compiler, by contrast, is given a procedure with no specification,
and the compiler typically reasons forward,  starting with the
weakest possible precondition, then using strongest postconditions to
compute assertions that hold at various points throughout the program.


Computing either strongest postconditions or weakest liberal
preconditions is a simple matter of calculation---except for loops.
Let's revisit the example of induction-variable elimination in
\secref{induction-var-elim}. 
If~a compiler wants to know what @i@~is, it can easily see that on the
first trip $@i@=0$, on the second trip $@i@=1$, and so on.
But it's all too easy for a compiler to get stuck trying to compute an
infinite disjunction like $\bigvee_{k=0}^{\infty} @i@=k$.
Compilers are no good at solving recursion
equations that require limits of infinite sequences.

People don't like infinite sequences either---instead they invent
\emph{loop invariants}. 
In the classic situation, using weakest preconditions, a loop
invariant~$I$ is an assertion that is \emph{stronger} than the weakest
precondition, is invariant under the loop, and when conjoined with
the loop's termination condition, implies the weakest precondition of
what follows the loop.
%%   and has these properties:
%%  \begin{itemize}
%%  \item
%%  Together with the loop-termination condition, the invariant~$I$ implies the weakest
%%  precondition of the code following the loop.
%%  \item
%%  The invariant~$I$ implies the weakest precondition of the loop body
%%  with respect to~$I$.
%%  In other words, $I$~is strong enought so that having $I$ hold at the
%%  beginning of the loop body is enough to
%%  imply that $I$~also holds at the end of the loop body.
%%  \end{itemize}
To be useful, a loop invariant must be neither too strong nor too
weak.
Much like finding a useful induction hypothesis for a proof, 
finding a loop invariant requires art and intuition.
In~all but the simplest cases, it is beyond the capabilities of any
compiler.

%%  Strongest postconditions also require a loop invariant, but with dual
%%  properties:
%%  \begin{itemize}
%%  \item
%%  The invariant~$I$ is weaker than the true strongest postcondition,
%%  but it is implied by the strongest postcondition of the code before
%%  the loop.
%%  \item
%%  Assuming invariant~$I$ holds on entry to the loop, the strongest
%%  postcondition of the loop body must imply~$I$.
%%  \item
%%  The strongest postcondition of the entire loop is deemed to be the
%%  conjunction of invariant~$I$ and the termination condition.
%%  \end{itemize}
%%  


So if we can't find loop invariants automatically, what's a poor
compiler writer to do?
The compiler writer wants assertions that justify interesting program
transformations. 
Such assertions can often be established by a dataflow analysis.
So to write a new dataflow analysis, a compiler writer
should \emph{invent a new logic} which is
\emph{expressive enough to justify interesting transformations}
but \emph{inexpressive enough that a fixed point
  is always reached in finitely many steps}.
With the logic in place, the transfer functions simply approximate
  weakest preconditions (for a backward analysis) or strongest
  postconditions (for a forward analysis) as best they can within the
  constraints of the logic.
Thus, the next 700 dataflow analyses will spring from 700 variations on
program logic, each of which is inexpressive in a different way.
\remark{Do we have examples or not?}

%%  Let's review the examples of \secref{example:xforms} in that light.
%%  \begin{itemize}
%%  \item
%%  Logic $\bigwedge_i x_i=k_i$.
%%  \begin{itemize}
%%  \item
%%  $i = 7 \join \true \equiv \true$ (no loss of information)
%%  \item
%%  $i = 7 \join i= 7 \equiv  i=7$ (no loss of information)
%%  \item
%%  $i = 7 \join i = 8 \implies \true$ (loss of information)
%%  \end{itemize}
%%  Because there are only finitely many variables, we can say only
%%  finitely many things at each program point, and we're guaranteed to
%%  reach a fixed point.
%%  \item
%%  Example: induction-variable elimination.
%%  \begin{itemize}
%%  \item
%%  What is the proposition?
%%  \item
%%  Key point: be able to express the intermediate state where the
%%  invariant is temporarily violated (first add 1 to @i@, then add 4
%%  to~@p@).
%%  \end{itemize}
%%  \end{itemize}


%\clearpage

\section{Conclusion}

Compiler textbooks make dataflow analysis and optimization appear
difficult and complicated.
We make dataflow simple not by using a single magic
ingredient, but by applying ideas that are well understood by functional
programmers and others who reason formally about programs.
These ideas
make it possible to think simple thoughts about classical code improvements.
\begin{itemize}
\item
We acknowledge only one program-analysis technique: the solution of
recursion equations.
Like our colleagues working in imperative languages, we solve
recursion equations by iterating to a fixed point.
Most equations relate
properties of program states; some relate properties of continuations.
\item
When looking at properties of program states, we consider only two
relations: weakest liberal precondition and strongest 
postcondition.
\john{Why is this tied to program states only?}
In the compiler literature, these relations correspond respectively to
``backward dataflow problems'' and ``forward dataflow problems.''
\simon{Can we give an exmple of a property of program states which is
neither, just by way of contrast; ie this we cannot do.}
%%  \item
%%  In a language that admits loops, iterating weakest preconditions or
%%  strongest postconditions typically does not reach a fixed point in
%%  finitely many steps; hence the need for loop invariants
%%  \cite{hoare:axiomatic-basis,dijkstra:discipline,gries:science-programming}.
%%  In \secref{logic-reconciled}, we show that we can guarantee to reach a
%%  fixed point by limiting what we can express in the logic.\remark{needs
%%  a fix}
%%  We~show that many classical analyses can be explained this way;
%%  the great diversity of classical analyses corresponds to a great
%%  diversity of inexpressive logics.
%%  This view leads us to a unifying principle:
%%  \emph{To implement a code-improving transformation, find the least
%%    expressive logic that can justify the transformation, then use that
%%    logic to compute strongest postconditions, which justify the
%%    transformation locally.}
\item
We consider only three program transformation techniques:
substitution of equals for equals, 
insertion of assignments to unobserved variables, 
and removal of assignments to unobserved variables. \simon{No such
restrictions were placed on the functions in @ForwardRewrites@.
If we expect such restrictions to be observed we should say so!}
Substituting equals for equals may require reasoning about program
states; for example, if we have a program point at which variable~$x$
is always~7, we are justified in substituting~7 for~$x$.
\remark{We can also justify substitution of \emph{labels} in goto
  statements by reasoning about continuations.  The introduction is
  probably not the place to mention this fact.}
Inserting and removing assignments require reasoning about program
continuations; an assignment may be inserted or removed if its results
are not used by its continuation.

%%  Some compiler texts treat the removal of unreachable code as a
%%  code-improving transformation in its own right.
%%  In~our framework, unreachable code becomes unreachable in the
%%  garbage-collection sense, so no special effort is required to remove
%%  it.
\item
These simple program transformations may be composed to form more
complex tranformations.
For example, both ``code motion'' and ``induction-variable
elimination'' are implemented in three stages: insert new assignments;
substitute equals for equals; remove unneeded assignments
(\secref{induction-var-elim}). 

\simon{I'd make these two bullet.  There are two things going
on.  First, a big transformation can be done by composing phases
(pass 1, 2, 3).  Benefit: modularity, plug and play.

Second each phase consists of lots of individual
rewrites, and it's OK to apply only some of these individual rewrites.
Hence we can isolate bugs to a very fine granularity.}

One virtue of implementing complex transformations by composing simple
transformations is that each simple transformation leaves the program
in a consistent state. \simon{Is that the same as ``leaves the semantics
of the program unchanged''?}
It~is therefore safe to use 
``optimization fuel'' to limit the number of \emph{simple}
transformations, which means that when we isolate a fault in the optimizer
(\secref{vpoiso}), we have the luxury of debugging a simple
 transformation, not a complex one.
\end{itemize}

We also blend proven implementation technqiues
in a way that
makes it possible to
write simple code
to implement classical code improvements.
\begin{itemize}
\item
To {analyze} programs, we use a purely applicative representation of
control-flow graphs, inspired by Huet's zipper
\cite{huet:zipper,ramsey-dias:applicative-flow-graph}. 
%
% important, but no longer mentioned in this paper:
%
%%  \item
%%  To \emph{construct} programs, we use a different representation of
%%  flow graphs, one which hides the complexity of the zipper and which
%%  provides a constant-time operation for joining flow graphs in
%%  sequence.
%%  It is inspired in part by Hughes's \citeyearpar{hughes:novel-lists}
%%  representation of lists, which supports a constant-time append operation.
\item
Improving on our own prior work, we make all flow-graph
types fully polymorphic in the representations of
assignments and control-flow operations.
%%  Although our polymorphic representations have been instantiated only
%%  with the low-level intermediate code used by the Glasgow Haskell
%%  Compiler, they are intended eventually to be instantiated with
%%  machine-dependent representations of target-machine instructions, as
%%  part of a larger project of refactoring GHC's back ends.
%
Making the control-flow graph polymorphic seems obvious in retrospect,
but we underestimated the degree to which polymorphism forces us to
separate concerns.
Even though at present the polymorphic types have but one instance
apiece, introducing polymorphism has made the code far simpler, easier
to understand, and easier to maintain. \simon{Is it possible to substantiate
this claim by examples?}
%
% gen and kill are history
%
%%\item
%%Judicious use of Haskell type classes makes is possible to write
%%weakest precondition or strongest postcondition using the ``transfer
%%equations'' that are familiar from compiler textbooks.
%%If you like, you can even write overloaded @gen@ and @kill@ functions.
%%The benefit is that it is easy to compare the actual code with the
%%abstract treatments found in textbooks\ifgenkill \ (\secref{gen-kill})\fi.
\item
Finally, we use both polymorphism and type classes to implement a single,
polymorphic,
higher-order ``dataflow engine,'' which composes analyses and transformations
\cite{lerner-grove-chambers:2002}. 
By~putting the sophisticated algorithms in the dataflow engine, 
we make it very easy to create a new analysis or transformation:
create a lattice representation for the assertions that can be
expressed in the logic;
create transfer functions that approximate weakest preconditions or
strongest postconditions;
and 
create rewrite functions that use the assertions to justify
interesting program transformations.  \simon{Isn't this
re-usability what Chambers was on about in his original paper?
In what way is it easier for us to create new analyses than it is for him?}
\end{itemize}
These ideas and techniques form a powerful toolkit for
analyzing and transforming imperative programs.
With this toolkit, a compiler writer can quickly get to the real
intellectual work of code improvement: identifying interesting
transformations and the assertions that can justify them.

\makeatother

\providecommand\includeftpref{\relax} %% total bafflement -- workaround
\IfFileExists{nrbib.tex}{\bibliography{cs,ramsey}}{\bibliography{cs,ramsey,simon,jd}}
\bibliographystyle{plainnatx}

\clearpage

\appendix

\section{Dataflow-engine functions}


\begin{figure*}
%%  forward_sol
%%          :: forall m l a . 
%%             (DebugNodes m l, LastNode l, Outputable a)
%%          => (forall a . Fuel -> Maybe a -> Maybe a)
%%  		-- Squashes proposed rewrites if there is
%%  		-- no more fuel; OR if we are doing a pure
%%  		-- analysis, so totally ignore the rewrite
%%  		-- ie. For pure-analysis the fn is (\_ _ -> Nothing)
%%          -> RewritingDepth	-- Shallow/deep
%%          -> PassName
%%          -> BlockEnv a		-- Initial set of facts
%%          -> ForwardTransfers m l a
%%          -> ForwardRewrites m l a
%%          -> a			-- Entry fact
%%          -> Graph m l
%%          -> Fuel
%%          -> DFM a (ForwardFixedPoint m l a (), Fuel)
\begin{code}
forward_sol squash rewrite name start_facts transfers rewrites = fixed_point
 where 
   fixed_point in_fact g fuel =
     do { setAllFacts start_facts
        ; (a, fuel) <- solve getExitFact in_fact g fuel
        ; facts <- getAllFacts
        ; let fp = ... facts ...
        ; return (fp, fuel)
        }

   solve :: DFM a b -> a -> Graph m l -> Fuel -> DFM a (b, Fuel)
   solve finish in_fact (Graph entry blockenv) fuel =
     let blocks = G.postorder_dfs_from blockenv entry
         set_or_save = mk_set_or_save (isJust . lookupBlockEnv blockenv)
         set_successor_facts (Block id _ tail) fuel =
           do { idfact <- getFact id
              ; (last_outs, fuel) <-
                  case squash fuel $ fr_first rewrites id idfact of
                    Nothing -> solve_tail (ft_first_out transfers id idfact) tail fuel
                    Just g -> do { (a, fuel) <- subAnalysis' $
                                     case rewrite of
                                       RewriteDeep -> solve getExitFact idfact g (oneLessFuel fuel)
                                       RewriteShallow ->
                                         do { a <- anal_f getExitFact idfact g
                                            ; return (a, oneLessFuel fuel) }
                                  ; solve_tail a tail fuel }
              ; set_or_save last_outs
              ; return fuel }

     in do { (last_outs, fuel) <- solve_tail in_fact entry fuel
           ; set_or_save last_outs                                    
           ; fuel <- run "forward" name set_successor_facts blocks fuel
           ; b <- finish
           ; return (b, fuel)
           }

   solve_tail in' (G.ZTail m t) fuel =
     case squash fuel $ fr_middle rewrites m in' of
       Nothing -> solve_tail (ft_middle_out transfers m in') t fuel
       Just g -> do { (a, fuel) <- subAnalysis' $                                     
                         case rewrite of                                              
                           RewriteDeep -> solve getExitFact in' g (oneLessFuel fuel)  
                           RewriteShallow -> do { a <- anal_f getExitFact in' g       
                                                ; return (a, oneLessFuel fuel) }      
                    ; solve_tail a t fuel                                             
                    }

   solve_tail in' (G.ZLast l) fuel = 
     case squash fuel $ either_last rewrites in' l of
       Nothing -> case l of LastOther l -> return (ft_last_outs transfers l in', fuel)
                            LastExit -> do { setExitFact in' ; return (LastOuts [], fuel) }
       Just g -> do { (last_outs :: LastOuts a, fuel) <- subAnalysis' $           
                        case rewrite of                                               
                          RewriteDeep -> solve lastOutFacts in' g (oneLessFuel fuel)  
                          RewriteShallow -> do { los <- anal_f lastOutFacts in' g     
                                               ; return (los, fuel) }                 
                    ; return (last_outs, fuel)                                        
                    }

   anal_f :: DFM a b -> a -> Graph m l -> DFM a b
   anal_f finish in' g = do { fwd_pure_anal name emptyBlockEnv transfers in' g; finish }
   either_last rewrites in' (LastExit) = in'
   either_last rewrites in' (LastOther l) = fr_last rewrites l in'
\end{code}
\caption{The forward solver}
\end{figure*}

\begin{figure*}
%%  forward_rew
%%          :: forall m l a . 
%%             (DebugNodes m l, LastNode l, Outputable a)
%%          => (forall a . Fuel -> Maybe a -> Maybe a)
%%          -> RewritingDepth
%%          -> BlockEnv a
%%          -> PassName
%%          -> ForwardTransfers m l a
%%          -> ForwardRewrites m l a
%%          -> a
%%          -> Graph m l
%%          -> Fuel
%%          -> DFM a (ForwardFixedPoint m l a (Graph m l), Fuel)
\begin{code}
forward_rew squash depth xstart_facts name transfers rewrites in_factx gx fuelx = fixed_pt_and_fuel
  where
    fixed_pt_and_fuel =
        do { (a, g, fuel) <- rewrite xstart_facts getExitFact in_factx gx fuelx
           ; facts <- getAllFacts
           ; let fp = ... facts ... g ...
           ; return (fp, fuel)
           }
    solve = forward_sol squash
    rewrite :: BlockEnv a -> DFM a b -> a -> Graph m l -> Fuel -> DFM a (b, Graph m l, Fuel)
    rewrite start finish in_fact g fuel =
      let Graph entry blockenv = g
          blocks = G.postorder_dfs_from blockenv entry
      in do { solve depth name start transfers rewrites in_fact g fuel
            ; eid <- freshBlockId "temporary entry id"
            ; (rewritten, fuel) <- rew_tail (ZFirst eid emptyStackInfo) in_fact entry emptyBlockEnv fuel
            ; (rewritten, fuel) <- rewrite_blocks blocks rewritten fuel
            ; a <- finish
            ; return (a, lgraphToGraph (LGraph eid 0 rewritten), fuel)
            }
    don't_rewrite facts finish in_fact g fuel =
        do  { solve depth name facts transfers rewrites in_fact g fuel
            ; a <- finish
            ; return (a, g, fuel)
            }
    inner_rew :: DFM a f -> a -> Graph m l -> Fuel -> DFM a (f, Graph m l, Fuel)
    inner_rew f i g fu = getAllFacts >>= \facts -> inner_rew' facts f i g fu
        where inner_rew' = case depth of RewriteShallow -> don't_rewrite
                                         RewriteDeep    -> rewrite
    rewrite_blocks :: [Block m l] -> (BlockEnv (Block m l)) -> Fuel -> DFM a (BlockEnv (Block m l), Fuel)
    rewrite_blocks [] rewritten fuel = return (rewritten, fuel)
    rewrite_blocks (G.Block id off t : bs) rewritten fuel =
      do { let h = ZFirst id off
         ; a <- getFact id
         ; case squash fuel $ fr_first rewrites id a of
             Nothing -> do { (rewritten, fuel) <- rew_tail h (ft_first_out transfers id a) t rewritten fuel
                           ; rewrite_blocks bs rewritten fuel }
             Just g  -> do { markGraphRewritten
                           ; (outfact, g, fuel) <- inner_rew getExitFact a g fuel
                           ; let (blocks, h) = splice_head' h g
                           ; (rewritten, fuel) <- rew_tail h outfact t (blocks `plusBlockEnv` rewritten) fuel
                           ; rewrite_blocks bs rewritten fuel }
         }

    rew_tail head in' (G.ZTail m t) rewritten fuel =
      case squash fuel $ fr_middle rewrites m in' of
        Nothing -> rew_tail (G.ZHead head m) (ft_middle_out transfers m in') t rewritten fuel
        Just g -> do { markGraphRewritten
                     ; (a, g, fuel) <- inner_rew getExitFact in' g fuel
                     ; let (blocks, h) = G.splice_head' head g
                     ; rew_tail h a t (blocks `plusBlockEnv` rewritten) fuel }
    rew_tail h in' (G.ZLast l) rewritten fuel = 
      case squash fuel $ either_last rewrites in' l of
        Nothing -> return (insertBlock (zipht h (G.ZLast l)) rewritten, fuel)
        Just g -> do { markGraphRewritten
                     ; ((), g, fuel) <- inner_rew (return ()) in' g fuel
                     ; let g' = G.splice_head_only' h g
                     ; return (G.lg_blocks g' `plusBlockEnv` rewritten, fuel) }
    either_last rewrites in' (LastExit) = in'
    either_last rewrites in' (LastOther l) = fr_last rewrites l in'
\end{code}
\caption{The forward rewriter}
\end{figure*}






\end{document}
