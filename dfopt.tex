\newif\ifpagetuning \pagetuningtrue  % adjust page breaks

\newif\ifnoauthornotes\noauthornotestrue
\newif\iftimestamp\timestamptrue  % show MD5 stamp of paper

\IfFileExists{timestamp.tex}{}{\timestampfalse}




\newif\ifgenkill\genkillfalse  % have a section on gen and kill
\genkilltrue


\newif\ifnotesinmargin \notesinmargintrue 
\IfFileExists{notesinline.tex}{\notesinmarginfalse}{\relax}

\documentclass[blockstyle,preprint,natbib,nocopyrightspace]{sigplanconf}

\newcommand\vfilbreak[1][\baselineskip]{%
  \vskip 0pt plus #1 \penalty -200 \vskip 0pt plus -#1 }

\usepackage{alltt}
\usepackage{array}
\newcommand\lbr{\char`\{}
\newcommand\rbr{\char`\}}
 
\clubpenalty=10000
\widowpenalty=10000

\usepackage{verbatim} % allows to define \begin{smallcode}
\newenvironment{smallcode}{\par\unskip\small\verbatim}{\endverbatim}

\newcommand\lineref[1]{line~\ref{line:#1}}
\newcommand\linepairref[2]{lines \ref{line:#1}~and~\ref{line:#2}}
\newcommand\linerangeref[2]{\mbox{lines~\ref{line:#1}--\ref{line:#2}}}
\newcommand\Lineref[1]{Line~\ref{line:#1}}
\newcommand\Linepairref[2]{Lines \ref{line:#1}~and~\ref{line:#2}}
\newcommand\Linerangeref[2]{\mbox{Lines~\ref{line:#1}--\ref{line:#2}}}

\makeatletter

\let\c@table=\c@figure % one counter for tables and figures, please

\newcommand\setlabel[1]{%
  \setlabel@#1!!\@endsetlabel
}
\def\setlabel@#1!#2!#3\@endsetlabel{%
  \ifx*#1*% line begins with label or is empty
     \ifx*#2*% line is empty
        \verbatim@line{}%
     \else
       \@stripbangs#3\@endsetlabel%
       \label{line:#2}%
     \fi
  \else
     \@stripbangs#1!#2!#3\@endsetlabel%
  \fi
}
\def\@stripbangs#1!!\@endsetlabel{%
  \verbatim@line{#1}%
}


\verbatim@line{hello mama}

\newcounter{codeline}
\newenvironment{numberedcode}
  {\endgraf
     \def\verbatim@processline{%
        \noindent
        \expandafter\ifx\expandafter+\the\verbatim@line+  % blank line
               {\small\textit{\def\rmdefault{cmr}\rmfamily\phantom{00}\phantom{: \,}}}%
            \else
               \refstepcounter{codeline}%
               {\small\textit{\def\rmdefault{cmr}\rmfamily\phantom{00}\llap{\arabic{codeline}}: \,}}%
            \fi
        \expandafter\setlabel\expandafter{\the\verbatim@line}%
        \the\verbatim@line\par}%
   \verbatim
   }
   {\endverbatim}

\makeatother

\newcommand\arrow{\rightarrow}

\newcommand\join{\sqcup}
\newcommand\slotof[1]{\ensuremath{s_{#1}}}
\newcommand\tempof[1]{\ensuremath{t_{#1}}}
\let\tempOf=\tempof
\let\slotOf=\slotof

\makeatletter
\newcommand{\nrmono}[1]{%
  {\@tempdima = \fontdimen2\font\relax
   \texttt{\spaceskip = 1.1\@tempdima #1}}}
\makeatother

\usepackage{times}  % denser fonts
\renewcommand{\ttdefault}{aett} % \texttt that goes better with times fonts
\usepackage{enumerate}
\usepackage{url}
\usepackage{graphicx}
\usepackage{natbib}  % redundant for Simon
\bibpunct();A{},
\let\cite\citep
\let\citeyearnopar=\citeyear
\let\citeyear=\citeyearpar

\usepackage[ps2pdf,bookmarksopen,breaklinks,pdftitle=dataflow-made-simple]{hyperref}

\newcommand\naive{na\"\i ve}

\usepackage{amsfonts}
\newcommand\naturals{\ensuremath{\mathbb{N}}}
\newcommand\true{\ensuremath{\mathbf{true}}}
\newcommand\implies{\supseteq}  % could use \Rightarrow?

\newcommand\PAL{\mbox{C{\texttt{-{}-}}}}
\newcommand\high[1]{\mbox{\fboxsep=1pt \smash{\fbox{\vrule height 6pt
   depth 0pt width 0pt \leavevmode \kern 1pt #1}}}}

\usepackage{tabularx}

% Put figures in boxes --- WHY??? --NR
\usepackage{float}
\floatstyle{boxed}
\restylefloat{figure}
\restylefloat{table}



% Set \noauthornotestrue to suppress notes
\noauthornotesfalse
%\newcommand{\qed}{QED}
\ifnotesinmargin
  \long\def\authornote#1{%
      \ifvmode
         \marginpar{\raggedright\hbadness=10000
         \parindent=8pt \parskip=2pt
         \def\baselinestretch{0.8}\tiny
         \itshape\noindent #1\par}%
      \else
          \unskip\raisebox{-3.5pt}{\rlap{$\scriptstyle\diamond$}}%
          \marginpar{\raggedright\hbadness=10000
         \parindent=8pt \parskip=2pt
         \def\baselinestretch{0.8}\tiny
         \itshape\noindent #1\par}%
      \fi}
\else
  % Simon: please set \notesinmarginfalse on the first line
  \long\def\authornote#{{\em #1\/}}
\fi
\ifnoauthornotes
  \def\authornote#1{\unskip\relax}
\fi

\newcommand{\simon}[1]{\authornote{SLPJ: #1}}
\newcommand{\norman}[1]{\authornote{NR: #1}}
\let\remark\norman
\def\finalremark#1{\relax}
% \let \finalremark \remark % uncomment after submission
\newcommand{\john}[1]{\authornote{JD: #1}}
\newcommand{\todo}[1]{\textbf{To~do:} \emph{#1}}
\newcommand\delendum[1]{\unskip\relax}

\newcommand\secref[1]{Section~\ref{sec:#1}}
\newcommand\secreftwo[2]{Sections \ref{sec:#1}~and~\ref{sec:#2}}
\newcommand\seclabel[1]{\label{sec:#1}}

\newcommand\figref[1]{Figure~\ref{fig:#1}}
\newcommand\figlabel[1]{\label{fig:#1}}

\newcommand\tabref[1]{Table~\ref{tab:#1}}
\newcommand\tablabel[1]{\label{tab:#1}}


\newcommand{\CPS}{\textbf{StkMan}}    % Not sure what to call it.


\usepackage{code}   % At-sign notation

\iftimestamp
\input{timestamp}
\preprintfooter{\mdfivestamp}
\fi

\begin{document}
\title{Dataflow Optimization Made Simple}
%\subtitle{\today}

\authorinfo{Norman Ramsey}{Tufts University}{nr@cs.tufts.edu}
\authorinfo{Jo\~ao Dias}{Tufts University}{dias@cs.tufts.edu}
\authorinfo{Simon Peyton Jones}{Microsoft Research}{simonpj@microsoft.com}


\maketitle
 
\begin{abstract}
We present a Haskell library that makes it easy for compiler writers
to implement program transformations based on dataflow analyses.
The compiler writer must identify (a)~a family of logical assertions
on which the transformation will be based;
(b)~an {approximate}
representation of such assertions, which
must have a lattice structure such that every assertion can be increased at
most finitely many times;
(c)~transfer functions that approximate weakest preconditions or
strongest postconditions over the assertions; and
(d)~rewrite functions whose soundness is justified by the assertions.
To~guide compiler writers,
we show how dataflow analyses are related to
seminal work on program 
correctness. \simon{The ``next 700'' section sort of does so, but I'm not 
sure it deserves mention in the abstract.}
Finally, our library uses the algorithm of 
\citet{lerner-grove-chambers:2002}, which enables compiler writers to
compose very simple analyses and transformations in a way that achieves
the same precision as complex, handwritten
``super-analyses.''
Our library is the workhorse of a new
back end for the Glasgow Haskell Compiler.
\end{abstract}

\makeatactive   %  Enable @foo@ notation

\section{Introduction}

\ifpagetuning\enlargethispage{\baselineskip}\fi

If you write a compiler for an imperative language, you can exploit
many years' work on code improvement, also called
``optimization.''
But the work is typically presented
as a collection of apparently unrelated analyses and
transformations, each with its own name---and 
it's not always easy to % not everyone can easily
remember how something like
``copy propagation'' differs from ``constant propagation.''
Many presentations obscure fundamental principles of
code improvement.

%The contribution of this paper is to elucidate a large body of work on code
%improvement; the body of work known as ``dataflow optimization.''
This paper makes two contributions:
\begin{itemize}
\item
We show that the ad-hoc ``optimization zoo'' consists mostly of special
cases of reasoning techniques that have long been understood and used
by semanticists and functional programmers:
assertions about states, assertions about continuations, and
substitution of equals for equals (\secref{next-700}).
What distinguishes dataflow optimization from classic formal reasoning
about programs is that in dataflow optimization, all assertions are
computed automatically, and they are
\emph{approximated}. 
\item
We embody our ideas in a library that makes it not just
possible but \emph{easy} to adapt imperative, dataflow-based
code-improvement techniques to a purely functional compiler.
Analyses and transformations, which use the library,
are small, simple, and easy to get right.
The library itself
\ifpagetuning uses \else is built around \fi
 a sophisticated algorithm which is
hard to get right but is 
written once and reused.
\end{itemize}

We consider code improvements over low-level
imperative codes, including intermediate languages and machine
languages.
As \citet{benitez-davidson:portable-optimizer} have shown, all the
classical scalar and loop optimizations can be performed over such
codes.
Moreover, any compiled functional program is
translated to such a code.


We introduce the subject by analyzing and transforming example code 
(\secref{example:xforms}),
thinking about and justifying classical optimizations using
Hoare logic and substitution of equals for equals.
To~support our claim that we make dataflow optimization easy, 
we spend most of the paper explaining how
to create new dataflow analyses or transformations
(\secref{making-simple}) and showing complete implementations of significant
analyses (\secref{example-analyses}) and transformations
(\secref{example-rewrites}) from the Glasgow Haskell Compiler.
We spend less space on our library's implementation (\secref{engine}).


\section{Dataflow analysis {\&} transformation by \rlap{example}}

\seclabel{example:transforms}
\seclabel{example:xforms}

In dataflow optimization, code-improving transformations are justified
by assertions about programs;
such assertions are often computed using
strongest postconditions or weakest liberal preconditions.
The most typical transformations are
insertion of assignments to unobserved variables,
substitution of equals for equals, 
and
removal of assignments to unobserved variables,
all of which preserve semantics.
Insertion and removal can be composed to achieve the effect called
``code motion.''
The examples below express classical code
improvements by composing small analyses and transformations.



\ifpagetuning \enlargethispage{1\baselineskip} \fi 
    % gets bottoms of columns to match

\subsection{Simple transformations}

\seclabel{constant-propagation}

Here is a sequence of assignments separated by assertions.
We compute the assertions by starting with the weakest possible
assertion (@true@) and computing strongest postconditions.
% \footnote
{Variables do not alias.}
\begin{verbatim}
    { true }
  x = 7;
    { x == 7 }
  y = 8: 
    { x == 7 && y == 8 }
  z = x + y;
\end{verbatim}
\delendum{SLPJ asks: Could we add $x=7,y=8,z=15$ as a final assertion?
We can but we should not, because a \naive\ sp function 
would produce the assertion $x=7\land y = 8
\land z = x+y$. To reach the point you desire, some sort of simplifier
would be required, and it is better to let the conclusion emerge
naturally as  the code is rewritten.}
In the assignment to~@z@, the assertion @x == 7@ justifies
substituting 7~for~@x@, leaving @z = 7 + y@.  
This transformation is traditionally called ``constant propagation.''
We may also substitute 8~for~@y@.
Finally, because @7 + 8 == 15@, we may again substitute equals for
equals, leaving the final assignment as
\begin{verbatim}
  z = 15;
\end{verbatim}
The final transformation, although also an instance of substituting equals
for equals, has a different name: ``constant folding.''

\subsection{A complex transformation}

\finalremark{It's a pity that this transformation occupies nearly the
entire second page,  
and then plays no subsequent role in the paper whatsoever.
One possibility: move it to ``the next 700'' section, as a substantiating example
to the claims made there.
But then we'd need another example here... well the sink/reload example of
Section 4 might be perfect.}

\seclabel{induction-var-elim}


The loop optimization known as ``induction-variable elimination'' 
can be composed from simpler transformations.
We begin by showing a simple loop, FORTRAN style,
although the code is~C:
\begin{verbatim}
  struct pixel { double r, g, b; };
  double sum_r(struct pixel a[], int n) {
    double x = 0.0;
    int i;
    for (i = 0; i < n; i++)
      x += a[i].r;
    return x;
  }
\end{verbatim}
To explain the improvement we wish to make, we show the same
code at the machine level.
We write machine-level examples using our low-level compiler-target
language,~{\PAL} % tuning for line breaks
\cite{peyton-jones-ramsey:garbage-collection:inproceedings,peyton-jones-ramsey:exceptions}: 
\begin{verbatim}
  sum_r("address" bits32 a, bits32 n) {
       bits64 x; bits32 i;
       x = 0.0;
       i = 0;
   L1: if (i >= n) goto L2;
       x = %fadd(x, bits64[a+i*24]);
       i = i + 1;
       goto L1;
   L2: return x; 
  }
\end{verbatim}
The code improvement called ``induction-variable elimination''
replaces~@i@ with a new variable~@p@ so that we can avoid repeating
the computation @a+i*24@ on each iteration through the loop.
The new variable~@p@ is intended to satisfy the invariant
\begin{verbatim}
   { p == a + i * 24 }
\end{verbatim}
The variable @i@ is also used in the loop-termination test.
To rewrite that test, 
we introduce a new variable @lim@ satisfying
the invariant % line breaking
@lim == a + n * 24@,
so that @i >= n@ if and only if @p >= lim@.

We implement the code improvement as a sequence of transformations.
After each transformation, the observable behavior of the program is unchanged.
%
Our first transformation declares @p@ and @lim@ and inserts suitable
assignments. 
New code is \high{boxed}\,.
\begin{alltt}
  sum_r("address" bits32 a, bits32 n) \lbr
       bits64 x; bits32 i; \high{bits32 p, lim;}
       x = 0.0;
       i = 0; \high{p = a; lim = a + n * 24;}
   L1: if (i >= n) goto L2;
       x = %fadd(x, bits64[a+i*24]);
       i = i + 1; \high{p = p + 24;}
       goto L1;
   L2: return x; 
  \rbr
\end{alltt}

As written, the assignments to @p@~and~@lim@ have no
effect on the program, but they enable the compiler to establish the assertions
@p == a + i * 24@ and @(i >= n) == (p >= lim)@.
On~the basis of these assertions, the compiler substitutes equals for
equals, resulting in the new code in boxes below:
\begin{alltt}
  sum_r("address" bits32 a, bits32 n) \lbr
       bits64 x; bits32 i; bits32 p, lim;
       x = 0.0;
       i = 0; p = a; lim = a + n * 24;
   L1: if (\high{p >= lim}) goto L2;
       x = %fadd(x, bits64[\kern1pt{}\high{p}\kern1pt{}]);
       i = i + 1; p = p + 24;
       goto L1;
   L2: return x; 
  \rbr
\end{alltt}

At this point, the compiler switches from reasoning about states to
reasoning about continuations.
In~particular, we reason about whether the value of a variable can be
used by a continuation; this reasoning is called ``liveness analysis.''
A~\naive\ analysis would show that although @i@~is not live at
label~@L2@, it is nevertheless live immediately after 
the assignment
@i = i + 1@ in the loop body,
because the value of~@i@ could be used by the next iteration of the
loop.
But we use Lerner, Grove, and Chambers's
\citeyearpar{lerner-grove-chambers:2002} algorithm to
\emph{interleave} liveness analysis with 
``dead-assignment elimination.'' \seclabel{interleave-introduced}%
Dead-assignment elimination removes an assignment if the variable
assigned to is not live, that is, if it cannot be used by the
assignment's continuation.
As~explained by Lerner, Grove, and Chambers, no sequential
composition of liveness analysis and dead-assignment elimination can
get rid of these assignments to~@i@, but interleaving analysis with
transformation does the trick.\footnote
{An experienced reader might be tempted to modify the
liveness analysis so that
@i = i + 1@ is not considered a ``use'' of~@i@ if @i@~is itself
dead.
This~modification is tantamount to writing a single, combined dataflow
pass that understands \emph{both} liveness analysis and dead-code
elimination.
In~this particular example, writing a single, combined pass presents
few difficulties, but the approach does not scale:
most combined passes are more complicated than the examples shown here;
the cost of writing combined passes does not scale linearly with
the number of individual passes;
combined passes often cannot be composed;
and 
some combined passes require nonstandard, handwritten traversals of
the control-flow graph.
\citet{lerner-grove-chambers:2002} discuss these issues in detail;
\citet{click-cooper} show the advantages of combining separate passes.}
\secref{dfengine} describes our implementation of their interleaving
method, which eliminates the boxed assignments to~@i@:
\begin{alltt}
sum_r("address" bits32 a, bits32 n) \lbr
     bits64 x; \high{bits32 i;} bits32 p, lim;
     x = 0.0;
     \high{i = 0;} p = a; lim = a + n * 24;
 L1: if ({p >= lim}) goto L2;
     x = %fadd(x, bits64[{p}]);
     \high{i = i + 1;} p = p + 24;
     goto L1;
 L2: return x; 
\rbr
\end{alltt}

After the insertion of assignments to @p@~and~@lim@, the substitution
of equals for equals, and the removal of newly dead assignments
to~@i@, we have ``eliminated the induction variable:''
\begin{alltt}
sum_r("address" bits32 a, bits32 n) \lbr
     bits64 x; bits32 p, lim;
     x = 0.0;
     p = a; lim = a + n * 24;
 L1: if ({p >= lim}) goto L2;
     x = %fadd(x, bits64[{p}]);
     p = p + 24;
     goto L1;
 L2: return x; 
\rbr
\end{alltt}

  
\section {Making dataflow simple}

\seclabel{making-simple}

\seclabel{create-analysis}

The goal of dataflow optimization is to compute valid
assertions, then use those assertions to justify code-improving
transformations.
%
% only Don Knuth knows why, but the paragraph break gets us three
% bullets on this page where a single paragraph gets only two!
%
Assertions are represented as
\emph{dataflow facts}.
Dataflow facts relate to
 traditional 
program logic as follows:
\begin{itemize}
\item
A dataflow fact is usually equivalent to an assertion about program state or
about a continuation.
For example, in \secref{constant-propagation}, @x == 7@ is a dataflow
fact that describes the program state. 


%%  There are two kinds of dataflow facts.
%%  The first kind is an assertion about the paths from the procedure
%%  entry to a program point;
%%  these facts are computed by a forward dataflow analysis.
%%  A common special case is an assertion about state at a program point,
%%  such as the assertion @x == 7@ in \secref{constant-propagation}.
%%  % Assertions about program state are usually sufficient to show that
%%  % a transformation preserves semantics,
%%  % but to decide whether a transformation will improve the code,
%%  % we sometimes need an assertion that describes how the program
%%  % state was established.
%%  
%%  The second kind of dataflow fact is an assertion about paths from the program point
%%  to the procedure exit;
%%  these facts are computed by a backward dataflow analysis.
%%  In the parlance of functional programmers, these dataflow facts are assertions on
%%  \emph{continuations}.
%%  
%%  Assertions about state are easy to formalize, but path properties are harder;
%%  we describe path properties informally in our assertions.
\item
A~set of dataflow facts forms a lattice.
To ensure that analysis terminates,
it is enough if
no fact has more than finitely many distinct facts above it.
\item
Each analysis or transformation may use a different lattice of
dataflow facts.
\end{itemize}

An assertion about a continuation is an assertion about paths
\emph{from} a program point 
to the procedure {exit};
such assertions are established by a \emph{backward dataflow analysis}.
An~assertion about paths \emph{to} a program point from the procedure
{entry} is established by a \emph{forward dataflow analysis}.
In an important special case,
an assertion may say simply
that all paths to a point establish a predicate, such as @x == 7@
above, which describes the program 
state at that point.


A~program point is represented as an edge in
a \emph{control-flow graph}.\footnote
{We discuss only intraprocedural optimization;
interprocedural optimizations are the work of the GHC~inliner
\cite{peyton-jones:secrets-inliner}.} 
The edges connect nodes, each of which represents a label, an assignment, or
a control transfer.

To write a dataflow \emph{analysis}, the compiler
writer must 
\begin{itemize}
\item
Choose a representation~$F$ of dataflow facts and a logical interpretation
thereof.
\item
Implement lattice operations over~$F$ (\secref{lattices}).
\item
Write \emph{transfer functions} that relate dataflow facts before and
after each type of node (\secref{tffuns}).
\delendum{I'd italicise key words from all three bullets, or none. NR:
It's not a question of bullets; the key concepts which are possibly
new to readers are transfer
functions and rewrite functions, which is why they are italicized.}
\end{itemize}

To write a \emph{transformation}
based on an analysis, the compiler writer
must also
create a \emph{rewrite function}, which is presented with a
flow-graph node and with the dataflow facts on the edges coming
into that node.
The function either suggests that the node should be replaced with a
fresh subgraph, or it leaves the node alone.
The rewrite function uses incoming facts to guarantee that
any proposed replacement preserves semantics.
For example, in \secref{constant-propagation} the fact @x == 7@ is
used to justify replacing @z = x + y@ with @z = 7 + y@.

\begin{table}
\centerline{%
\begin{tabular}{@{}>{\raggedright\arraybackslash}p{1.03in}>{\scshape}c>{\scshape}
      c>{\raggedright\arraybackslash}p{1.2in}@{}}
&\multicolumn1{r}{\llap{\emph{Specified}}\hspace*{-0.3em}}&
\multicolumn1{l}{\hspace*{-0.4em}\rlap{\emph{Implemented}}}&\\
\multicolumn1{c}{\emph{Part of optimizer}}
&\multicolumn1{c}{\emph{by}}&
\multicolumn1{c}{\emph{by}}&
\multicolumn1{c}{\emph{How many}}%
\\[5pt]
Control-flow graphs& Us & Us & One \\
Nodes in a control-flow graph & You & You & Two types per intermediate language \\[3pt]
Dataflow lattice & Us & You & One per logic \\[3pt]
Transfer functions & Us & You & One set per analysis \\
Rewrite functions & Us & You & One set per transformation \\[3pt]
Iterative solver functions & Us & Us & Two (forward \&\ backward) \\
Solve-and-rewrite functions & Us & Us & Two (forward \&\ backward) \\
\end{tabular}%
}
\caption{Parts of an optimizer}
\tablabel{parts}
\end{table}



Our library defines the types of lattice operations,
control-flow graphs, transfer functions, and rewrite functions.
\emph{All}~these types are parameterized by the types of
nodes in the control-flow graph, so
the library can be used with many intermediate languages.
The function types are also parameterized by the type of dataflow
facts, so a compiler writer may define different analyses, 
using different types of facts,
all operating over one type of graph. 

\tabref{parts} shows how our code interacts with client code that you
might write.
Client code passes
lattice
operations, transfer functions, and rewrite functions
to the part of our library we call the \emph{dataflow engine}.
The dataflow engine includes
\emph{solver functions}, 
which use a forward or backward \emph{analysis} to compute 
a dataflow fact for each program point (\secref{zdfSolveFwd}).
It also includes \emph{rewrite functions}, which
use a forward or backward \emph{transformation} to compute facts and to
rewrite a control-flow graph in light of those facts
(\secref{rewrites}).
Its implementation is sketched in \secref{dfengine}.
\delendum{Interface?  Perhpas ``The signatures and expected usage of these
functions is described in xxx, while their implementation is sketched
in yyy''.  NR: I like the idea of parallel structure, but the paper is
not really very parallel here.  Faced with either repeating the
section references or distorting what those sections are about, I've
chosen instead to abandon parallel structure.}



\begin{figure}
\begin{code}
data ChangeFlag = NoChange | SomeChange
data TxRes a    = TxRes ChangeFlag a
data DataflowLattice a = DataflowLattice
 {fact_bot        :: a,
  fact_add_to     :: a -> a -> TxRes a,
  fact_name       :: String } -- for debugging
\end{code}
\caption{Representation of a dataflow lattice} \figlabel{lattice-type} \figlabel{lattice}
\end{figure}


\subsection{Dataflow lattices}

\seclabel{lattices}

As an example, 
we present a lattice of facts about constant propagation.
At any program point, a standard constant-propagation analysis
computes exactly one of three
facts about a variable~$x$:
\begin{itemize}
\item
The analysis shows that
$x = k$, where $k$~is a compile-time constant of type @Const@.
\item
The analysis shows that $x$~is \emph{not} a compile-time constant.
We~notate this fact as $x = \top$.
\item
The analysis shows nothing about~$x$, which we notate $x=\bot$.
\end{itemize}
The bottom element of the lattice is~$x=\bot$, and
\remark{I have a note saying ``it turns out that,'' but I've forgotten where
this phrase was supposed to be inserted.}
the join operation \emph{approximates} disjunction:\simon{No emphasis needed.}
a~disjunction of two inconsistent facts is represented by~$x=\top$.
\simon{I'd still like to give an intuition here about
\emph{why} disjunction is the key operation, and/or where is is used.
NR: Maybe in \secref{next-700}?
%For example: ``Disjunction is to combine constant-propagation facts
%arriving at a label; for example, if two blocks branch to L, and $x=7$ holds
%at one branch, while $x=8$ holds at the, then the disjunction $x=7 \lor x=8$ holds
%at L.  In the particular language of facts chosen above, this disjunction
%approximated by $x=\top$.''  
}%
\simon{This would be a better place for the footnote. NR: Which footnote?}
Here are some examples:
\begin{itemize}
\item
$i = 7 \lor i=\bot \equiv i=7$ (no loss of information)
\item
$i = 7 \lor i= 7 \equiv  i=7$ (no loss of information)
\item
$i = 7 \lor i = 8 \equiv i = \top$ (loss of information)\footnote
{Client code determines how much information is lost.
For example, in a similar analysis for a functional language,
you might wish to track whether a value~$v$ is known to
be the application of a value constructor~$C_i$.
In this analysis, there's no need to limit the representation to a
single constructor or to~$\top$;
you could choose to represent such facts as ``$v$~is
the application of a constructor drawn from the set $\{C_1, C_2,
C_4\}$.''
Because every algebraic data type has finitely many constructors,
there are finitely many sets and therefore finitely many facts in the
lattice, so a dataflow analysis over this lattice would always reach a
fixed point.
}
%%  The compiler writer gets to choose how much information is lost;
%%  in a different analysis, it might be useful to 
%%  choose a lattice which can say
%%  that the value of a variable is one of,
%%  say, at most three constants.

\end{itemize}

The lattice  used by the analysis is the Cartesian product of the
lattices for all the local variables.
We~represent this lattice as a finite map from variable
to a value of type @Maybe Const@.
If~a variable $x$ is not in the domain of the map then $x=\bot$;
if $x$~maps to @Nothing@ then $x=\top$; if $x$~maps to $@Just@\;k$ then
$x=k$.


The dataflow engine uses the lattice join operation in a stylized way.
Joins occur at labels.
If~$f_{\mathit{id}}$ is the fact currently associated with the
label~$\mathit{id}$, 
and if a transfer function propagates a new fact~$f_{\mathit{new}}$
into the label~$\mathit{id}$, 
the dataflow engine replaces $f_{\mathit{id}}$ with
the join  $f_{\mathit{new}} \join f_{\mathit{id}}$.
Furthermore, the dataflow engine wants to know if
  $f_{\mathit{new}} \join f_{\mathit{id}} = f_{\mathit{id}}$,
because if not, it has not reached a fixed point.

In the example above, in any one program there are only finitely many variables;
only finitely many facts are computed at any program point;
and any one fact can increase at most twice.
These properties
ensure that the dataflow engine will
reach a fixed point.



When computing a join, 
it is typically cheap to learn if the join
is equal to one of the arguments.
We therefore use the nonstandard representation of lattice operations
shown in \figref{lattice}.
The join operation~$\join$ and equality test~$=$ are represented by a
single function called @fact_add_to@.
The term $@fact_add_to@\;f_{\mathit{new}}\;f_{\mathit{id}}$ is equal to
$@TxRes NoChange@\; f_{\mathit{id}}$ if $f_{\mathit{new}} \join f_{\mathit{id}} = f_{\mathit{id}}$
and is equal to
$@TxRes SomeChange@\; (f_{\mathit{new}} \join f_{\mathit{id}})$ otherwise.
The @fact_bot@ value is the bottom element, 
and @fact_name@  is used for debugging.

\begin{figure}
\begin{code}
newtype LastOuts a = LastOuts [(BlockId, a)]
data ForwardTransfers mid last a = ForwardTransfers
 {ft_first_out  :: BlockId -> a -> a,
  ft_middle_out :: mid     -> a -> a,
  ft_last_outs  :: last    -> a -> LastOuts a} 

data BackTransfers mid last a = BackTransfers
 {bt_first_in  :: BlockId -> a              -> a,
  bt_middle_in :: mid     -> a              -> a,
  bt_last_in   :: last    -> (BlockId -> a) -> a} 
\end{code}
\caption{Transfer functions for forward and backward analyses.}
\figlabel{transfers}
%
% elided: 
%    ft_exit_out   ::            a -> a
%
\end{figure}



\subsection{Transfer functions} \seclabel{tffuns}

A~transfer function is presented with dataflow facts on edges coming
into a node, and it computes dataflow facts on outgoing edges.
To~understand transfer functions, we must 
understand how the library organize the nodes and edges of a control-flow graph.

\seclabel{graph.intro}

A~control-flow graph is a collection of labelled \emph{basic blocks}.
Each basic block is a sequence beginning with a \emph{first node},
containing zero or more \emph{middle nodes},
and ending in a \emph{last node}.
A~first node is always a label;
a~typical middle node assigns to a register or memory
location;
a~typical last node is a conditional, unconditional, or indirect branch.
The client code choose the types of middle and last nodes, and in this
way defines its intermediate
representation.
(An~optimizer also works with \emph{subgraphs}, which may be missing a
first node or a last node, as discussed in \secref{subgraphs}.)




First nodes are the only targets of control transfers;
middle nodes never perform control transfers;
and
last nodes always perform control transfers.
Hence, a~first node has arbitrarily many predecessors and exactly one
successor;
a~middle node has exactly one predecessor and one successor;
and a last node has exactly one predecessor and arbitrarily many
successors. 

These constraints on number of predecessors and successors determine
the signatures of 
transfer functions, 
which are shown in \figref{transfers}.
For each type of node (first, middle, last) and for each kind of
analysis (forward, backward), there is a distinct transfer function.
Functions are grouped by kind of analysis, and each group is
parameterized over a dataflow fact of type~@a@ and over the types
@mid@ and @last@ of middle and last nodes.  


The type @BlockId@ represents a label.
Because a fact in a forward analysis typically represents an assertion
about program state,
 and because passing a label does not change
program state, the transfer function @ft_first_out@ is typically 
@flip const@---a variation on
the
identity function.\footnote
{One counterexample is in an
analysis, similar to a dominator analysis,
which we use to help convert {\PAL} to continuation-passing style.
The  analysis
helps us not to duplicate code that is common to multiple continuations.
}
For a middle node, the transfer function @ft_middle_out@ is given a
node and a precondition and returns an approximation of the strongest
postcondition. 
For a last node, different postconditions may be propagated to
different successors; for example, the true and false successors of a
conditional branch may accumulate information implied by the truth or
falsehood of the condition.
A~collection of (successor, fact) pairs is represented by a value of
type @LastOuts a@.




In a forward analysis, the dataflow engine starts with the fact at the
beginning of a block and applies transfer functions to the nodes in
that block until eventually the transfer function for the last node
computes the facts that are propagated to the block's successors.
For example, in the basic block
\begin{verbatim}
  L1: x = 7;
      y = 8;
      z = x + y;
      goto L2;
\end{verbatim}
a forward analysis would propagate the fact 
$@x == 7@ \land @y == 8@$, which we will call $f_{\mathit{new}}$,
along the edge to~@L2@. 
%%  \remark{We've elected not to get to the level of detail where
%%  we show how propagating a fact~$f$ through \mbox{@x = 7;@} results in a new
%%  fact either $(f \setminus @x@) \land @x == 7@$.}
The dataflow engine then \emph{replaces} the current fact
at~@L2@~($f_{\mathtt{L2}}$) with the lattice join $f_{\mathit{new}}
\join f_{\mathtt{L2}}$. 
The dataflow engine iterates over the blocks repeatedly, creating new
facts~$f$ and joining them with facts $f_{\mathit{id}}$ until
\mbox{$f \join f_{\mathit{id}} = f_{\mathit{id}}$} at every label~$\mathit{id}$.
When the facts at labels stop changing, the dataflow
engine has reached a fixed point.
%\remark{Promissory note: compose this analysis with two
%transformations: constant propagation and constant folding}


%% \ifpagetuning\enlargethispage{0.5\baselineskip}\fi


\subsection{Running the dataflow engine}

\seclabel{zdfSolveFwd}

Given lattice operations of type @DataflowLattice a@
and transfer functions of type @ForwardTransfers m l a@,
the compiler writer can run the analysis by calling our library
function @zdfSolveFwd@, which is part of our dataflow engine:
\begin{code}
 zdfSolveFwd 
  :: PassName               -- Name of this analysis
  -> DataflowLattice a      -- Lattice
  -> ForwardTransfers m l a -- Transfer functions
  -> a                      -- Input fact
  -> Graph m l              -- Control-flow graph
  -> DFMonad (FwdFixedPoint m l a)
\end{code}
The function is polymorphic in the types of middle and last nodes
@m@~and~@l@ and in the type of the dataflow fact~@a@.
The first three arguments characterize the analysis.
The next argument is the dataflow fact that holds on entry to the
graph;
because a procedure's caller may establish some facts about
parameters or about the stack,
this fact
is not always~$\bot$.
The last argument to @zdfSolveFwd@ is the graph, and the result is a 
monadic computation that produces a
fixed point.\footnote
{The type of @zdfSolveFwd@ has pedagogical lies;
truth is told in~\secref{engine-truth}.}
%%  \remark{One of the lies is
%%  that I've deliberately conflated the three representations of graphs.}
The \emph{dataflow monad}, written
@DFMonad@,
%This monad keeps track of the current dataflow fact at each label,
provides a supply of fresh labels
and
also helps implement the fault-isolation strategy described
in \secref{vpoiso}.


The @FwdFixedPoint@ data structure is a big bag
of information about the solution.
The most significant information is
a finite map from each block label to the dataflow fact that holds at
the label, which is extracted using function @zdfFpFacts@:
% Simon really wants these type signatures!
\begin{verbatim}
type BlockEnv a = Data.Map BlockId a
zdfFpFacts :: FwdFixedPoint m l a -> BlockEnv a
\end{verbatim}






\iffalse
% never tell the whole truth!
\begin{code}
class DataflowSolverDirection
        transfers fixedpt where
  zdfSolveFrom :: (DebugNodes m l, Outputable a)
    => BlockEnv a        -- Init facts
    -> PassName          -- Analysis name
    -> DataflowLattice a -- Lattice
    -> transfers m l a   -- Transfers
    -> a                 -- Input fact
    -> Graph m l         -- CFG
    -> DFMonad (fixedpt m l a ())
\end{code}
\fi



%%  The main contribution of this paper is to present an interface which
%%  enables many powerful program transformations based on dataflow
%%  analysis while keeping the individual dataflow passes as simple as
%%  possible.
%%  We keep the \emph{concepts} simple by relating dataflow facts and
%%  transfer functions to classic work in program correctness, as
%%  discussed in \secref{next-700}.
%%  We keep the \emph{implementations of dataflow passes} simple by pushing
%%  as much work as
%%  possible into the dataflow engine, which is implemented just once.
%%  We have also made the dataflow engine and its interface polymorphic in
%%  the types of 
%%  the nodes that appear in the control-flow graph \secref{polymorphic-framework}.
%%  Parametricity ensures separation of concerns between the dataflow
%%  engine and the individual dataflow passes.


\section{Example analysis passes}

\seclabel{example-analyses}


\newcommand\T{\rule{0pt}{0.6ex}}
\newcommand\B{\rule[-0.05ex]{0pt}{0pt}}
\newcolumntype{C}{>{\begin{minipage}{5.35in}}l<{\end{minipage}}} % code
\newcolumntype{L}{>{\Large\bfseries}m{1.3in}<{\centering}}       % label
\newenvironment{codetable}
  {\setcounter{codeline}{0}%
   \let\code=\numberedcode
   \let\endcode=\endnumberedcode
   \begin{tabular}{CL}%
  }
  {\end{tabular}}

\begin{figure*}
\setcounter{codeline}{0}
\begin{codetable}
\T\begin{numberedcode}
!UniverseMinus!data AvailVars = UniverseMinus VarSet
!AvailVars!               | AvailVars     VarSet
!extendAvail!extendAvail  :: AvailVars -> LocalVar  -> AvailVars  -- add var to set
delFromAvail :: AvailVars -> LocalVar  -> AvailVars  -- remove var from set
elemAvail    :: AvailVars -> LocalVar  -> Bool       -- set membership
interAvail   :: AvailVars -> AvailVars -> AvailVars  -- set intersection
!smallerAvail!smallerAvail :: AvailVars -> AvailVars -> Bool       -- compare sizes
\end{numberedcode}%
\B
& Dataflow fact and operations\\
\hline

\T\begin{code}
availVarsLattice :: DataflowLattice AvailVars
availVarsLattice = DataflowLattice "reloaded registers" empty add
    where empty = UniverseMinus emptyVarSet
          add new old = let join = interAvail new old in
                        if join `smallerAvail` old then aTx join else noTx join
\end{code}%
\B
& Lattice\\
\hline

%%  \T\begin{code}
%%  agen  :: UserOfLocalVars    a => a -> AvailVars -> AvailVars
%%  akill :: DefinerOfLocalVars a => a -> AvailVars -> AvailVars
%%  agen  a avail = foldVarsUsed extendAvail  avail a
%%  akill a avail = foldVarsDefd delFromAvail avail a
%%  \end{code}\B
%%  & Gen/Kill \mbox{functions}\\
%%  \hline
%%  
\T\begin{code}
avail_vars_transfer :: ForwardTransfers Middle Last AvailVars
!avail.first!avail_vars_transfer = ForwardTransfers (flip const) middleAvail lastAvail

middleAvail :: Middle -> AvailVars -> AvailVars
!reload1!middleAvail (MidAssign (CmmLocal x) (CmmLoad l) avail
!reload2!                 | l `isStackSlotOf` x = extendAvail avail x
!assign.avail.1!middleAvail (MidAssign lhs _expr) avail = 
!assign.avail.2!  foldVarsDefd delFromAvail avail lhs  -- remove variables defined in 'lhs'
!store.avail.spill.1!middleAvail (MidStore l (CmmReg (CmmLocal x))) avail
!store.avail.spill.2!                 | l `isStackSlotOf` x = avail
!store.avail.otherslot.1!middleAvail (MidStore l _) avail 
!store.avail.otherslot.2!  | isStackSlot l = delFromAvail avail (variableOfSlot l)
!store.avail.other!middleAvail (MidStore {}) avail = avail

lastAvail :: Last -> AvailVars -> LastOuts AvailVars
!avail.LastCall!lastAvail (LastCall _ (Just k) _ _) _ = LastOuts [(k, AvailVars emptyVarSet)]
lastAvail l avail = LastOuts $ map (\id -> (id, avail)) $ succs l
\end{code}%
\B
& Transfer \mbox{functions}\\
\hline

\T\begin{code}
cmmAvailableVars :: Graph Middle Last -> BlockEnv AvailVars
cmmAvailableVars g = zdfFpFacts soln
!avail.solve.1!  where soln = zdfSolveFwd "available variables" availVarsLattice 
!avail.solve.2!               avail_vars_transfer (fact_bot availVarsLattice) g
\end{code}%
\B
& Available-variables analysis\\
%\hline

\end{codetable}
% \caption{Available-variable analysis}
\caption{Dataflow analysis pass to compute available variables}
\figlabel{avail-all}
\figlabel{avail}
\figlabel{avail-lattice}
\figlabel{avail-gen-kill}
\figlabel{avail-transfers}
\figlabel{avail-running}
\end{figure*}
% Old captions' text:
% The dataflow fact for the available-reload analysis describes
%   the set of registers for which a reload is available.
%   We list the types of the functions that manipuate sets of available registers,
%   as well as the definition of the lattice.
% The standard gen and kill functions for available expressions
% The transfer functions for the available-reloads analysis.
% Running the available-reloads analysis and extracting the results with \texttt{zdfFpFacts}
% The rewrite functions to insert redundant reloads immediately before uses

% Probably no space for the implementations:
% interAvail (UniverseMinus s) (UniverseMinus s') =
%   UniverseMinus (s `plusVarSet`  s')
% interAvail (AvailVars     s) (AvailVars     s') =
%   AvailVars (s `timesVarSet` s')
% interAvail (AvailVars     s) (UniverseMinus s') =
%   AvailVars (s  `minusVarSet` s')
% interAvail (UniverseMinus s) (AvailVars     s') =
%   AvailVars (s' `minusVarSet` s )
% 
% smallerAvail (AvailVars _) (UniverseMinus _) = True
% smallerAvail (UniverseMinus _) (AvailVars _) = False
% smallerAvail (AvailVars     s) (AvailVars    s')  =
%   sizeVarSet s < sizeVarSet s'
% smallerAvail (UniverseMinus s) (UniverseMinus s') =
%   sizeVarSet s > sizeVarSet s'
% 
% extendAvail (UniverseMinus s) r =
%   UniverseMinus (deleteFromVarSet s r)
% extendAvail (AvailVars     s) r =
%   AvailVars (extendVarSet s r)
% 
% delFromAvail (UniverseMinus s) r =
%   UniverseMinus (extendVarSet s r)
% delFromAvail (AvailVars     s) r =
%   AvailVars (deleteFromVarSet s r)
% 
% elemAvail (UniverseMinus s) r =
%   not $ elemVarSet r s
% elemAvail (AvailVars     s) r =
%   elemVarSet r s


We claim that our library
makes it easy to write compiler passes based on dataflow.
In~this section we provide evidence for that claim by showing
implementations of two analyses;
related transformations appear in \secref{example-rewrites}. 
The example analyses help solve a real problem in the Glasgow Haskell
Compiler:
because most calls are tail calls, GHC uses no 
callee-saves registers.
Therefore, at each (rare) non-tail call, all live
variables must be spilled to the stack.
In order to reduce register pressure,
such variables are spilled as early as possible and reloaded as late as possible.

To illustrate the results of the example analyses and transformations,
here is a contrived example program in the style of \secref{example:xforms}:
\begin{alltt}
f (bits32 a) \lbr
  bits32 w, x, y, z;  // local variables
  x = a * a;
  w = a + a + a;
  y = g(w);           // call; x must be spilled
  z = y + y;
  if (y > 0) \lbr
    return z;
  \rbr else \lbr
    return z + x;
  \rbr
\rbr
\end{alltt}
A~spill and a reload should be inserted as follows:%
\seclabel{spill-reload-example} % space is deliberate
\newcommand\bigstrut{%
  \leavevmode\vrule width 0pt height 11pt depth 6pt }
\begin{alltt}
f (bits32 a) \lbr
  bits32 w, x, y, z;  
  x = a * a;
  \high{SPILL x;}
  w = a + a + a;    // no register pressure from x
  y = g(w);
  z = y + y;        // no register pressure from x
  if (y > 0) \lbr
    return z;       // x does not need reloading
  \rbr else \lbr
    \high{RELOAD x;}
    return z + x;
  \rbr
\rbr
\end{alltt}
Although the @SPILL@ and @RELOAD@ operations are introduced because of
the call to @g(a)@, they are moved as far from the call as possible:
the variable~@x@ is spilled immediately after being assigned @a * a@,
and @x@ is reloaded not
immediately after the call to~@g@, but just before its use in the
expression @z + x@.
 On the control-flow path to @return z@, @x@~needn't be reloaded
at all.

\ifpagetuning\enlargethispage{0.5\baselineskip}\fi 
  % this one has national-security importance!



Spills and reloads are inserted in the right places
by a sequence of three dataflow passes:
\begin{enumerate}
\item
\label{insert-spills}
A backward analysis computes liveness
to identify the variables that should be spilled at call sites.
An accompanying transformation inserts reloads immediately after each call
site and inserts spills not immediately before call sites, but
rather immediately after the appropriate reaching definitions.
\item
\label{reload-duplication}
A forward analysis finds ``available variables'' which have been reloaded
from the stack, and an accompanying transformation
inserts redundant reloads before their uses.
By keeping variables on the stack longer, this pass reduces register pressure.
% \simon{Why is the second pass a second pass?  The first
% pass added spills and reloads in; could the first pass not have 
% added the reloads immediately before the
% reloaded variables are used as well?  Maybe the text can say?
% I think the answer is that this is a \emph{forwards} analysis, since
% it is propagating forward the information about which variables currently
% have an up-to-date stack slot.}
\item
\label{remove-dead-reloads}
A backward analysis computes liveness,
and an accompanying transformation, dead-assignment elimination,
removes redundant reloads.
\end{enumerate}
Pass~\ref{insert-spills} is not shown in this paper.
Passes
\ref{reload-duplication}~and~\ref{remove-dead-reloads} cooperate to ``sink''
the reloads as far as possible from the call site.
The analyses used in
passes~\ref{reload-duplication}~and~\ref{remove-dead-reloads}
are described in \secreftwo{avail}{liveness};
the transformations are described in
\secreftwo{sink-reloads}{dead-code-elimination}.
%\simon{I would put the forward references
%to the sub-sections in the bullets themselves; they are more easy to
%find there.
%  No room, and not worthy of a bullet! ---NR}



\subsection{Available variables: a forward analysis} 

\seclabel{avail}


%%%%%%%%%%%%%%%%%%%%%%%%%%%%%%%%%%%%%%%%%%%%%%%%%%%%%%%%%%%%%%%%%%
%
%  desperately trying to get figures to emerge in a decent order
%
\begin{figure*}
\begin{codetable}
\T\begin{code}
!Live!type Live = VarSet
\end{code}%
\B
& Dataflow fact\\
\hline

\T\begin{code}
!liveLattice!liveLattice :: DataflowLattice Live
liveLattice = DataflowLattice "live LocalReg's" emptyVarSet add
  where add new old =
          let join = unionVarSets new old in
!liveLattice.end!          (if sizeVarSet join > sizeVarSet old then aTx else noTx) join
\end{code}%
\B
& Lattice\\
\hline

%%  \T\begin{code}
%%  gen  :: UserOfLocalVars    a => a -> Live -> Live
%%  kill :: DefinerOfLocalVars a => a -> Live -> Live 
%%  gen  a live = foldVarsUsed extendVarSet  live a
%%  kill a live = foldVarsDefd delFromVarSet live a
%%  \end{code}\B
%%  & Gen/Kill \mbox{functions}\\
%%  \hline

\T\begin{code}
liveTransfers :: BackwardTransfers Middle Last Live
liveTransfers = BackwardTransfers (flip const) middleLiveness lastLiveness

middleLiveness :: Middle -> Live -> Live
lastLiveness   :: Last -> (BlockId -> Live) -> Live
!middleLiveness!middleLiveness m = addUsed m . remDefd m
!lastLiveness!lastLiveness   l = addUsed l . remDefd l . lastLiveOut l 

!liveness.addUsed.sig!addUsed :: UserOfLocalVars    a => a -> Live -> Live
remDefd :: DefinerOfLocalVars a => a -> Live -> Live 
addUsed a live = foldVarsUsed extendVarSet  live a
!liveness.remDefd.def!remDefd a live = foldVarsDefd delFromVarSet live a

lastLiveOut :: Last -> (BlockId -> Live) -> Live
!lastLiveOut.1!lastLiveOut l env = last l 
!live.lastBranch!  where last (LastBranch id)        = env id 
!live.condBranch!        last (LastCondBranch _ t f) = unionVarSets (env t) (env f)
!live.lastSwitch!        last (LastSwitch _ tbl)     = unionManyVarSets $ map env (catMaybes tbl)
!live.lastCall!        last (LastCall { })         = emptyVarSet
\end{code}%
\B
& Transfer \mbox{functions}\\
\hline

\T\begin{code}
cmmLiveness :: Graph Middle Last -> BlockEnv Live
cmmLiveness g = zdfFpFacts soln
!live.zdfSolveBwd!   where soln = zdfSolveBwd "liveness" liveLattice liveTransfers emptyVarSet g
\end{code}%
\B
& Liveness \mbox{analysis}\\
\end{codetable}
\caption{Dataflow analysis pass to compute liveness}
\figlabel{liveness-all}
\figlabel{liveness}
\figlabel{live-lattice}
\figlabel{live-transfers}
\figlabel{live-running}
\end{figure*}
%
%
%%%%%%%%%%%%%%%%%%%%%%%%%%%%%%%%%%%%%%%%%%%%%%%%%%%%%%%%%%%%%%%%%%



To~understand the available-variables analysis, you must know that for
each variable~$x$, 
there is a stack slot~\slotof x that is used to save the value of~$x$.
%
If the variable and the stack slot hold the same value,
that is if $x = \slotof x$,
then it is \emph{safe} to insert a reload.


%\ifpagetuning\enlargethispage{0.5\baselineskip}\fi

To sink a reload of a variable $x$, we insert redundant reloads immediately
before uses of~$x$.
%We use an analysis that identifies not only when it is safe to insert
%such a reload, but when the inserted reload will make an earlier reload redundant.
%Specifically, 
It is \emph{profitable} to insert a reload before a use of~$x$ only if, 
on every path to the use, the most recent definition of $x$~is a reload from
$\slotof x$.
\delendum{I'm confused.  Surely that definition of profitable is also what we
mean by safe?  NR: No---safety could be establish by an arbitrary
assignment to~$x$ followed by a store to $\slotof x$.}
Safety an profitability are incomparable;
the dataflow fact computed by our analysis is the set of variables for
which it is \emph{both}
safe \emph{and} profitable to insert a reload.
The lattice-join operation is set intersection,\footnote
{Because the assertion of interest is an ``all-paths'' property.}
and the bottom element
is the universal set containing all variables.
It~is optimistic to assume that any variable can be safely reloaded from the stack,
but this optimistic assumption is what enables us to improve the
code---and if the assumption is not justified, the transfer functions
will correct it before
the analysis reaches a fixed point.
\simon{I'm guessing that you intend this sentence to address my long bleat:
``I'm puzzled about why you are treating this example so differently
to constant-prop in Section 3.1.  It looks almost identical to me.  We could
keep a fact for every variable: $x=\bot$ means nothing is known; $x=s_x$ means
x's stack slot is up to date; $x=\top$ means x's stack slot is out of date.
Then keep a finite map as we do for constant prop.  If there is a difference
that drives the rep you have here, let's say so. If the difference is
purely accidental, we should eliminate it.  (Or maybe we don't have time, in
which case we should remark that there is no diff.)''.
For the paper, let's leave it as it is, but this note is to remind us
that I am still unhappy on this point.  I think the difficulty is that we
want to represent \emph{both} the universal set \emph{and} the empty set.
But I'm unclear about the trade-offs in approximation that arise 
from this either-or representation.
%\par
NR \&\ JD: \figref{avail} has only 24~lines of code.
Perhaps you could write an alternative implementation?
}

%\ifpagetuning\enlargethispage{\baselineskip}\fi
   % restoring this saves about 5 lines on page 9


Because universal sets can be awkward to manipulate, we represent a
set either as
$\mathtt{UniverseMinus}\;s$, which stands for all variables except
those in the set~$s$,
or $\mathtt{AvailVars}\;s$, which stands for the variables in the set~$s$
(\figref{avail-lattice}, \linepairref{UniverseMinus}{AvailVars}).
%
%
The bottom element is @UniverseMinus emptyVarSet@.
To manipulate sets of variables, we provide the functions declared in
\linerangeref{extendAvail}{smallerAvail} of \figref{avail}.



%%  Among many other uses, an available-expressions analysis can be~used
%%  in code-motion optimizations.
%%  For example, when making a function call, we insert
%%  spills and reloads to save and restore the values of local variables
%%  around the call site.
%%  It is easy to insert the reloads at the function-call return site,
%%  but to avoid register pressure, it would be better to leave the variable
%%  where it was spilled on the stack.
%%  Rather than complicating the code that inserts the spills and
%%  reloads around call sites,
%%  we write an analysis to insert a redundant reload immediately
%%  before a reloaded variable is used.
%%  Then, we rely on a dead-code elimination to~remove
%%  the early reload.


%%In~the lattice of available reloads (\figref{avail-lattice}),
%%the~join is computed by taking the intersection of two sets
%%of available reloads,
%%The join operation (@add@) returns a \emph{transaction} @aTx@
%%if the join has not yet reached a fixed point;
%%otherwise, it returns @noTx@ to indicate that the dataflow fact
%%did not change.
%%The bottom element of the lattice is the universe of all variables,
%%reflecting the initial assumption that all variables
%%have available reloads.
%%\john{Shouldn't it be @AvailRels@ instead of @AvailVars@?}




% \ifpagetuning\enlargethispage{0.5\baselineskip}\fi

The most interesting part of the pass is the @middleAvail@ transfer
function in \figref{avail-transfers}.\finalremark
{Let us revise the paper to pretend that global variables
don't exist.}
This transfer function is specialized to the type of @Middle@ node
used in~GHC.
In~GHC, a~middle node is either @MidAssign@, which assigns to a local
or global variable, or @MidStore@, which assigns to memory.
%\par
%\ifpagetuning{\enlargethispage{\baselineskip}}\fi
%\kern-8pt
\begin{itemize}
\item
Lines \ref{line:reload1} and \ref{line:reload2}
identify an assignment that reloads local
variable~@x@ from its stack slot.\finalremark{I propose the compiler be
modified to use @isStackSlotOf@ as I've written. JD~approves.}
After such an assignment, $x = \slotof x$,
and the last definition of $x$ is a reload,
so @x@~is added to the set of available variables.
% No other variable is affected.
\item
On \linepairref{assign.avail.1}{assign.avail.2},
an assignment to a local variable means that the
variable is no longer necessarily equal to the value in its stack
slot, so if @lhs@ is a local variable, it is removed from the set of
available variables.
Function @foldVarsDefd@ is a function which, in this
instance, calls @delFromAvail@ if @lhs@ is a local variable and does
nothing if @lhs@ is a global variable.

%\ifpagetuning\smallskip\fi  
  % more even spacing because \subsection 4.2 forces page break

%% \remark{To Simon and John: I've removed ``overloaded but am a bit nervous
%% because @foldVarsDef@ shows up as overloaded later in the paper, and I
%% had thought to signpost it here.  Your thoughts?}
\item 
There are three cases for @MidStore@ nodes.
\Linepairref{store.avail.spill.1}{store.avail.spill.2}
match a node that spills a variable~$\mathtt{x}$ to the stack.
After such a node, $\mathtt{x} = \slotOf {\mathtt{x}}$,
but the node is not a reload instruction,
so @x@~is not added to the set of available variables.
%
\Linepairref{store.avail.otherslot.1}{store.avail.otherslot.2}
match a node that writes any \emph{other} value to a stack slot,
after which the variable associated with that slot is no longer available.
%
\Lineref{store.avail.other} matches a store to a location that is not
a stack slot, which leaves the set of available variables unchanged.
%%
%%
%%
%%
% \footnote
% {Although @MidStore@ may overwrite a stack slot \slotof x, GHC
% carefully arranges that all stores to \slotof x have the form
% $\slotof{\mathtt{x}}= @x@$.
% These stores could be used to extend the set of available variables,
% but it is not useful to do so.}
% \simon{Why not? asks the reader.  Perhaps
% because such saves immediately precede calls?}
% \remark{How do we know that @MidStore@ doesn't
% destroy a stack slot??  I've put in a footnote but it will probably be
% simpler to fix the code.}
% \simon{Ah, this harks backe to the start of this sub-section, where
% we say ``to understand the analysis, you must know that...''.  Another
% thing you must know is that $s_x$ is used exclusively for $x$.  That's
% all, I think.}
\end{itemize}
%% \john{I've expunged MidComment from our example} % well done ----NR


The transfer function for a last node checks to see if the node is a
function call (\lineref{avail.LastCall}); if so, the set of
available variables at the call's continuation is empty.
Other last nodes do not change values of variables or stack slots, 
so the set of available variables remains unchanged.
%
A~first node has no effect on program state, so its transfer function
is @flip const@ (\lineref{avail.first}).

%%Using the @agen@ and @akill@ functions, we define the transfer functions
%%for the available-reloads analysis (see~\figref{avail-transfers}):
%%\begin{itemize}
%%\item \emph{First nodes}:
%%  A first node cannot define a variable,
%%  so the set of available reloads is the same before and after a first node.
%%  We return the set unchanged using @flip const@.
%%\item \emph{Middle nodes}:
%%  If a middle node reloads a value from a register's slot on the stack (@RegSlot@),
%%  then we add the register the register to the set of available reloads.
%%  For any other assignment to a register, we remove the register from the
%%  set of available reloads.
%%  Similarly, because a function call can overwrite the value of any local variable,
%%  the set of available reloads is empty after a~function call.
%%  Any other middle node leaves the set of available reloads unchanged.
%%\item \emph{Last nodes}:
%%  If the last node is a function call, the outgoing set of available reloads
%%  is empty for every successor basic block.
%%  Otherwise, the last node cannot modify any variables,
%%  so the set of available reloads remains unchanged.
%%\end{itemize}

Given the lattice and the transfer functions,
we perform the available-variables analysis by calling
the dataflow-engine function @zdfSolveFwd@ (\figref{avail},
\linerangeref{avail.solve.1}{avail.solve.2}). 
\simon{But this is really a lie. We actually call the transformation 
function!  I'm not quite sure how to fix this pedagogical point.
NR: It's true that we don't actually have a use for the results of an
independent available-variables analysis, other than to drive the
transformation in the next section.  But just because we have no use
for the results at present does not  make the example analysis
incorrect or invalid, and I think the pedagogy is sound.
}
%The function @zdfFpFacts@ returns 
%a finite map from basic-block IDs to the set of available variables
%at the beginning of each block.
Except for the implementations of the set operations on
\linerangeref{extendAvail}{smallerAvail}, 
\figref{avail} shows the \emph{entire} analysis.

\subsection{Liveness: a backward analysis} 

\seclabel{liveness}

The assertion computed by 
a backward dataflow analysis applies to a
\emph{continuation} at a program point, not to a state.
The classic example is liveness analysis;
the assertion of interest is that at a particular program point,
the answer produced by the continuation does not depend on
the value of a particular variable~$x$.
If~so, $x$~is said to be \emph{dead} at that point.
If the answer produced by the continuation \emph{might} depend on the
value of~$x$, $x$~is \emph{live}.\footnote
{Liveness cannot be decided accurately; it reduces to the halting problem.
As usual, we approximate liveness by reachability.}

In a modern compiler, liveness analysis supports many program
transformations,
including
dead-assignment elimination,
which removes assigments to dead variables, 
and register allocation, which
ensures that if two variables are 
live at the same time, they are not assigned to the same register. 

The dataflow fact we use to represent liveness assertions is the set of
live variables (\figref{live-lattice}, \lineref{Live}).
The bottom element of the lattice is the empty set, and the join
operation is set union (\figref{live-lattice},
\linerangeref{liveLattice}{liveLattice.end}); 
a~variable is deemed live after a node if it is live on \emph{any} edge leaving that
node.

The transfer functions for liveness rely on two auxiliary functions
@addUsed@ and @remDefd@ (\figref{liveness}, 
\linerangeref{liveness.addUsed.sig}{liveness.remDefd.def}).
A~transfer function is given a set of variables live on the edges
going out of the node.
It~removes from that set any variable
defined by the node, then adds any variable used by the
node.
For~example, if the node is
\begin{verbatim}
  i = n - 1;
\end{verbatim}
then @i@ is not live just before the node, since if we start the
program just before the assignment to~@i@, the answer cannot 
depend on the value of~@i@, which is about to be overwritten.
But the answer \emph{might} depend on the value of~@n@, so
@n@~is considered live before the assignment.
If a variable appears on both sides of an
assignment, as in \ifpagetuning{\looseness=-1 \par}\fi
\begin{verbatim}
  i = i + 1;
\end{verbatim}
then the answer might depend on~@i@, so @i@~is considered live
before the assignment.
The transfer functions
therefore
remove defined variables \emph{before} adding used variables
(\figref{liveness}, \linepairref{middleLiveness}{lastLiveness}). 

For a last node, function @lastLiveOut@ consults the solution in
progress (parameter~@env@ on \lineref{lastLiveOut.1}) to find out what
variables are live at the \emph{successors} of a 
last node. 
For an unconditional branch, we look up the live set at the label
branched to (\lineref{live.lastBranch});
for a conditional branch, we look at both true and false edges
(\lineref{live.condBranch}), 
 and
for a switch, we consider every possible target of the
branch (\lineref{live.lastSwitch}).
The~remaining case (\lineref{live.lastCall}) is a call, 
and since a call destroys the values of all local variables, no
local variables are live at its continuation.

Given the lattice and the transfer functions,
we perform liveness analysis by calling
the dataflow-engine function @zdfSolveBwd@ (\figref{liveness},
\lineref{live.zdfSolveBwd}). 
%The function @zdfFpFacts@ returns 
%a finite map from basic-block IDs to the set of live variables
%at the beginning of each block.
\figref{liveness} shows the \emph{entire} analysis.
(Overloaded functions
@foldVarsUsed@ and @foldVarsDefd@
are used throughout the back end and
are not considered part of liveness analysis.)
\simon{I still want to see their type signatures in a figure somewhere!!
NR: Agreed in principle, but pray suggest where you mean by ``somewhere.''
SLPK: Plus any other non-obvious functions we use but do not define.
NR: Pray enumerate the functions that offend you.
}
%
% types of fold functions agreed to in principle, but in practice
% there is no good place to put them.



\ifgenkill
\subsection{Writing transfer functions using {\mdseries\texttt{gen}} and
{\mdseries\texttt{kill}}}

%\remark{I've tried to rewrite this section without the abstract guff;
%Simon, can you let us know what you think?}


\seclabel{gen-kill}

If you pick up a compiler textbook, you might see dataflow
analysis explained in terms of functions called \texttt{gen} and
\texttt{kill}, 
which say what dataflow facts are ``generated'' and
``killed'' by each node.
Our design is compatible with explanations based on \texttt{gen} and
\texttt{kill};
first you define functions \texttt{gen} and \texttt{kill}, then you
use them to write the transfer functions.
%%  Using our design, you can stillIt's~possible to write transfer functions i thi
%%  In~a forward analysis, for example an assignment might establish an
%%  assertion (\texttt{gen}), or it might change state in such a way that an
%%  assertion no longer holds (@kill@), or both.
%%  If~you want to transliterate specifications that use @gen@ and @kill@,
%%  it's easy.
For example, the functions @addUsed@ and @remDefd@ in \figref{liveness}'s
liveness analysis correspond directly to the traditional @gen@ and
@kill@ functions.
In the available-variables analysis of \figref{avail}, @gen@ and
@kill@ can be defined as follows:
\begin{smallcode}
gen  :: UserOfLocalVars    a => a -> AvailVars -> AvailVars
kill :: DefinerOfLocalVars a => a -> AvailVars -> AvailVars
gen  a avail = foldVarsUsed extendAvail  avail a
kill a avail = foldVarsDefd delFromAvail avail a
\end{smallcode}
Using @gen@ and @kill@ entails no loss of efficiency;
for example, if the right-hand side of \lineref{reload2} of \figref{avail}
were written
@gen x avail@,
then GHC~would 
identify and % \ifpagetuning\else identify and \fi
inline the type-specific instance of @foldVarsUsed@, which for 
\ifpagetuning\else the case of \fi
a local variable is the identity function, so @gen x avail@
would reduce to @extendAvail avail x@.
\fi

\section{Using dataflow facts to rewrite graphs\ifpagetuning\else, with examples\fi}

\seclabel{rewrites}

\seclabel{example-rewrites}

\begin{figure}
\begin{code}
type Rewrite mid last = Maybe (Graph mid last)
data ForwardRewrites mid last a = ForwardRewrites
 {fr_first  :: BlockId -> a -> Rewrite mid last,
  fr_middle :: mid     -> a -> Rewrite mid last,
  fr_last   :: last    -> a -> Rewrite mid last} 

data BackwardRewrites mid last a = BackwardRewrites
 {br_first  :: BlockId  -> a  -> Rewrite mid last,
  br_middle :: mid      -> a  -> Rewrite mid last,
  br_last   :: last ->
               (BlockId -> a) -> Rewrite mid last} 
\end{code}
\caption{Types of forward and backward rewrite functions.}
\figlabel{rewrites}
\end{figure}


\begin{figure*}
\begin{codetable}
\T\begin{code}
availRewrites :: ForwardRewrites Middle Last AvailVars
availRewrites = ForwardRewrites first middle last
!avail.rewrites.first!  where first _ _ = Nothing
        middle m avail = maybe_reload_before avail m (mkMiddle m)
!avail.rewrites.last!        last   l avail = maybe_reload_before avail l (mkLast l)
!maybe.reload.before.1!        maybe_reload_before avail node tail =
            let used = filterVarsUsed (elemAvail avail) node
            in  if isEmptyVarSet used then Nothing
!maybe.reload.before.2!                else Just $ reloadTail used tail
        reloadTail vars t = foldl rel t $ varSetToList vars
!mkMiddle!          where rel t r = mkMiddle (reload r) <*> t
\end{code}
\B
& Rewrite \mbox{functions}\\
\hline

\T\begin{code}
insertLateReloads :: (Graph Middle Last) -> DFMonad (Graph Middle Last)
insertLateReloads g = liftM zdfFpContents result
!insertLateReloads.1!  where result = zdfRewriteFwd RewriteShallow "insert late reloads"
                               availVarsLattice avail_vars_transfer
!insertLateReloads.2!                               availRewrites (fact_bot availVarsLattice) g
\end{code}
& Late-reload insertion\\
\end{codetable}
\caption{Late-reload insertion, which relies on the analysis of \figref{avail}}
\figlabel{avail-rewrites}
\end{figure*}


We compute dataflow facts in order to enable code-improving
transformations. \simon{Our library provides a function
that performs (interleaved) analysis and transformation on the
graph, which we describe in this section. New paragraph.}
The heart of a transformation is a triple of
\emph{rewrite functions};
type declarations for rewrite functions are
shown in \figref{rewrites}. 
%
A~rewrite function is given a dataflow fact and a node~$n$.
It~may choose to replace node~$n$ with a \emph{replacement graph}~$g$,
in which case it 
returns $@Just@\;g$, or it may do nothing, in which case it returns @Nothing@.
If it returns $@Just@\;g$, it must guarantee that given
the assertions represented by incoming dataflow facts,
graph~$g$ is observationally equivalent to node~$n$.
\simon{Doesn't the rewrite have to be monotonic wrt the analysis?
NR~\&~JD: we're pretty sure not.  For scrutiny, we've reinserted your long comment,
but in your example, once $x$ is deemed live it stays live, because
facts at labels increase monotonically.
(Also, the only interesting partial order we're aware of is Tony Hoare's
``implements'' relation.  That might actually be permissible, but we
have used the stronger requirement of observational equivalence.
In~code generation these will be equivalent because of
determinism. ---NR)
}
%%  
%%  
%%  If~a rewrite function returns $@Just@\;g$, the incoming dataflow fact
%%  must guarantee that replacing the node with~$g$ does not change the
%%  observable behavior of the program.
%%  
%%%%    \simon{The rewrite functions must presumably satisfy
%%%%    some monotonicity property.  Something like: given a more informative
%%%%    fact, the rewrite function will rewrite a node to a more informative graph
%%%%    (in the fact lattice.).
%%%%    \textbf{NR}: actually the only obligation of the rewrite function is
%%%%    to preserve observable behavior.  There's no requirement that it be
%%%%    monotonic or indeed that it do anything useful.  It just has to
%%%%    preserve semantics (and be a pure function of course).
%%%%    \textbf{SLPJ} In that case I think I could cook up a program that
%%%%    would never reach a fixpoint. Imagine a liveness analysis with a loop;
%%%%    x is initially unused anywhere.
%%%%    At some assignment node inside the loop, the rewriter behaves as follows: 
%%%%    if (and only if) x is dead downstream, 
%%%%    make it alive by rewriting the assignment to mention x.
%%%%    Now in each successive iteration x will go live/dead/live/dead etc.  I
%%%%    maintain my claim that rewrite functions must satisfy some
%%%%    monotonicity property.
%%%%    \textbf{JD}: in the example you cite, monotonicity of facts at labels
%%%%    means x cannot go live/dead/live/dead etc.  The only way we can think
%%%%    of not to terminate is infinite ``deep rewriting.''
%%%%    }

To write a program transformation,
the compiler writer must 
\begin{itemize}
\item
Create a dataflow lattice and transfer functions for the supporting
analysis, as described in \secref{create-analysis}. 
\item
Create rewrite functions for first, middle, and last nodes.
\end{itemize}
The
compiler writer can then use our
library function @zdfRewriteFwd@ (from our dataflow engine) to
transform a control-flow 
graph:
\begin{code}
  zdfRewriteFwd 
    :: RewritingDepth         -- Rewrite recursively?
    -> PassName               -- Name of this pass
    -> DataflowLattice a      -- Lattice
    -> ForwardTransfers m l a -- Transfer functions
    -> ForwardRewrites m l a  -- Rewrite functions
    -> a                      -- Input fact
    -> Graph m l              -- Control-flow graph
    -> DFMonad (FwdFixedPoint m l a (Graph m l))
\end{code}
%%  
%%  class DataflowSolverDirectiontransfers fixedpt =>
%%        DataflowDirection
%%          transfers fixedpt rewrites where
%%    zdfRewriteFwd :: (DebugNodes m l, Outputable a)
%%      => RewritingDepth    -- Recursive rewrites?
%%      -> BlockEnv a        -- Init facts
%%      -> PassName          -- Analysis name
%%      -> DataflowLattice a -- Lattice
%%      -> transfers m l a   -- Transfers
%%      -> rewrites m l a    -- Input fact
%%      -> a                 -- Input fact
%%      -> Graph m l         -- CFG
%%      -> DFMonad (fixedpt m l a (Graph m l))
%%
%%
Function @zdfRewriteFwd@ is like @zdfSolveFwd@ in
\secref{zdfSolveFwd}, but it uses and produces extra
information:\seclabel{engine-truth} 
\begin{itemize}
\item
The @RewritingDepth@ parameter controls recursive rewriting;
if~a graph produced by a rewrite function should not be further rewritten,
rewriting is \emph{shallow};
if~a graph produced by a rewrite function can be rewritten again,
rewriting is \emph{deep}.
Deep rewriting is essential to achieve the results of
\citet{lerner-grove-chambers:2002}, e.g., to remove the induction
variable from the loop in the example in \secref{induction-var-elim}.
When deep rewriting is used, the rewrite functions must
ensure that the graphs they produce are not rewritten indefinitely.
\item
Function @zdfRewriteFwd@ requires rewrite functions as well as transfer
functions.
\item
The type constructor @FwdFixedPoint@ has a fourth
type parameter,\footnote
{We lied to you above.}
which is a value contained in the fixed point.
The~value can be extracted using function @zdfFpContents@, which has
type @FwdFixedPoint m l a b -> b@.
Here the type parameter~@b@ is instantiated to @Graph m l@: the fixed point
contains the rewritten graph.
In~@zdfSolveFwd@, @b@~is instantiated with
the unit type~@()@.
\simon{We can't make this point until later, where we say that solve is
implemented using rewrite.  NR: I don't understand why we can't make
this point now---we are revealing that @FwdFixedPoint@ has an
additional type parameter, and we say what value it takes on in the
earlier example.} 
\end{itemize}


\seclabel{dfengine-spec}

Function
@zdfRewriteFwd@ implements
interleaved analysis and transformation 
\remark{First mention of
interleaving is in \secref{interleave-introduced} on
page~\pageref{interleave-introduced}.}
in two phases \citep{lerner-grove-chambers:2002}:\seclabel{solver-phase}
\begin{itemize}
\item
In the first phase, when a rewrite function proposes to replace a
node, the replacement graph is analyzed recursively, and the results
of that analysis are used as the new dataflow
fact(s) flowing out of the original node.
Then the replacement
graph is \emph{thrown away}; only the facts remain.
If,~during iteration, the original node is analyzed again, perhaps
with a different input fact, the rewrite function may propose
a different replacement or even no replacement at all.
As described in \secref{subgraphs}, every replacement graph is a
\emph{subgraph}, not a complete graph.

The first phase is called the \emph{solver}.
It computes a fixed point of the dataflow analysis
\emph{as if} nodes were replaced, while never actually replacing a node.
%%%%    \simon{The rewrite functions must presumably satisfy
%%%%    some monotonicity property.  Something like: given a more informative
%%%%    fact, the rewrite function will rewrite a node to a more informative graph
%%%%    (in the fact lattice.).
%%%%    \textbf{NR}: actually the only obligation of the rewrite function is
%%%%    to preserve observable behavior.  There's no requirement that it be
%%%%    monotonic or indeed that it do anything useful.  It just has to
%%%%    preserve semantics (and be a pure function of course).
%%%%    \textbf{SLPJ} In that case I think I could cook up a program that
%%%%    would never reach a fixpoint. Imagine a liveness analysis with a loop;
%%%%    x is initially unused anywhere.
%%%%    At some assignment node inside the loop, the rewriter behaves as follows: 
%%%%    if (and only if) x is dead downstream, 
%%%%    make it alive by rewriting the assignment to mention x.
%%%%    Now in each successive iteration x will go live/dead/live/dead etc.  I
%%%%    maintain my claim that rewrite functions must satisfy some
%%%%    monotonicity property.
%%%%    \textbf{JD}: in the example you cite, monotonicity of facts at labels
%%%%    means x cannot go live/dead/live/dead etc.  The only way we can think
%%%%    of not to terminate is infinite ``deep rewriting.''
%%%%    }
\item
Once the solver is complete, the resulting fixed point is sound,
and the facts in the fixed point are used by the second phase, in~which
no dataflow facts change, but
each replacement proposed by a rewrite function is actually
performed.
This phase is called the \emph{rewriter}.
\end{itemize}

%The solver executes a complicated algorithm with complicated control-flow
%that depends on the lattice, transfer functions, and rewrite functions;
Facts computed by the solver depend on graphs produced by rewrite
functions, which in turn depend on facts computed by the solver.
How~do we know this algorithm is sound, or even if it terminates?
A~proof requires a POPL paper
\cite{lerner-grove-chambers:2002}, but we can give some
intuition:
\begin{itemize} 
\item
The algorithm is sound because, given the incoming dataflow facts,
each rewrite must preserve the observable behavior of the program.
Therefore, a sound analysis of the rewritten graph
can only generate dataflow facts that could have been
generated by a more complicated analysis of the original graph.
\item
No matter what the transfer functions and rewrite functions do,
the dataflow engine uses the dataflow lattice's join operation to ensure that
facts at labels never decrease. 
As~long as the lattice permits no fact to increase infinitely many
times, analysis is guaranteed to terminate.
\end{itemize}
Thus to guarantee soundness and termination, client code must supply a
lattice with no infinite ascending chains, sound transfer functions,
and sound rewrite functions.
Moreover, either the rewrite functions must not return replacement
graphs that contain nodes which are rewritten indefinitely,
or the client code must specify shallow rewriting, which tells the
dataflow engine never to rewrite a replacement graph.

Why bother with such a complex algorithm?
\citet{lerner-grove-chambers:2002} write
\begin{quote}
\emph{Previous efforts to exploit [the mutually beneficial
interactions of dataflow analyses] either (1)~iteratively performed
each individual analysis until no further improvements are discovered
or (2)~developed [handwritten] ``super-analyses'' that manually
combine conceptually separate analyses. We have devised a new approach
that allows analyses to be defined independently while still enabling
them to be combined automatically and profitably. Our approach avoids
the loss of precision associated with iterating individual analyses
and the implementation difficulties of manually writing a
super-analysis.}
\end{quote}



%%  \simon{But do we apply rewrites even before the analysis reaches a fixed point?
%%  If so, what property do the rewrites have to satisfy to ensure soundness?
%%  If not, even a single rewrite might destroy the fixed-point property of the
%%  current facts.  Or perhaps we iterate the analysis to a fixpoint, and only \emph{then}
%%  do rewriting? If so, do we need the transfer functions at that stage?
%%  
%%  Also the fixed-point of the analysis relies on upward chains. What if
%%  the rewrite pushed it downward?  Or is it the case that a rewrite must
%%  change a node $n$ into a graph $g$ so 
%%  that $\mathit{fwdtrans}(n) \leq \mathit{fwdtrans}(g)$?
%%  
%%  Also the fixpoint calculation requires multiple passses; do the 
%%  rewrites then apply multiple times?
%%  
%%  I'm deliberately playing the role of the reader here, and not peeking at
%%  the code.  I don't think it's enough to say ``go look at Chambers paper''; 
%%  I suggest we say enough (half a column would do it) to address the obvious
%%  questions and point to Chambers for details.
%%  
%%  
%%  \textbf{NR}: Good questions, but let's have a forward reference to \secref{dfengine}}


Interleaving analysis with transformation makes it
possible to create startlingly simple transformations.
In the rest of this section we present two examples:
\secref{sink-reloads} shows how to insert a reload instruction just
before each use of each spilled variable, and
\secref{dead-code-elim} shows how to eliminate dead assignments.
When these two transformations are run in sequence, the effect is to
sink reloads and produce programs like the example shown in
\secref{spill-reload-example}. 





%%  After defining the lattice, the transfer functions, and the rewrite functions,
%%  the client runs the analysis by invoking the dataflow framework
%%  (see~\figref{framework-fns}).
%%  The function @zdfSolveFrom@ performs an analysis on an input control-flow graph,
%%  using a dataflow lattice and a set of transfer functions.
%%  The additional arguments to the function provide
%%  the name of the analysis,
%%  the initial set of dataflow facts (usually empty),
%%  and the initial fact (usually bottom)
%%  that flows into either the entry or exit of the graph,
%%  depending on whether the transfers define a forward or backward analysis.
%%  The result of the function is the fixed point of the analysis,
%%  which stores the dataflow fact on entry to each basic block.
%%  \john{Maybe we should export a simple version of these functions to clients?
%%    Do they always do the obvious things with the initial facts and in-fact?
%%    Initial facts can reuse results of a previous analysis, but then you lose
%%    interleaving.}
%%  
%%  To combine an analysis and a transformation,
%%  the client calls the @zdfRewriteFrom@ function,
%%  which takes the same arguments as @zdfSolveFrom@,
%%  with the addition of the set of rewrite functions
%%  and a parameter (@RewritingDepth@) that decides whether the result
%%  of a rewrite function should be considered for further rewriting.
%%  The result of the function is not only the fixed point of the
%%  analysis interleaved with the transformation,
%%  but also the transformed control-flow graph.
%%  



\begin{figure*}
\begin{codetable}
\T\begin{code}
deadRewrites = BackwardRewrites nothing middleRemoveDeads nothing
!deadRewrites.1!  where nothing _ _ = Nothing
        middleRemoveDeads :: Middle -> CmmLive -> Maybe (Graph Middle Last)
!elim.dead.1!        middleRemoveDeads (MidAssign (CmmLocal x) _) live
!elim.dead.2!            | not (x `elemVarSet` live) = Just emptyGraph
!deadRewrites.2!        middleRemoveDeads _ _ = Nothing
\end{code}
\B
& Rewrite \mbox{functions}\\
\hline

\T\begin{code}
removeDeadAssignments :: (Graph Middle Last) -> DFMonad (Graph Middle Last)
removeDeadAssignments g = liftM zdfFpContents result
!rewriteBwd.1!     where result = zdfRewriteBwd RewriteDeep "dead-assignment elim"
!rewriteBwd.2!                                  liveLattice liveTransfers rewrites emptyVarSet g
\end{code}
& \mbox{Dead-code} elimination\\
\end{codetable}
\caption{Dead-assignment elimination, which relies on the analysis of
\figref{liveness}} 
\figlabel{dead-elim}
\end{figure*}


\subsection{Sinking reloads: a forward transformation}

\finalremark{Incidentally, I wonder if we should
use record notation when constructing @ForwardRewrites@?}

\seclabel{sink-reloads}

We can use the available-variables analysis of \secref{avail} to
insert reloads
immediately before uses of variables.
The transformation is implemented by the rewrite functions on
\linerangeref{avail.rewrites.first}{avail.rewrites.last} of \figref{avail-rewrites}.
A~first node uses no variables and so is never rewritten.
For middle and last nodes, @maybe_reload_before@ 
(\linerangeref{maybe.reload.before.1}{maybe.reload.before.2})
computes @used@, which is the set
of variables used in the node which are both safe and profitable to
reload. 
If that set is not empty, function
@reloadTail@ replaces @node@ with a new graph in which @node@ is
preceded by a (redundant) reload for each variable in the set~@used@.
%%%%% This rewrite function \emph{must} be used with shallow
%%%%% rewriting. % redundant

Rewrite functions create graphs using functions like these
(\figref{avail-rewrites}, \lineref{mkMiddle}):
\begin{code}
mkLabel  :: BlockId -> Graph m l
mkMiddle :: m       -> Graph m l
mkLast   :: l       -> Graph m l
(<*>)    :: Graph m l -> Graph m l -> Graph m l
\end{code}
%% \simon{Aha!  Here is where we lie by pretending that @Graph@ = @AGraph@.  Ok.}
%% \simon{Does @mkLast@ really have a @LastNode l@ context? How does it use it?}
%%        Yes, it really does, but it's used only for prettyprinting
%%        the graph (for debugging).  So I've expunged it from the
%%        paper. ---NR
The infix @<*>@ function is graph concatenation.

Our transformation is implemented by the call to @zdfRewriteFwd@
on \linerangeref{insertLateReloads.1}{insertLateReloads.2} of \figref{avail-rewrites}.
Rewriting is shallow, so a graph returned by
@maybe_reload_before@ is not itself rewritten.
(If~it \emph{were} rewritten, a nonempty @used@ set would make the
compiler insert an infinite sequence of reloads before @node@.)
Once the reloads are inserted, the original reloads immediately
following the call site are dead, and they can be eliminated by our
next transformation, dead-assignment elimination.

\subsection{Dead-assignment elimination: a backward \rlap{transformation}}


\seclabel{dead-code-elimination}
\seclabel{dead-code-elim}

\seclabel{bwd-rewrite}


\def\liveout{$\mathit{live_{out}}$}

We use the liveness analysis of \secref{liveness} to identify
assignments which modify only variables that are not used.
Such \emph{dead assignments} can be removed without changing the
observable behavior of the program.
The removal is implemented by the rewrite functions on
\linerangeref{deadRewrites.1}{deadRewrites.2} of \figref{dead-elim}. 
First and last nodes are not assignments and so are never
rewritten.
A~middle node is rewritten to the empty graph if and only if it is an
assignment to a dead variable (\linerangeref{elim.dead.1}{elim.dead.2}).
On \linepairref{rewriteBwd.1}{rewriteBwd.2}, we call @zdfRewriteBwd@.
That's the whole thing.\finalremark
{JD: Need to run this version of the code in anger.}
%
\finalremark{In this space we should have some guff about
composing transformations, which should refer to the example on
eliminating the induction variable.
More generally, list some places dead-assignment elim is used and
include \secref{induction-var-elim}.
}




\begin{figure*}
\setcounter{codeline}{0}
\begin{numberedcode}
!FactKont!type FactKont a b = a          -> Fuel -> DFM a b
type LOFsKont a b = LastOuts a -> Fuel -> DFM a b
!FuelKont!type FuelKont a b =               Fuel -> DFM a b

!forward.sol.sig!forward_sol :: forall m l a . LastNode l => (forall a . Fuel -> Maybe a -> Maybe (Fuel, a)) 
            -> RewritingDepth -> PassName -> BlockEnv a -> ForwardTransfers m l a
            -> ForwardRewrites m l a -> a -> Graph m l -> Fuel -> DFM a (Maybe a, Fuel)
!forward.sol.args!forward_sol with_fuel depth name start_facts transfers rewrites in_fact g fuel =
!forward.sol.setAllFacts!     do { setAllFacts start_facts ; solve_Ox g in_fact fuel }
   where 
!solve.first.sig!     solve_first :: BlockId -> FactKont a b -> FuelKont a b
     solve_mid   :: m       -> FactKont a b -> FactKont a b
!solve.last.sig!     solve_last  :: l       -> LOFsKont a b -> FactKont a b

!set.last.sig!     set_last :: LOFsKont a Fuel
!set.last.*!     set_last (LastOutFacts l) fuel = do { mapM_ (uncurry setFact) l; return fuel }

!solve.block!     solve_block :: Block m l -> FuelKont a Fuel
!solve.Ox!     solve_Ox    :: Graph m l -> FactKont a (Maybe a, Fuel)

!solve.mid.1!     solve_mid m k in' fuel =
!solve.mid.case!       case with_fuel fuel $ fr_middle rewrites in' m of
!solve.mid.Nothing!         Nothing -> k (ft_middle_out transfers in' m) fuel
!solve.mid.Just!         Just (fuel, g) ->
!solve.subAnalysis!           do { (a, fuel) <- subAnalysis $
                   case depth of
!solve.OO!                     RewriteDeep    -> solve_OO g (uncurry return) in' fuel
!solve.rewrite.shallow.1!                     RewriteShallow -> do { a <- anal_f_OO g in'; return (a, fuel) }
!solve.mid.*!              ; k a fuel }
\end{numberedcode}
\caption{Excerpts from the forward solver}
\figlabel{solver-excerpts}
\end{figure*}





\section{The dataflow engine}
\seclabel{engine}
\seclabel{dfengine}

\simon{The earlier sections promised that we'd reveal the lies.
Do we?  I see no mention of @LastNode@ for example, which is rather important
for polymorphism.  Indeed, a subsection on that point might be a good way
to substantiate the claims of the last bullet of the conclusion.}

Above, in sections \ref{sec:making-simple}
through~\ref{sec:rewrites},
we use our library to create analyses and transformations.
In this section, we sketch the implementation of the main part of our
library: the dataflow engine.
The dataflow engine is not as simple as the title of this paper might
lead you to expect.
The \emph{interface} (lattice, transfer functions, rewrite functions) is simple;
as shown above, compiler passes written \emph{using} the engine
are simple;
and even the individual parts of the engine are simple---but the
totality is complex.

\remark{Simon asked what new insights are gained.
Compared with our earlier work, using CPS dramatically simplifies the
dataflow engine.}

The dataflow engine is implemented in two layers.
The lower layer comprises the \emph{fuel monad} and the \emph{inner dataflow monad}.
The fuel monad provides a way to suppress
optimization selectively in order to isolate faults, as discussed
below (\secref{vpoiso}). 

The inner dataflow monad has type @DFM a@, where @a@~is the type of a
dataflow fact.
It~is a state monad which keeps track of the values of dataflow facts
as the engine iterates.
Its~operations include % listed as per SLPJ!
\begin{code}
getFact     :: BlockId -> DFM a a
setFact     :: BlockId -> a -> DFM a ()
getAllFacts :: DFM a (BlockEnv a)
setAllFacts :: BlockEnv a -> DFM a ()
subAnalysis :: DFM a b -> DFM a b
runDFM :: DataflowLattice a -> DFM a b -> DFMonad b
\end{code}
The~interesting operation is @subAnalysis@, which
runs an analysis in a fresh copy of the current state, then reverts to
the previous state.
%%  Using pure code and a monad makes this version much easier to get
%%  right than our Objective Caml version
%%  \cite{ramsey-dias:applicative-flow-graph}. 
%%  \john{Presumably this sentence disappears if the italic text is made permanent.}

The fuel monad and inner dataflow monad are private to the
implementation of the library; 
what is exposed in the interface is a combined monad called @DFMonad@,
and a @run@ function that lifts computations of type @DFMonad a@
to~@IO a@. \simon{``combined''??  There is no fuel moand in Fig 8!}
% cannot afford to mention @Fuel@ yet

The upper layer of the dataflow engine comprises four functions:
a forward solver, a forward rewriter,
a backward solver, and a backward rewriter.
These functions are new and are significantly simpler than in our
previous work \cite{ramsey-dias:applicative-flow-graph}.
In~\secreftwo{forward-solver}{forward-rewriter},
we show parts of the forward solver and rewriter. 


%%  Note that the dataflow engine is the only part of the system that is
%%  hard to get right---this is where all the hair is.
%%  Prime benefit of our system is that once this is right, everything is
%%  easy (and indeed is just logic, strongest postcondition, or weakest
%%  precondition). 
%%  

\subsection{Throttling the dataflow engine using ``optimization
  fuel''}

\seclabel{vpoiso}

We have extended Lerner, Grove, and Chambers's optimization-combining algorithm with
Whalley's \citeyearpar{whalley:isolation} algorithm for isolating
faults.
Whalley's algorithm is used to test a faulty optimizer;
it automatically
finds the first rewrite that introduces a fault.
It works by giving the optimizer a finite supply of \emph{optimization
fuel}.
Each time a rewrite function proposes to replace a node, one unit of @Fuel@ is
consumed.
When the optimizer runs out of fuel, further rewrites are suppressed.
Because each rewrite leaves the observable behavior of the
program unchanged, it is safe to suppress rewrites at
any point.
In~normal operation, the optimizer has unlimited fuel, but during
debugging, a fault can be isolated quickly by doing a binary search on
the size of the fuel supply.
To control the fuel supply in a purely functional setting, we use
the fuel monad.
%% , which works with
%% computations of type @Fuel -> (a, Fuel)@.


\subsection{Aside: Representing graphs and subgraphs}

\seclabel{subgraphs}

\delendum{There is a confusion of terminology.  In this section 
we say that a ``graph'' may also contain two special incomplete blocks.
But later we use the terms ``subgraph'' and ``replacement graph'' to mean
one of these more general graphs; and ``full graph'' to mean (I think)
a closed/closed graph.   Let's define these terms, and use them consistently.
I think we can define a ``full graph'' as closed/closed, and a ``subgraph'' as
one of the other three combinations.
}

\remark{I've added a \P\ defining ``replacement graph'' and
``subgraph.''
A full graph is open/closed, but that's a somewhat arbitrary design
choice that I'd prefer not to expose in this paper.}

To explain how graphs are analyzed and rewritten, we must explain how
they are represented.
\secref{graph.intro} oversimplifies.
A~graph \emph{is} a collection of basic blocks, 
and a typical basic block \emph{does} have a first node followed by zero or
more middle nodes followed by a last node.
But a graph may also contain two special, incomplete
blocks:
\begin{itemize}
\item
An \emph{entry sequence} is a sequence of zero or more middle nodes
ending in a last node (i.e., a control transfer).
If~a graph has an entry sequence it is \emph{enterable} or
\emph{open at the entry}.
\item
An \emph{exit sequence} is a first node followed by zero or more
middle nodes, but \emph{not} ending in a last node---control ``falls
off the end.''
If~a graph has an exit sequence it is \emph{exitable} or
\emph{open at the exit}.
\end{itemize}
The representation of a graph is therefore a triple:
an optional entry sequence, a @BlockEnv@ containing basic blocks,
and an optional exit sequence.
As~a special case, a single sequence of middle nodes also forms a
graph open at both ends.

A~graph proposed by a rewrite function as a replacement for a node is
a \emph{replacement graph}.
Every replacement graph is also a \emph{subgraph}, i.e., a graph that
can be analyzed or rewritten independently, but that is connected to
an outer graph by control-flow edges.
A~subgraph has the same representation as a graph, including 
optional entry and exit sequences.

To~describe the presence or absence of entry and exit sequences, we
refer to a graph's \emph{shape} as open/open (enterable and exitable),
open/closed (enterable but not exitable), or closed/open (not
enterable, but exitable).  
In
the dataflow engine, the shape constrains the rewrite
functions:\footnote {Shape is also used when graphs are constructed:
% graphs provide a constant-time concatentation operator @<*>@, and 
for
example, if $g_1$'s shape is $e$/open and $g_2$'s shape is open/$x$,
then the concatenation $g_1 \mathbin{\mbox{@<*>@}} g_2$ is well defined and has shape
$e$/$x$.  (The exit sequence of~$g_1$ is spliced to the entry sequence
of~$g_2$ to form a new, complete basic block.)}
\begin{itemize}
\item
A first node with @BlockId@~$l$ must be rewritten to a closed/open
replacement graph which also begins with @BlockId@~$l$.
\item
A middle node must be rewritten to an open/open graph.
\item
A last node must be rewritten to an open/closed graph, in 
which all control-flow
paths end in explicit control transfers.
\end{itemize}
These conditions are necessary and sufficient to ensure that 
every replacement graph can be spliced in place of the node it is derived from.


\subsection{The forward solver}

\seclabel{forward-solver}

\finalremark{What does the reader gain from here on?}

To avoid duplicating code,
\emph{the dataflow engine implements only composed
analysis and transformation}.
Pure analysis is a special case in which no node is
ever rewritten.

\simon{I keep tripping over a nasty misunderstanding here.
We say that ``the dataflow engine implements only composed
analysis and transformation'', but then we provide (a)
a solver, and (b) a rewriter.  Which apparently contradicts.

I believe that the story is that the solver needs to perform rewriting
to get the right answer; it just doesn't retain the rewritten graph.
Well, that's ok, but it's quite confusing.  Moreover, a simple (but
perhaps less efficient) way to write the solver would be to call the
rewriter, and simply discard the returned graph.  Correct?

NR: \emph{Incorrect}.  The rewriter calls the solver to do most of the work.
%So {\tt forward\_sol} is simply an efficiency hack on {\tt forward\_rew}.
%If that is so, perhaps we should simply present the forward rewriter,
%thereby avoiding the confusion altogether.  Admittedly the code is slightly
%more complicated, but not much.
}


\finalremark{How and where do we say what's new over
\citet{ramsey-dias:applicative-flow-graph}?}
%showed a backward solver and rewriter that kept dataflow facts in
%mutable cells.
%\emph{[New stuff is pure, polymorphic, and shows how to use
%    fuel]}\remark{fix me}


\figref{solver-excerpts} shows excerpts from the forward solver 
@forward_sol@, which is used to implement the solver phase of
@zdfRewriteFwd@ and also to implement
@zdfSolveFwd@.
Because @forward_sol@ is used to solve replacement graphs as well as
full graphs, it
takes a few more parameters than @zdfSolveFwd@
(\figref{solver-excerpts}, \linerangeref{forward.sol.sig}{forward.sol.args}).
\begin{itemize}
\item
In the solver phase described in \secref{solver-phase},
if a node is rewritten, the replacement graph is used to compute new
dataflow facts, and then the replacement graph is thrown away.
But sometimes we want to prevent a node from being rewritten,
either because
we are implementing a pure analysis, or because the optimizer
is out of fuel. 
We~do so using parameter @with_fuel@, which can prevent rewriting by
returning @Nothing@.
\item
Analysis of a subgraph starts with known facts, not
bottom facts; they are passed as % parameter
@start_facts@
% (\lineref{forward.sol.args}) 
and
 set on \lineref{forward.sol.setAllFacts}.
\item
We track the amount of fuel remaining using
parameter @fuel@.
Parameter @fuel@ comes last because the fuel monad works with
functions of type @Fuel -> (a, Fuel)@.
\end{itemize}
%A~fixed point is computed by initializing the facts using
%@setAllFacts@, which is an operation in the dataflow monad.
%
\iffalse
Function @solve@, on \linerangeref{solve.1}{solve.*} of
\figref{solver-excerpts}, 
takes an input fact, a graph, and a fuel supply; it~returns a pair
containing the exit fact and the 
remaining fuel supply.
It~also has a side effect on the state stored in the inner dataflow monad:
it brings the facts associated with labels up to a fixed point.
\fi



\begin{figure*}
%%  forward_rew
%%          :: forall m l a . 
%%             (DebugNodes m l, LastNode l, Outputable a)
%%          => (forall a . Fuel -> Maybe a -> Maybe a)
%%          -> RewritingDepth
%%          -> BlockEnv a
%%          -> PassName
%%          -> ForwardTransfers m l a
%%          -> ForwardRewrites m l a
%%          -> a
%%          -> Graph m l
%%          -> Fuel
%%          -> DFM a (FwdFixedPoint m l a (Graph m l), Fuel)
%%  forward_rew squash depth xstart_facts name transfers rewrites in_factx gx fuelx = fixed_pt_and_fuel
%%    where
%%      fixed_pt_and_fuel =
%%          do { (a, g, fuel) <- rewrite xstart_facts getExitFact in_factx gx fuelx
%%             ; facts <- getAllFacts
%%             ; let fp = ... facts ... g ...
%%             ; return (fp, fuel)
%%             }
%%  
%%  
%%      rewrite_blocks (Block id t : bs) rewritten fuel =
%%        ... rew_tail h (ft_first_out transfers id a) t rewritten fuel ...
%%      rewrite_blocks [] rewritten fuel = return (rewritten, fuel)
\setcounter{codeline}{0}
\begin{numberedcode}
!GraphFactKont!type GraphFactKont  m l a b = Graph m l -> a -> Fuel -> DFM a b
!GraphKont!type GraphKont      m l a b = Graph m l      -> Fuel -> DFM a b

!rew.first!      rew_first :: BlockId -> GraphFactKont m l a b -> GraphKont     m l a b
      rew_mid   :: m       -> GraphFactKont m l a b -> GraphFactKont m l a b
!rew.last!      rew_last  :: l       -> GraphKont     m l a b -> GraphFactKont m l a b

!rew.mid.1!      rew_mid m k head in' fuel =
        case with_fuel fuel $ fr_middle rewrites in' m of
          Nothing -> k (head <*> mkMiddle m) (ft_middle_out transfers in' m) fuel
          Just (fuel, g) ->
!rew.subAnalysis!              do { (g, a, fuel) <- subAnalysis $
                      case depth of
                        RewriteDeep    -> sar_OO g (\g a f -> return (g, a, f)) in' fuel
!rew.anal.f.OO!                        RewriteShallow -> do { a <- anal_f_OO g in'; return (g, a, fuel) }
!rew.mid.*!                 ; k (head <*> g) a fuel }

!sar.OO!      sar_OO :: Graph m l -> GraphFactKont m l a b -> FactKont a b
\end{numberedcode}
\caption{Excerpts from the forward rewriter}
\figlabel{rewriter-excerpts}
\end{figure*}

\vfilbreak[3\baselineskip]

The implementation of @zdfSolveFwd@ using @forward_sol@ looks
like this:
\begin{smallcode}
zdfSolveFwd name lattice transfers in_fact g = 
  runDFM lattice $ ffp () $ 
  fwd_pure_anal name emptyBlockEnv transfers in_fact g
fwd_pure_anal name env transfers in_fact g =
  forward_sol (\_ _ -> Nothing) undefined name env 
              transfers undefined in_fact g undefined >>= 
  return . fst
ffp :: b -> DFM a (Maybe a) -> DFM a (FwdFixedPoint m l a b)
\end{smallcode}
Funtion @fwd_pure_anal@ is the special case of pure analysis; it is
also used to implement function @anal_f_OO@ on
\lineref{solve.rewrite.shallow.1} of \figref{solver-excerpts}.
Function~@ffp@ (not shown) extracts a @FwdFixedPoint@ from the
state stored in the inner dataflow monad.

\simon{I found myself deep in fuel administration when looking 
at the code.  Of all things that ought to be hidden in a monad, surely 
fuel is the primary example?  Yet you have deliberately and carefully 
made it back into an explicitly passed parameter.  Why?

NR:~How is this state different from all other states?
Consider two iterations of @solve\_blocks@.
The first iteration starts with dataflow facts~$\rho$ and fuel supply~$f$.
At the end of the iteration the facts are~$\rho'$ and the new
fuelsupply is $f'$.
The second iteration starts with \emph{facts}~$\rho'$ but with
\emph{fuel}~$f$ (no prime).
That's why I treated fuel as pure.
I~guess we need a paragraph on this.
}

Function @forward_sol@ is composed of many parts, all written in
continuation-passing style.
The types of the
continuations are shown on \linerangeref{FactKont}{FuelKont} of
\figref{solver-excerpts}. 
\begin{itemize}
\item
Type @FactKont a b@ describes a context following a first node or middle
node: in a forward analysis, we expect a fact of type~@a@
to flow out of the node.
The rest of the analysis consumes that
fact and the remaining fuel and produces a computation in the inner
dataflow monad (@DFM a@) whose answer type is type~@b@.
%Type @FactKont a b@ also describes the context following the analysis of
%any replacement graph that is open at the exit.
\item
Type @LOFsKont a b@ describes a context following a last node.
The type is dictated by the type of the transfer function
@ft_last_outs@ in \figref{transfers}:
since as many facts flow out of a last node as there are control-flow
edges leaving that node, the context expects those facts to have type
@LastOuts a@.
\item
Type @FuelKont a b@ describes a context \emph{before} a first node (or a
basic block).
The forward analysis of a basic block does \emph{not} start with a
fact flowing in; it starts with the fact associated with that block's
label in the inner dataflow monad.
Thus a @FuelKont@ expects no fact as argument; it gets its fact by
calling the monadic operation @getFact@.
\end{itemize}
\simon{I like these continuations, but I'm very puzzled about the
fuel.  If @FuelKont@ takes @Fuel@ in, where does it return the depleted @Fuel@?
It must come out eventualy, because it's needed in the rest of the program.
And if it always comes out, then we should say so:
@FuelKont a b = Fuel -> DFM a (Fuel, b)@.
And once you do that, it's plain that @FuelKont@ is just a state monad,
and I can't see why it isn't part of @DFM@ in the first place.}

\Linerangeref{solve.first.sig}{solve.last.sig} of
\figref{solver-excerpts} give the types of
solvers for first, middle, and last nodes.
These types show that we have written dataflow analysis in Strachey's
style:
the analysis of each kind of thing is a function from continuations to
continuations. 

The result types of @solve_first@, @solve_mid@, and @solve_last@
compose nicely, so that for any basic block we can compute a function
of type
@LOFsKont a b -> FuelKont a b@.
But once we have analyzed a block, we no longer want to pass facts
from one function to another---we want to update the facts stored in
the inner dataflow monad.
Stored facts are updated by function @set_last@ on
\linerangeref{set.last.sig}{set.last.*} of 
\figref{solver-excerpts}, which calls @setFact@ on each fact
flowing out of a block, then returns the remaining~@Fuel@.
By applying the composition
of @solve_first@, @solve_mid@, and @solve_last@ to @set_last@, we get
function @solve_block@, whose type is given on \lineref{solve.block}.
Function @solve_block@ is given a block and an amount of
fuel,
it runs for its side effects on the inner dataflow monad,
and
it returns the remaining fuel.

In the rest of the solver,
function @solve_blocks@ (not shown) takes a list of blocks and
produces a continuation of type @FuelKont a Fuel@, which
runs @solve_block@ repeatedly,
updating stored facts until they reach a
fixed point.
Using @solve_blocks@, we have writren solvers for subgraphs which are open/open,
open/closed, or closed/open.

Finally, @solve_Ox@ (type on \lineref{solve.Ox}, implementation not
shown), which provides the external interface, 
solves a graph~@g@ which is open/open or open/closed.
This function is run primarily for its
side effects on the inner
dataflow monad.
It also returns the remaining fuel, and if it is open at the exit, it
returns an exit fact.

Ultimately, \emph{all} solver functions
are composed of applications of @solve_first@,
@solve_mid@, and @solve_last@.
They are very similar, so we show just one,
@solve_mid@, on
\linerangeref{solve.mid.1}{solve.mid.*} of \figref{solver-excerpts}.
On~\lineref{solve.mid.case}, a rewrite function is passed an
input fact~@in'@ and a middle node~@m@.
If~the rewrite function proposes no replacement graph, 
or if no more fuel is available, the application of @with_fuel@
returns @Nothing@, and continuation~@k@ is given the output fact
(computed by @ft_middle_out@ on \lineref{solve.mid.Nothing}) and
the current supply of fuel.

The interesting case occurs on \linerangeref{solve.mid.Just}{solve.mid.*},
when the rewrite function 
proposes a replacement graph~@g@.
The application of @with_fuel@ produces~@g@ together with a reduced
supply of fuel.
\begin{enumerate}
\item
If we are doing \emph{deep} rewriting, then as @g@~is analyzed
it may be rewritten further.
Analysis with rewriting is done
on~\lineref{solve.OO}
by a recursive call to @solve_OO@ (not shown).
(Because @g@~replaces a middle node, it is open at entry and exit.)
The recursive call gets continuation @uncurry return@, and the
resulting @FactKont a (a, Fuel)@ is given
the input fact and fuel supply.
The output fact and remaining
fuel supply are computed in the inner dataflow monad and bound on
\lineref{solve.subAnalysis}. 

The monadic computation is wrapped in @subAnalysis@ (also on
\lineref{solve.subAnalysis}), which rolls back the mutations
made by @solve_OO@ as it iterates to a fixed point.
Applying @subAnalysis@ leaves the state of the inner dataflow monad
unchanged, so that the only effect of the sub-analysis is to produce
 the pair @(a, fuel)@.
\item
If we are doing \emph{shallow} rewriting,  new graph~@g@ must not be
rewritten, but we must still find a fixed point of the transfer
equations.
We compute that fixed point using @anal_f_OO@ (not shown), which  recursively calls
@forward_sol@ using the @with_fuel@ function
@\ _ _ -> Nothing@, 
thereby doing no rewriting and consuming no fuel
(\lineref{solve.rewrite.shallow.1}).
\simon{This business of passing in a different @with\_fuel@ seems terribly
clumsy to me.  The obvious thing would be to add a third constructor @NoRewrite@
to the @RewritingDepth@ type, so that we could call @forward\_sol@ saying
``don't do any rewriting at all''.  Would that even elimiate the higher order
@with\_fuel@ parameter altogether?  What is it used for? 
NR:~If you re-examine the structure of the case expressions, you'll
see that @NoRewrite@ as a third constructor would leave to
inexhaustive pattern matching.  A Boolean would do.
Function @with\_fuel@ is also
used to decrement the fuel supply.}

\end{enumerate}
%%  Whether rewriting is shallow or deep, the new graph~@g@ is analyzed as
%%  a ``sub-analysis'' in the inner dataflow monad
%%  (\lineref{solver.rewrite.1}), which means it has fresh 
%%  state for tracking facts and deciding whether analysis has reached a
%%  fixed point.
%%  State changes made in a sub-analysis cannot be observed by the outer
%%  analysis. 
%%  In~fact, after the sub-analysis all the new state \emph{and the new
%%    graph} are thrown away.
%%  As~\linerangeref{solver.rewrite.1}{solver.tail.*} show, the new
%%  graph~@g@ is used only to compute the new 
%%  dataflow fact~@a@ and the diminished
%%  supply of fuel.
Whether rewriting is shallow or deep, the application on
\lineref{solve.mid.*} uses the continuation~@k@ to solve the rest of
the graph, given the output fact and the remaining fuel supply.



%%\begin{itemize}
%%\item
%%\item
%%The internal @solve@ function is higher-order in the parameter
%%@finish@, which extracts from the inner dataflow monad either the unique
%%exit fact or the set of @LastOuts@, depending on context.
%%\item
%%The function @set_or_save@ calls @setFact@ for @BlockId@s located
%%within graph~@g@ and calls @addLastOutFact@ for @BlockId@s located
%%outside graph~@g@.
%%\end{itemize}



\subsection{The forward rewriter}

\seclabel{forward-rewriter}

The forward rewriter closely resembles the
forward solver, but because the rewriter passes the rewritten graph as
an accumulating parameter, the types of the continuations are
different, as shown on \linepairref{GraphFactKont}{GraphKont} of
\figref{rewriter-excerpts}. 
\Linerangeref{rew.first}{rew.last} give the types of the base-case
rewriting functions; because these functions are called \emph{after}
the solver has set the state of the inner dataflow monad to a fixed
point, they need not propagate facts out of a last node, so there is
no analog of @LOFsKont@.

Function @rew_mid@ on \linerangeref{rew.mid.1}{rew.mid.*} of
\figref{rewriter-excerpts} is very similar to function @solve_mid@ on 
\linerangeref{solve.mid.1}{solve.mid.*} of \figref{solver-excerpts}.
\Lineref{rew.mid.1} shows the additional parameter @head@, which
contains the (rewritten) graph preceding middle node~@m@.
When no rewrite is proposed, the only change to the code is that
continuation~@k@ takes the additional parameter
@head <*> mkMiddle m@,
which is the graph formed by concatenating graph @head@ and node~@m@. 
When a rewrite is proposed, the sub-analysis computes not just an
output fact and a new fuel supply but also a possibly rewritten graph
(\linerangeref{rew.subAnalysis}{rew.anal.f.OO}).
Rewriting proceeds with the new graph @head <*> g@
(\lineref{rew.mid.*}).
\remark{Need to be sure that we say @<*>@ is splicing and cite ML
workshop paper}

The recursive solve-and-rewrite function @sar_OO@ (type on
\lineref{sar.OO}, implementation not shown) 
has no counterpart in the solver.
It first calls the solver to set the dataflow facts to a fixed
point, then calls rewrite functions to rewrite the graph based on
those facts.
For graphs of other shapes,
there are similar functions.





%%  \subsection{The inner dataflow monad}
%%  
%%  The primary purpose of the dataflow monad is to keep track of 
%%  dataflow facts as the engine iterates.
%%  Dataflow facts are found in three places:
%%  \begin{itemize}
%%  \item
%%  There is a dataflow fact associated with every labelled basic block in
%%  the current graph;
%%  the dataflow monad maintains this association in a finite map.
%%  The functions @getFact@ and @setFact@ query and update this map.
%%  \item
%%  The current graph may be a subgraph of a larger graph, in which case a
%%  forward dataflow pass may produce dataflow facts that flow to labelled
%%  blocks that are outside the current graph.
%%  These facts must be retained and propagated even if the current graph
%%  is abandoned; such facts are added with @addLastOutFact@ and recovered
%%  with @bareLastOuts@.
%%  \item
%%  Finally, a foraward dataflow pass over a subgraph may propagate a fact forward by
%%  ``falling off the end;'' such a fact is set with @setExitFact@ and
%%  recovered with @getExitFact@.
%%  \end{itemize}
%%  In addition to keeping track of facts, 
%%  the dataflow monad provides a number of other facilities to manage
%%  changes in state as graphs are rewritten and facts climb the dataflow
%%  lattice:
%%  \begin{itemize}
%%  \item
%%  The monad keeps track of whether any fact has changed.
%%  \item
%%  It provides a @subanalysis@ function which makes it possible to
%%   analyze a subgraph using the current set of facts, then discard any
%%   changes in state that may have resulted from the analysis of the
%%   subgraph.
%%  \item
%%  It provides a supply of fresh @BlockId@s, which are available for use
%%  by rewrite functions.
%%  \item
%%  It tracks the supply of \emph{optimization fuel}.
%%  As~shown below, when fuel runs out, the dataflow engine stops
%%  calling rewrite functions, effectively halting optimization.
%%  Binary search on the size of the fuel supply enables the compiler to
%%  identify unsound rewrites quickly \cite{whalley:isolation}.
%%  \end{itemize}











\section{The next 700 dataflow analyses}

\seclabel{logic}
\seclabel{next-700}

%%  \simon{I know that this section is Dear to the Ramsey Heart, but
%%  I still can't figure out what it is trying to say to the reader.
%%
%%  [For the record: one reason this section is dear to my heart is
%%  that after my talk at the Computing Lab in 2007, Alan Mycroft
%%  encouraged me in the strongest possible terms to get this
%%  explanation published. ---NR]
%%  
%%  Let me try one stab at what I think you might be trying to say.
%%  In this paper we have described a re-usable, polymorphic 
%%  framework (concretely, a library)
%%  for the analysis and transformation of control flow graphs.
%%  This is not a specialised tool; on the contrary we claim that
%%  a huge variety of optimisations can be viewed through the lens of
%%  dataflow optimisation.
%%  
%%  If that's the thrust, then the generalities of ths section would
%%  be stronger if they were supported with a list of dragon-book stuff
%%  that are all special cases. 
%%  
%%  But I may have misunderstood.}
%%  
%%  \remark{Here's what I'm trying to say: Gentle Reader,
%%  Bad people may have told you that dataflow analysis is new,
%%  special, difficult, and intimidating.  But it's not really---it's
%%  stuff you knew about all along.  Why, if you already understand the
%%  classic material on program correctness, you know how to write a new
%%  dataflow analysis---just invent an approximation of logic that has no
%%  infinite ascending chains.
%%  
%%  I could indeed go through the dragon book, but there's no time.}
%%  \simon{Well that is certainly better. Why not say that?  It's much 
%%  shorter too!  Evidence of the claim that data flow analysis is presented
%%  as special difficult intimidating might be good.}


Most of this paper is devoted to the implementation of dataflow
analyses and their associated transformations.
But we also promise ideas.
The~main idea is that dataflow analysis is not unique,
difficult, or intimidating---it's a special case of 
the seminal work of
\citet{floyd:assigning-meanings},
\citet{hoare:axiomatic-basis},
and
\citet{dijkstra:discipline}
on semantics of programming languages and
 formal methods of program construction.
If~you like this work, you are ready to invent a new
dataflow analysis:
just define an approximation of program logic
using a lattice with no infinite ascending chains.
% just approximate program logic using a lattice with no infinite ascending chains.



%%  \section{Logical view of optimization}
%%  
%%  Connection to Hoare logic:
%%   - facts derived by forward analysis as assertions on state
%%   - facts derived by backward analysis as assertions on continuation.
%%  
%%  Examples on board.

To~illuminate the connection, let's compare the formal methods a 
person uses to 
\emph{construct} a new program with the formal methods a compiler uses
to \emph{analyze} a program.
%The comparison is a bit tricky,
%because 
%People and compilers 
%typically work in opposite directions.
Dijkstra says people should start with a specification (a postcondition
we must establish and a precondition we may assume), then construct a
procedure body from back 
to front, computing weakest preconditions until one is implied by the
procedure's precondition. 
%%  
%%  
%%  
%%  
%%  The most highly developed
%%  methodology for constructing a procedure (Dijkstra's) starts with a
%%  postcondition~$Q$ that the procedure must establish and a
%%  precondition~$P$ that the procedure may assume.
%%  Using Dijkstra's method, a programmer 
%%  works backwards, adding a statement, computing the weakest
%%  precondition of that statement with respect to~$Q$,
%%  and continuing with more statements until
%%  eventually the weakest precondition of the procedure body is implied by~$P$.
A~compiler, by contrast, is given a procedure with no specification,
and most analyses reason forward,  starting with the
weakest possible precondition and 
computing strongest postconditions that hold at various points throughout the program.
% using strongest postconditions to
% compute assertions that hold at various points throughout the program.

Computing strongest postconditions or weakest
preconditions is a simple matter of calculation---except for loops.
In~a loop, such as in the example of induction-variable elimination from
\secref{induction-var-elim}, 
%If~a compiler wants to know what @i@~is, it can easily see that on the
%first trip $@i@=0$, on the second trip $@i@=1$, and so on.
it's easy for a compiler to get stuck trying to compute an
infinite disjunction like $\bigvee_{k=0}^{\infty} @i@=k$.
Compilers are no good at solving recursion
equations that require limits of infinite sequences.
%
People don't like infinite sequences either, which is why they invent
\emph{loop invariants}. 
%%  
%%  In the classic situation, using weakest preconditions, a loop
%%  invariant~$I$ is an assertion that is \emph{stronger} than the weakest
%%  precondition, is invariant under the loop, and when conjoined with
%%  the loop's termination condition, implies the weakest precondition of
%%  what follows the loop.
%%  %%   and has these properties:
%%  %%  \begin{itemize}
%%  %%  \item
%%  %%  Together with the loop-termination condition, the invariant~$I$ implies the weakest
%%  %%  precondition of the code following the loop.
%%  %%  \item
%%  %%  The invariant~$I$ implies the weakest precondition of the loop body
%%  %%  with respect to~$I$.
%%  %%  In other words, $I$~is strong enought so that having $I$ hold at the
%%  %%  beginning of the loop body is enough to
%%  %%  imply that $I$~also holds at the end of the loop body.
%%  %%  \end{itemize}
%%  To be useful, a loop invariant must be neither too strong nor too
%%  weak.
%Much like finding a useful induction hypothesis for a proof, 
But finding a loop invariant requires art and intuition;
in~all but the simplest cases, it is beyond
what a compiler can do.


%%  Strongest postconditions also require a loop invariant, but with dual
%%  properties:
%%  \begin{itemize}
%%  \item
%%  The invariant~$I$ is weaker than the true strongest postcondition,
%%  but it is implied by the strongest postcondition of the code before
%%  the loop.
%%  \item
%%  Assuming invariant~$I$ holds on entry to the loop, the strongest
%%  postcondition of the loop body must imply~$I$.
%%  \item
%%  The strongest postcondition of the entire loop is deemed to be the
%%  conjunction of invariant~$I$ and the termination condition.
%%  \end{itemize}
%%  


So if we can't find loop invariants automatically, what's a poor
compiler writer to do?
We~want assertions that justify interesting program
transformations. 
Such assertions can often be established by a dataflow analysis.
So to write a new dataflow analysis, we
should \emph{invent a new logic} which is
\emph{expressive enough to justify interesting transformations}
but \emph{inexpressive enough that no fact can increase more than
finitely many times}.
With the logic in place, the transfer functions simply approximate
  weakest preconditions (for a backward analysis) or strongest
  postconditions (for a forward analysis) as best they can
given the expressive power of the logic.
From~this point of view, we can understand the traditional
``optimization zoo'' as arising from a collection of approximate,
inexpressive logics, each with its own implementation of strongest postcondition
or weakest precondition. 
The next 700 dataflow analyses may well
arise from 700 more variations on 
program logic, each of which can be inexpressive in a different way.

%%  Let's review the examples of \secref{example:xforms} in that light.
%%  \begin{itemize}
%%  \item
%%  Logic $\bigwedge_i x_i=k_i$.
%%  \begin{itemize}
%%  \item
%%  $i = 7 \join \true \equiv \true$ (no loss of information)
%%  \item
%%  $i = 7 \join i= 7 \equiv  i=7$ (no loss of information)
%%  \item
%%  $i = 7 \join i = 8 \implies \true$ (loss of information)
%%  \end{itemize}
%%  Because there are only finitely many variables, we can say only
%%  finitely many things at each program point, and we're guaranteed to
%%  reach a fixed point.
%%  \item
%%  Example: induction-variable elimination.
%%  \begin{itemize}
%%  \item
%%  What is the proposition?
%%  \item
%%  Key point: be able to express the intermediate state where the
%%  invariant is temporarily violated (first add 1 to @i@, then add 4
%%  to~@p@).
%%  \end{itemize}
%%  \end{itemize}


%\clearpage

\section{Conclusion}

Compiler textbooks make dataflow analysis and optimization appear
difficult and complicated.
We make dataflow simple not by using a single magic
ingredient, but by applying ideas that are well understood by functional
programmers. %% and others who reason formally about programs.
These ideas
make it possible to think simple thoughts about classical code improvements.
\simon{It would be good to give cross-refs to plces in the paper that
substantiate these claims.}
\begin{itemize}
\item
We acknowledge only one program-analysis technique: the solution of
recursion equations.
Like our colleagues working in imperative languages, we solve
recursion equations by iterating to a fixed point.
Many equations relate
properties of program states; some relate properties of paths through
programs. 
\item
We consider only two
relations: weakest liberal precondition and strongest 
postcondition, which
 correspond 
%\ifpagetuning\else
respectively
%\fi
to
``backward dataflow problems'' and ``forward dataflow
problems.''\finalremark
{Can we give an example of a property of program states which is
neither, just by way of contrast; ie this we cannot do.}
%%  \item
%%  In a language that admits loops, iterating weakest preconditions or
%%  strongest postconditions typically does not reach a fixed point in
%%  finitely many steps; hence the need for loop invariants
%%  \cite{hoare:axiomatic-basis,dijkstra:discipline,gries:science-programming}.
%%  In \secref{logic-reconciled}, we show that we can guarantee to reach a
%%  fixed point by limiting what we can express in the logic.\remark{needs
%%  a fix}
%%  We~show that many classical analyses can be explained this way;
%%  the great diversity of classical analyses corresponds to a great
%%  diversity of inexpressive logics.
%%  This view leads us to a unifying principle:
%%  \emph{To implement a code-improving transformation, find the least
%%    expressive logic that can justify the transformation, then use that
%%    logic to compute strongest postconditions, which justify the
%%    transformation locally.}
\item
Although our implementation allows graph nodes to be rewritten in any
way that preserves semantics, we concentrate on
only three program-transformation techniques:
substitution of equals for equals, 
insertion of assignments to unobserved variables, 
and removal of assignments to unobserved variables. 
Substitution of equals for equals is often justified by properties of program
states; for example, if variable~$x$
is always~7, we may substitute~7 for~$x$.\finalremark
{We can also justify substitution of \emph{labels} in goto
  statements by reasoning about continuations.  This is
  probably not the place to mention this fact.}
Insertion and removal of assignments are often justified by properties
of paths through programs;
for example, if an assignment's continuation does not use the variable
assigned~to, that assignment may be removed.

%%  Some compiler texts treat the removal of unreachable code as a
%%  code-improving transformation in its own right.
%%  In~our framework, unreachable code becomes unreachable in the
%%  garbage-collection sense, so no special effort is required to remove
%%  it.
\item
Complex program transformations should be composed from simple
transformations. 
For example, both ``code motion'' and ``induction-variable
elimination'' should be implemented in three stages: insert new assignments;
substitute equals for equals; remove unneeded assignments
(\secref{induction-var-elim}). 

\item 
Because each rewrite leaves the semantics
of the program unchanged, 
we can use 
``optimization fuel'' to limit the number of rewrites.
 When we isolate a fault in the optimizer
(\secref{vpoiso}), we therefore have the luxury of debugging a single
 rewrite, not a complex transformation.
\end{itemize}

We also blend proven implementation techniques
in a way that
makes it easy
to implement classical code improvements.
\begin{itemize}
\item
We use the algorithm of \citet{lerner-grove-chambers:2002} to 
compose analyses and transformations.
This~algorithm makes it easy to compose complex transformations
from simple ones.

We have create a new implementation of the algorithm, which uses
continuation-passing style.
Written in this style, each part of algorithm is implemented by many
compositions of three relatively simple functions.
Because the individual functions are simple, and because we can
compare our code with standard denotational semantics for imperative
languages, we have more confidence in this new implementation than in
any previous implementation. 
\item
Our code is pure.
Inspired by Huet's~\citeyearpar{huet:zipper} zipper,
we use an applicative representation of
control-flow graphs
\cite{ramsey-dias:applicative-flow-graph}. 
We~improve on our prior work by making the state of analysis
explicit in the inner dataflow monad;
these two functional data structures make it easy to implement the hard
part of Lerner, Grove, and Chamber's algorithm: discarding replacement
graphs and their associated state at need.
Managing state explicitly, instead of
using ML's built-in state monad, 
makes it especially easy to implement such
operations as independent sub-analysis of a replacement graph. \simon{This 
last sentence is a repeat of the previous one.}
%
% important, but no longer mentioned in this paper:
%
%%  \item
%%  To \emph{construct} programs, we use a different representation of
%%  flow graphs, one which hides the complexity of the zipper and which
%%  provides a constant-time operation for joining flow graphs in
%%  sequence.
%%  It is inspired in part by Hughes's \citeyearpar{hughes:novel-lists}
%%  representation of lists, which supports a constant-time append operation.
\item
We also make 
the interface to our dataflow library polymorphic in the
representations of 
assignments and control-flow operations.
%%  Although our polymorphic representations have been instantiated only
%%  with the low-level intermediate code used by the Glasgow Haskell
%%  Compiler, they are intended eventually to be instantiated with
%%  machine-dependent representations of target-machine instructions, as
%%  part of a larger project of refactoring GHC's back ends.
%
This design seems obvious in retrospect,
but we underestimated the degree to which polymorphism would force us to
separate concerns.
Introducing polymorphism has made the code simpler, easier
to understand, and easier to maintain.\finalremark
{SLPJ: Is it possible to substantiate this claim by [more] examples?}
In particular, it forced us to make explicit \emph{exactly} what the
dataflow engine must know about flow-graph nodes.
(It must be able to enumerate the successors of a last
node, and it must be able to create an unconditional branch to the
label of its choice.)
\end{itemize}
%
% gen and kill are history
%
%%\item
%%Judicious use of Haskell type classes makes is possible to write
%%weakest precondition or strongest postcondition using the ``transfer
%%equations'' that are familiar from compiler textbooks.
%%If you like, you can even write overloaded @gen@ and @kill@ functions.
%%The benefit is that it is easy to compare the actual code with the
%%abstract treatments found in textbooks\ifgenkill \ (\secref{gen-kill})\fi.
Using our tools,
you can create a new code improvement in three steps:
create a lattice representation for the assertions you want to
express;
create transfer functions that approximate weakest preconditions or
strongest postconditions;
and 
create rewrite functions that use your assertions to justify
interesting program transformations.  
You can get quickly to the real 
intellectual work of code improvement: identifying interesting
transformations and the assertions that can justify them.

\makeatother

\providecommand\includeftpref{\relax} %% total bafflement -- workaround
\IfFileExists{nrbib.tex}{\bibliography{cs,ramsey}}{\bibliography{cs,ramsey,simon,jd}}
\bibliographystyle{plainnatx}

\subsubsection*{Relationship to previously published work}

Our ML~workshop paper \citep{ramsey-dias:applicative-flow-graph} describes
an applicative flow graph which is
an ancestor of the one used in this paper. 
That paper also presents an implementation of dataflow analysis,
but the implementation described in \secref{engine} above is entirely
now, and even
the~interface has evolved significantly.
 The analyses of
\secreftwo{create-analysis}{example-analyses}, the transformations of
\secref{rewrites}, and the 
explanation in \secref{next-700} are also entirely new.

If~this submission is accepted, we will find a better way to work
this information into the text.


\clearpage

\appendix

\section{Dataflow-engine functions}


\begin{figure*}
\setcounter{codeline}{0}
\begin{numberedcode}
\end{numberedcode}
\caption{The forward solver}
\end{figure*}

\begin{figure*}
\setcounter{codeline}{0}
\begin{numberedcode}
\end{numberedcode}
\caption{The forward rewriter}
\end{figure*}






\end{document}
