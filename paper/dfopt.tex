\newif\ifhaskellworkshop\haskellworkshoptrue
\newif\ifbanner\bannertrue


\iffalse

Payload: alternatives for functions polymorphic in node type

  (n C O -> a, n C O -> b, n C O -> c)

vs

  (forall e x . n e x -> ShapeTag e x, forall e x . n e x -> result e x)

where 

  data ShapeTag e x where
    First  :: ShapeTag C O
    Middle :: ShapeTag O O
    Last   :: ShapeTag O C

  result C O = a
  result O O = b
  result O C = c

The function returning ShapeTag can be passed in a type-class
dictionary. 




Say something about the proliferation of heavyweight type signatures
required for GADT pattern matches.  When the signature is three times
the size of the function, something is wrong...


I'm going to leave this point out, because in all the client code
we've written, we are (a) matching only on a node, and (b) matching on
all cases.  In this scenario the GADT exhaustiveness checker provides
no additional benefits.  Indeed, GADTs can be an irritant to the
client: in many pattern matches, GADTs make it impossible to default
cases.


I was thinking again about unwrapped nodes, cons-lists, snoc-lists,
and tree fringe.  I think there's an analogy with ordinary lists, and
the analogy is 'concatMap' vs 'fold'.  Just as with lists, the common
case of 'concatMap (\a -> [a])' does a lot of unnecessary wrapping and
unwrapping of elements.  We nevertheless prefer 'concatMap' because it
provides better separation of concerns.  But this analogy suggests
several ideas:

  - Can block processing be written in higher-order fashion?

  - Can it be done in both styles (concatMap *and* fold)?

  - Once we have framed the problem in these terms, can we write
    fold-style cold that is not too hard to understand and maintain?

  - Finally, is there an analog of stream fusion that would work for
    control-flow graphs, enabling us to write some graph combinators
    that be both perspicuous and efficient?

These are good observations for the paper and for future work.


----------------------------------------------------------------


P.S. The three of us should have a nice little Skype chat about
higher-rank types.  A lot of the questions still at issue boil down to
the following observations:

  - It's really convenient for the *implementation* of Hoopl to put
    forall to the left of an arrow.  Polymorphic functions as
    arguments are powerful and keen, and they make it really easy for
    Hoopl to do its job.

  - It's a bid inconvenient for a *client* of Hoopl to be forced to
    supply functions that are polymorphic in shape.  All constructors
    have to be written out by hand; defaulting is not possible.  (Or
    at least John and I haven't figured out how to do it.)

  - Using higher-rank types in Hoopl's interface makes some very
    desirable things downright impossible:

      - One can't write a generic dominator analysis with type

           dom :: NonLocal n => FwdPass n Dominators

      - One can't write a generic debugging wrapper of type

           debug :: (Show n, Show f)
                 => TraceFn -> WhatToTrace -> FwdPass n f -> FwdPass n f

      - One can't write a combinator of type

           product :: FwdPass n f -> FwdPass n f' -> FwdPass n (f, f')

    I submit that these things are all very desirable.

I'm therefore strongly in favor of removing the *requirement* that a
FwdPass include transfer and rewrite functions that are polymorphic in
shape.  It will be a bit tedious to return to triples of functions,
but for those lucky clients who *can* write polymorphic functions and
who wish to, we can provide injection functions.

I should stress that I believe the tedium can be contained within
reasonable bounds---for example, the arfXXX functions that are
internal to Hoopl all remain polymorphic in shape (although not
higher-rank any longer).

\fi


%
% TODO
%
%
% AGraph = graph under construction
% Graph = replacement graph
%
%%%  Hoopl assumes that the replacement graph for a node open on exit
%%%  doesn't contain any additional exits
%%%  
%%%  introduce replacement graph in section 3, after graph.
%%%  it has arbitrarily complicated internal control flow, but only
%%%  one exit (if open on exit)
%%%  
%%%  a rquirement on the client that is not checked statically.


%%  advances over prior work
%%  
%%  (if open/closed) -- detect bad splices at compile time
%%  constant-time access to exit sequence
%%  (indeed exit  node)
%%  
%%  directly splicable in constant amortized time (vs Hughes
%%  technique)
%%  
%%  polymorphic in node type (type classes)
%%  
%%  



\iffalse % submission abstract

We present Hoopl, a Haskell library that makes it easy for compiler
writers to implement program transformations based on dataflow
analyses. The compiler writer must identify (a) logical assertions on
which the transformation will be based; (b) a representation of such
assertions, which should form a lattice of finite height; (c) transfer
functions that approximate weakest preconditions or strongest
postconditions over the assertions; and (d) rewrite functions whose
soundness is justified by the assertions. Hoopl uses the algorithm of
Lerner, Grove, and Chambers (2002), which can compose very simple
analyses and transformations in a way that achieves the same precision
as complex, handwritten "superanalyses." Hoopl will be the workhorse
of a new back end for the Glasgow Haskell Compiler (version 6.12,
forthcoming).

Because the major claim in the paper is that Hoopl makes it easy to
implement program transformations, the paper is filled with examples,
which are written in Haskell.  The paper also sketches the
implementation of Hoopl, including some excerpts from the
implementation. 

\fi


% ====> THIS IS A DERIVED FILE; DO NOT EDIT IT <======
% ===========> SIMON, THIS MEANS YOU! <===============
     % If you are Simon, run un-preprocessed code
     % It not Simon, use version with def-use information

\newif\ifpagetuning \pagetuningtrue  % adjust page breaks

\newif\ifnoauthornotes\noauthornotestrue
\newif\iftimestamp\timestamptrue  % show MD5 stamp of paper

\timestampfalse % it's submission time

\IfFileExists{timestamp.tex}{}{\timestampfalse}

\newif\ifcutting \cuttingtrue % cutting down to submission size


\newif\ifgenkill\genkillfalse  % have a section on gen and kill
\genkillfalse


\newif\ifnotesinmargin \notesinmargintrue 
% \IfFileExists{notesinline.tex}{\notesinmarginfalse}{\relax}

\documentclass[blockstyle,preprint,natbib,nocopyrightspace]{sigplanconf}

\newcommand\ourlib{Hoopl}
   % higher-order optimization library
   % ('Hoople' was taken -- see hoople.org)
\let\hoopl\ourlib

% l2h substitution ourlib Hoopl
% l2h substitution hoopl Hoopl

\newcommand\fs{\ensuremath{\mathit{fs}}} % dataflow facts, possibly plural

\newcommand\vfilbreak[1][\baselineskip]{%
  \vskip 0pt plus #1 \penalty -200 \vskip 0pt plus -#1 }

\usepackage{alltt}
\usepackage{array}
\newcommand\lbr{\char`\{}
\newcommand\rbr{\char`\}}
 
\clubpenalty=10000
\widowpenalty=10000

\usepackage{verbatim} % allows to define \begin{smallcode}
\newenvironment{smallcode}{\par\unskip\small\verbatim}{\endverbatim}

\newcommand\lineref[1]{line~\ref{line:#1}}
\newcommand\linepairref[2]{lines \ref{line:#1}~and~\ref{line:#2}}
\newcommand\linerangeref[2]{\mbox{lines~\ref{line:#1}--\ref{line:#2}}}
\newcommand\Lineref[1]{Line~\ref{line:#1}}
\newcommand\Linepairref[2]{Lines \ref{line:#1}~and~\ref{line:#2}}
\newcommand\Linerangeref[2]{\mbox{Lines~\ref{line:#1}--\ref{line:#2}}}

\makeatletter

\let\c@table=
           \c@figure % one counter for tables and figures, please

\newcommand\setlabel[1]{%
  \setlabel@#1!!\@endsetlabel
}
\def\setlabel@#1!#2!#3\@endsetlabel{%
  \ifx*#1*% line begins with label or is empty
     \ifx*#2*% line is empty
        \verbatim@line{}%
     \else
       \@stripbangs#3\@endsetlabel%
       \label{line:#2}%
     \fi
  \else
     \@stripbangs#1!#2!#3\@endsetlabel%
  \fi
}
\def\@stripbangs#1!!\@endsetlabel{%
  \verbatim@line{#1}%
}


\verbatim@line{hello mama}

\newcommand{\numberedcodebackspace}{0.5\baselineskip}

\newcounter{codeline}
\newenvironment{numberedcode}
  {\endgraf
     \def\verbatim@processline{%
        \noindent
        \expandafter\ifx\expandafter+\the\verbatim@line+  % blank line
               %{\small\textit{\def\rmdefault{cmr}\rmfamily\phantom{00}\phantom{: \,}}}%
            \else
               \refstepcounter{codeline}%
               {\small\textit{\def\rmdefault{cmr}\rmfamily\phantom{00}\llap{\arabic{codeline}}: \,}}%
            \fi
        \expandafter\setlabel\expandafter{\the\verbatim@line}%
        \expandafter\ifx\expandafter+\the\verbatim@line+  % blank line
          \vspace*{-\numberedcodebackspace}\par%
        \else
          \the\verbatim@line\par
        \fi}%
   \verbatim
   }
   {\endverbatim}

\makeatother

\newcommand\arrow{\rightarrow}

\newcommand\join{\sqcup}
\newcommand\slotof[1]{\ensuremath{s_{#1}}}
\newcommand\tempof[1]{\ensuremath{t_{#1}}}
\let\tempOf=\tempof
\let\slotOf=\slotof

\makeatletter
\newcommand{\nrmono}[1]{%
  {\@tempdima = \fontdimen2\font\relax
   \texttt{\spaceskip = 1.1\@tempdima #1}}}
\makeatother

\usepackage{times}  % denser fonts
\renewcommand{\ttdefault}{aett} % \texttt that goes better with times fonts
\usepackage{enumerate}
\usepackage{url}
\usepackage{graphicx}
\usepackage{natbib}  % redundant for Simon
\bibpunct();A{},
\let\cite\citep
\let\citeyearnopar=\citeyear
\let\citeyear=\citeyearpar

\usepackage[ps2pdf,bookmarksopen,breaklinks,pdftitle=dataflow-made-simple]{hyperref}

\newcommand\naive{na\"\i ve}
\newcommand\Naive{Na\"\i ve}

\usepackage{amsfonts}
\newcommand\naturals{\ensuremath{\mathbb{N}}}
\newcommand\true{\ensuremath{\mathbf{true}}}
\newcommand\implies{\supseteq}  % could use \Rightarrow?

\newcommand\PAL{\mbox{C{\texttt{-{}-}}}}
\newcommand\high[1]{\mbox{\fboxsep=1pt \smash{\fbox{\vrule height 6pt
   depth 0pt width 0pt \leavevmode \kern 1pt #1}}}}

\usepackage{tabularx}

%%
%% 2009/05/10: removed 'float' package because it breaks multiple
%% \caption's per {figure} environment.   ---NR
%%
%%  % Put figures in boxes --- WHY??? --NR
%%  \usepackage{float}
%%  \floatstyle{boxed}
%%  \restylefloat{figure}
%%  \restylefloat{table}



% ON LINE THREE, set \noauthornotestrue to suppress notes (or not)

%\newcommand{\qed}{QED}
\ifnotesinmargin
  \long\def\authornote#1{%
      \ifvmode
         \marginpar{\raggedright\hbadness=10000
         \parindent=8pt \parskip=2pt
         \def\baselinestretch{0.8}\tiny
         \itshape\noindent #1\par}%
      \else
          \unskip\raisebox{-3.5pt}{\rlap{$\scriptstyle\diamond$}}%
          \marginpar{\raggedright\hbadness=10000
         \parindent=8pt \parskip=2pt
         \def\baselinestretch{0.8}\tiny
         \itshape\noindent #1\par}%
      \fi}
\else
  % Simon: please set \notesinmarginfalse on the first line
  \long\def\authornote#1{{\em #1\/}}
\fi
\ifnoauthornotes
  \def\authornote#1{\unskip\relax}
\fi

\newcommand{\simon}[1]{\authornote{SLPJ: #1}}
\newcommand{\norman}[1]{\authornote{NR: #1}}
\let\remark\norman
\def\finalremark#1{\relax}
% \let \finalremark \remark % uncomment after submission
\newcommand{\john}[1]{\authornote{JD: #1}}
\newcommand{\todo}[1]{\textbf{To~do:} \emph{#1}}
\newcommand\delendum[1]{\relax\ifvmode\else\unskip\fi\relax}

\newcommand\secref[1]{Section~\ref{sec:#1}}
\newcommand\secreftwo[2]{Sections \ref{sec:#1}~and~\ref{sec:#2}}
\newcommand\seclabel[1]{\label{sec:#1}}

\newcommand\figref[1]{Figure~\ref{fig:#1}}
\newcommand\figreftwo[2]{Figures \ref{fig:#1}~and~\ref{fig:#2}}
\newcommand\figlabel[1]{\label{fig:#1}}

\newcommand\tabref[1]{Table~\ref{tab:#1}}
\newcommand\tablabel[1]{\label{tab:#1}}


\newcommand{\CPS}{\textbf{StkMan}}    % Not sure what to call it.


\usepackage{code}   % At-sign notation

\iftimestamp
\input{timestamp}
\preprintfooter{\mdfivestamp}
\fi

\hyphenation{there-by}

\begin{document}
\title{\ourlib: Dataflow Optimization Made Simple}
%\subtitle{\today}

%\titlebanner{\textsf{\mdseries\itshape Submitted to the 2010 ACM Symposium on Principles
%    of Programming Languages (POPL)}}

\ifnoauthornotes
\makeatletter
\let\HyPsd@Warning=
                \@gobble
\makeatother
\fi

% Jo�o


\authorinfo{Norman Ramsey}{Tufts University}{nr@cs.tufts.edu}
\authorinfo{Jo\~ao Dias}{Tufts University}{dias@cs.tufts.edu}
\authorinfo{Simon Peyton Jones}{Microsoft Research}{simonpj@microsoft.com}


\maketitle
 
\begin{abstract}
We present \ourlib, a Haskell library that makes it easy for compiler writers
to implement program transformations based on dataflow analyses.
The compiler writer must identify (a)~logical assertions
on which the transformation will be based;
(b)~a~representation of such assertions, which
\ifcutting
should form a lattice of finite height;
\else
must have a lattice structure
 such that every assertion can be increased at
most finitely many times;
\fi
(c)~transfer functions that approximate weakest preconditions or
strongest postconditions over the assertions; and
(d)~rewrite functions whose soundness is justified by the assertions.
%%  To~guide compiler writers,
%%  we show how dataflow analyses are related to
%%  seminal work on program 
%%  correctness. \simon{The ``next 700'' section sort of does so, but I'm not 
%%  sure it deserves mention in the abstract.}
\ourlib\ uses the algorithm of 
\citet{lerner-grove-chambers:2002}, which 
% enables compiler writers to
can
compose very simple analyses and transformations in a way that achieves
the same precision as complex, handwritten
``super-analyses.''
\ourlib\ will be the workhorse of a new
back end for the Glasgow Haskell Compiler (version~6.12, forthcoming).

\emph{Reviewers:} code examples are indexed at {\small\url{http://bit.ly/jkr3K}}
%%% Source: http://www.cs.tufts.edu/~nr/drop/popl-index.pdf
\end{abstract}

\makeatactive  

\section{Introduction}

If you write a compiler for an imperative language, you can exploit
many years' work on code improvement
(``optimization'').
The work is typically presented
as a long list of analyses and
transformations, each with a different name.
This presentation makes optimization appear complex and difficult.
Another source of complexity is the need for synergistic combinations
of optimizations; you may have to write one ``super-analysis'' per
combination. 

But optimization doesn't have to be complicated.
Most optimizations
work by applying well-understood techniques for
reasoning about programs:
assertions about states, assertions about continuations, and
substitution of equals for equals. 
What makes optimization different from classic reasoning techniques
is that in dataflow optimization, assertions are approximated,
and all assertions are computed automatically.

This paper presents \ourlib\ (higher-order optimization library), 
a~Haskell library that makes it easy to implement dataflow optimizations.
Our contributions are as follows:
\begin{itemize}
\item
\ourlib\ defines a simple interface for implementing analyses and transformations:
you provide representations for assertions and for functions that
transform assertions, 
and \ourlib\ computes assertions
 by setting up and solving recursion
equations.
Additional functions you provide use computed
assertions to justify program transformations.
Analyses and transformations built on \ourlib\ 
are small, simple, and easy to get right.
\item
% using LGC, simple implementations of complicated super-analyses using interleaving and speculative
% rewriting
Using the sophisticated algorithm of \citet{lerner-grove-chambers:2002},
%which is \emph{not} easy to get right,
\ourlib\ can perform super-analyses by \emph{interleaving}
simple analyses and transformations.
Interleaving is tricky to implement,
but by using 
generalized algebraic data types 
and continuation-passing style,
our new implementation expresses the algorithm with a clarity and a
degree of 
static checking that has not 
previously been achieved.
\item
\ourlib\ helps you write correct optimizations:
it
statically rules out transformations that violate invariants
of the control-flow graph,
and dynamically it can help find the first transformation that introduces a fault
in a test program \cite{whalley:isolation}.
\item
\ourlib's polymorphic, higher-order design makes it reusable
with many languages.
\hoopl\ is designed to help optimize imperative code
with arbitrary control flow,
including low-level intermediate languages and machine languages.
As \citet{benitez-davidson:portable-optimizer} have shown, all the
classic scalar and loop optimizations can be performed over such
codes.
\end{itemize}






% %Many presentations obscure fundamental principles of
% %code improvement.
% %
% But~most optimizations
% work by applying reasoning techniques that have long been understood
% and used 
% to reason about programs:
% assertions about states, assertions about continuations, and
% substitution of equals for equals.  % (\secref{next-700}).
% % What distinguishes dataflow optimization from classic formal reasoning
% % about programs is that in dataflow optimization, all assertions are
% % computed automatically, and they are
% % \emph{approximated}. 
% 
% This paper presents \ourlib\ (higher-order optimization library), 
% a Haskell library the implements dataflow optimizations using
% simple representations of assertions and the functions that transform
% them.
% Its contributions are as follows:
% %The contribution of this paper is to elucidate a large body of work on code
% %improvement; the body of work known as ``dataflow optimization.''
% %This paper makes two contributions:
% \begin{itemize}
% \item
% \ourlib\ is engineered to make it \emph{easy} to implement
% dataflow-based 
% code-improvement techniques, even in a purely functional setting.
% When built on \ourlib, analyses and transformations
% are small, simple, and easy to get right.
% \item 
% Using the sophisticated algorithm of \citet{lerner-grove-chambers:2002},
%  which is \emph{not} easy to get right,
% \ourlib\ \emph{interleaves} analysis and transformation.
% Our new implementation, which uses generalized
% algebraic data 
% types % \cite{xi:guarded-recursive} 
% and continuation-passing style,
% %By~exploiting Strachey's ideas about compositional semantics and
% %the extra type checking possible with GADTs, we implement
% expresses the algorithm with a clarity and a degree of
% static checking that has not 
% previously been achieved.
% \item
% \ourlib's polymorphic, higher-order design makes it easier to reuse
% these techniques than ever before.
% \end{itemize}
% %
% \hoopl\ is designed to help optimize imperative procedures (after
% inlining). 
% \ourlib\ supports local-level codes with arbitrary control flow,
% including intermediate 
% languages and machine 
% languages.
% As \citet{benitez-davidson:portable-optimizer} have shown, all the
% classic scalar and loop optimizations can be performed over such
% codes.



We introduce dataflow optimization by analyzing and transforming
example code (\secref{example:xforms}),
thinking about and justifying classic optimizations using
Hoare logic and substitution of equals for equals.
To~support our claim that \ourlib\ makes dataflow optimization easy, 
we explain how
to create new dataflow analyses and transformations
(\secref{making-simple}), and we show complete implementations of significant
analyses (\secref{example-analyses}) and transformations
(\secref{example-rewrites}) from the Glasgow Haskell Compiler.
We also sketch a new implementation of interleaving (\secref{engine}).


\section{Dataflow analysis {\&} transformation by \rlap{example}}

\seclabel{example:transforms}
\seclabel{example:xforms}

In dataflow optimization, code-improving transformations are justified
by assertions about programs;
such assertions are often computed using
strongest postconditions or weakest liberal preconditions.
Typical transformations are
to insert assignments to unobserved variables,
to substitute equals for equals, 
and
to remove assignments to unobserved variables.
Insertion and removal can be composed to achieve 
``code motion.''
\ourlib\  expresses classic code
improvements by composing simple \ifcutting\else analyses and \fi
transformations. 




\subsection{Simple transformations}

\seclabel{constant-propagation}

Here is a sequence of assignments separated by assertions.
We~compute assertions by starting with the weakest 
\ifcutting\else
possible
\fi
assertion (\texttt{true}) and computing strongest postconditions.
% \footnote
{Variables do not alias.}
\begin{verbatim}
    { true }
  x = 7;
    { x == 7 }
  y = 8: 
    { x == 7 && y == 8 }
  z = x + y;
\end{verbatim}
\delendum{SLPJ asks: Could we add $x=7,y=8,z=15$ as a final assertion?
We can but we should not, because a \naive\ sp function 
would produce the assertion $x=7\land y = 8
\land z = x+y$. To reach the point you desire, some sort of simplifier
would be required, and it is better to let the conclusion emerge
naturally as  the code is rewritten.}
In the assignment to~@z@, the assertion @x == 7@ justifies
\label{haskell.firstuse.z}% automated use
\label{haskell.firstuse.x}% automated use
\label{haskell.firstuse.==}% automated use
substituting~7 for~@x@, leaving @z = 7 + y@.  
\label{haskell.firstuse.y}% automated use
\label{haskell.firstuse.+}% automated use
This transformation is traditionally called ``constant propagation.''
We may also substitute 8~for~@y@.
Finally, because @7 + 8 == 15@, we may again substitute equals for
equals, leaving the final assignment as
\begin{verbatim}
  z = 15;
\end{verbatim}
The final transformation, although it also substitutes equals
for equals, has a different name: ``constant folding.''

\subsection{A complex transformation}


\finalremark{
2.2: we are rather blas\'e about where the invariant comes from.  Implying
perhaps that it comes from a simple Hoopl analysis or something.  That is
not what we  intend to imply --- perhaps we should say that the invariant
and the insertion of the assignments to 'p' and 'lim' are handled some
other way.
}

\finalremark{It's a pity that this transformation occupies nearly the
entire second page,  
and then plays no subsequent role in the paper whatsoever.
One possibility: move it to ``the next 700'' section, as a substantiating example
to the claims made there.
But then we'd need another example here... well the sink/reload example of
Section 4 might be perfect.}

\seclabel{induction-var-elim}


The loop optimization known as ``induction-variable elimination'' 
can be composed from simpler transformations.
We begin by showing a loop that sums red pixels from an array:
\begin{verbatim}
  struct pixel { double r, g, b; };
  double sum_r(struct pixel a[], int n) {
    double x = 0.0;
    int i;
    for (i = 0; i < n; i++)
      x += a[i].r;
    return x;
  }
\end{verbatim}
To explain induction-variable elimination, we show the same
code at the machine level, using our low-level compiler-target
language,~{\PAL}
\cite{peyton-jones-ramsey:single-intermediate}: 
\begin{verbatim}
  sum_r("address" bits32 a, bits32 n) {
       bits64 x; bits32 i;
       x = 0.0;
       i = 0;
   L1: if (i >= n) goto L2;
       x = %fadd(x, bits64[a+i*24]);
       i = i + 1;
       goto L1;
   L2: return x;
  }
\end{verbatim}
Induction-variable elimination
replaces~@i@ with a new variable~@p@, helping us
\label{haskell.firstuse.i}% automated use
\label{haskell.firstuse.p}% automated use
to remove the computation @a+i*24@ from the loop.
\label{haskell.firstuse.a}% automated use
\label{haskell.firstuse.*}% automated use
Variable~@p@ is intended to satisfy the invariant
\begin{verbatim}
   { p == a + i * 24 }
\end{verbatim}
Variable @i@ is also used in the loop-termination test.
To rewrite that test, 
we introduce another new variable \texttt{lim} satisfying
the invariant
\icode|lim == a + n * 24|,
so that @i >= n@ if and only if \icode|p >= lim|.
\label{haskell.firstuse.n}% automated use
\label{haskell.firstuse.>=}% automated use

\ifpagetuning \enlargethispage{0.9\baselineskip} \fi 
  


We implement the code improvement as a sequence of transformations.
After each transformation, the observable behavior of the program is unchanged.
%
Our first transformation declares @p@ and \texttt{lim} and inserts suitable
assignments. 
New code is \high{boxed}\,.
\begin{alltt}
  sum_r("address" bits32 a, bits32 n) \lbr
       bits64 x; bits32 i; \high{bits32 p, lim;}
       x = 0.0;
       i = 0; \high{p = a; lim = a + n * 24;}
   L1: if (i >= n) goto L2;
       x = %fadd(x, bits64[a+i*24]);
       i = i + 1; \high{p = p + 24;}
       goto L1;
   L2: return x;
  \rbr
\end{alltt}

As written, the assignments to @p@~and~\texttt{lim} have no
effect on the program, but they establish the assertions
@p == a + i * 24@ and \icode|(i >= n) == (p >= lim)|.
On~the basis of these assertions, the compiler substitutes equals for
equals, resulting in the new code in boxes below:
\begin{alltt}
  sum_r("address" bits32 a, bits32 n) \lbr
       bits64 x; bits32 i; bits32 p, lim;
       x = 0.0;
       i = 0; p = a; lim = a + n * 24;
   L1: if (\high{p >= lim}) goto L2;
       x = %fadd(x, bits64[\kern1pt{}\high{p}\kern1pt{}]);
       i = i + 1; p = p + 24;
       goto L1;
   L2: return x;
  \rbr
\end{alltt}

Here the compiler switches from reasoning about states to
reasoning about continuations.
In~particular, we reason about whether the value of a variable can be
used by a continuation; this reasoning is called ``liveness analysis.''
\Naive\ analysis would show that although @i@~is not live at
label~\texttt{L2}, it is nevertheless live immediately after 
the assignment
@i = i + 1@ in the loop body,
because the value of~@i@ could be used by the next iteration of the
loop.
But~we use Lerner, Grove, and Chambers's
\citeyearpar{lerner-grove-chambers:2002} algorithm to
\emph{interleave} liveness analysis with 
``dead-assignment elimination.'' \seclabel{interleave-introduced}%
Dead-assignment elimination removes an assignment if the variable
assigned to is not live, that is, if it cannot be used by the
assignment's continuation.
\ifcutting
No
\else
As~explained by Lerner, Grove, and Chambers, no
\fi
sequential
composition of liveness analysis and dead-assignment elimination can
get rid of these assignments to~@i@, but interleaving analysis with
transformation does the trick.\footnote
{You might be tempted to modify the
liveness analysis so that
@i = i + 1@ is not considered a ``use'' of~@i@ if @i@~is itself
dead.
This~modification is tantamount to writing a single ``super-analysis''
that \emph{combines} liveness analysis and dead-code
elimination.
In~this case, writing a super-analysis is easy,
but the approach does not scale:
most super-analyses are more complicated than the examples shown here;
the cost of writing a super-analysis does not scale linearly with
the number of analyses combined;
super-analyses often cannot be composed;
and 
some super-analyses require nonstandard, handwritten traversals of
the control-flow graph.
\citet{lerner-grove-chambers:2002} discuss these issues in detail;
\citet{click-cooper} show both the advantages of and the programming cost
of combining analyses.} 
Interleaving%
\ifcutting
\  (\secref{dfengine}) 
\else, as sketched in 
\secref{dfengine}, \fi
eliminates the boxed assignments to~@i@:
\begin{alltt}
sum_r("address" bits32 a, bits32 n) \lbr
     bits64 x; \high{bits32 i;} bits32 p, lim;
     x = 0.0;
     \high{i = 0;} p = a; lim = a + n * 24;
 L1: if ({p >= lim}) goto L2;
     x = %fadd(x, bits64[{p}]);
     \high{i = i + 1;} p = p + 24;
     goto L1;
 L2: return x;
\rbr
\end{alltt}

After the insertion of assignments to @p@~and~\texttt{lim}, the substitution
of equals for equals, and the removal of newly dead assignments
to~@i@, we have ``eliminated the induction variable:''
\begin{alltt}
sum_r("address" bits32 a, bits32 n) \lbr
     bits64 x; bits32 p, lim;
     x = 0.0;
     p = a; lim = a + n * 24;
 L1: if ({p >= lim}) goto L2;
     x = %fadd(x, bits64[{p}]);
     p = p + 24;
     goto L1;
 L2: return x;
\rbr
\end{alltt}

  
%\ifpagetuning \enlargethispage{0.9\baselineskip} \fi 


\section {Making dataflow simple}

\seclabel{making-simple}

\seclabel{create-analysis}

The goal of dataflow optimization is to compute valid
assertions, then use those assertions to justify code-improving
transformations.
%
% only Don Knuth knows why, but the paragraph break gets us three
% bullets on this page where a single paragraph gets only two!
%
Assertions are represented as
\emph{dataflow facts}.
Dataflow facts relate to
 traditional 
program logic:
\begin{itemize}
\item
A dataflow fact is usually equivalent to an assertion about program state or
about a continuation.
For example, in \secref{constant-propagation}, @x == 7@ is a dataflow
fact that describes the program state. 


%%  There are two kinds of dataflow facts.
%%  The first kind is an assertion about the paths from the procedure
%%  entry to a program point;
%%  these facts are computed by a forward dataflow analysis.
%%  A common special case is an assertion about state at a program point,
%%  such as the assertion @x == 7@ in \secref{constant-propagation}.
%%  % Assertions about program state are usually sufficient to show that
%%  % a transformation preserves semantics,
%%  % but to decide whether a transformation will improve the code,
%%  % we sometimes need an assertion that describes how the program
%%  % state was established.
%%  
%%  The second kind of dataflow fact is an assertion about paths from the program point
%%  to the procedure exit;
%%  these facts are computed by a backward dataflow analysis.
%%  In the parlance of functional programmers, these dataflow facts are assertions on
%%  \emph{continuations}.
%%  
%%  Assertions about state are easy to formalize, but path properties are harder;
%%  we describe path properties informally in our assertions.
\item
A~set of dataflow facts forms a lattice.
To ensure that analysis terminates,
it is enough if
no fact has more than finitely many distinct facts above it.
\item
Each analysis or transformation may use a different lattice of
dataflow facts.
\end{itemize}

An assertion about a continuation is an assertion about paths
\emph{from} a program point 
to the procedure {exit};
such assertions are established by a \emph{backward dataflow analysis}.
An~assertion about paths \emph{to} a program point from the procedure
{entry} is established by a \emph{forward dataflow analysis}.
As~an important special case,
an assertion, such as @x == 7@ above,
 may say simply
that all paths to a point establish a predicate
which describes the program state at that point.


A~program point is represented as an edge in
a \emph{control-flow graph}.
%%\footnote
%%{We discuss only intraprocedural optimization, done after inlining.}
%%
%%interprocedural optimizations are the work of the GHC~inliner
%%\cite{peyton-jones:secrets-inliner}.} 
%% 
%%  Note: trying to eliminate reviewer's impression that our work is
%%  conceived for and limited to GHC.  S&S's citation count must
%%  suffer.  ---NR
%%
Edges connect nodes, each of which represents a label, an assignment, or
a control transfer.

To write a dataflow \emph{analysis}, you must 
\begin{itemize}
\item
Choose a representation~$F$ of dataflow facts and a logical interpretation
thereof.
\item
Implement lattice operations over~$F$ (\secref{lattices}).
\item
Write \emph{transfer functions} that relate dataflow facts before and
after each type of node (\secref{tffuns}).
\delendum{I'd italicize key words from all three bullets, or none. NR:
It's not a question of bullets; the key concepts which are possibly
new to readers are transfer
functions and rewrite functions, which is why they are italicized.}
\end{itemize}

To write a \emph{transformation}
based on an analysis, you
must also
create a \emph{rewrite function}, which is presented with a
flow-graph node and with the dataflow facts on the edges coming
into that node (\secref{rewrite-functions}).
The function either proposes to replace the node with a
fresh subgraph, or it leaves the node alone.
If the function proposes a replacement,
the replacement must preserve
semantics;
preservation may be justified by incoming 
\ifcutting \else dataflow \fi
facts.
For example, in \secref{constant-propagation} the fact @x == 7@ 
justifies replacing @z = x + y@ with @z = 7 + y@.

\begin{table}
\centerline{%
\begin{tabular}{@{}>{\raggedright\arraybackslash}p{1.03in}>{\scshape}c>{\scshape}
      c>{\raggedright\arraybackslash}p{1.2in}@{}}
&\multicolumn1{r}{\llap{\emph{Specified}}\hspace*{-0.3em}}&
\multicolumn1{l}{\hspace*{-0.4em}\rlap{\emph{Implemented}}}&\\
\multicolumn1{c}{\emph{Part of optimizer}}
&\multicolumn1{c}{\emph{by}}&
\multicolumn1{c}{\emph{by}}&
\multicolumn1{c}{\emph{How many}}%
\\[5pt]
Control-flow graphs& Us & Us & One \\
Nodes in a control-flow graph & You & You & Two datatypes per intermediate language \\[3pt]
Dataflow fact~$F$    & You & You & One datatype per logic \\
Lattice operations & Us & You & One set per logic \\[3pt]
Transfer functions & Us & You & One set per analysis \\
Rewrite functions & Us & You & One set per \rlap{transformation} \\[3pt]
Iterative solver functions & Us & Us & Two (forward \&\ backward) \\
Solve-and-rewrite functions & Us & Us & Two (forward \&\ backward) \\
\end{tabular}%
}
\caption{Parts of an optimizer built with \ourlib}
\tablabel{parts}
\end{table}



\tabref{parts} shows how \ourlib\ interacts with your client code.
\ourlib\ defines the types of
control-flow graphs, 
lattice operations,
transfer functions, 
and 
rewrite functions.
{All}~these types are parameterized by the types of
nodes in the control-flow graph, which \emph{you} get to define, so you can use
\ourlib\ with many intermediate languages (\tabref{parts}).
Function types are also parameterized by the type of dataflow
facts, so you can define different analyses, 
using different types of facts,
all operating over one type of graph. 

\ifpagetuning\enlargethispage{0.5\baselineskip}\fi

To run an~optimization, you~pass lattice
operations, transfer functions, and rewrite functions
to one of \ourlib's
\emph{solver functions} or \emph{rewrite functions}---\ourlib's 
\emph{dataflow engine}.
A~solver function 
uses a forward or backward \emph{analysis} to compute 
a dataflow fact for each program point (\secref{zdfSolveFwd}).
A~rewrite function
uses a forward or backward \emph{transformation} to compute facts and to
rewrite a control-flow graph in light of those facts
(\secref{rewrites}).
\ifcutting\else
The engine's implementation is sketched in \secref{dfengine}.
\fi
\delendum{Interface?  Perhaps ``The signatures and expected usage of these
functions is described in xxx, while their implementation is sketched
in yyy''.  NR: I like the idea of parallel structure, but the paper is
not really very parallel here.  Faced with either repeating the
section references or distorting what those sections are about, I've
chosen instead to abandon parallel structure.}



\begin{figure}
\ifcutting
\begin{code}
data ChangeFlag = NoChange | SomeChange
data DataflowLattice a = DataflowLattice
 {fact_bot        :: a,
  fact_add_to     :: a -> a -> (a, ChangeFlag) }
\end{code}
\else
\begin{code}
--data ChangeFlag = NoChange | SomeChange
--data DataflowLattice a = DataflowLattice
-- {fact_bot        :: a,
--  fact_add_to     :: a -> a -> (a, ChangeFlag),
--  fact_name       :: String } -- for debugging
\end{code}
\fi
\caption{Representation of a dataflow lattice} \figlabel{lattice-type} \figlabel{lattice}
\label{haskell.def.fact:unadd:unto}% automated definition
\label{haskell.def.fact:unbot}% automated definition
\label{haskell.def.DataflowLattice}% automated definition
\label{haskell.def.SomeChange}% automated definition
\label{haskell.def.NoChange}% automated definition
\label{haskell.def.ChangeFlag}% automated definition
\label{haskell.firstuse.fact:unname}% automated use
\label{haskell.firstuse.String}% automated use
\label{haskell.firstuse.ChangeFlag}% automated use
\label{haskell.firstuse.NoChange}% automated use
\label{haskell.firstuse.SomeChange}% automated use
\label{haskell.firstuse.DataflowLattice}% automated use
\label{haskell.firstuse.fact:unbot}% automated use
\label{haskell.firstuse.fact:unadd:unto}% automated use
\end{figure}


\subsection{Dataflow lattices}

\seclabel{lattices}

As an example, 
we present a lattice of facts about constant propagation.
At any program point, a standard constant-propagation analysis
computes exactly one of three
facts about a variable~$x$:
\begin{itemize}
\item
The analysis shows that
$x = k$, where $k$~is a compile-time constant of type \texttt{Const}.
\item
The analysis shows that $x$~is \emph{not} a compile-time constant.
We~notate this fact as $x = \top$.
\item
The analysis shows nothing about~$x$, which we notate $x=\bot$.
\end{itemize}
The bottom element of the lattice is~$x=\bot$, and
the join operation~$\join$ approximates disjunction,
the logical operation that combines facts flowing to a single label.
A~disjunction of two inconsistent facts is represented by~$x=\top$%
\ifcutting,
so for example 
$x = 7 \lor x = 8$ is approximated by $x = \top$, 
losing information.%
\else
Here are some examples:
\begin{itemize}
\item
$i = 7 \lor i=\bot \equiv i=7$ (no loss of information)
\item
$i = 7 \lor i= 7 \equiv  i=7$ (no loss of information)
\item
$i = 7 \lor i = 8 \equiv i = \top$ (loss of information)%
\fi
\footnote
{Your client code determines how much information is lost.
For example, in a similar analysis for a functional language,
you might track whether a value\ifcutting\else~$v$\fi\ is 
the result of applying a constructor from any finite set~$\{C_i\}$.
\ifcutting
\else
In this analysis, you needn't limit the representation to a
single constructor or to~$\top$;
you could choose to represent such facts as ``$v$~is
the application of a constructor drawn from the set $\{C_1, C_2,
C_4\}$.''
Because every algebraic data type has finitely many constructors,
there are finitely many sets and therefore finitely many facts in the
lattice, so a dataflow analysis over this lattice would always reach a
fixed point.
\fi
}
%%  The compiler writer gets to choose how much information is lost;
%%  in a different analysis, it might be useful to 
%%  choose a lattice which can say
%%  that the value of a variable is one of,
%%  say, at most three constants.
\ifcutting\else
\end{itemize}
\fi

The lattice used by the analysis is the Cartesian product of the
lattices for all the local variables.
We~represent this lattice as a finite map from a variable
to a value of type \icode|Maybe Const|.
A~variable $x$ is not in the domain of the map iff $x=\bot$;
$x$~maps to @Nothing@ iff $x=\top$; $x$~maps to $@Just@\;k$ iff
\label{haskell.firstuse.Nothing}% automated use
\label{haskell.firstuse.Just}% automated use
$x=k$.

\ifcutting\else
Any one procedure has only finitely many variables;
only finitely many facts are computed at any program point;
and in this lattice any one fact can increase at most twice.
These properties
ensure that the dataflow engine will
reach a fixed point.
\fi


\ourlib's dataflow engine uses
\ifcutting joins
\else the lattice's join operation
\fi in a stylized way.
Joins occur at labels.
If~$f_{\mathit{id}}$ is the fact currently associated with the
label~$\mathit{id}$, 
and if a transfer function propagates a new fact~$f_{\mathit{new}}$
into the label~$\mathit{id}$, 
the dataflow engine replaces $f_{\mathit{id}}$ with
the join  $f_{\mathit{new}} \join f_{\mathit{id}}$.
Furthermore, the dataflow engine wants to know if
  $f_{\mathit{new}} \join f_{\mathit{id}} = f_{\mathit{id}}$,
because if not, the analysis has not reached a fixed point.



When computing a join, 
it is often cheap to learn if the join
is equal to one of the arguments.
We therefore use a nonstandard representation of lattice operations,
\iffalse
(\figref{lattice}).
\else
as shown in \figref{lattice}.
\fi
The~join operation~$\join$ and equality test~$=$ are represented by a
single function called @fact_add_to@.
The term $@fact_add_to@\;f_{\mathit{new}}\;f_{\mathit{id}}$ is equal to
$(f_{\mathit{id}}, @NoChange@)$ 
if $f_{\mathit{new}} \join f_{\mathit{id}} = f_{\mathit{id}}$
and is equal to
\mbox{$(f_{\mathit{new}} \join f_{\mathit{id}}, @SomeChange@)$} otherwise.
The @fact_bot@ value is the bottom element\ifcutting.
\label{haskell.def.fact:unname}% automated definition
\else, 
and @fact_name@  is used for debugging.
\fi

\ifpagetuning\enlargethispage{0.5\baselineskip}\fi



\subsection{Transfer functions} \seclabel{tffuns}

A~transfer function is presented with dataflow facts on edges coming
into a node, and it computes dataflow facts on outgoing edges.
To~understand transfer functions, we must 
understand how \ourlib\ organizes the nodes and edges of a control-flow graph.

\seclabel{graph.intro}

A~control-flow graph is a collection of \emph{basic blocks}, each
labelled with a~@BlockId@.
\label{haskell.def.BlockId}% automated definition
\label{haskell.firstuse.BlockId}% automated use
A~basic block is a sequence beginning with a \emph{first node},
containing zero or more \emph{middle nodes},
and ending in a \emph{last node}.
(An~optimizer also works with \emph{subgraphs}, which,
as~discussed in \secref{subgraphs},  may omit an initial
first node or a final last node.)
A~first node is always a @BlockId@;
a~typical middle node assigns to a register or memory
location;
and
a~typical last node is a conditional, unconditional, or indirect branch.
You choose the types of middle and last nodes to suit your
intermediate representation; if~these types are @m@~and~@l@, the type
\label{haskell.firstuse.m}% automated use
\label{haskell.firstuse.l}% automated use
of a basic block is @Block m l@.
\label{haskell.def.Block}% automated definition
\label{haskell.firstuse.Block}% automated use




First nodes are the only targets of control transfers;
middle nodes never perform control transfers;
and
last nodes always perform control transfers.
So~a~first node has arbitrarily many predecessors and exactly one
successor;
a~middle node has exactly one predecessor and one successor;
and a last node has exactly one predecessor and arbitrarily many
successors. 

These constraints on number of predecessors and successors determine
the signatures of 
transfer functions, 
which are shown in \figref{transfers}.
For each type of node (first, middle, last) and for each kind of
analysis (forward, backward), there is a distinct transfer function.
Functions are grouped by kind of analysis, and each group is
parameterized over a dataflow fact of type~@a@ and over the types
@m@~and~@l@ of middle and last nodes.  

\begin{figure}
\begin{code}
newtype LastOuts a = LastOuts [(BlockId, a)]
data ForwardTransfers m l a = ForwardTransfers
 {ft_first_out  :: BlockId -> a -> a,
  ft_middle_out :: m       -> a -> a,
  ft_last_outs  :: l       -> a -> LastOuts a} 

data BackTransfers m l a = BackTransfers
 {bt_first_in  :: BlockId -> a              -> a,
  bt_middle_in :: m       -> a              -> a,
  bt_last_in   :: l       -> (BlockId -> a) -> a} 
\end{code}
\caption{Transfer functions for forward and backward analyses.}
\figlabel{transfers}
\ifpagetuning\vspace*{-1ex}\fi
%
% elided: 
%    ft_exit_out   ::            a -> a
%
\label{haskell.def.bt:unlast:unin}% automated definition
\label{haskell.def.bt:unmiddle:unin}% automated definition
\label{haskell.def.bt:unfirst:unin}% automated definition
\label{haskell.def.BackTransfers}% automated definition
\label{haskell.def.ft:unlast:unouts}% automated definition
\label{haskell.def.ft:unmiddle:unout}% automated definition
\label{haskell.def.ft:unfirst:unout}% automated definition
\label{haskell.def.l}% automated definition
\label{haskell.def.m}% automated definition
\label{haskell.def.ForwardTransfers}% automated definition
\label{haskell.def.LastOuts}% automated definition
\label{haskell.firstuse.LastOuts}% automated use
\label{haskell.firstuse.ForwardTransfers}% automated use
\label{haskell.firstuse.^}% automated use
\label{haskell.firstuse.ft:unfirst:unout}% automated use
\label{haskell.firstuse.ft:unmiddle:unout}% automated use
\label{haskell.firstuse.ft:unlast:unouts}% automated use
\label{haskell.firstuse.BackTransfers}% automated use
\label{haskell.firstuse.bt:unfirst:unin}% automated use
\label{haskell.firstuse.bt:unmiddle:unin}% automated use
\label{haskell.firstuse.bt:unlast:unin}% automated use
\end{figure}


A fact in a forward analysis typically represents an assertion
about program state,
 and because a label does not change
program state, the transfer function @ft_first_out@ is often
@flip const@---a variation on
\label{haskell.firstuse.flip}% automated use
\label{haskell.firstuse.const}% automated use
the
identity function.\footnote
{Not every fact is about program state,
so not every forward analysis can ignore labels.
For example, dominator analysis and other all-paths analyses often
compute a set of labels through which control may (or must) pass.
}
For a middle node, the transfer function @ft_middle_out@ is given a
node and a precondition and returns an approximation of the strongest
postcondition. 
For a last node, different postconditions may be propagated to
different successors; for example, the true and false successors of a
conditional branch may accumulate information implied by the truth or
falsehood of the condition.
A~collection of (successor, fact) pairs is represented by a value of
type @LastOuts a@ (\figref{transfers}).




In a forward analysis, the dataflow engine starts with the fact at the
beginning of a block and applies transfer functions to the nodes in
that block until eventually the transfer function for the last node
computes the facts that are propagated to the block's successors.
For example, in the block
\begin{verbatim}
  L1: x = 7;
      y = 8;
      z = x + y;
      goto L2;
\end{verbatim}
a forward analysis would propagate the fact 
$@x@ = 7 \land @y@ = 8$, which we will call $f_{\mathit{new}}$,
along the edge to~\texttt{L2}. 
%%  \remark{We've elected not to get to the level of detail where
%%  we show how propagating a fact~$f$ through \mbox{@x = 7;@} results in a new
%%  fact either $(f \setminus @x@) \land @x == 7@$.}
The dataflow engine then \emph{replaces} the current fact
at~\texttt{L2}~($f_{\mathtt{L2}}$) with the lattice join $f_{\mathit{new}}
\join f_{\mathtt{L2}}$. 
The dataflow engine iterates over the blocks repeatedly, creating new
facts~$f$ and joining them with facts $f_{\mathit{id}}$ until
\mbox{$f \join f_{\mathit{id}} = f_{\mathit{id}}$} at every label~$\mathit{id}$.
When the facts at labels stop changing, the dataflow
engine has reached a fixed point.
%\remark{Promissory note: compose this analysis with two
%transformations: constant propagation and constant folding}


\ifpagetuning\enlargethispage{0.5\baselineskip}\fi


\subsection{Running the dataflow engine}

\seclabel{zdfSolveFwd}

\iftrue
Given lattice operations of type @DataflowLattice a@ (\figref{lattice})
together with
transfer functions of type @ForwardTransfers m l a@
(\figref{transfers}),
\else
Given lattice operations of type @DataflowLattice a@
and 
transfer functions of type @ForwardTransfers m l a@
(\figreftwo{lattice}{transfers}),
\fi
you can run the corresponding analysis by calling \ourlib\
function @zdfSolveFwd@, which is a~part of our dataflow engine
\label{haskell.firstuse.zdfSolveFwd}% automated use
(a~backward analysis calls function @zdfSolveBwd@, which has a similar type):
\label{haskell.def.zdfSolveBwd}% automated definition
\label{haskell.firstuse.zdfSolveBwd}% automated use
\begingroup\hfuzz=1pt\relax
\begin{code}
 zdfSolveFwd 
  :: HavingSuccessors l     -- Find successors of l
  => PassName               -- Name of the analysis
  -> DataflowLattice a      -- Lattice
  -> ForwardTransfers m l a -- Transfer functions
  -> a                      -- Input fact
  -> Graph m l              -- Control-flow graph
  -> FwdFixedPoint m l a ()
\end{code}
\label{haskell.def.zdfSolveFwd}% automated definition
\label{haskell.def.PassName}% automated definition
\label{haskell.firstuse.FwdFixedPoint}% automated use
\label{haskell.firstuse.Graph}% automated use
\label{haskell.firstuse.PassName}% automated use
\label{haskell.firstuse.HavingSuccessors}% automated use
The function is polymorphic in the types of middle and last nodes
@m@~and~@l@ and in the type of the dataflow fact~@a@.
Polymorphism allows \ourlib\ to work with any intermediate
language, as long as the type of last node @l@ satisfies the constraint
@HavingSuccessors l@ by providing a function
\label{haskell.def.HavingSuccessors}% automated definition
\ifcutting
@succs@ of type @l -> [BlockId]@,
\label{haskell.def.succs}% automated definition
\label{haskell.firstuse.succs}% automated use
\else
\begin{code}
  succs :: l -> [BlockId]
\end{code}
\fi
which gives the labels of the blocks to which a last node of type~@l@
might transfer control.

\endgroup

After the type constraint, 
the first three arguments to @zdfSolveFwd@ characterize the analysis.
The next argument is the dataflow fact that holds on entry to the
graph;
because a procedure's caller may establish some facts about
parameters or about the stack,
this fact
is not always~$\bot$.
The last argument to @zdfSolveFwd@ is the graph, and the result is a 
fixed point.

\ifpagetuning\enlargethispage{0.5\baselineskip}\fi


The @FwdFixedPoint@ data structure,
\label{haskell.def.FwdFixedPoint}% automated definition
whose final type parameter~@()@ is 
explained in~\secref{engine-truth},
 is a big bag
of information about a solution.
The most significant information is
a finite map from each block label to the dataflow fact that holds at
the label, which is extracted using function @zdfFpFacts@:
\label{haskell.firstuse.zdfFpFacts}% automated use
\label{haskell.def.emptyBlockEnv}% automated definition
\begin{code}
type BlockEnv a = Data.Map BlockId a
zdfFpFacts :: FwdFixedPoint m l a g -> BlockEnv a
\end{code}
\label{haskell.def.BlockEnv}% automated definition
\label{haskell.def.zdfFpFacts}% automated definition
\label{haskell.firstuse.g}% automated use
\label{haskell.firstuse..}% automated use
\label{haskell.firstuse.Data.Map}% automated use
\label{haskell.firstuse.BlockEnv}% automated use





%%  \iffalse
%%  % never tell the whole truth!
%%  \begin{code}
%%  class DataflowSolverDirection
%%          transfers fixedpt where
%%    zdfSolveFrom :: (DebugNodes m l, Outputable a)
%%      => BlockEnv a        -- Init facts
%%      -> PassName          -- Analysis name
%%      -> DataflowLattice a -- Lattice
%%      -> transfers m l a   -- Transfers
%%      -> a                 -- Input fact
%%      -> Graph m l         -- CFG
%%      -> fixedpt m l a ()
%%  \end{code}
%%  \fi



%%  The main contribution of this paper is to present an interface which
%%  enables many powerful program transformations based on dataflow
%%  analysis while keeping the individual dataflow passes as simple as
%%  possible.
%%  We keep the \emph{concepts} simple by relating dataflow facts and
%%  transfer functions to classic work in program correctness, as
%%  discussed in \secref{next-700}.
%%  We keep the \emph{implementations of dataflow passes} simple by pushing
%%  as much work as
%%  possible into the dataflow engine, which is implemented just once.
%%  We have also made the dataflow engine and its interface polymorphic in
%%  the types of 
%%  the nodes that appear in the control-flow graph \secref{polymorphic-framework}.
%%  Parametricity ensures separation of concerns between the dataflow
%%  engine and the individual dataflow passes.





\section {Related work}

While dataflow analysis and optimization are covered
by a vast literature, 
\emph{design} of optimizers, the topic of this paper, is covered
relatively sparsely.
We therefore focus on foundations.

When transfer functions are monotone and lattices are finite in height,
iterative dataflow analysis converges to a fixed point
\cite{kam-ullman:global-iterative-analysis}. 
If~the lattice's join operation distributes over transfer
functions,
this fixed point is equivalent to a join-over-all-paths solution to
the recursive dataflow equations
\cite{kildall:unified-optimization}.\footnote
{Kildall uses meets, not joins.  
Lattice orientation is conventional, and conventions have changed.
We use Dana Scott's
orientation, in which higher elements carry more information.}
\citet{kam-ullman:monotone-flow-analysis} generalize to some
monotone functions.
Each~client of \hoopl\ must guarantee monotonicity,
but for transfer functions that
approximate weakest preconditions or strongest postconditions,
monotonicity falls out naturally.

\ifcutting
\citet{cousot:abstract-interpretation:1977}
\else
\citet{cousot:abstract-interpretation:1977,cousot:systematic-analysis-frameworks}
\fi
introduce abstract interpretation as a technique for developing
lattices for program analysis.
\citet{schmidt:data-flow-analysis-model-checking} shows that
an all-paths dataflow problem can be viewed as model checking an
abstract interpretation.

The soundness of interleaving analysis and transformation,
even when some speculative transformations are not performed on later
iterations, was shown by
\citet{lerner-grove-chambers:2002}.
\ifcutting\else
Muchnick \citeyearpar{muchnick:compiler-implementation} 
presents many examples of both particular analyses and related
algorithms.
\fi


\newcommand\T{\rule{0pt}{0.6ex}}
\newcommand\B{\rule[-0.05ex]{0pt}{0pt}}
\renewcommand\B{\relax\par\unskip\vspace*{0.8ex}%
  \hrule height 0pt depth 0pt \relax}
\renewcommand\T{\relax\par\unskip\vspace*{1.0ex}%
  \hrule height 0pt depth 0pt \relax}
\newcolumntype{C}{>{\begin{minipage}{5.35in}}l<{\end{minipage}}}
%\newcolumntype{L}{>{\T\Large\bfseries\indent}m{1.3in}%
%                  <{\fillindent \parfillskip=0pt \parskip=0pt \endgraf}}       % label
\newcolumntype{L}{>{\T\Large\bfseries}m{1.3in}<{\centering}}      
\newcommand\fillindent{\parindent=0pt \leftskip=0pt plus 1fill}
\newcommand\ltab[1]{\begin{tabular}{@{}c@{}}#1\\\end{tabular}}
\label{haskell.firstuse.c}% automated use
\renewcommand\ltab[1]{{\let\\=\relax#1}}
\newenvironment{codetable}
  {\setcounter{codeline}{0}%
   \let\code=\numberedcode
   \let\endcode=\endnumberedcode
   \begin{tabular}{CL}%
  }
  {\end{tabular}}

\begin{figure*}\hfuzz=1pt
\setcounter{codeline}{0}
\begin{codetable}
\T
\begin{numberedcode}
!AvailVars!data AvailVars = UniverseMinus VarSet | AvailVars VarSet
!extendAvail!extendAvail  :: AvailVars -> LocalVar  -> AvailVars  -- add var to set
delFromAvail :: AvailVars -> LocalVar  -> AvailVars  -- remove var from set
elemAvail    :: AvailVars -> LocalVar  -> Bool       -- set membership
interAvail   :: AvailVars -> AvailVars -> AvailVars  -- set intersection
!smallerAvail!smallerAvail :: AvailVars -> AvailVars -> Bool       -- compare sizes
\end{numberedcode}
\B
& \ltab{Dataflow fact\\ and operations}\\
\hline

\T\hfuzz=45pt
\begin{code}
availVarsLattice :: DataflowLattice AvailVars
availVarsLattice = DataflowLattice empty add
    where empty = UniverseMinus emptyVarSet
          add new old = let join = interAvail new old in
                        (if join `smallerAvail` old then SomeChange else NoChange, join)
\end{code}
\B
& Lattice\\
\hline

%%  \T
%%  \begin{code}
%%  agen  :: UserOfLocalVars    a => a -> AvailVars -> AvailVars
%%  akill :: DefinerOfLocalVars a => a -> AvailVars -> AvailVars
%%  agen  a avail = foldVarsUsed extendAvail  avail a
%%  akill a avail = foldVarsDefd delFromAvail avail a
%%  \end{code}\B
%%  & Gen/Kill \mbox{functions}\\
%%  \hline
%%  
\T\hfuzz=87pt
\begin{code}
availTransfers :: ForwardTransfers CmmMiddle CmmLast AvailVars
!avail.first!availTransfers = ForwardTransfers (flip const) middleAvail lastAvail

middleAvail :: CmmMiddle -> AvailVars -> AvailVars
!reload1!middleAvail (MidAssign (CmmLocal x) (CmmLoad l) avail | l `isStackSlotOf` x = extendAvail avail x
!assign.avail.1!middleAvail (MidAssign lhs _expr) avail = foldVarsDefd delFromAvail avail lhs
!store.avail.spill.1!middleAvail (MidStore l (CmmVar (CmmLocal x))) avail | l `isStackSlotOf` x = avail
!store.avail.otherslot.1!middleAvail (MidStore l _) avail | isStackSlot l = delFromAvail avail (varOfSlot l)
!store.avail.other!middleAvail (MidStore _ _) avail = avail

lastAvail :: CmmLast -> AvailVars -> LastOuts AvailVars
!avail.LastCall!lastAvail (LastCall _ (Just k) _ _) _ = LastOuts [(k, AvailVars emptyVarSet)]
lastAvail l avail = LastOuts $ map (\id -> (id, avail)) $ succs l
\end{code}
\B
& \hfuzz=1.8pt \mbox{\phantom{SPACE}}
\mbox{\phantom{XXXX}Transfer} \mbox{\phantom{XXXX}functions}\\
% \ltab{Transfer \\Functions}\\
\hline

\T
\begin{code}
cmmAvailableVars :: Graph CmmMiddle CmmLast -> BlockEnv AvailVars
cmmAvailableVars g = zdfFpFacts fp
!avail.solve.1!  where fp = zdfSolveFwd "available variables" availVarsLattice 
!avail.solve.2!                 availTransfers (fact_bot availVarsLattice) g
\end{code}
\B
&
\def\baselinestretch{0.8}\hspace{-0.3in}\parbox{1.6in}{\center Available-variables analysis}
%\B
\\
%\hline

\end{codetable}
% \caption{Available-variable analysis}
\caption{Dataflow analysis pass to compute available variables}
\figlabel{avail-all}
\figlabel{avail}
\figlabel{avail-lattice}
\figlabel{avail-gen-kill}
\figlabel{avail-transfers}
\figlabel{avail-running}
\label{haskell.def.fp}% automated definition
\label{haskell.def.cmmAvailableVars}% automated definition
\label{haskell.def.lastAvail}% automated definition
\label{haskell.def.:unexpr}% automated definition
\label{haskell.def.lhs}% automated definition
\label{haskell.def.avail}% automated definition
\label{haskell.def.middleAvail}% automated definition
\label{haskell.def.availTransfers}% automated definition
\label{haskell.def.join}% automated definition
\label{haskell.def.old}% automated definition
\label{haskell.def.new}% automated definition
\label{haskell.def.add}% automated definition
\label{haskell.def.empty}% automated definition
\label{haskell.def.availVarsLattice}% automated definition
\label{haskell.def.smallerAvail}% automated definition
\label{haskell.def.interAvail}% automated definition
\label{haskell.def.elemAvail}% automated definition
\label{haskell.def.delFromAvail}% automated definition
\label{haskell.def.extendAvail}% automated definition
\label{haskell.def.UniverseMinus}% automated definition
\label{haskell.def.AvailVars}% automated definition
\label{haskell.firstuse.cmmAvailableVars}% automated use
\label{haskell.firstuse.fp}% automated use
\label{haskell.firstuse.availTransfers}% automated use
\label{haskell.firstuse.CmmMiddle}% automated use
\label{haskell.firstuse.CmmLast}% automated use
\label{haskell.firstuse.middleAvail}% automated use
\label{haskell.firstuse.lastAvail}% automated use
\label{haskell.firstuse.MidAssign}% automated use
\label{haskell.firstuse.CmmLocal}% automated use
\label{haskell.firstuse.CmmLoad}% automated use
\label{haskell.firstuse.avail}% automated use
\label{haskell.firstuse.isStackSlotOf}% automated use
\label{haskell.firstuse.lhs}% automated use
\label{haskell.firstuse.:unexpr}% automated use
\label{haskell.firstuse.foldVarsDefd}% automated use
\label{haskell.firstuse.MidStore}% automated use
\label{haskell.firstuse.CmmVar}% automated use
\label{haskell.firstuse.isStackSlot}% automated use
\label{haskell.firstuse.varOfSlot}% automated use
\label{haskell.firstuse.LastCall}% automated use
\label{haskell.firstuse.k}% automated use
\label{haskell.firstuse.map}% automated use
\label{haskell.firstuse.id}% automated use
\label{haskell.firstuse.$}% automated use
\label{haskell.firstuse.:bs}% automated use
\label{haskell.firstuse.availVarsLattice}% automated use
\label{haskell.firstuse.empty}% automated use
\label{haskell.firstuse.add}% automated use
\label{haskell.firstuse.emptyVarSet}% automated use
\label{haskell.firstuse.new}% automated use
\label{haskell.firstuse.old}% automated use
\label{haskell.firstuse.join}% automated use
\label{haskell.firstuse.AvailVars}% automated use
\label{haskell.firstuse.UniverseMinus}% automated use
\label{haskell.firstuse.VarSet}% automated use
\label{haskell.firstuse.extendAvail}% automated use
\label{haskell.firstuse.LocalVar}% automated use
\label{haskell.firstuse.delFromAvail}% automated use
\label{haskell.firstuse.elemAvail}% automated use
\label{haskell.firstuse.Bool}% automated use
\label{haskell.firstuse.interAvail}% automated use
\label{haskell.firstuse.smallerAvail}% automated use
\end{figure*}

\section{Example analysis passes}

\seclabel{example-analyses}


\ourlib\
makes it easy to write compiler passes based on dataflow.
To show \emph{how} easy, we present
two analyses;
related transformations appear in \secref{example-rewrites}. 
The examples help solve a real problem in the Glasgow Haskell
Compiler:
because most calls are tail calls, GHC uses no 
callee-saves registers.
Therefore, at each (rare) non-tail call, all live
variables must be spilled to the stack.
\ifcutting\else
To reduce register pressure,
such variables are spilled as early as possible and reloaded as late as
possible. 
\fi

To illustrate the results of the example analyses and transformations,
here is a contrived example program in the style of \secref{example:xforms}:
\begin{alltt}
f (bits32 a) \lbr
  bits32 w, x, y, z;  // local variables
  x = a * a;
  w = a + a + a;
  y = g(w);           // call; x must be spilled
  z = y + y;
  if (y > 0) \lbr
    return z;
  \rbr else \lbr
    return z + x;
  \rbr
\rbr
\end{alltt}
%\ifpagetuning\par\vfilbreak\fi % this is hell, but if we break here,
%  downstream suffers
A~spill and a reload should be inserted as follows:%
\seclabel{spill-reload-example}
\newcommand\bigstrut{%
  \leavevmode\vrule width 0pt height 11pt depth 6pt }
\begin{alltt}
f (bits32 a) \lbr
  bits32 w, x, y, z;  
  x = a * a;
  \high{SPILL x;}
  w = a + a + a;    // no register pressure from x
  y = g(w);
  z = y + y;        // no register pressure from x
  if (y > 0) \lbr
    return z;       // x does not need reloading
  \rbr else \lbr
    \high{RELOAD x;}
    return z + x;
  \rbr
\rbr
\end{alltt}
Although the \texttt{SPILL} and \texttt{RELOAD} operations are introduced because of
the call to @g(a)@, they are moved as far from the call as possible:
@x@~is spilled immediately after being assigned @a * a@,
and @x@~is reloaded not
immediately after the call to~@g@, but just before its use in the
expression @z + x@.
On the control-flow path to @return z@, @x@~needn't be reloaded
\label{haskell.firstuse.return}% automated use
at all.




Spills and reloads are inserted 
\ifcutting\else in the right places \fi
by a sequence of 
\ifcutting\else three \fi
dataflow passes:
\finalremark{Uses of ``passes'' is not explained and is not very consistent.
Would be nice to say ``pass = analysis + (possibly degenerate)
transformation.''}%
\begin{enumerate}
\item
\label{insert-spills}
A backward analysis computes liveness
to identify the variables that should be spilled at call sites
(\secref{liveness} and \figref{liveness}).
An accompanying transformation (not shown) inserts reloads immediately
after each call 
site and inserts spills not immediately before call sites, but
rather immediately after the reaching definitions.
\item
\label{reload-duplication}
A forward analysis finds
``available variables'' which have been reloaded 
from the stack (\secref{avail} and \figref{avail}), 
and an accompanying transformation
inserts redundant reloads before their uses
(\secref{sink-reloads} and \figref{avail-rewrites}).
By keeping variables on the stack longer, this pass reduces register pressure.
% \simon{Why is the second pass a second pass?  The first
% pass added spills and reloads in; could the first pass not have 
% added the reloads immediately before the
% reloaded variables are used as well?  Maybe the text can say?
% I think the answer is that this is a \emph{forwards} analysis, since
% it is propagating forward the information about which variables currently
% have an up-to-date stack slot.}
\item
\label{remove-dead-reloads}
A backward analysis (the same as in pass~\ref{insert-spills}) computes
liveness, 
and an accompanying transformation (\figref{dead-elim} in
\secref{dead-code-elimination}), dead-assignment elimination, 
removes redundant reloads.
\end{enumerate}
%%%%%Pass~\ref{insert-spills} is not shown in this paper.
Passes
\ref{reload-duplication}~and~\ref{remove-dead-reloads} cooperate to ``sink''
reloads away from the call site.
%%  The analyses used in
%%  passes~\ref{reload-duplication}~and~\ref{remove-dead-reloads}
%%  are described in \secreftwo{avail}{liveness};
%%  the transformations are described in
%%  \secreftwo{sink-reloads}{dead-code-elimination}.


%% \ifpagetuning \enlargethispage\baselineskip \fi
%%%%%%%%%%%%%%%%%%%%%%%%%%%%%%%%%%%%%%%%%%%%%%%%%%%%%%%%%%%%%%%%%%
%
%  desperately trying to get figures to emerge in a decent order
%
\begin{figure*}\hfuzz=1pt
\begin{codetable}
\T
\begin{code}
!Live!type Live = VarSet
\end{code}
\B
& Dataflow fact\\
\hline

\T\hfuzz=21pt
\begin{code}
!liveLattice!liveLattice :: DataflowLattice Live
liveLattice = DataflowLattice emptyVarSet add
  where add new old =
          let join = unionVarSets new old in
!liveLattice.end!          (if sizeVarSet join > sizeVarSet old then SomeChange else NoChange, join)
\end{code}
\B
& Lattice\\
\hline

%%  \T
%%  \begin{code}
%%  gen  :: UserOfLocalVars    a => a -> Live -> Live
%%  kill :: DefinerOfLocalVars a => a -> Live -> Live 
%%  gen  a live = foldVarsUsed extendVarSet  live a
%%  kill a live = foldVarsDefd delFromVarSet live a
%%  \end{code}\B
%%  & Gen/Kill \mbox{functions}\\
%%  \hline

\T\hfuzz=7pt
\begin{code}
liveTransfers :: BackTransfers CmmMiddle CmmLast Live
liveTransfers = BackTransfers (flip const) middleLiveness lastLiveness

middleLiveness :: CmmMiddle -> Live -> Live
lastLiveness   :: CmmLast -> (BlockId -> Live) -> Live
!middleLiveness!middleLiveness m = addUsed m . remDefd m
!lastLiveness!lastLiveness   l = addUsed l . remDefd l . lastLiveOut l 

!liveness.addUsed.sig!addUsed :: UserOfLocalVars    a => a -> Live -> Live
remDefd :: DefinerOfLocalVars a => a -> Live -> Live 
addUsed a live = foldVarsUsed extendVarSet  live a
!liveness.remDefd.def!remDefd a live = foldVarsDefd delFromVarSet live a

lastLiveOut :: CmmLast -> (BlockId -> Live) -> Live
!lastLiveOut.1!lastLiveOut l env = last l 
!live.lastBranch!  where last (LastBranch id)        = env id 
!live.condBranch!        last (LastCondBranch _ t f) = unionVarSets (env t) (env f)
!live.lastSwitch!        last (LastSwitch _ tbl)     = unionManyVarSets $ map env (catMaybes tbl)
!live.lastCall!        last (LastCall { })         = emptyVarSet
\end{code}
\B
& Transfer \mbox{functions}\\
\hline

\T
\begin{code}
cmmLiveness :: Graph CmmMiddle CmmLast -> BlockEnv Live
cmmLiveness g = zdfFpFacts fp
!live.zdfSolveBwd!   where fp = zdfSolveBwd "liveness" liveLattice liveTransfers emptyVarSet g
\end{code}
\B
& Liveness \mbox{analysis}\\
\end{codetable}
\caption{Dataflow analysis pass to compute liveness}
\figlabel{liveness-all}
\figlabel{liveness}
\figlabel{live-lattice}
\figlabel{live-transfers}
\figlabel{live-running}
\label{haskell.def.cmmLiveness}% automated definition
\label{haskell.def.catMaybes}% automated definition
\label{haskell.def.tbl}% automated definition
\label{haskell.def.last}% automated definition
\label{haskell.def.env}% automated definition
\label{haskell.def.lastLiveOut}% automated definition
\label{haskell.def.remDefd}% automated definition
\label{haskell.def.live}% automated definition
\label{haskell.def.addUsed}% automated definition
\label{haskell.def.lastLiveness}% automated definition
\label{haskell.def.middleLiveness}% automated definition
\label{haskell.def.liveTransfers}% automated definition
\label{haskell.def.liveLattice}% automated definition
\label{haskell.def.Live}% automated definition
\label{haskell.firstuse.cmmLiveness}% automated use
\label{haskell.firstuse.liveTransfers}% automated use
\label{haskell.firstuse.middleLiveness}% automated use
\label{haskell.firstuse.lastLiveness}% automated use
\label{haskell.firstuse.addUsed}% automated use
\label{haskell.firstuse.remDefd}% automated use
\label{haskell.firstuse.lastLiveOut}% automated use
\label{haskell.firstuse.UserOfLocalVars}% automated use
\label{haskell.firstuse.DefinerOfLocalVars}% automated use
\label{haskell.firstuse.live}% automated use
\label{haskell.firstuse.foldVarsUsed}% automated use
\label{haskell.firstuse.extendVarSet}% automated use
\label{haskell.firstuse.delFromVarSet}% automated use
\label{haskell.firstuse.env}% automated use
\label{haskell.firstuse.last}% automated use
\label{haskell.firstuse.LastBranch}% automated use
\label{haskell.firstuse.LastCondBranch}% automated use
\label{haskell.firstuse.t}% automated use
\label{haskell.firstuse.f}% automated use
\label{haskell.firstuse.LastSwitch}% automated use
\label{haskell.firstuse.tbl}% automated use
\label{haskell.firstuse.unionManyVarSets}% automated use
\label{haskell.firstuse.catMaybes}% automated use
\label{haskell.firstuse.liveLattice}% automated use
\label{haskell.firstuse.unionVarSets}% automated use
\label{haskell.firstuse.sizeVarSet}% automated use
\label{haskell.firstuse.>}% automated use
\label{haskell.firstuse.Live}% automated use
\end{figure*}
%
%
%%%%%%%%%%%%%%%%%%%%%%%%%%%%%%%%%%%%%%%%%%%%%%%%%%%%%%%%%%%%%%%%%%


\subsection{Choosing node types for GHC}

To~show that \ourlib\ works at scale,
we present examples that
have been implemented and tested in GHC.  
GHC's low-level intermediate code, called @Cmm@, is a 
\label{haskell.def.Cmm}% automated definition
\label{haskell.firstuse.Cmm}% automated use
 subset of 
the portable assembly language~{\PAL}
\cite{peyton-jones-ramsey:single-intermediate}.
We~specialize \ourlib\ to GHC by instantiating type parameters
@m@~and~@l@ with GHC's types @CmmMiddle@ and @CmmLast@.

\ifpagetuning
\penalty-10000

\enlargethispage{1.0\baselineskip}
\fi


A~middle node stores the value of an expression:
\begin{code}
data CmmMiddle 
  = MidAssign CmmVar  CmmExpr -- store in variable
  | MidStore  CmmExpr CmmExpr -- store in memory
\end{code}
\label{haskell.def.CmmMiddle}% automated definition
\label{haskell.def.MidAssign}% automated definition
\label{haskell.def.MidStore}% automated definition
\label{haskell.firstuse.CmmExpr}% automated use
Type @CmmVar@ represents a variable, which may be local (@CmmLocal@
\label{haskell.def.CmmVar}% automated definition
\label{haskell.def.CmmLocal}% automated definition
@LocalVar@) or global 
\label{haskell.def.LocalVar}% automated definition
(@CmmGlobal GlobalVar@).
\label{haskell.def.CmmGlobal}% automated definition
\label{haskell.def.GlobalVar}% automated definition
\label{haskell.firstuse.CmmGlobal}% automated use
\label{haskell.firstuse.GlobalVar}% automated use
Type @CmmExpr@ represents a pure expression; 
\label{haskell.def.CmmExpr}% automated definition
among its constructors are @CmmLoad@ (a~value from memory)
\label{haskell.def.CmmLoad}% automated definition
and @CmmVar@ (the value of a variable).

A~last node represents a control transfer; constructors include
unconditional, conditional, and indirect branches, as well as a call:
\begin{code}
data CmmLast
  = LastBranch     BlockId
  | LastCondBranch CmmExpr BlockId BlockId
  | LastSwitch     CmmExpr [Maybe BlockId]
  | LastCall ...     -- arguments omitted
\end{code}
\label{haskell.def.CmmLast}% automated definition
\label{haskell.def.LastBranch}% automated definition
\label{haskell.def.LastCondBranch}% automated definition
\label{haskell.def.LastSwitch}% automated definition
\label{haskell.def.LastCall}% automated definition
\label{haskell.firstuse.Maybe}% automated use








\subsection{Available variables: a forward analysis supporting pass 2} 

\seclabel{avail}

\finalremark{We mean to do liveness first,
but that will require some editing, especially moving the explanation
of the overloaded fold functions.}



To~understand the available-variables analysis, you must know that 
each variable~$x$ is related to a stack slot~\slotof x, which is used to
save the value of~$x$. 
(GHC~represents 
the relation using Haskell functions @isStackSlot@,
@varOfSlot@, and @isStackSlotOf@.)
\label{haskell.def.isStackSlotOf}% automated definition
\label{haskell.def.isStackSlot}% automated definition
\label{haskell.def.varOfSlot}% automated definition
If the variable and the stack slot hold the same value,
that is if $x = \slotof x$,
then it is \emph{safe} to insert a reload.

To sink a reload of a variable $x$, we insert redundant reloads immediately
before uses of~$x$.
%We use an analysis that identifies not only when it is safe to insert
%such a reload, but when the inserted reload will make an earlier reload redundant.
%Specifically, 
It is \emph{profitable} to insert a reload before a use of~$x$ only if, 
on every path to the use, the most recent definition of $x$~is a reload from
$\slotof x$.
\delendum{I'm confused.  Surely that definition of profitable is also what we
mean by safe?  NR: No---safety could be establish by an arbitrary
assignment to~$x$ followed by a store to $\slotof x$.}
Safety and profitability are incomparable;
the dataflow fact computed by our analysis is the set we call
\emph{available} variables, for
which it is
safe \emph{and} profitable to insert a reload.
Because the assertion of interest is an ``all-paths'' property, 
the lattice-join operation is set intersection,
and the bottom element
is the universal set containing all variables.
%%It~is optimistic to assume that any variable can be safely reloaded from the stack,
%%but this optimistic assumption is what enables us to improve the
%%code---and if the assumption is not justified, the transfer functions
%%will correct it before
%%the analysis reaches a fixed point.
\delendum{%I'm guessing that you intend this sentence to address my long bleat:
I'm puzzled about why you are treating this example so differently
to constant-prop in Section 3.1.  It looks almost identical to me.  We could
keep a fact for every variable: $x=\bot$ means nothing is known; $x=s_x$ means
x's stack slot is up to date; $x=\top$ means x's stack slot is out of date.
Then keep a finite map as we do for constant prop.  If there is a difference
that drives the rep you have here, let's say so. If the difference is
purely accidental, we should eliminate it.

I am still unhappy on this point.  I think the difficulty is that we
want to represent \emph{both} the universal set \emph{and} the empty set.
But I'm unclear about the trade-offs in approximation that arise 
from this either-or representation.

NR: The lattice here has only two values per variable: either it is
safe and profitable to reload $x$ at this point (an all-paths
property) or it isn't.   A finite map to @Bool@ would be suitable (and
is traditional) but requires both the tedium and the run-time cost of
enumerating all the variables.
}

%\ifpagetuning\enlargethispage{\baselineskip}\fi
  


Instead of the usual mutable bit vectors, we use a purely functional
representation of sets---one in which we can represent the set of all
variables without enumerating them.
A~set is either 
$\mathtt{UniverseMinus}\,s$, which stands for all variables except
those in the set~$s$,
or $\mathtt{AvailVars}\,s$, which stands for the variables in the set~$s$
(\figref{avail-lattice}, \lineref{AvailVars}).
%
%
The bottom element is @UniverseMinus emptyVarSet@.
To manipulate these sets, we provide the functions declared in
\linerangeref{extendAvail}{smallerAvail} of \figref{avail}.
%
\label{haskell.def.delFromVarSet}% automated definition
\label{haskell.def.extendVarSet}% automated definition
\label{haskell.def.unionVarSets}% automated definition
\label{haskell.def.sizeVarSet}% automated definition
\label{haskell.def.emptyVarSet}% automated definition
\label{haskell.def.unionManyVarSets}% automated definition
\label{haskell.def.isEmptyVarSet}% automated definition
\label{haskell.def.elemVarSet}% automated definition
\label{haskell.def.VarSet}% automated definition



%%  Among many other uses, an available-expressions analysis can be~used
%%  in code-motion optimizations.
%%  For example, when making a function call, we insert
%%  spills and reloads to save and restore the values of local variables
%%  around the call site.
%%  It is easy to insert the reloads at the function-call return site,
%%  but to avoid register pressure, it would be better to leave the variable
%%  where it was spilled on the stack.
%%  Rather than complicating the code that inserts the spills and
%%  reloads around call sites,
%%  we write an analysis to insert a redundant reload immediately
%%  before a reloaded variable is used.
%%  Then, we rely on a dead-code elimination to~remove
%%  the early reload.






\ifpagetuning
\enlargethispage{1.0\baselineskip} 
\fi

% \ifpagetuning\enlargethispage{0.5\baselineskip}\fi

The most interesting part of the analysis is the @middleAvail@ transfer
function in \figref{avail-transfers}.\finalremark
{Let us revise the paper to pretend that global variables
don't exist.}
\begin{itemize}
\item
\Lineref{reload1} 
identifies an assignment that reloads local
variable~@x@ from its stack slot.\finalremark{I propose the compiler be
modified to use @isStackSlotOf@ as I've written. JD~approves.}
After such an assignment, $x = \slotof x$,
and the last definition of $x$ is a reload,
so @x@~is added to the set of available variables.
% No other variable is affected.
\item
On \lineref{assign.avail.1},
an assignment to a local variable means that the
variable need not be equal to the value in its stack
slot, so if @lhs@ is a local variable, it is removed from the set of
available variables.
The conditional removal is done by applying @foldVarsDefd@ to @delFromAvail@;
@foldVarsDefd@ is an overloaded function which, along with its dual,
is used throughout the back end:
\begin{code}
foldVarsUsed :: UserOfLocalVars a 
        => (b -> LocalVar -> b) -> b -> a -> b
foldVarsDefd :: DefinerOfLocalVars a 
        => (b -> LocalVar -> b) -> b -> a -> b
\end{code}
\label{haskell.def.foldVarsUsed}% automated definition
\label{haskell.def.UserOfLocalVars}% automated definition
\label{haskell.def.foldVarsDefd}% automated definition
\label{haskell.def.DefinerOfLocalVars}% automated definition
\label{haskell.firstuse.b}% automated use
On \lineref{assign.avail.1},  if @lhs@ is a local variable,
@foldVarsDefd@ calls @delFromAvail@;
if @lhs@ is global, @foldVarsDefd@ does nothing.


%\ifpagetuning\smallskip\fi  
 

%% \remark{To Simon and John: I've removed ``overloaded but am a bit nervous
%% because @foldVarsDef@ shows up as overloaded later in the paper, and I
%% had thought to signpost it here.  Your thoughts?}
\item 
There are three cases for @MidStore@ nodes.
\Lineref{store.avail.spill.1}
matches a node that spills a variable~$\mathtt{x}$ to the stack.
After such a node, $\mathtt{x} = \slotOf {\mathtt{x}}$,
but the node is not a reload instruction,
so @x@~is not added to the set of available variables.
%
\Lineref{store.avail.otherslot.1}
matches a node that writes any \emph{other} value to a stack slot,
after which the variable associated with that slot is no longer available.
%
\Lineref{store.avail.other} matches a store to a location that is not
a stack slot, which leaves the set of available variables unchanged.
%%
%%
%%
%%
% \footnote
% {Although @MidStore@ may overwrite a stack slot \slotof x, GHC
% carefully arranges that all stores to \slotof x have the form
% $\slotof{\mathtt{x}}= @x@$.
% These stores could be used to extend the set of available variables,
% but it is not useful to do so.}
% \simon{Why not? asks the reader.  Perhaps
% because such saves immediately precede calls?}
% \remark{How do we know that @MidStore@ doesn't
% destroy a stack slot??  I've put in a footnote but it will probably be
% simpler to fix the code.}
% \simon{Ah, this harks backe to the start of this sub-section, where
% we say ``to understand the analysis, you must know that...''.  Another
% thing you must know is that $s_x$ is used exclusively for $x$.  That's
% all, I think.}
\end{itemize}
%% \john{I've expunged MidComment from our example} % well done ----NR


The transfer function for a last node checks to see if the node is a
function call (\lineref{avail.LastCall}); if so, the set of
available variables at the call's continuation is empty.
Other last nodes do not change values of variables or stack slots, 
so the set of available variables remains unchanged.
%
A~first node has no effect on program state, so its transfer function
is @flip const@ (\lineref{avail.first}).

%%Using the @agen@ and @akill@ functions, we define the transfer functions
%%for the available-reloads analysis (see~\figref{avail-transfers}):
%%\begin{itemize}
%%\item \emph{First nodes}:
%%  A first node cannot define a variable,
%%  so the set of available reloads is the same before and after a first node.
%%  We return the set unchanged using @flip const@.
%%\item \emph{Middle nodes}:
%%  If a middle node reloads a value from a register's slot on the stack (@RegSlot@),
%%  then we add the register the register to the set of available reloads.
%%  For any other assignment to a register, we remove the register from the
%%  set of available reloads.
%%  Similarly, because a function call can overwrite the value of any local variable,
%%  the set of available reloads is empty after a~function call.
%%  Any other middle node leaves the set of available reloads unchanged.
%%\item \emph{Last nodes}:
%%  If the last node is a function call, the outgoing set of available reloads
%%  is empty for every successor basic block.
%%  Otherwise, the last node cannot modify any variables,
%%  so the set of available reloads remains unchanged.
%%\end{itemize}

Given the lattice and the transfer functions,
we can perform the analysis by calling
the \ourlib\ function @zdfSolveFwd@ (\figref{avail},
\linerangeref{avail.solve.1}{avail.solve.2}). 
\delendum{But this is really a lie. We actually call the transformation 
function!  I'm not quite sure how to fix this pedagogical point.
NR: It's true that we don't actually have a use for the results of an
independent available-variables analysis, other than to drive the
transformation in the next section.  But just because we have no use
for the results at present does not  make the example analysis
incorrect or invalid, and I think the pedagogy is sound.
}
%The function @zdfFpFacts@ returns 
%a finite map from basic-block IDs to the set of available variables
%at the beginning of each block.
Except for the implementations of the set operations on
\linerangeref{extendAvail}{smallerAvail}, 
\figref{avail} shows the \emph{entire} analysis.

\subsection{Liveness: a backward analysis supporting passes 1~and~3} 

\seclabel{liveness}

The assertion computed by 
a backward dataflow analysis applies to a
\emph{continuation} at a program point.
The classic example is liveness analysis;
the assertion of interest is that at a particular program point,
the answer produced by the continuation does not depend on
the value of a particular variable~$x$.
If~so, $x$~is said to be \emph{dead} at that point.
If the answer produced by the continuation \emph{might} depend on the
value of~$x$, $x$~is \emph{live}.\footnote
{Liveness cannot be decided accurately; it reduces to the halting problem.
As usual, we approximate liveness by reachability.}

In a modern compiler, liveness analysis supports many program
transformations,
including
dead-assignment elimination,
which removes assignments to dead variables, 
and register allocation, which
ensures that if two variables are 
live at the same time, they are not assigned to the same register. 

The dataflow fact we use to represent liveness assertions is the set of
live variables (\figref{live-lattice}, \lineref{Live}).
The bottom element of the lattice is the empty set, and the join
operation is set union (\figref{live-lattice},
\linerangeref{liveLattice}{liveLattice.end}); 
a~variable is deemed live after a node if it is live on \emph{any} edge leaving that
node.

The transfer functions for liveness rely on two auxiliary functions
@addUsed@ and @remDefd@ (\figref{liveness}, 
\linerangeref{liveness.addUsed.sig}{liveness.remDefd.def}).
A~transfer function is given a set of variables live on the edges
going out of the node.
It~removes from that set any variable
defined by the node, then adds any variable used by the
node%
\ifcutting\ \else.
For~example, if the node is
\begin{verbatim}
  i = n - 1;
\end{verbatim}
then @i@ is not live just before the node, since if we start the
program just before the assignment to~@i@, the answer cannot 
depend on the value of~@i@, which is about to be overwritten.
But the answer \emph{might} depend on the value of~@n@, so
@n@~is considered live before the assignment.
If a variable appears on both sides of an
assignment, as in \ifpagetuning{\looseness=-1 \par}\fi
\begin{verbatim}
  i = i + 1;
\end{verbatim}
then the answer might depend on~@i@, so @i@~is considered live
before the assignment.
The transfer functions
therefore
remove defined variables \emph{before} adding used variables
\fi
(\figref{liveness}, \linepairref{middleLiveness}{lastLiveness}). 

For a last node, function @lastLiveOut@ consults the solution in
progress (parameter~@env@ on \lineref{lastLiveOut.1}) to find out what
variables are live at the \emph{successors} of a 
last node. 
For an unconditional branch, we look up the live set at the label
branched to (\lineref{live.lastBranch});
for a conditional branch, we look at both true and false edges
(\lineref{live.condBranch}), 
 and
for a switch, we consider every possible target of the
branch (\lineref{live.lastSwitch}).
The~remaining case (\lineref{live.lastCall}) is a call, 
and since a call destroys the values of all local variables, no
local variables are live at its continuation.

Given the lattice and the transfer functions,
we perform liveness analysis by calling
the dataflow-engine function @zdfSolveBwd@ (\figref{liveness},
\lineref{live.zdfSolveBwd}). 
%The function @zdfFpFacts@ returns 
%a finite map from basic-block IDs to the set of live variables
%at the beginning of each block.
\figref{liveness} shows the \emph{entire} analysis.



%%%%    \ifgenkill
%%%%    \subsection{Writing transfer functions using {\mdseries\texttt{gen}} and
%%%%    {\mdseries\texttt{kill}}}
%%%%    
%%%%    %\remark{I've tried to rewrite this section without the abstract guff;
%%%%    %Simon, can you let us know what you think?}
%%%%    
%%%%    
%%%%    \seclabel{gen-kill}
%%%%    
%%%%    If you pick up a compiler textbook, you might see dataflow
%%%%    analysis explained in terms of functions called \texttt{gen} and
%%%%    \texttt{kill}, 
%%%%    which say what dataflow facts are ``generated'' and
%%%%    ``killed'' by each node.
%%%%    Our design is compatible with explanations based on \texttt{gen} and
%%%%    \texttt{kill};
%%%%    first you define functions \texttt{gen} and \texttt{kill}, then you
%%%%    use them to write the transfer functions.
%%%%    %%  Using our design, you can stillIt's~possible to write transfer functions i thi
%%%%    %%  In~a forward analysis, for example an assignment might establish an
%%%%    %%  assertion (\texttt{gen}), or it might change state in such a way that an
%%%%    %%  assertion no longer holds (@kill@), or both.
%%%%    %%  If~you want to transliterate specifications that use @gen@ and @kill@,
%%%%    %%  it's easy.
%%%%    For example, the functions @addUsed@ and @remDefd@ in \figref{liveness}'s
%%%%    liveness analysis correspond directly to the traditional @gen@ and
%%%%    @kill@ functions.
%%%%    In the available-variables analysis of \figref{avail}, @gen@ and
%%%%    @kill@ can be defined as follows:
%%%%    \begin{smallcode}
%%%%    gen  :: UserOfLocalVars    a => a -> AvailVars -> AvailVars
%%%%    kill :: DefinerOfLocalVars a => a -> AvailVars -> AvailVars
%%%%    `gen  a avail = foldVarsUsed extendAvail  avail a
%%%%    `kill a avail = foldVarsDefd delFromAvail avail a
%%%%    \end{smallcode}
%%%%    Using @gen@ and @kill@ entails no loss of efficiency;
%%%%    for example, if the right-hand side of \lineref{reload2} of \figref{avail}
%%%%    were written
%%%%    @gen x avail@,
%%%%    then GHC~would 
%%%%    identify and % \ifpagetuning\else identify and \fi
%%%%    inline the type-specific instance of @foldVarsUsed@, which for 
%%%%    \ifpagetuning\else the case of \fi
%%%%    a local variable is the identity function, so @gen x avail@
%%%%    would reduce to @extendAvail avail x@.
%%%%    \fi

\section{Using dataflow facts to rewrite graphs\ifpagetuning\else, with examples\fi}

\seclabel{rewrites}

\seclabel{example-rewrites}


We compute dataflow facts in order to enable code-improving
transformations on control-flow graphs.
A~dataflow fact may enable a rewrite function to replace a node by a
\emph{subgraph}. 
A~subgraph is a graph that may not define all the labels to which it refers.
A~valuable, novel property of our implementation is that it uses
Haskell's static type system to control which subgraphs may replace
which nodes.
Before explaining how to
transform graphs, we
explain how graphs and subgraphs are represented.


\subsection{Representing graphs and subgraphs}

\seclabel{subgraphs}

As mentioned in \secref{graph.intro},
a~graph is a collection of basic blocks, 
and a basic block is normally a first node followed by zero or
more middle nodes followed by a last node.
But
a graph may also contain two special, incomplete
blocks:
\begin{itemize}
\item
A~graph may begin with 
an \emph{entry sequence}: zero or more middle nodes
followed by a last node (i.e., a control transfer).
Such a graph is \emph{open at the entry}.  

%% \ifpagetuning \enlargethispage{0.5\baselineskip} \fi

\item
A~graph may end with
an \emph{exit sequence}: a first node followed by zero or more
middle nodes, but \emph{not} followed by a last node.
Such a graph is \emph{open at the exit} 
(control ``falls
off the end'').
\end{itemize}
Our general type of graph, called @GF@, therefore takes \emph{four}
\label{haskell.firstuse.GF}% automated use
type parameters (\figref{subgraphs}):
@m@~is the type of a middle node;
@l@~is the type of a last node;
@entry@ is either type~@O@ or type~@C@, depending on whether the graph
\label{haskell.firstuse.entry}% automated use
\label{haskell.firstuse.O}% automated use
\label{haskell.firstuse.C}% automated use
is open or closed at the entry;
and
@exit@ is type~@O@ or type~@C@, depending on whether the graph
\label{haskell.firstuse.exit}% automated use
is open or closed at the exit.
The instantiations of type parameters @entry@ and @exit@ specify the
graph's \emph{shape}, which we refer to in shorthand.
For example,
a~full @Graph@, which represents a function or
procedure, is open at the entry and closed at the exit, or
simply ``open/closed.''

\begin{figure}
\begin{code}
type O  -- marks graph as open   at entry or exit
type C  -- marks graph as closed at entry or exit
type GF m l entry exit -- graph or subgraph
type Graph m l = GF m l O C
\end{code}

\caption{Types of graphs and subgraphs}
\figlabel{subgraphs}
\label{haskell.def.Graph}% automated definition
\label{haskell.def.exit}% automated definition
\label{haskell.def.entry}% automated definition
\label{haskell.def.GF}% automated definition
\label{haskell.def.C}% automated definition
\label{haskell.def.O}% automated definition
\end{figure}




Graphs are created using these functions:
\begin{code}
mkLabel    :: BlockId        -> GF m l C O
mkMiddle   :: m              -> GF m l O O
mkLast     :: l              -> GF m l O C
(<*>)      :: GF m l e a     -> GF m l a x 
                             -> GF m l e x
emptyGraph :: GraphClosure a => GF m l a a
\end{code}
\label{haskell.def.mkLabel}% automated definition
\label{haskell.def.mkMiddle}% automated definition
\label{haskell.def.mkLast}% automated definition
\label{haskell.def.<*>}% automated definition
\label{haskell.def.emptyGraph}% automated definition
\label{haskell.firstuse.GraphClosure}% automated use
\label{haskell.firstuse.emptyGraph}% automated use
\label{haskell.firstuse.<*>}% automated use
\label{haskell.firstuse.e}% automated use
\label{haskell.firstuse.mkLast}% automated use
\label{haskell.firstuse.mkMiddle}% automated use
\label{haskell.firstuse.mkLabel}% automated use
The infix @<*>@ function is graph concatenation; the exit of the first
argument must match the entry of the next (both open or both closed).
The @emptyGraph@ is a left and right unit of concatenation;
the constraint @GraphClosure a@ is satisfied only by types @O@~and~@C@.
\label{haskell.def.GraphClosure}% automated definition

%% If $g_1$'s shape is $e$/open and $g_2$'s shape is open/$x$,
%% then the concatenation $g_1 \mathbin{\mbox{@<*>@}} g_2$ is well
%% defined and has shape 
%% $e$/$x$.  (The exit sequence of~$g_1$ is spliced to the entry sequence
%% of~$g_2$ to form a new, complete basic block.)


%% \ifpagetuning \enlargethispage{0.5\baselineskip} \fi

A graph is normally represented by a triple:
an optional entry sequence, a @BlockEnv@ containing basic blocks,
and an optional exit sequence.
As~a special case, a single sequence of middle nodes also forms a
graph open at both entry and exit.

This new representation improves significantly on our previous work
\cite{ramsey-dias:applicative-flow-graph}:
\begin{itemize}
\item
We can find the exit point of a graph in constant time.
\item
We can concatenate data structures 
in near-constant amortized time.
Previously, we had to resort to Hughes's
\citeyearpar{hughes:lists-representation:article} technique, representing
a graph as a function.
\item
Most important, errors in concatenation are ruled out at
compile-compile time by Haskell's static
type system.
In~earlier implementations, such errors were not detected until
the compiler~ran, at which point \ourlib\ tried to compensate
for the errors%
\ifcutting---but
\else
by inserting branch instructions and then issued a warning message.
But
\fi
the compensation code harbored subtle faults%
\ifcutting\else, which were discovered while developing a new back end
for~GHC\fi. 
\hfuzz=1.6pt
\end{itemize}



%% A~graph proposed by a rewrite function as a replacement for a node is
%% a \emph{replacement graph}.
%% Every replacement graph is also a \emph{subgraph}, i.e., a graph that
%% can be analyzed or rewritten independently, but that is connected to
%% an outer graph by control-flow edges.
%% A~subgraph has the same representation as a graph, including 
%% optional entry and exit sequences.
%% \simon{I don't find the definition of replacement graph and 
%% subgraph satisfactory.  Is the ONLY difference between a graph
%% and a sub-graph its purpose in the eye of the beholder?  I think
%% perhaps the key point is this: a full graph has no free labels, 
%% whereas a subgraph may have
%% free labels?  Are subgraphs o/c or c/c or anything like that?
%% What about replacement graphs?}



\begin{figure}
\hfuzz=5.7pt
\begin{code}
type Rewrite m l e x = Maybe (GF m l e x)
data ForwardRewrites m l a = ForwardRewrites
 {fr_first  :: BlockId -> a -> Rewrite m l C O,
  fr_middle :: m       -> a -> Rewrite m l O O,
  fr_last   :: l       -> a -> Rewrite m l O C} 

data BackwardRewrites m l a = BackwardRewrites
 {br_first  :: BlockId -> a      -> Rewrite m l C O,
  br_middle :: m       -> a      -> Rewrite m l O O,
  br_last   :: l -> (BlockId->a) -> Rewrite m l O C} 
\end{code}
\caption{Types of forward and backward rewrite functions.}
\figlabel{rewrites}
\label{haskell.def.br:unlast}% automated definition
\label{haskell.def.br:unmiddle}% automated definition
\label{haskell.def.br:unfirst}% automated definition
\label{haskell.def.BackwardRewrites}% automated definition
\label{haskell.def.fr:unlast}% automated definition
\label{haskell.def.fr:unmiddle}% automated definition
\label{haskell.def.fr:unfirst}% automated definition
\label{haskell.def.ForwardRewrites}% automated definition
\label{haskell.def.Rewrite}% automated definition
\label{haskell.firstuse.Rewrite}% automated use
\label{haskell.firstuse.ForwardRewrites}% automated use
\label{haskell.firstuse.fr:unfirst}% automated use
\label{haskell.firstuse.fr:unmiddle}% automated use
\label{haskell.firstuse.fr:unlast}% automated use
\label{haskell.firstuse.BackwardRewrites}% automated use
\label{haskell.firstuse.br:unfirst}% automated use
\label{haskell.firstuse.br:unmiddle}% automated use
\label{haskell.firstuse.br:unlast}% automated use
\end{figure}



\subsection{Rewrite functions}
\seclabel{rewrite-functions}

\ourlib\ transforms its graphs by composing
 transfer functions (\secref{tffuns}) with
\emph{rewrite functions}, whose
types are
shown in \figref{rewrites}. 
%
A~rewrite function is given a dataflow fact and a node~$n$.
It~may choose to replace node~$n$ with a \emph{replacement graph}~$g$,
in which case it 
returns $@Just@\;g$, or it may do nothing, in which case it returns @Nothing@.
If it returns $@Just@\;g$, it must guarantee that given
the assertions represented by incoming dataflow facts,
graph~$g$ is observationally equivalent to node~$n$.
\finalremark{Doesn't the rewrite have to be have the following property:
for a forward analysis/transform, if (rewrite P s) = Just s',
then (transfer P s $\sqsubseteq$ transfer P s').
For backward: if (rewrite Q s) = Just s', then (transfer Q s' $\sqsubseteq$ transfer Q s).
Works for liveness.
``It works for liveness, so it must be true'' (NR).
If this is true, it's worth a QuickCheck property!
}%
\finalremark{Version 2, after further rumination.  Let's define
$\scriptstyle \mathit{rt}(f,s) = \mathit{transform}(f, \mathit{rewrite}(f,s))$.
 Then $\mathit{rt}$ should
be monotonic in~$f$.  We think this is true of liveness, but we are not sure
whether it's just a generally good idea, or whether it's actually a 
precondition for some (as yet unarticulated) property of Hoopl to hold.}%

%%  
%%  
%%  If~a rewrite function returns $@Just@\;g$, the incoming dataflow fact
%%  must guarantee that replacing the node with~$g$ does not change the
%%  observable behavior of the program.
%%  
%%%%    \simon{The rewrite functions must presumably satisfy
%%%%    some monotonicity property.  Something like: given a more informative
%%%%    fact, the rewrite function will rewrite a node to a more informative graph
%%%%    (in the fact lattice.).
%%%%    \textbf{NR}: actually the only obligation of the rewrite function is
%%%%    to preserve observable behavior.  There's no requirement that it be
%%%%    monotonic or indeed that it do anything useful.  It just has to
%%%%    preserve semantics (and be a pure function of course).
%%%%    \textbf{SLPJ} In that case I think I could cook up a program that
%%%%    would never reach a fixpoint. Imagine a liveness analysis with a loop;
%%%%    x is initially unused anywhere.
%%%%    At some assignment node inside the loop, the rewriter behaves as follows: 
%%%%    if (and only if) x is dead downstream, 
%%%%    make it alive by rewriting the assignment to mention x.
%%%%    Now in each successive iteration x will go live/dead/live/dead etc.  I
%%%%    maintain my claim that rewrite functions must satisfy some
%%%%    monotonicity property.
%%%%    \textbf{JD}: in the example you cite, monotonicity of facts at labels
%%%%    means x cannot go live/dead/live/dead etc.  The only way we can think
%%%%    of not to terminate is infinite ``deep rewriting.''
%%%%    }

A~rewrite function may replace a node only with a graph of the same shape:
\begin{itemize}
\item
\iffalse
A first node with @BlockId@~$L$ must be rewritten to a closed/open
replacement graph which also begins with @BlockId@~$L$.
\delendum{Whoa!  What does it mean to say that a graph ``begins with $L$''?
Perhaps we just mean that the replacement graph must include a block labelled $L$?}
%% Simon's right---but the details are too grotty to include. ---NR
\else
A first node must be rewritten to a closed/open
 graph.
\finalremark{Revisit graphs ``beginning with $L$''}
\fi
\item
A middle node must be rewritten to an open/open graph.
\item
A last node must be rewritten to an open/closed graph.
\end{itemize}
These conditions, which are enforced by the static type system
 (\figref{rewrites}), 
 are necessary and sufficient to ensure that 
every replacement graph can be spliced in place of the node it replaces.

\subsection{Running the dataflow engine}

\overfullrule=10pt

To write a program transformation,
you must 
\begin{itemize}
\item
Create a dataflow lattice and transfer functions for the supporting
analysis, as described in \secref{create-analysis}. 
\item
Create rewrite functions for first, middle, and last nodes.
\end{itemize}
You can then use \ourlib\ function @zdfRewriteFwd@ to transform a
\label{haskell.firstuse.zdfRewriteFwd}% automated use
control-flow graph (a~backward transformation uses function @zdfRewriteBwd@,
\label{haskell.def.zdfRewriteBwd}% automated definition
\label{haskell.firstuse.zdfRewriteBwd}% automated use
which has a similar type):\begingroup\hfuzz=1pt\relax
\begin{code}
 zdfRewriteFwd 
  :: HavingSuccessors l     -- Find successors of l
  => RewritingDepth         -- Rewrite recursively?
  -> PassName               -- Name of this pass
  -> DataflowLattice a      -- Lattice
  -> ForwardTransfers m l a -- Transfer functions
  -> ForwardRewrites  m l a -- Rewrite functions
  -> a                      -- Input fact
  -> Graph m l              -- Graph or subgraph
  -> FuelMonad (FwdFixedPoint m l a (Graph m l))
\end{code}
\label{haskell.def.zdfRewriteFwd}% automated definition
\label{haskell.firstuse.FuelMonad}% automated use
\label{haskell.firstuse.RewritingDepth}% automated use
%%  
%%  class DataflowSolverDirectiontransfers fixedpt =>
%%        DataflowDirection
%%          transfers fixedpt rewrites where
%%    zdfRewriteFwd :: (DebugNodes m l, Outputable a)
%%      => RewritingDepth    -- Recursive rewrites?
%%      -> BlockEnv a        -- Init facts
%%      -> PassName          -- Analysis name
%%      -> DataflowLattice a -- Lattice
%%      -> transfers m l a   -- Transfers
%%      -> rewrites m l a    -- Input fact
%%      -> a                 -- Input fact
%%      -> Graph m l         -- CFG
%%      -> FuelMonad (fixedpt m l a (Graph m l))
%%
%%
Function @zdfRewriteFwd@ is like @zdfSolveFwd@ in
\secref{zdfSolveFwd}, but it uses and produces extra
information:\seclabel{engine-truth} 
\endgroup
\begin{itemize}
\item
Function @zdfRewriteFwd@ requires rewrite functions as well as transfer
functions.
\item
The @RewritingDepth@ parameter controls recursive rewriting;
\label{haskell.def.RewritingDepth}% automated definition
if~a graph produced by a rewrite function should not be further rewritten,
rewriting is \emph{shallow};
\label{haskell.def.RewriteShallow}% automated definition
if~a graph produced by a rewrite function can be rewritten again,
rewriting is \emph{deep}.
\label{haskell.def.RewriteDeep}% automated definition
\ifcutting\else
Deep rewriting is essential to achieve the results of
\citet{lerner-grove-chambers:2002}, e.g., to remove the induction
variable from the loop in the example in \secref{induction-var-elim}.\john{Deep rewriting isn't needed for this example; interleaving is...}
\fi
\ifcutting\else
When deep rewriting is used, the rewrite functions must
ensure that the graphs they produce are not rewritten indefinitely.
\fi
\item
In the result type, the fourth type parameter of type constructor
@FwdFixedPoint@ is a value contained in the fixed point.
The~value is extracted using function @zdfFpContents@, which has
\label{haskell.def.zdfFpContents}% automated definition
\label{haskell.firstuse.zdfFpContents}% automated use
type @FwdFixedPoint m l a b -> b@.
Here the type parameter~@b@ is instantiated to @Graph m l@: the fixed point
contains the rewritten graph.
%%  In~@zdfSolveFwd@, @b@~is instantiated with
%%  the unit type~@()@.
%%  \simon{We can't make this point until later, where we say that solve is
%%  implemented using rewrite.  NR: I don't understand why not.
%%  We are explaining the meaning of the final type parameter to
%%  @FwdFixedPoint@, as promised.}
\item
Rewriting is monadic.
A~@FuelMonad@ holds resources needed to
\label{haskell.def.FuelMonad}% automated definition
rewrite nodes into subgraphs:
a~supply of fresh labels and a supply of \emph{optimization fuel}
(\secref{fuel}). 
\end{itemize}


\begin{figure*}\hfuzz=1pt
\begin{codetable}
\T
\begin{code}
availRewrites :: ForwardRewrites CmmMiddle CmmLast AvailVars
availRewrites = ForwardRewrites first middle last
!avail.rewrites.first!  where first _ _ = Nothing
        middle m avail = maybe_reload_before avail m (mkMiddle m)
!avail.rewrites.last!        last   l avail = maybe_reload_before avail l (mkLast l)
!maybe.reload.before.1!        maybe_reload_before avail node tail =
            let used = filterVarsUsed (elemAvail avail) node
            in  if isEmptyVarSet used then Nothing
!maybe.reload.before.2!                else Just $ reloadTail used tail
        reloadTail vars t = foldl rel t $ varSetToList vars
!mkMiddle!            where rel t r = mkMiddle (reload r) <*> t
\end{code}
\B
& Rewrite \mbox{functions}\\
\hline

\T\hfuzz=21pt
\begin{code}
insertLateReloads :: Graph CmmMiddle CmmLast -> FuelMonad (Graph CmmMiddle CmmLast)
insertLateReloads g = liftM zdfFpContents fp
!insertLateReloads.1!  where fp = zdfRewriteFwd RewriteShallow "insert late reloads" availVarsLattice
!insertLateReloads.2!               availTransfers availRewrites (fact_bot availVarsLattice) g
\end{code}
& Late-reload insertion\\
\end{codetable}
\caption{Late-reload insertion, which relies on the analysis of \figref{avail}}
\figlabel{avail-rewrites}
\label{haskell.def.insertLateReloads}% automated definition
\label{haskell.def.varSetToList}% automated definition
\label{haskell.def.rel}% automated definition
\label{haskell.def.vars}% automated definition
\label{haskell.def.reloadTail}% automated definition
\label{haskell.def.used}% automated definition
\label{haskell.def.node}% automated definition
\label{haskell.def.maybe:unreload:unbefore}% automated definition
\label{haskell.def.middle}% automated definition
\label{haskell.def.first}% automated definition
\label{haskell.def.availRewrites}% automated definition
\label{haskell.firstuse.insertLateReloads}% automated use
\label{haskell.firstuse.liftM}% automated use
\label{haskell.firstuse.RewriteShallow}% automated use
\label{haskell.firstuse.availRewrites}% automated use
\label{haskell.firstuse.first}% automated use
\label{haskell.firstuse.middle}% automated use
\label{haskell.firstuse.maybe:unreload:unbefore}% automated use
\label{haskell.firstuse.node}% automated use
\label{haskell.firstuse.tail}% automated use
\label{haskell.firstuse.used}% automated use
\label{haskell.firstuse.filterVarsUsed}% automated use
\label{haskell.firstuse.isEmptyVarSet}% automated use
\label{haskell.firstuse.reloadTail}% automated use
\label{haskell.firstuse.vars}% automated use
\label{haskell.firstuse.foldl}% automated use
\label{haskell.firstuse.rel}% automated use
\label{haskell.firstuse.varSetToList}% automated use
\label{haskell.firstuse.r}% automated use
\label{haskell.firstuse.reload}% automated use
\end{figure*}


\seclabel{dfengine-spec}

Function
@zdfRewriteFwd@ implements
interleaved analysis and transformation 
in two phases \citep{lerner-grove-chambers:2002}:\seclabel{solver-phase}
\begin{itemize}
\item
In the first phase, when a rewrite function proposes to replace a
node~$n$, the replacement graph is analyzed recursively, and the results
of that analysis are used as the new dataflow
fact(s) flowing out of node~$n$.
Then the replacement
graph is \emph{thrown away}; only the facts remain.
(In~other words, rewriting is \emph{speculative}.)
If,~on a later iteration, node~$n$ is analyzed again, perhaps
with a different input fact, the rewrite function may propose
a different replacement or even no replacement at~all.
%%As described in \secref{subgraphs}, every replacement graph is a
%%\emph{subgraph}, not a complete graph.

The first phase is called the \emph{iterator}.
It computes a fixed point of the dataflow analysis
\emph{as if} nodes were replaced, while never actually replacing a node.
%%%%    \simon{The rewrite functions must presumably satisfy
%%%%    some monotonicity property.  Something like: given a more informative
%%%%    fact, the rewrite function will rewrite a node to a more informative graph
%%%%    (in the fact lattice.).
%%%%    \textbf{NR}: actually the only obligation of the rewrite function is
%%%%    to preserve observable behavior.  There's no requirement that it be
%%%%    monotonic or indeed that it do anything useful.  It just has to
%%%%    preserve semantics (and be a pure function of course).
%%%%    \textbf{SLPJ} In that case I think I could cook up a program that
%%%%    would never reach a fixpoint. Imagine a liveness analysis with a loop;
%%%%    x is initially unused anywhere.
%%%%    At some assignment node inside the loop, the rewriter behaves as follows: 
%%%%    if (and only if) x is dead downstream, 
%%%%    make it alive by rewriting the assignment to mention x.
%%%%    Now in each successive iteration x will go live/dead/live/dead etc.  I
%%%%    maintain my claim that rewrite functions must satisfy some
%%%%    monotonicity property.
%%%%    \textbf{JD}: in the example you cite, monotonicity of facts at labels
%%%%    means x cannot go live/dead/live/dead etc.  The only way we can think
%%%%    of not to terminate is infinite ``deep rewriting.''
%%%%    }
\item
When the iterator finishes, the resulting fixed point is sound,
and the facts in the fixed point are used by the second phase, in~which
no dataflow facts change, but rewrites are not speculative:
each replacement proposed by a rewrite function is actually
performed.
This phase is therefore called the \emph{actualizer}.
\end{itemize}

%The iterator executes a complicated algorithm with complicated control-flow
%that depends on the lattice, transfer functions, and rewrite functions;
Facts computed by the iterator depend on graphs produced by rewrite
functions, which in turn depend on facts computed by the iterator.
How~do we know this algorithm is sound, or even if it terminates?
A~proof requires its own POPL paper
\cite{lerner-grove-chambers:2002}, but we can give some
intuition:
\begin{itemize} 
\item
The algorithm is sound because, given the incoming dataflow facts,
each rewrite must preserve the observable behavior of the program.
A~sound analysis of the rewritten graph
may generate only dataflow facts that could have been
generated by a more complicated analysis of the original graph.
\item
No matter what the transfer functions and rewrite functions do,
the dataflow engine uses the dataflow lattice's join operation to ensure that
facts at labels never decrease. 
As~long as
\ifcutting
 no fact may
\else
 the lattice permits no fact to 
\fi
increase infinitely many
times, analysis
\ifcutting
 terminates.
\else
 is guaranteed to terminate.
\fi
\end{itemize}
Thus to guarantee soundness and termination, client code must supply 
sound transfer functions,
sound rewrite functions,
and
a
lattice with no infinite ascending chains.
And unless client code specifies shallow rewriting,
\ifcutting
\else which tells the
dataflow engine never to rewrite a replacement graph,
\fi
rewrite functions must not return replacement
graphs which contain nodes that could be rewritten indefinitely.


Why use such a complex algorithm?
\ifcutting Because interleaving \else
\citet{lerner-grove-chambers:2002} write
\begin{quote}
\emph{Previous efforts to exploit [the mutually beneficial
interactions of dataflow analyses] either (1)~iteratively performed
each individual analysis until no further improvements are discovered
or (2)~developed [handwritten] ``super-analyses'' that manually
combine conceptually separate analyses. We have devised a new approach
that allows analyses to be defined independently while still enabling
them to be combined automatically and profitably. Our approach avoids
the loss of precision associated with iterating individual analyses
and the implementation difficulties of manually writing a
super-analysis.}
\end{quote}



%%  \simon{But do we apply rewrites even before the analysis reaches a fixed point?
%%  If so, what property do the rewrites have to satisfy to ensure soundness?
%%  If not, even a single rewrite might destroy the fixed-point property of the
%%  current facts.  Or perhaps we iterate the analysis to a fixpoint, and only \emph{then}
%%  do rewriting? If so, do we need the transfer functions at that stage?
%%  
%%  Also the fixed-point of the analysis relies on upward chains. What if
%%  the rewrite pushed it downward?  Or is it the case that a rewrite must
%%  change a node $n$ into a graph $g$ so 
%%  that $\mathit{fwdtrans}(n) \leq \mathit{fwdtrans}(g)$?
%%  
%%  Also the fixpoint calculation requires multiple passses; do the 
%%  rewrites then apply multiple times?
%%  
%%  I'm deliberately playing the role of the reader here, and not peeking at
%%  the code.  I don't think it's enough to say ``go look at Chambers paper''; 
%%  I suggest we say enough (half a column would do it) to address the obvious
%%  questions and point to Chambers for details.
%%  
%%  
%%  \textbf{NR}: Good questions, but let's have a forward reference to \secref{dfengine}}

Interleaving
\fi
 analysis with transformation makes it
possible to implement useful  transformations using startlingly simple
client code.
In the rest of this section we present two examples:
\secref{sink-reloads} shows how to insert a reload instruction just
before each use of each spilled variable, and
\secref{dead-code-elim} shows how to eliminate dead assignments.
When these two transformations are run in sequence, the effect is to
sink reloads and produce programs like the example shown in
\secref{spill-reload-example}. 





%%  After defining the lattice, the transfer functions, and the rewrite functions,
%%  the client runs the analysis by invoking the dataflow framework
%%  (see~\figref{framework-fns}).
%%  The function @zdfSolveFrom@ performs an analysis on an input control-flow graph,
%%  using a dataflow lattice and a set of transfer functions.
%%  The additional arguments to the function provide
%%  the name of the analysis,
%%  the initial set of dataflow facts (usually empty),
%%  and the initial fact (usually bottom)
%%  that flows into either the entry or exit of the graph,
%%  depending on whether the transfers define a forward or backward analysis.
%%  The result of the function is the fixed point of the analysis,
%%  which stores the dataflow fact on entry to each basic block.
%%  \john{Maybe we should export a simple version of these functions to clients?
%%    Do they always do the obvious things with the initial facts and in-fact?
%%    Initial facts can reuse results of a previous analysis, but then you lose
%%    interleaving.}
%%  
%%  To combine an analysis and a transformation,
%%  the client calls the @zdfRewriteFrom@ function,
%%  which takes the same arguments as @zdfSolveFrom@,
%%  with the addition of the set of rewrite functions
%%  and a parameter (@RewritingDepth@) that decides whether the result
%%  of a rewrite function should be considered for further rewriting.
%%  The result of the function is not only the fixed point of the
%%  analysis interleaved with the transformation,
%%  but also the transformed control-flow graph.
%%  



\begin{figure*}\hfuzz=1pt
\begin{codetable}\hfuzz=21pt
\T
\begin{code}
deadRewrites = BackwardRewrites nothing middleRemoveDeads nothing
!deadRewrites.1!  where nothing _ _ = Nothing
        middleRemoveDeads :: CmmMiddle -> VarSet -> Maybe (Graph CmmMiddle CmmLast)
!elim.dead.1!        middleRemoveDeads (MidAssign (CmmLocal x) _) live
!elim.dead.2!            | not (x `elemVarSet` live) = Just emptyGraph
!deadRewrites.2!        middleRemoveDeads _ _ = Nothing
\end{code}
\B
& Rewrite \mbox{functions}\\
\hline

\T\hfuzz=40pt
\begin{code}
removeDeadAssignments :: Graph CmmMiddle CmmLast -> FuelMonad (Graph CmmMiddle CmmLast)
removeDeadAssignments g = liftM zdfFpContents fp
!rewriteBwd.1!     where fp = zdfRewriteBwd RewriteDeep "dead-assignment elim" liveLattice
!rewriteBwd.2!                  liveTransfers deadRewrites emptyVarSet g
\end{code}
& \vspace*{12pt}\mbox{Dead-code} elimination\\
\end{codetable}
\caption{Dead-assignment elimination, which relies on the analysis of
\figref{liveness}} 
\figlabel{dead-elim}
\label{haskell.def.removeDeadAssignments}% automated definition
\label{haskell.def.middleRemoveDeads}% automated definition
\label{haskell.def.nothing}% automated definition
\label{haskell.def.deadRewrites}% automated definition
\label{haskell.firstuse.removeDeadAssignments}% automated use
\label{haskell.firstuse.RewriteDeep}% automated use
\label{haskell.firstuse.deadRewrites}% automated use
\label{haskell.firstuse.nothing}% automated use
\label{haskell.firstuse.middleRemoveDeads}% automated use
\label{haskell.firstuse.not}% automated use
\label{haskell.firstuse.elemVarSet}% automated use
\end{figure*}


\subsection{Sinking reloads: a forward transformation}

\finalremark{Incidentally, I wonder if we should
use record notation when constructing @ForwardRewrites@?}

\seclabel{sink-reloads}

We use the available-variables analysis of \secref{avail} to
insert reloads
immediately before uses of variables.
The transformation is implemented by the rewrite functions on
\linerangeref{avail.rewrites.first}{avail.rewrites.last} of \figref{avail-rewrites}.
A~first node uses no variables and so is never rewritten.
For middle and last nodes, @maybe_reload_before@ 
(\linerangeref{maybe.reload.before.1}{maybe.reload.before.2})
computes @used@, which is the set
of variables used in the node that are both safe and profitable to
reload. 
%
\label{haskell.def.filterVarsUsed}% automated definition
%
If that set is not empty, function
@reloadTail@ replaces @node@ with a new graph in which @node@ is
preceded by a (redundant) reload for each variable in the set~@used@.
A~reload node is created by function @reload@ (\lineref{mkMiddle}),
\label{haskell.def.reload}% automated definition
which has type @LocalVar -> CmmMiddle@.
%%%%% This rewrite function \emph{must} be used with shallow
%%%%% rewriting. % redundant

Our transformation is implemented by the call to @zdfRewriteFwd@
on \linerangeref{insertLateReloads.1}{insertLateReloads.2} of \figref{avail-rewrites}.
Rewriting is shallow, so a graph containing reload nodes
is not itself rewritten.
(If~it \emph{were} rewritten, a nonempty @used@ set would make the
compiler insert an infinite sequence of reloads before @node@.)
Once the reloads are inserted, the original reloads
\ifcutting\else immediately following the call site
\fi
are dead, and they can be eliminated by our
next transformation, dead-assignment elimination.

\subsection{Dead-assignment elimination: a backward \rlap{transformation}}


\seclabel{dead-code-elimination}
\seclabel{dead-code-elim}

\seclabel{bwd-rewrite}


\def\liveout{$\mathit{live_{out}}$}

We use the liveness analysis of \secref{liveness} to identify
assignments
to local variables that 
are not live.
Such \emph{dead assignments} can be removed without changing the
observable behavior of the program.
The removal is implemented by the rewrite functions on
\linerangeref{deadRewrites.1}{deadRewrites.2} of \figref{dead-elim}. 
First and last nodes are not assignments and so are never
rewritten.
A~middle node is rewritten to the empty graph if and only if it is an
assignment to a dead variable (\linerangeref{elim.dead.1}{elim.dead.2}).
On \linepairref{rewriteBwd.1}{rewriteBwd.2}, we call @zdfRewriteBwd@.
That's the whole thing.\finalremark
{JD: Need to run this version of the code in anger.}
%
\finalremark{In this space we should have some guff about
composing transformations, which should refer to the example on
eliminating the induction variable.
More generally, list some places dead-assignment elim is used and
include \secref{induction-var-elim}.
}




\begin{figure*}
\def\numberedcodebackspace{0.7\baselineskip}
\setcounter{codeline}{0}
\begin{numberedcode}
!FactKont!type FactKont a b = a          -> DFM a b
type LOFsKont a b = LastOuts a -> DFM a b
!Kont!type Kont     a b =               DFM a b

!forward.sol.sig!fwd_iter :: forall m l e x a . HavingSuccessors l => (forall b . Maybe b -> DFM a (Maybe b))
         -> RewritingDepth -> PassName -> BlockEnv a -> ForwardTransfers m l a
!forward.sol.zmaybe!         -> ForwardRewrites m l a -> ZMaybe e a -> GF m l e x -> DFM a (ZMaybe x a)
!forward.sol.args!fwd_iter with_fuel depth name start_facts transfers rewrites in_fact g =
!forward.sol.setAllFacts!     do { setAllFacts start_facts ; iter_ex g in_fact }
!solve.ex.sig!   where iter_ex    :: GF m l e x -> ZMaybe e a -> DFM a (ZMaybe x a)

!solve.first.sig!         iter_first :: BlockId             -> FactKont a b -> Kont     a b
!solve.mid.sig!         iter_mid   :: m                   -> FactKont a b -> FactKont a b
!solve.last.sig!         iter_last  :: l                   -> LOFsKont a b -> FactKont a b

!solve.block.sig!         iter_block :: BlockId -> [m] -> l -> LOFsKont a b -> Kont     a b
!solve.block.code!         iter_block f ms l = iter_first f . flip (foldr iter_mid) ms . iter_last l

!set.last.sig!         set_last :: LOFsKont a ()
!set.last.*!         set_last (LastOuts l) = mapM_ (uncurry setFact) l

!solve.mid.1!         iter_mid m k in' =
!solve.mid.case!           (with_fuel $ fr_middle rewrites in' m) >>= \x -> case x of
!solve.mid.Nothing!             Nothing -> k (ft_middle_out transfers in' m)
!solve.mid.Just!             Just g  -> do { a <- subAnalysis $ case depth of
!solve.OO!                                                  RewriteDeep    -> iter_OO g return in'
!solve.rewrite.shallow.1!                                                  RewriteShallow -> anal_f_OO g in'
!solve.mid.*!                           ; k a }
\end{numberedcode}
\caption{Excerpts from the forward iterator}
\figlabel{iterator-excerpts}
% !solve.block!     iter_block (Block id tail) = iter_first id $ iter_tail tail $ set_last
% omitted, much as I would have liked to include it...
% !solve.OO.def!         iter_OO    :: GF m l O O -> FactKont a b -> FactKont a b

\label{haskell.def.in'}% automated definition
\label{haskell.def.iter:unmid}% automated definition
\label{haskell.def.set:unlast}% automated definition
\label{haskell.def.ms}% automated definition
\label{haskell.def.iter:unblock}% automated definition
\label{haskell.def.iter:unlast}% automated definition
\label{haskell.def.iter:unfirst}% automated definition
\label{haskell.def.iter:unex}% automated definition
\label{haskell.def.in:unfact}% automated definition
\label{haskell.def.rewrites}% automated definition
\label{haskell.def.transfers}% automated definition
\label{haskell.def.start:unfacts}% automated definition
\label{haskell.def.name}% automated definition
\label{haskell.def.depth}% automated definition
\label{haskell.def.fwd:uniter}% automated definition
\label{haskell.def.Kont}% automated definition
\label{haskell.def.LOFsKont}% automated definition
\label{haskell.def.FactKont}% automated definition
\label{haskell.firstuse.FactKont}% automated use
\label{haskell.firstuse.DFM}% automated use
\label{haskell.firstuse.LOFsKont}% automated use
\label{haskell.firstuse.Kont}% automated use
\label{haskell.firstuse.fwd:uniter}% automated use
\label{haskell.firstuse.ZMaybe}% automated use
\label{haskell.firstuse.with:unfuel}% automated use
\label{haskell.firstuse.depth}% automated use
\label{haskell.firstuse.name}% automated use
\label{haskell.firstuse.start:unfacts}% automated use
\label{haskell.firstuse.transfers}% automated use
\label{haskell.firstuse.rewrites}% automated use
\label{haskell.firstuse.in:unfact}% automated use
\label{haskell.firstuse.setAllFacts}% automated use
\label{haskell.firstuse.iter:unex}% automated use
\label{haskell.firstuse.iter:unfirst}% automated use
\label{haskell.firstuse.iter:unmid}% automated use
\label{haskell.firstuse.iter:unlast}% automated use
\label{haskell.firstuse.iter:unblock}% automated use
\label{haskell.firstuse.ms}% automated use
\label{haskell.firstuse.foldr}% automated use
\label{haskell.firstuse.set:unlast}% automated use
\label{haskell.firstuse.mapM:un}% automated use
\label{haskell.firstuse.uncurry}% automated use
\label{haskell.firstuse.setFact}% automated use
\label{haskell.firstuse.in'}% automated use
\label{haskell.firstuse.>>=}% automated use
\label{haskell.firstuse.subAnalysis}% automated use
\label{haskell.firstuse.iter:unOO}% automated use
\label{haskell.firstuse.anal:unf:unOO}% automated use
\end{figure*}





\section{\ourlib's dataflow engine}
\seclabel{engine}
\seclabel{dfengine}




\delendum{The earlier sections promised that we'd reveal the lies.
Do we?  I see no mention of @HavingSuccessors@ for example, which is rather important
for polymorphism.  Indeed, a subsection on that point might be a good way
to substantiate the claims of the last bullet of the conclusion.}

In sections \ref{sec:making-simple}
through~\ref{sec:rewrites},
we use \ourlib\ to create analyses and transformations.
Here we sketch the implementation of the main part of \ourlib:
the dataflow engine.
While a full description of the implementation is beyond the scope of
this paper, a sketch 
demonstrates the new ideas that make this implementation simpler
than the original:
using pure functional code throughout;
using an explicit state monad to manage the computation of fixed
points;
giving each type of graph node its own analysis function, 
which also performs speculative rewriting;
and
using continuation-passing style to stitch these functions together.
We~sketch the implementation from the bottom~up:
\ourlib's fuel monad,
the monad that holds dataflow facts,
an~iterator,
and
an~actualizer.



%The dataflow engine comprises four functions:
%a forward iterator, a forward actualizer,
%a backward iterator, and a backward actualizer.


\subsection{Throttling the dataflow engine using ``optimization fuel''}

\seclabel{vpoiso}
\seclabel{fuel}

We have extended Lerner, Grove, and Chambers's optimization-combining algorithm with
Whalley's \citeyearpar{whalley:isolation} algorithm for isolating
faults.
Whalley's algorithm is used to test a faulty optimizer;
it~automatically
finds the first rewrite that introduces a fault in a test program.
It works by giving the optimizer a finite supply of \emph{optimization fuel}.
Each time a rewrite function proposes to replace a node, one unit of fuel is
consumed.
When the optimizer runs out of fuel, further rewrites are suppressed.
Because each rewrite leaves the observable behavior of the
program unchanged, it is safe to suppress rewrites at
any point.
In~normal operation, the optimizer has unlimited fuel, but during
debugging, a fault can be isolated quickly by doing a binary search on
the size of the fuel supply.
% To control the fuel supply in a purely functional setting, we use
% the fuel monad.
%% , which works with
%% computations of type @Fuel -> (a, Fuel)@.
The fuel supply is stored in a state monad (@FuelMonad@), which
also holds a supply of fresh labels.
Fresh labels are used for making new
blocks.


\subsection{A monad for dataflow effects}

\seclabel{dataflow-monad}

\iftrue
In addition to fuel, each analysis and
transformation keeps track of the values of dataflow facts.  
Facts and fuel are stored in a
\emph{dataflow monad}, a state-transformer monad whose state includes
a private \emph{environment} mapping labels to facts, as well as
the global supplies of fuel and fresh labels.
\else
In addition to the fuel supply,
each analysis and transformation keeps
track of the values of dataflow
facts; these facts, stored in an \emph{environment}, are managed by
 a \emph{dataflow monad}.
\fi
A value in the dataflow monad has type @DFM a b@, where @a@~is the type of a
\label{haskell.def.DFM}% automated definition
dataflow fact and @b@ is the type of the value returned by the monadic action.
% And to track fuel, a dataflow monad is also a fuel monad.

%%  In addition to the fuel supply,
%%  each analysis and transformation 
%%  uses an \emph{environment} which
%%  keeps
%%  track of the values of dataflow
%%  facts; the {environment}
%%  is managed by a \emph{dataflow monad}.
%%  A value in the dataflow monad has type @`DFM a b@, where @a@~is the type of a
%%  dataflow fact and @b@ is the type of the value returned by the monadic action.
%%  The state of a dataflow monad
%%  \ifcutting
%%  also includes supplies of fuel and labels.
%%  \else
%%  includes the shared state of the
%%  fuel monad.
%%  \fi

Operations on the dataflow monad include{\hfuzz=10.5pt
\begin{code}
getFact           :: BlockId ->      DFM a a
setFact           :: BlockId -> a -> DFM a ()
getAllFacts       :: DFM a (BlockEnv a)
setAllFacts       :: BlockEnv a -> DFM a ()
useOneFuel        :: DFM a ()
fuelExhausted     :: DFM a Bool
subAnalysis       :: DFM a b -> DFM a b
withDuplicateFuel :: DFM a b -> DFM a b
runDFM :: DataflowLattice a -> DFM a b -> FuelMonad b
\end{code}
\label{haskell.def.getFact}% automated definition
\label{haskell.def.setFact}% automated definition
\label{haskell.def.getAllFacts}% automated definition
\label{haskell.def.setAllFacts}% automated definition
\label{haskell.def.useOneFuel}% automated definition
\label{haskell.def.fuelExhausted}% automated definition
\label{haskell.firstuse.runDFM}% automated use
\label{haskell.firstuse.withDuplicateFuel}% automated use
\label{haskell.firstuse.fuelExhausted}% automated use
\label{haskell.firstuse.useOneFuel}% automated use
\label{haskell.firstuse.getAllFacts}% automated use
\label{haskell.firstuse.getFact}% automated use
A~computation} in the dataflow monad has two significant side effects:
it may \emph{increase stored facts} (according to a lattice ordering)
and it may \emph{consume fuel}.
The~two most interesting operations in the monad are used to control
those effects:
\begin{itemize}
\item
Computation @subAnalysis c@ computes the same results as~@c@ and
\label{haskell.def.subAnalysis}% automated definition
consumes the same fuel as~@c@, but it does not change any stored
dataflow facts.
\item
Computation @withDuplicateFuel c@ computes the same results as~@c@ and
\label{haskell.def.withDuplicateFuel}% automated definition
changes the same stored facts as~@c@, but it consumes fuel from a
\emph{copy} of the fuel supply.
The inner computation~@c@ may run out of fuel, but afterward,
@withDuplicateFuel@ restores the original fuel supply.
Using @withDuplicateFuel@ has enabled us to eliminate fuel from
arguments and results, making an implementation which
is less error-prone and \emph{much} easier to
read than the one by \citet{ramsey-dias:applicative-flow-graph}. 
\end{itemize}
%%  Using pure code and a monad makes this version much easier to get
%%  right than our Objective Caml version
%%  \cite{ramsey-dias:applicative-flow-graph}. 
%%  \john{Presumably this sentence disappears if the italic text is made permanent.}
Function @runDFM@ runs a single analysis or transformation, then
\label{haskell.def.runDFM}% automated definition
abandons the dataflow facts and returns the result in the fuel monad.
Only @FuelMonad@ is exposed to the client;
the dataflow monad is private to \ourlib.
Using the dataflow monad, \ourlib's iterators and actualizers
are significantly simpler than those in our
previous work \cite{ramsey-dias:applicative-flow-graph}.
In~\secreftwo{forward-iterator}{forward-actualizer},
we show parts of the forward iterator and actualizer. 


%%  Note that the dataflow engine is the only part of the system that is
%%  hard to get right---this is where all the hair is.
%%  Prime benefit of our system is that once this is right, everything is
%%  easy (and indeed is just logic, strongest postcondition, or weakest
%%  precondition). 
%%  


\subsection{The forward iterator}

\seclabel{forward-iterator}

An \emph{iterator} does dataflow analysis with speculative rewriting.
Analysis begins an dataflow monad whose
environment maps all labels to bottom facts.
For each block in the control-flow graph, the iterator begins with the
dataflow facts flowing into one end of the block 
(in a forward analysis, the first node; in a
backward analysis, the last node),
then uses the transfer functions and rewrite functions to compute the
dataflow facts flowing 
out the other end of the block.
The outflowing facts are joined with the facts previously stored in the
environment, and when the facts in the environment stop changing, the
iterator terminates. 

The iterator interleaves analysis and speculative rewriting
     \cite{lerner-grove-chambers:2002}. 
At a node~$n$, the iterator passes~$n$ and
any incoming dataflow facts~\fs\ to a rewriting function.
If node~$n$ is rewritten to a graph~$g$, 
the iterator continues with the same dataflow facts~\fs\
     flowing into graph~$g$.
     After graph~$g$ is analyzed,
     it is discarded;
%, and the original node~$n$ is restored;
     only the facts flowing out of~$g$ persist.


\finalremark{What does the reader gain from here on?}

\delendum{I keep tripping over a nasty misunderstanding here.
We say that ``the dataflow engine implements only composed
analysis and transformation'', but then we provide (a)
a solver, and (b) a actualizer.  Which apparently contradicts.

NR: Apparently so.  Maybe you can think of better names.
Here's the story: 
\begin{itemize}
\item
The iterator implements composed analysis and
transformation.  The output is a set of facts and a graph.
The iterator keeps the facts and discards the graph.  (And to save
allocations, it never builds the graph in the first place.)
The iterator is both iterative and recursive.
\item
The actualizer calls the iterator, then in a \emph{single} pass, rewrites
the graph.  The actualizer is not iterative, but it is recursive.
\end{itemize}
Ideas for a better way to tell this story?

SLPJ: 
I believe that the story is that the iterator needs to perform rewriting
to get the right answer; it just doesn't retain the rewritten graph.
Well, that's ok, but it's quite confusing.  Moreover, a simple (but
perhaps less efficient) way to write the iterator would be to call the
actualizer, and simply discard the returned graph.  Correct?

NR: \emph{Incorrect}.  The actualizer calls the iterator to do most of the work.
%So {\tt fwd\_iter} is simply an efficiency hack on {\tt forward\_rew}.
%If that is so, perhaps we should simply present the forward actualizer,
%thereby avoiding the confusion altogether.  Admittedly the code is slightly
%more complicated, but not much.
}


\begin{figure*}
%%  forward_rew
%%          :: forall m l a . 
%%             (DebugNodes m l, HavingSuccessors l, Outputable a)
%%          => (forall a . Fuel -> Maybe a -> Maybe a)
%%          -> RewritingDepth
%%          -> BlockEnv a
%%          -> PassName
%%          -> ForwardTransfers m l a
%%          -> ForwardRewrites m l a
%%          -> a
%%          -> Graph m l
%%          -> Fuel
%%          -> DFM a (FwdFixedPoint m l a (Graph m l), Fuel)
%%  forward_rew squash depth xstart_facts name transfers rewrites in_factx gx fuelx = fixed_pt_and_fuel
%%    where
%%      fixed_pt_and_fuel =
%%          do { (a, g, fuel) <- rewrite xstart_facts getExitFact in_factx gx fuelx
%%             ; facts <- getAllFacts
%%             ; let fp = ... facts ... g ...
%%             ; return (fp, fuel)
%%             }
%%  
%%  
%%      rewrite_blocks (Block id t : bs) rewritten fuel =
%%        ... ar_tail h (ft_first_out transfers id a) t rewritten fuel ...
%%      rewrite_blocks [] rewritten fuel = return (rewritten, fuel)
\setcounter{codeline}{0}
\def\numberedcodebackspace{0.7\baselineskip}
\begin{numberedcode}
!GraphFactKont!type GraphFactKont  m l e x a b = GF m l e x -> a -> DFM a b
!GraphKont!type GraphKont      m l e x a b = GF m l e x      -> DFM a b

!rew.first!      ar_first :: BlockId -> GraphFactKont m l e O a b -> GraphKont     m l e C a b
      ar_mid   :: m       -> GraphFactKont m l e O a b -> GraphFactKont m l e O a b
!rew.last!      ar_last  :: l       -> GraphKont     m l e C a b -> GraphFactKont m l e O a b

!rew.mid.1!      ar_mid m k head in' =
        (with_fuel $ fr_middle rewrites in' m) >>= \x -> case x of
          Nothing -> k (head <*> mkMiddle m) (ft_middle_out transfers in' m)
!rew.subAnalysis!          Just g  -> do { (g, a) <- subAnalysis $
                             case depth of
                               RewriteDeep    -> iar_OO g (curry return) in'
!rew.anal.f.OO!                               RewriteShallow -> do { a <- anal_f_OO g in'; return (g, a) }
!rew.mid.*!                        ; k (head <*> g) a }

!iar.OO!      iar_OO :: GF m l O O -> GraphFactKont m l O O a b -> FactKont a b
\end{numberedcode}
\caption{Excerpts from the forward actualizer}
\figlabel{actualizer-excerpts}
\label{haskell.def.iar:unOO}% automated definition
\label{haskell.def.ar:unmid}% automated definition
\label{haskell.def.ar:unlast}% automated definition
\label{haskell.def.ar:unfirst}% automated definition
\label{haskell.def.GraphKont}% automated definition
\label{haskell.def.GraphFactKont}% automated definition
\label{haskell.firstuse.GraphFactKont}% automated use
\label{haskell.firstuse.GraphKont}% automated use
\label{haskell.firstuse.ar:unfirst}% automated use
\label{haskell.firstuse.ar:unmid}% automated use
\label{haskell.firstuse.ar:unlast}% automated use
\label{haskell.firstuse.head}% automated use
\label{haskell.firstuse.iar:unOO}% automated use
\label{haskell.firstuse.curry}% automated use
\end{figure*}



\finalremark{How and where do we say what's new over
\citet{ramsey-dias:applicative-flow-graph}?}
%showed a backward iterator and actualizer that kept dataflow facts in
%mutable cells.
%\emph{[New stuff is pure, polymorphic, and shows how to use
%    fuel]}\remark{fix me}


\figref{iterator-excerpts} shows excerpts from the forward iterator 
@fwd_iter@.
\begin{itemize}
\item
The @with_fuel@ parameter 
\label{haskell.def.with:unfuel}% automated definition
is called on the result of each rewriting function (e.g. \lineref{solve.mid.case}).
It consumes fuel; or if no fuel is available,
it prevents any nodes from being rewritten.
\item
Analysis of a subgraph starts with known facts, not
bottom facts; they are passed as
@start_facts@
% (\lineref{forward.sol.args}) 
and
 set on \lineref{forward.sol.setAllFacts}.
\item
A~forward analysis requires an entry fact @in_fact@ if and only if the
graph being analyzed is open at the entry.
Similarly, the analysis produces an output fact if and only if the
graph being analyzed is open at the exit.
We~express these constraints using the generalized algebraic data type
@ZMaybe@ (\figref{iterator-excerpts}, \lineref{forward.sol.zmaybe}):
\begin{code}
  data ZMaybe ex a where
    ZJust    :: a -> ZMaybe O a
    ZNothing ::      ZMaybe C a
\end{code}
\label{haskell.def.ZMaybe}% automated definition
\label{haskell.def.ex}% automated definition
\label{haskell.def.ZJust}% automated definition
\label{haskell.def.ZNothing}% automated definition
\label{haskell.firstuse.ZNothing}% automated use
\label{haskell.firstuse.ZJust}% automated use
\label{haskell.firstuse.ex}% automated use
Using @ZMaybe@ to construct the types of the input and output facts
has simplified our implementation of the dataflow engine and has
eliminated dynamic tests of the shapes of subgraphs.
\end{itemize}
%A~fixed point is computed by initializing the facts using
%@setAllFacts@, which is an operation in the dataflow monad.
%
%\iffalse
%Function @solve@, on \linerangeref{solve.1}{solve.*} of
%\figref{iterator-excerpts}, 
%takes an input fact, a graph, and a fuel supply; it~returns a pair
%containing the exit fact and the 
%remaining fuel supply.
%It~also has a side effect on the state stored in the inner dataflow monad:
%it brings the facts associated with labels up to a fixed point.
%\fi


\vfilbreak[3\baselineskip]

%
\delendum{Hang on!  What kind of graph does {\tt zdfSolveFwd} take?
I assume a full graph, closed at the entry!  So what is this pesky {\tt in\_fact}??}
%

% JD didn't see the point:
% Function @zdfSolveFwd@ calls @fwd_iter@:
% \begin{smallcode}
% zdfSolveFwd name ^lattice ^transfers in_fact g = 
%   runWithoutFuel $ runDFM lattice $ ffp () $ 
%   fwd_pure_anal name emptyBlockEnv transfers in_fact g
% fwd_pure_anal name env transfers in_fact g =
%   fwd_iter (\_ _ -> Nothing) undefined name env 
%               transfers undefined in_fact g
% ffp :: b -> DFM a (ZMaybe x a) 
%     -> DFM a (FwdFixedPoint m l a b)
% runWithoutFuel :: FuelMonad a -> a
% \end{smallcode}
% Funtion @`fwd_pure_anal@ is the special case of pure analysis; it is
% also used to implement function @`anal_f_OO@ on
% \lineref{solve.rewrite.shallow.1} of \figref{iterator-excerpts}.
% Function~@`ffp@ (not shown) extracts a @FwdFixedPoint@ from the
% state stored in the dataflow monad.
% Function @`runWithoutFuel@ (not shown) exploits lazy evaluation to run
% a computation in the fuel monad while guaranteeing that the
% computation uses no fuel and no labels.
% \john{The previous paragraph was a big interruption in my reading.
% Is it serving a Higher Purpose,
% or can we put this code at the bottom of Figure 10
% and give a briefer description inline? It seems to me that the only salient points
% are that zdfSolveFwd calls forwardsol with empty rewrites,
% then extracts the results.}


The function @iter_ex@ (type on \lineref{solve.ex.sig}, implementation not
shown), solves a graph or subgraph~@g@.
Where the graph is open, @iter_ex@ converts @ZMaybe@ facts to actual
facts---the static type system precludes the possibility of a missing
or superfluous fact.

The iterator is composed of functions written in continuation-passing style:
the result of analyzing part of a graph is a function from
continuations to continuations. 
The types of the
continuations are shown on \linerangeref{FactKont}{Kont} of
\figref{iterator-excerpts}. 
\begin{itemize}
\item
Type @FactKont a b@ describes a context following a first node or middle
node: in a forward analysis, the context expects a fact of type~@a@
to flow out of the node.
The rest of the analysis consumes that
fact and produces a computation in the
dataflow monad (@DFM a@) with an answer of type~@b@.
%Type @FactKont a b@ also describes the context following the analysis of
%any replacement graph that is open at the exit.
\item
Type @LOFsKont a b@ describes a context following a last node.
The type is dictated by the type of the transfer function
@ft_last_outs@ in \figref{transfers}:
since as many facts flow out of a last node as there are control-flow
edges leaving that node, the context expects those facts to have type
@LastOuts a@.
\item
Type @Kont a b@ describes a context \emph{before} a first node (or a
basic block).
The dataflow fact flowing into the note is not passed as a parameter;
it is extracted from the dataflow monad's environment by calling
the monadic operation @getFact@.
\ifcutting\else
Thus, a @Kont@ expects no fact as argument.
\fi
\end{itemize}
\delendum{I like these continuations, but I'm very puzzled about the
fuel.  If @Kont@ takes fuel in, where does it return the depleted fuel?
It must come out eventually, because it's needed in the rest of the program.
And if it always comes out, then we should say so:
@Kont a b = DFM a b@.
And once you do that, it's plain that @Kont@ is just a state monad,
and I can't see why it isn't part of @DFM@ in the first place.}

Declarations of 
\ifpagetuning
continuation-passing
\fi
iterator functions for nodes are shown on
\linerangeref{solve.first.sig}{solve.last.sig} of
\figref{iterator-excerpts}. 
Function @iter_last@ on \lineref{solve.last.sig} maps a @LOFsKont@ to a @FactKont@;
@iter_mid@ on \lineref{solve.mid.sig} maps a @FactKont@ to another @FactKont@;
and
@iter_first@ on \lineref{solve.first.sig} maps a @FactKont@ to a @Kont@.
To analyze a basic block, with speculative rewriting, we compose these
three functions, as shown in function @iter_block@ on
\linepairref{solve.block.sig}{solve.block.code}.\footnote
{To simplify the example, we conceal \ourlib's
representation of blocks.}

In~code not shown here, 
function @iter_block@ is applied to continuation @set_last@
(\linepairref{set.last.sig}{set.last.*}), 
which updates the environment of facts stored in the dataflow monad.
The value @iter_block set_last@ is a computation of type
%\ifcutting\else
@Kont a ()@, which is
%\fi
@DFM a ()@.
This computation reads the stored fact flowing into a block,
propagates facts
%\ifcutting\else
 through the block
%\fi
using transfer functions and
speculative rewriting, and finally updates the stored facts flowing out to
the block's successors.
Iterator functions for graphs%
%\ifcutting, \else
\ and subgraphs,
%\fi
like @iter_ex@,
perform such a
computation for every block, then repeat until stored facts stop changing.
Each iteration
runs under @withDuplicateFuel@, so
@fwd_iter@
\emph{simulates} the effects of a fuel limit, but it does
not actually consume fuel.


%%  \iffalse %%%%%%%%%%%%%%%%%%%%%%%%%%%%%%%%%%%%%%%%%%%%%%%%%%%%%%%%%%%%%%%%
%%  
%%  
%%  % The result types of @iter_first@, @iter_mid@, and @iter_last@
%%  % compose nicely, so that for any basic block we can compute a function
%%  % of type
%%  % @LOFsKont a b -> Kont a b@.
%%  Once we have analyzed a block, we update the facts stored in
%%  the dataflow monad.
%%  \ifcutting
%%   Function @set_last@
%%  (\linerangeref{set.last.sig}{set.last.*}) calls @setFact@ on each fact
%%  flowing out of a block.
%%  \else
%%  The facts are updated by function @set_last@
%%  (\linerangeref{set.last.sig}{set.last.*}), which calls @setFact@ on each fact
%%  flowing out of a block.
%%  \fi
%%  By applying the composition
%%  of @iter_first@, @iter_mid@, and @iter_last@ to @set_last@, we get
%%  the implementation of @iter_block@.\footnote
%%  {Similar
%%    compositions of @iter\_first@, @iter\_mid@, and @iter\_last@
%%    are used to analyze the partial blocks in a graph that is open
%%    at entry or exit.}
%%  The iterator calls @iter_block@ repeatedly
%%  \ifcutting\else
%%   over all the blocks
%%   in the control-flow graph
%%  \fi
%%  until facts stop changing.
%%  \fi %%%%%%%%%%%%%%%%%%%%%%%%%%%%%%%%%%%%%%%%%%%%%%%%%%%%%%%%%%%%%%%%


\ifpagetuning\enlargethispage{1.1\baselineskip}

Computations in @fwd_iter@, such as @iter_block@, are compositions
of @iter_first@, @iter_mid@, and @iter_last@.
Because these three functions so
resemble one another, we show only
one:
@iter_mid@, on
\linerangeref{solve.mid.1}{solve.mid.*} of \figref{iterator-excerpts}.
On~\lineref{solve.mid.case}, a~rewrite function gets an
input fact~@in'@ and a middle node~@m@.
If~the rewrite function proposes no replacement graph, 
or if no fuel is available, the application of @with_fuel@
returns @return Nothing@, and continuation~@k@ is given the output fact
(computed 
\iffalse
on \lineref{solve.mid.Nothing} by @ft_middle_out@).
\else
by @ft_middle_out@ on \lineref{solve.mid.Nothing}).
\fi
% and the current supply of fuel.
%
The interesting case occurs on \linerangeref{solve.mid.Just}{solve.mid.*},
when the rewrite function 
proposes a replacement graph~@g@.
Function @with_fuel@ decrements the fuel supply and produces~@g@.
\begin{enumerate}
\item
If we are doing \emph{deep} rewriting, then as @g@~is analyzed,
it may be rewritten further.
Because @g@~replaces a middle node, it is open at entry and exit,
so it is analyzed and rewritten 
on~\lineref{solve.OO}
by a recursive call to @iter_OO@ 
\label{haskell.def.iter:unOO}% automated definition
(implementation not shown;
type \nrmono{GF m l O O -> FactKont a b -> FactKont a b}).
The recursive call gets continuation @return@, and the
resulting @FactKont a a@ is given
the input fact.
The output fact is computed in a sub-analysis
and bound on
\lineref{solve.mid.Just}. 
%
Function @subAnalysis@ rolls back the facts mutated
by @iter_OO@%
\ifcutting
\else
as it iterates to a fixed point%
\fi, but @subAnalysis@
does account for 
fuel consumed by @iter_OO@.
\item
If we are doing \emph{shallow} rewriting,  the new graph~@g@ must not be
rewritten, but we must still find a fixed point of the transfer
equations.
We compute that fixed point using @anal_f_OO@ (\lineref{solve.rewrite.shallow.1}).
\label{haskell.def.anal:unf:unOO}% automated definition
Function @anal_f_OO@ (not shown) recursively calls
@fwd_iter@ using
\nrmono{with\char`\_fuel = \char`\\{ }\char`\_ -> return Nothing}, 
and so it does no rewriting and consumes no fuel.
\finalremark{We hope that cunning transfer functions will make this
higher-order function moot.}
\delendum{This business of passing in a different @with\_fuel@ seems terribly
clumsy to me.  The obvious thing would be to add a third constructor \texttt{NoRewrite}
to the @RewritingDepth@ type, so that we could call @fwd\_iter@ saying
``don't do any rewriting at all''.  Would that even eliminate the higher order
@with\_fuel@ parameter altogether?  What is it used for? 
NR:~If you re-examine the structure of the case expressions, you'll
see that \texttt{NoRewrite} as a third constructor would leave to
inexhaustive pattern matching.  A Boolean would do.
Function @with\_fuel@ is also
used to decrement the fuel supply.
Combining the decrement with the test in a higher-order function
simplified the code significantly (and made it impossible to forget to
decrement).
}
%
\end{enumerate}
Whether rewriting is shallow or deep, the application on
\lineref{solve.mid.*} solves the rest of the graph by applying the
continuation~@k@%
\ifcutting.
\else\
to the new fact~@a@.
\fi







Function @zdfSolveFwd@ is implemented by calling @fwd_iter@
with the transfer functions given, with undefined rewrite functions, and with 
parameter @with_fuel = \ _ -> return Nothing@.



\ifpagetuning\enlargethispage{1.1\baselineskip}\fi

\subsection{The forward actualizer}



\seclabel{forward-actualizer}


An~iterator returns dataflow facts, leaving the graph
unchanged. 
An~\emph{actualizer} takes facts and a graph, and in a single
pass uses rewrite functions to create a new graph.
The~actualizer also uses transfer functions to materialize facts
on edges within basic blocks.


The forward actualizer closely resembles the
forward iterator, but because the actualizer passes a rewritten graph as
an accumulating parameter, the continuations have
different types, as shown on \linepairref{GraphFactKont}{GraphKont} of
\figref{actualizer-excerpts}. 
When the actualizer runs, the dataflow monad already contains a fixed
point, so there is no need to propagate facts out of a block, and
so no continuation analogous 
to @LOFsKont@.

The functions that actualize rewrites are again in
continuation-passing style;
\linerangeref{rew.first}{rew.last}  of
\figref{actualizer-excerpts} give the types of the base-case
functions.
%
We show only @ar_mid@ (\linerangeref{rew.mid.1}{rew.mid.*}).
It~is much like function @iter_mid@ on 
\linerangeref{solve.mid.1}{solve.mid.*} of \figref{iterator-excerpts}.
\Lineref{rew.mid.1} shows the additional parameter @head@, which
contains the (rewritten) graph preceding middle node~@m@.
When no rewrite is proposed, the only change to the code is that
continuation~@k@ takes the additional parameter
@head <*> mkMiddle m@,
which is the graph formed by concatenating graph @head@ and node~@m@. 
When a rewrite is proposed, the sub-analysis computes not just an
output fact but also a possibly rewritten graph
(\linerangeref{rew.subAnalysis}{rew.anal.f.OO}).
Rewriting proceeds with the new graph @head <*> g@
(\lineref{rew.mid.*}).

The recursive iterate-and-actualize-rewrites function  @iar_OO@ (type on
\lineref{iar.OO}, implementation not shown) 
has no counterpart in the iterator.
It calls the iterator to set the dataflow facts to a fixed
point (using a duplicate fuel supply), then calls actualize-rewrite
functions to 
rewrite the graph based on 
those facts (using the shared fuel supply). 
Similar functions apply to graphs of other shapes; for example,
@iar_OC@ is used to implement
\label{haskell.def.iar:unOC}% automated definition
\label{haskell.firstuse.iar:unOC}% automated use
@zdfRewriteFwd@.


\section{Conclusions}

Compiler textbooks make dataflow optimization appear
difficult and complicated.
In~this paper, we show how to engineer a library, \ourlib, which makes
it easy to build analyses and transformations based on dataflow.
\ourlib\ makes dataflow simple not by using a single magic
ingredient, but by applying ideas that are well understood by 
the programming-language community.
\begin{itemize}
\item
We acknowledge only one program-analysis technique: the solution of
recursion equations over assertions.
%Like our colleagues working in imperative languages, 
We solve the equations by iterating to a fixed point.
% Many equations relate
% properties of program states; some relate properties of paths through
% programs. 
\item
We consider only two
ways of relating assertions: weakest liberal precondition and strongest 
postcondition, which
 correspond 
%\ifpagetuning\else
%respectively
%\fi
to
\ifcutting
``backward'' and ``forward'' dataflow
problems.
\else
``backward dataflow problems'' and ``forward dataflow
problems.''
\fi
\finalremark
{Can we give an example of a property of program states which is
neither, just by way of contrast; ie this we cannot do.}
%%  \item
%%  In a language that admits loops, iterating weakest preconditions or
%%  strongest postconditions typically does not reach a fixed point in
%%  finitely many steps; hence the need for loop invariants
%%  \cite{hoare:axiomatic-basis,dijkstra:discipline,gries:science-programming}.
%%  In \secref{logic-reconciled}, we show that we can guarantee to reach a
%%  fixed point by limiting what we can express in the logic.\remark{needs
%%  a fix}
%%  We~show that many classic analyses can be explained this way;
%%  the great diversity of classic analyses corresponds to a great
%%  diversity of inexpressive logics.
%%  This view leads us to a unifying principle:
%%  \emph{To implement a code-improving transformation, find the least
%%    expressive logic that can justify the transformation, then use that
%%    logic to compute strongest postconditions, which justify the
%%    transformation locally.}
\item
Although our implementation allows graph nodes to be rewritten in any
way that preserves semantics, we describe
three program-transformation techniques:
substitution of equals for equals, 
insertion of assignments to unobserved variables, 
and removal of assignments to unobserved variables
(\secref{example:transforms}).
Substitution of equals for equals is often justified by properties of program
states; for example, if variable~$x$
is always~7, we may substitute~7 for~$x$.\finalremark
{We can also justify substitution of \emph{labels} in goto
  statements by reasoning about continuations.  This is
  probably not the place to mention this fact.}
Insertion and removal of assignments are often justified by properties
of paths through programs;
for example, if an assignment's continuation does not use the variable
assigned~to, that assignment may be removed.

%%  Some compiler texts treat the removal of unreachable code as a
%%  code-improving transformation in its own right.
%%  In~our framework, unreachable code becomes unreachable in the
%%  garbage-collection sense, so no special effort is required to remove
%%  it.
\item
Complex program transformations should be composed from simple
transformations. 
For example, both ``code motion'' and ``induction-variable
elimination'' can be implemented in three stages: insert new assignments;
substitute equals for equals; remove unneeded assignments
(\secref{induction-var-elim}). 

\item 
Because each rewrite leaves the semantics
of the program unchanged, 
we can use 
``optimization fuel'' to limit the number of rewrites.
 When we isolate a fault 
\ifcutting\else in the optimizer \fi
(\secref{vpoiso}), we 
\ifcutting have to debug just \else therefore have the luxury of debugging \fi
 a single
 rewrite, not a complex transformation.
\end{itemize}

We also build on proven implementation techniques
in a way that
makes it easy
to implement classic code improvements.
\begin{itemize}
\item
We use the algorithm of \citet{lerner-grove-chambers:2002} to 
compose analyses and transformations.
This~algorithm makes it easy to compose complex transformations
from simple ones.

Using continuation-passing style and generalized algebraic data types,
we have created a new implementation%
\ifcutting\else\ of the algorithm\fi, 
which works by 
composing three relatively simple functions
(\secref{forward-iterator}). 
The functions are simple
because the static type of a node constrains the number of predecessors
and successors it may have.
And because we can
compare our code with a standard continuation semantics, we have more
confidence in this new implementation than in 
any previous implementation. 
\item
Our code is pure.
Inspired by Huet's~\citeyearpar{huet:zipper} zipper,
we use an applicative representation of
control-flow graphs
\cite{ramsey-dias:applicative-flow-graph}. 
We~improve on our prior work by storing changing dataflow facts
in an explicit dataflow monad,
which
makes it especially easy to implement such
operations as sub-analysis of a replacement graph
(\secref{dataflow-monad});
by using static types to guarantee that each replacement graph can be
spliced in place of the node it replaces
(\secreftwo{subgraphs}{rewrite-functions});
and by simplifying our implementation using continuation-passing style
(\secreftwo{forward-iterator}{forward-actualizer}). 
%
% important, but no longer mentioned in this paper:
%
%%  \item
%%  To \emph{construct} programs, we use a different representation of
%%  flow graphs, one which hides the complexity of the zipper and which
%%  provides a constant-time operation for joining flow graphs in
%%  sequence.
%%  It is inspired in part by Hughes's \citeyearpar{hughes:novel-lists}
%%  representation of lists, which supports a constant-time append operation.
\item
\ourlib\ is polymorphic in the
representations of 
assignments and control-flow operations.
%%  Although our polymorphic representations have been instantiated only
%%  with the low-level intermediate code used by the Glasgow Haskell
%%  Compiler, they are intended eventually to be instantiated with
%%  machine-dependent representations of target-machine instructions, as
%%  part of a larger project of refactoring GHC's back ends.
%
%This design seems obvious in retrospect,
%but we underestimated the degree to which polymorphism would force us to
%separate concerns.
%Introducing polymorphism has made the code simpler, easier
%to understand, and easier to maintain.
By forcing us to separate concerns, introducing polymorphism
made the code simpler, easier to understand, and easier to maintain.
\finalremark
{SLPJ: Is it possible to substantiate this claim by [more] examples?}
In particular, it forced us to make explicit \emph{exactly} what
\ourlib\ 
 must know about flow-graph nodes:
it must be able to find
targets of control-flow operations (constraint
@HavingSuccessors l@, \secref{zdfSolveFwd}).
\end{itemize}
%
% gen and kill are history
%
%%\item
%%Judicious use of Haskell type classes makes is possible to write
%%weakest precondition or strongest postcondition using the ``transfer
%%equations'' that are familiar from compiler textbooks.
%%If you like, you can even write overloaded @gen@ and @kill@ functions.
%%The benefit is that it is easy to compare the actual code with the
%%abstract treatments found in textbooks\ifgenkill \ (\secref{gen-kill})\fi.
Using \ourlib,
you can create a new code improvement in three steps:
create a lattice representation for the assertions you want to
express;
create transfer functions that approximate weakest preconditions or
strongest postconditions;
and 
create rewrite functions that use your assertions to justify
program transformations.  
You can get quickly to the real 
intellectual work of code improvement: identifying interesting
transformations and the assertions that justify them.

\finalremark{Don't forget acknowledgements!!!
Microsoft, Intel, NSF
}

\makeatother

\providecommand\includeftpref{\relax}
\IfFileExists{nrbib.tex}{\bibliography{cs,ramsey}}{\bibliography{cs,ramsey,simon,jd}}
\bibliographystyle{plainnatx}


\clearpage

\appendix


\section{Index of defined identifiers}

This appendix lists every nontrivial identifier used in the body of
the paper.  
For each identifier, we list the page on which that identifier is
defined or discussed---or when appropriate, the figure (with line
number where possible).
For those few identifiers not defined or discussed in text, we give
the type signature and the page on which the identifier is first
referred to.

Some identifiers used in the text are defined in the Haskell Prelude;
for those readers less familiar with Haskell, these identifiers are
listed in Appendix~\ref{sec:prelude}.

\newcommand\dropit[3][]{}

\newcommand\hsprelude[2]{\noindent
  \texttt{#1} defined in the Haskell Prelude\\}
\let\hsprelude\dropit

\newcommand\hspagedef[3][]{\noindent
  \texttt{#2} defined on page~\pageref{#3}.\\}
\newcommand\omithspagedef[3][]{\noindent
  \texttt{#2} not shown (but see page~\pageref{#3}).\\}
\newcommand\omithsfigdef[3][]{\noindent
  \texttt{#2} not shown (but see Figure~\ref{#3} on page~\pageref{#3}).\\}
\newcommand\hsfigdef[3][]{%
  \noindent
  \ifx!#1!%
    \texttt{#2} defined in Figure~\ref{#3} on page~\pageref{#3}.\\
  \else
    \texttt{#2} defined on \lineref{#1} of Figure~\ref{#3} on page~\pageref{#3}.\\
  \fi
}    
\newcommand\hstabdef[3][]{%
  \noindent
  \ifx!#1!
    \texttt{#2} defined in Table~\ref{#3} on page~\pageref{#3}.\\
  \else
    \texttt{#2} defined on \lineref{#1} of Table~\ref{#3} on page~\pageref{#3}.\\
  \fi
}    
\newcommand\hspagedefll[3][]{\noindent
  \texttt{#2} {let}- or $\lambda$-bound on page~\pageref{#3}.\\}
\newcommand\hsfigdefll[3][]{%
  \noindent
  \ifx!#1!%
    \texttt{#2} {let}- or $\lambda$-bound in Figure~\ref{#3} on page~\pageref{#3}.\\
  \else
    \texttt{#2} {let}- or $\lambda$-bound on \lineref{#1} of Figure~\ref{#3} on page~\pageref{#3}.\\
  \fi
}    

\newcommand\nothspagedef[3][]{\notdefd\ndpage{#1}{#2}{#3}}
\newcommand\nothsfigdef[3][]{\notdefd\ndfig{#1}{#2}{#3}}
\newcommand\nothslinedef[3][]{\notdefd\ndline{#1}{#2}{#3}}

\newcommand\ndpage[3]{\texttt{#2}~(p\pageref{#3})}
\newcommand\ndfig[3]{\texttt{#2}~(Fig~\ref{#3},~p\pageref{#3})}
\newcommand\ndline[3]{%
  \ifx!#1!%
      \ndfig{#1}{#2}{#3}%
  \else
      \texttt{#2}~(Fig~\ref{#3}, line~\lineref{#1}, p\pageref{#3})%
  \fi
}



\newif\ifundefinedsection\undefinedsectionfalse

\newcommand\notdefd[4]{%
  \ifundefinedsection
    , #1{#2}{#3}{#4}%
  \else
    \undefinedsectiontrue
    \par
    \section{Undefined identifiers}
    #1{#2}{#3}{#4}%
  \fi
}

\begingroup
\raggedright

\hsprelude{!}{Prelude}% context prelude
\hsprelude{\$}{Prelude}% context prelude
\hsprelude{\&}{Prelude}% context prelude
\hsprelude{\&\&}{Prelude}% context prelude
\hsprelude{*}{Prelude}% context prelude
\hsprelude{+}{Prelude}% context prelude
\hsprelude{++}{Prelude}% context prelude
\hsprelude{-}{Prelude}% context prelude
\hsprelude{.}{Prelude}% context prelude
\hsprelude{/}{Prelude}% context prelude
\hspagedef{<*>}{haskell.def.<*>}% context document
\hsprelude{==}{Prelude}% context prelude
\hsprelude{>}{Prelude}% context prelude
\hsprelude{>=}{Prelude}% context prelude
\hsprelude{>>}{Prelude}% context prelude
\hsprelude{>>=}{Prelude}% context prelude
\hsfigdefll{add}{haskell.def.add}% context figure
\hsfigdef{addUsed}{haskell.def.addUsed}% context figure
\hspagedef{anal\_f\_OO}{haskell.def.anal:unf:unOO}% context document
\hsfigdef[rew.first]{ar\_first}{haskell.def.ar:unfirst}% context figure
\hsfigdef[rew.last]{ar\_last}{haskell.def.ar:unlast}% context figure
\hsfigdef[rew.mid.1]{ar\_mid}{haskell.def.ar:unmid}% context figure
\hsfigdefll[reload1]{avail}{haskell.def.avail}% context figure
\hsfigdef{availRewrites}{haskell.def.availRewrites}% context figure
\hsfigdef[avail.first]{availTransfers}{haskell.def.availTransfers}% context figure
\hsfigdef[AvailVars]{AvailVars}{haskell.def.AvailVars}% context figure
\hsfigdef{availVarsLattice}{haskell.def.availVarsLattice}% context figure
\hsfigdef{BackTransfers}{haskell.def.BackTransfers}% context figure
\hsfigdef{BackwardRewrites}{haskell.def.BackwardRewrites}% context figure
\hspagedef{Block}{haskell.def.Block}% context document
\hspagedef{BlockEnv}{haskell.def.BlockEnv}% context document
\hspagedef{BlockId}{haskell.def.BlockId}% context document
\hsprelude{Bool}{Prelude}% context prelude
\hsfigdef{br\_first}{haskell.def.br:unfirst}% context figure
\hsfigdef{br\_last}{haskell.def.br:unlast}% context figure
\hsfigdef{br\_middle}{haskell.def.br:unmiddle}% context figure
\hsfigdef{bt\_first\_in}{haskell.def.bt:unfirst:unin}% context figure
\hsfigdef{bt\_last\_in}{haskell.def.bt:unlast:unin}% context figure
\hsfigdef{bt\_middle\_in}{haskell.def.bt:unmiddle:unin}% context figure
\hsfigdef{C}{haskell.def.C}% context figure
\omithsfigdef{catMaybes :: [Maybe a] -> [a]}{haskell.def.catMaybes}% context figure
\hsfigdef{ChangeFlag}{haskell.def.ChangeFlag}% context figure
\hspagedef{Cmm}{haskell.def.Cmm}% context document
\hsfigdef{cmmAvailableVars}{haskell.def.cmmAvailableVars}% context figure
\hspagedef{CmmExpr}{haskell.def.CmmExpr}% context document
\hspagedef{CmmGlobal}{haskell.def.CmmGlobal}% context document
\hspagedef{CmmLast}{haskell.def.CmmLast}% context document
\hsfigdef{cmmLiveness}{haskell.def.cmmLiveness}% context figure
\hspagedef{CmmLoad}{haskell.def.CmmLoad}% context document
\hspagedef{CmmLocal}{haskell.def.CmmLocal}% context document
\hspagedef{CmmMiddle}{haskell.def.CmmMiddle}% context document
\hspagedef{CmmVar}{haskell.def.CmmVar}% context document
\hsprelude{const}{Prelude}% context prelude
\hsprelude{curry}{Prelude}% context prelude
\hsprelude{Data.Map}{Prelude}% context prelude
\hsfigdef{DataflowLattice}{haskell.def.DataflowLattice}% context figure
\hsfigdef{deadRewrites}{haskell.def.deadRewrites}% context figure
\hspagedef{DefinerOfLocalVars}{haskell.def.DefinerOfLocalVars}% context document
\hsfigdef{delFromAvail}{haskell.def.delFromAvail}% context figure
\omithspagedef{delFromVarSet :: VarSet -> LocalVar -> VarSet}{haskell.def.delFromVarSet}% context document
\hsfigdefll[forward.sol.args]{depth}{haskell.def.depth}% context figure
\hspagedef{DFM}{haskell.def.DFM}% context document
\hsfigdef{elemAvail}{haskell.def.elemAvail}% context figure
\omithspagedef{elemVarSet :: LocalVar -> VarSet -> Bool}{haskell.def.elemVarSet}% context document
\hsfigdefll{empty}{haskell.def.empty}% context figure
\omithspagedef{emptyBlockEnv :: BlockEnv a}{haskell.def.emptyBlockEnv}% context document
\hspagedef{emptyGraph}{haskell.def.emptyGraph}% context document
\omithspagedef{emptyVarSet :: VarSet}{haskell.def.emptyVarSet}% context document
\hsfigdefll{entry}{haskell.def.entry}% context figure
\hsfigdefll[lastLiveOut.1]{env}{haskell.def.env}% context figure
\hspagedefll{ex}{haskell.def.ex}% context document
\hsfigdefll{exit}{haskell.def.exit}% context figure
\hsfigdefll[assign.avail.1]{\_expr}{haskell.def.:unexpr}% context figure
\hsfigdef[extendAvail]{extendAvail}{haskell.def.extendAvail}% context figure
\omithspagedef{extendVarSet :: VarSet -> LocalVar -> VarSet}{haskell.def.extendVarSet}% context document
\hsfigdef{fact\_add\_to}{haskell.def.fact:unadd:unto}% context figure
\hsfigdef{fact\_bot}{haskell.def.fact:unbot}% context figure
\hsfigdef[FactKont]{FactKont}{haskell.def.FactKont}% context figure
\omithspagedef{fact\_name :: DataflowLattice a -> String}{haskell.def.fact:unname}% context document
\hsprelude{False}{Prelude}% context prelude
\omithspagedef{filterVarsUsed :: UserOfLocalVars e => (LocalVar -> Bool) -> e -> VarSet}{haskell.def.filterVarsUsed}% context document
\hsfigdefll[avail.rewrites.first]{first}{haskell.def.first}% context figure
\hsprelude{flip}{Prelude}% context prelude
\hsprelude{foldl}{Prelude}% context prelude
\hsprelude{foldr}{Prelude}% context prelude
\hspagedef{foldVarsDefd}{haskell.def.foldVarsDefd}% context document
\hspagedef{foldVarsUsed}{haskell.def.foldVarsUsed}% context document
\hsfigdef{ForwardRewrites}{haskell.def.ForwardRewrites}% context figure
\hsfigdef{ForwardTransfers}{haskell.def.ForwardTransfers}% context figure
\hsfigdefll[avail.solve.1]{fp}{haskell.def.fp}% context figure
\hsfigdef{fr\_first}{haskell.def.fr:unfirst}% context figure
\hsfigdef{fr\_last}{haskell.def.fr:unlast}% context figure
\hsfigdef{fr\_middle}{haskell.def.fr:unmiddle}% context figure
\hsprelude{fst}{Prelude}% context prelude
\hsfigdef{ft\_first\_out}{haskell.def.ft:unfirst:unout}% context figure
\hsfigdef{ft\_last\_outs}{haskell.def.ft:unlast:unouts}% context figure
\hsfigdef{ft\_middle\_out}{haskell.def.ft:unmiddle:unout}% context figure
\hspagedef{fuelExhausted}{haskell.def.fuelExhausted}% context document
\hspagedef{FuelMonad}{haskell.def.FuelMonad}% context document
\hspagedef{FwdFixedPoint}{haskell.def.FwdFixedPoint}% context document
\hsfigdef[forward.sol.sig]{fwd\_iter}{haskell.def.fwd:uniter}% context figure
\hspagedef{getAllFacts}{haskell.def.getAllFacts}% context document
\hspagedef{getFact}{haskell.def.getFact}% context document
\hsfigdef{GF}{haskell.def.GF}% context figure
\hspagedef{GlobalVar}{haskell.def.GlobalVar}% context document
\hsfigdef{Graph}{haskell.def.Graph}% context figure
\hspagedef{GraphClosure}{haskell.def.GraphClosure}% context document
\hsfigdef[GraphFactKont]{GraphFactKont}{haskell.def.GraphFactKont}% context figure
\hsfigdef[GraphKont]{GraphKont}{haskell.def.GraphKont}% context figure
\hspagedef{HavingSuccessors}{haskell.def.HavingSuccessors}% context document
\hsprelude{head}{Prelude}% context prelude
\hspagedef{iar\_OC}{haskell.def.iar:unOC}% context document
\hsfigdef[iar.OO]{iar\_OO}{haskell.def.iar:unOO}% context figure
\hsprelude{id}{Prelude}% context prelude
\hsfigdefll[solve.mid.1]{in'}{haskell.def.in'}% context figure
\hsfigdefll[forward.sol.args]{in\_fact}{haskell.def.in:unfact}% context figure
\hsfigdef{insertLateReloads}{haskell.def.insertLateReloads}% context figure
\hsprelude{Int}{Prelude}% context prelude
\hsfigdef{interAvail}{haskell.def.interAvail}% context figure
\omithspagedef{isEmptyVarSet :: VarSet -> Bool}{haskell.def.isEmptyVarSet}% context document
\omithspagedef{isStackSlot :: CmmExpr -> Bool}{haskell.def.isStackSlot}% context document
\omithspagedef{isStackSlotOf :: CmmExpr -> LocalVar -> Bool}{haskell.def.isStackSlotOf}% context document
\hsfigdef[solve.block.sig]{iter\_block}{haskell.def.iter:unblock}% context figure
\hsfigdef[solve.ex.sig]{iter\_ex}{haskell.def.iter:unex}% context figure
\hsfigdef[solve.first.sig]{iter\_first}{haskell.def.iter:unfirst}% context figure
\hsfigdef[solve.last.sig]{iter\_last}{haskell.def.iter:unlast}% context figure
\hsfigdef[solve.mid.1]{iter\_mid}{haskell.def.iter:unmid}% context figure
\hsfigdef[solve.OO.def]{iter\_OO}{haskell.def.iter:unOO}% context figure
\hsfigdefll{join}{haskell.def.join}% context figure
\hsprelude{Just}{Prelude}% context prelude
\hsfigdef[Kont]{Kont}{haskell.def.Kont}% context figure
\hsfigdefll{l}{haskell.def.l}% context figure
\hsfigdefll[avail.rewrites.last]{last}{haskell.def.last}% context figure
\hsfigdef{lastAvail}{haskell.def.lastAvail}% context figure
\hspagedef{LastBranch}{haskell.def.LastBranch}% context document
\hspagedef{LastCall}{haskell.def.LastCall}% context document
\hspagedef{LastCondBranch}{haskell.def.LastCondBranch}% context document
\hsfigdef[lastLiveness]{lastLiveness}{haskell.def.lastLiveness}% context figure
\hsfigdef[lastLiveOut.1]{lastLiveOut}{haskell.def.lastLiveOut}% context figure
\hsfigdef{LastOuts}{haskell.def.LastOuts}% context figure
\hspagedef{LastSwitch}{haskell.def.LastSwitch}% context document
\hsfigdefll[assign.avail.1]{lhs}{haskell.def.lhs}% context figure
\hsprelude{liftM}{Prelude}% context prelude
\hsfigdef[Live]{Live}{haskell.def.Live}% context figure
\hsfigdefll{live}{haskell.def.live}% context figure
\hsfigdef[liveLattice]{liveLattice}{haskell.def.liveLattice}% context figure
\hsfigdef{liveTransfers}{haskell.def.liveTransfers}% context figure
\hspagedef{LocalVar}{haskell.def.LocalVar}% context document
\hsfigdef{LOFsKont}{haskell.def.LOFsKont}% context figure
\hsfigdefll{m}{haskell.def.m}% context figure
\hsprelude{map}{Prelude}% context prelude
\hsprelude{mapM\_}{Prelude}% context prelude
\hsprelude{Maybe}{Prelude}% context prelude
\hsfigdef[maybe.reload.before.1]{maybe\_reload\_before}{haskell.def.maybe:unreload:unbefore}% context figure
\hspagedef{MidAssign}{haskell.def.MidAssign}% context document
\hsfigdefll{middle}{haskell.def.middle}% context figure
\hsfigdef{middleAvail}{haskell.def.middleAvail}% context figure
\hsfigdef[middleLiveness]{middleLiveness}{haskell.def.middleLiveness}% context figure
\hsfigdef{middleRemoveDeads}{haskell.def.middleRemoveDeads}% context figure
\hspagedef{MidStore}{haskell.def.MidStore}% context document
\hspagedef{mkLabel}{haskell.def.mkLabel}% context document
\hspagedef{mkLast}{haskell.def.mkLast}% context document
\hspagedef{mkMiddle}{haskell.def.mkMiddle}% context document
\hsfigdefll[forward.sol.args]{name}{haskell.def.name}% context figure
\hsfigdefll{new}{haskell.def.new}% context figure
\hsfigdef{NoChange}{haskell.def.NoChange}% context figure
\hsfigdefll[maybe.reload.before.1]{node}{haskell.def.node}% context figure
\hsprelude{not}{Prelude}% context prelude
\hsprelude{Nothing}{Prelude}% context prelude
\hsfigdefll[deadRewrites.1]{nothing}{haskell.def.nothing}% context figure
\hsfigdef{O}{haskell.def.O}% context figure
\hsfigdefll{old}{haskell.def.old}% context figure
\hspagedef{PassName}{haskell.def.PassName}% context document
\hsfigdefll[mkMiddle]{rel}{haskell.def.rel}% context figure
\hspagedef{reload}{haskell.def.reload}% context document
\hsfigdef{reloadTail}{haskell.def.reloadTail}% context figure
\hsfigdef[liveness.remDefd.def]{remDefd}{haskell.def.remDefd}% context figure
\hsfigdef{removeDeadAssignments}{haskell.def.removeDeadAssignments}% context figure
\hsprelude{return}{Prelude}% context prelude
\hsfigdef{Rewrite}{haskell.def.Rewrite}% context figure
\hspagedef{RewriteDeep}{haskell.def.RewriteDeep}% context document
\hsfigdefll[forward.sol.args]{rewrites}{haskell.def.rewrites}% context figure
\hspagedef{RewriteShallow}{haskell.def.RewriteShallow}% context document
\hspagedef{RewritingDepth}{haskell.def.RewritingDepth}% context document
\hspagedef{runDFM}{haskell.def.runDFM}% context document
\hspagedef{setAllFacts}{haskell.def.setAllFacts}% context document
\hspagedef{setFact}{haskell.def.setFact}% context document
\hsfigdef[set.last.*]{set\_last}{haskell.def.set:unlast}% context figure
\omithspagedef{sizeVarSet :: VarSet -> Int}{haskell.def.sizeVarSet}% context document
\hsfigdef[smallerAvail]{smallerAvail}{haskell.def.smallerAvail}% context figure
\hsprelude{snd}{Prelude}% context prelude
\hsfigdef{SomeChange}{haskell.def.SomeChange}% context figure
\hsfigdefll[forward.sol.args]{start\_facts}{haskell.def.start:unfacts}% context figure
\hsprelude{String}{Prelude}% context prelude
\hspagedef{subAnalysis}{haskell.def.subAnalysis}% context document
\hspagedef{succs}{haskell.def.succs}% context document
\hsprelude{tail}{Prelude}% context prelude
\hsfigdefll[live.lastSwitch]{tbl}{haskell.def.tbl}% context figure
\hsfigdefll[forward.sol.args]{transfers}{haskell.def.transfers}% context figure
\hsprelude{True}{Prelude}% context prelude
\hsfigdef{TxRes}{haskell.def.TxRes}% context figure
\hsprelude{uncurry}{Prelude}% context prelude
\hsprelude{undefined}{Prelude}% context prelude
\omithspagedef{unionManyVarSets :: [VarSet] -> VarSet}{haskell.def.unionManyVarSets}% context document
\omithspagedef{unionVarSets :: VarSet -> VarSet -> VarSet}{haskell.def.unionVarSets}% context document
\hsfigdef[AvailVars]{UniverseMinus}{haskell.def.UniverseMinus}% context figure
\hsfigdefll{used}{haskell.def.used}% context figure
\hspagedef{useOneFuel}{haskell.def.useOneFuel}% context document
\hspagedef{UserOfLocalVars}{haskell.def.UserOfLocalVars}% context document
\omithspagedef{varOfSlot :: CmmExpr -> LocalVar}{haskell.def.varOfSlot}% context document
\hsfigdefll{vars}{haskell.def.vars}% context figure
\omithspagedef{VarSet\textrm{~(a~type)}}{haskell.def.VarSet}% context document
\omithsfigdef{varSetToList :: VarSet -> [LocalVar]}{haskell.def.varSetToList}% context figure
\hspagedef{withDuplicateFuel}{haskell.def.withDuplicateFuel}% context document
\hspagedef{with\_fuel}{haskell.def.with:unfuel}% context document
\hspagedef{zdfFpContents}{haskell.def.zdfFpContents}% context document
\hspagedef{zdfFpFacts}{haskell.def.zdfFpFacts}% context document
\hspagedef{zdfRewriteBwd}{haskell.def.zdfRewriteBwd}% context document
\hspagedef{zdfRewriteFwd}{haskell.def.zdfRewriteFwd}% context document
\hspagedef{zdfSolveBwd}{haskell.def.zdfSolveBwd}% context document
\hspagedef{zdfSolveFwd}{haskell.def.zdfSolveFwd}% context document
\hspagedef{ZJust}{haskell.def.ZJust}% context document
\hspagedef{ZMaybe}{haskell.def.ZMaybe}% context document
\hspagedef{ZNothing}{haskell.def.ZNothing}% context document
%
\ifundefinedsection.\fi

\undefinedsectionfalse


\renewcommand\hsprelude[2]{\noindent
  \ifundefinedsection
    , \texttt{#1}%
  \else
    \undefinedsectiontrue
    \par
    \section{Identifiers defined in Haskell Prelude}\label{sec:prelude}
    \texttt{#1}%
  \fi
}
\let\hspagedef\dropit
\let\omithspagedef\dropit
\let\omithsfigdef\dropit
\let\hsfigdef\dropit
\let\hstabdef\dropit
\let\hspagedefll\dropit
\let\hsfigdefll\dropit
\let\nothspagedef\dropit
\let\nothsfigdef\dropit
\let\nothslinedef\dropit

\hsprelude{!}{Prelude}% context prelude
\hsprelude{\$}{Prelude}% context prelude
\hsprelude{\&}{Prelude}% context prelude
\hsprelude{\&\&}{Prelude}% context prelude
\hsprelude{*}{Prelude}% context prelude
\hsprelude{+}{Prelude}% context prelude
\hsprelude{++}{Prelude}% context prelude
\hsprelude{-}{Prelude}% context prelude
\hsprelude{.}{Prelude}% context prelude
\hsprelude{/}{Prelude}% context prelude
\hspagedef{<*>}{haskell.def.<*>}% context document
\hsprelude{==}{Prelude}% context prelude
\hsprelude{>}{Prelude}% context prelude
\hsprelude{>=}{Prelude}% context prelude
\hsprelude{>>}{Prelude}% context prelude
\hsprelude{>>=}{Prelude}% context prelude
\hsfigdefll{add}{haskell.def.add}% context figure
\hsfigdef{addUsed}{haskell.def.addUsed}% context figure
\hspagedef{anal\_f\_OO}{haskell.def.anal:unf:unOO}% context document
\hsfigdef[rew.first]{ar\_first}{haskell.def.ar:unfirst}% context figure
\hsfigdef[rew.last]{ar\_last}{haskell.def.ar:unlast}% context figure
\hsfigdef[rew.mid.1]{ar\_mid}{haskell.def.ar:unmid}% context figure
\hsfigdefll[reload1]{avail}{haskell.def.avail}% context figure
\hsfigdef{availRewrites}{haskell.def.availRewrites}% context figure
\hsfigdef[avail.first]{availTransfers}{haskell.def.availTransfers}% context figure
\hsfigdef[AvailVars]{AvailVars}{haskell.def.AvailVars}% context figure
\hsfigdef{availVarsLattice}{haskell.def.availVarsLattice}% context figure
\hsfigdef{BackTransfers}{haskell.def.BackTransfers}% context figure
\hsfigdef{BackwardRewrites}{haskell.def.BackwardRewrites}% context figure
\hspagedef{Block}{haskell.def.Block}% context document
\hspagedef{BlockEnv}{haskell.def.BlockEnv}% context document
\hspagedef{BlockId}{haskell.def.BlockId}% context document
\hsprelude{Bool}{Prelude}% context prelude
\hsfigdef{br\_first}{haskell.def.br:unfirst}% context figure
\hsfigdef{br\_last}{haskell.def.br:unlast}% context figure
\hsfigdef{br\_middle}{haskell.def.br:unmiddle}% context figure
\hsfigdef{bt\_first\_in}{haskell.def.bt:unfirst:unin}% context figure
\hsfigdef{bt\_last\_in}{haskell.def.bt:unlast:unin}% context figure
\hsfigdef{bt\_middle\_in}{haskell.def.bt:unmiddle:unin}% context figure
\hsfigdef{C}{haskell.def.C}% context figure
\omithsfigdef{catMaybes :: [Maybe a] -> [a]}{haskell.def.catMaybes}% context figure
\hsfigdef{ChangeFlag}{haskell.def.ChangeFlag}% context figure
\hspagedef{Cmm}{haskell.def.Cmm}% context document
\hsfigdef{cmmAvailableVars}{haskell.def.cmmAvailableVars}% context figure
\hspagedef{CmmExpr}{haskell.def.CmmExpr}% context document
\hspagedef{CmmGlobal}{haskell.def.CmmGlobal}% context document
\hspagedef{CmmLast}{haskell.def.CmmLast}% context document
\hsfigdef{cmmLiveness}{haskell.def.cmmLiveness}% context figure
\hspagedef{CmmLoad}{haskell.def.CmmLoad}% context document
\hspagedef{CmmLocal}{haskell.def.CmmLocal}% context document
\hspagedef{CmmMiddle}{haskell.def.CmmMiddle}% context document
\hspagedef{CmmVar}{haskell.def.CmmVar}% context document
\hsprelude{const}{Prelude}% context prelude
\hsprelude{curry}{Prelude}% context prelude
\hsprelude{Data.Map}{Prelude}% context prelude
\hsfigdef{DataflowLattice}{haskell.def.DataflowLattice}% context figure
\hsfigdef{deadRewrites}{haskell.def.deadRewrites}% context figure
\hspagedef{DefinerOfLocalVars}{haskell.def.DefinerOfLocalVars}% context document
\hsfigdef{delFromAvail}{haskell.def.delFromAvail}% context figure
\omithspagedef{delFromVarSet :: VarSet -> LocalVar -> VarSet}{haskell.def.delFromVarSet}% context document
\hsfigdefll[forward.sol.args]{depth}{haskell.def.depth}% context figure
\hspagedef{DFM}{haskell.def.DFM}% context document
\hsfigdef{elemAvail}{haskell.def.elemAvail}% context figure
\omithspagedef{elemVarSet :: LocalVar -> VarSet -> Bool}{haskell.def.elemVarSet}% context document
\hsfigdefll{empty}{haskell.def.empty}% context figure
\omithspagedef{emptyBlockEnv :: BlockEnv a}{haskell.def.emptyBlockEnv}% context document
\hspagedef{emptyGraph}{haskell.def.emptyGraph}% context document
\omithspagedef{emptyVarSet :: VarSet}{haskell.def.emptyVarSet}% context document
\hsfigdefll{entry}{haskell.def.entry}% context figure
\hsfigdefll[lastLiveOut.1]{env}{haskell.def.env}% context figure
\hspagedefll{ex}{haskell.def.ex}% context document
\hsfigdefll{exit}{haskell.def.exit}% context figure
\hsfigdefll[assign.avail.1]{\_expr}{haskell.def.:unexpr}% context figure
\hsfigdef[extendAvail]{extendAvail}{haskell.def.extendAvail}% context figure
\omithspagedef{extendVarSet :: VarSet -> LocalVar -> VarSet}{haskell.def.extendVarSet}% context document
\hsfigdef{fact\_add\_to}{haskell.def.fact:unadd:unto}% context figure
\hsfigdef{fact\_bot}{haskell.def.fact:unbot}% context figure
\hsfigdef[FactKont]{FactKont}{haskell.def.FactKont}% context figure
\omithspagedef{fact\_name :: DataflowLattice a -> String}{haskell.def.fact:unname}% context document
\hsprelude{False}{Prelude}% context prelude
\omithspagedef{filterVarsUsed :: UserOfLocalVars e => (LocalVar -> Bool) -> e -> VarSet}{haskell.def.filterVarsUsed}% context document
\hsfigdefll[avail.rewrites.first]{first}{haskell.def.first}% context figure
\hsprelude{flip}{Prelude}% context prelude
\hsprelude{foldl}{Prelude}% context prelude
\hsprelude{foldr}{Prelude}% context prelude
\hspagedef{foldVarsDefd}{haskell.def.foldVarsDefd}% context document
\hspagedef{foldVarsUsed}{haskell.def.foldVarsUsed}% context document
\hsfigdef{ForwardRewrites}{haskell.def.ForwardRewrites}% context figure
\hsfigdef{ForwardTransfers}{haskell.def.ForwardTransfers}% context figure
\hsfigdefll[avail.solve.1]{fp}{haskell.def.fp}% context figure
\hsfigdef{fr\_first}{haskell.def.fr:unfirst}% context figure
\hsfigdef{fr\_last}{haskell.def.fr:unlast}% context figure
\hsfigdef{fr\_middle}{haskell.def.fr:unmiddle}% context figure
\hsprelude{fst}{Prelude}% context prelude
\hsfigdef{ft\_first\_out}{haskell.def.ft:unfirst:unout}% context figure
\hsfigdef{ft\_last\_outs}{haskell.def.ft:unlast:unouts}% context figure
\hsfigdef{ft\_middle\_out}{haskell.def.ft:unmiddle:unout}% context figure
\hspagedef{fuelExhausted}{haskell.def.fuelExhausted}% context document
\hspagedef{FuelMonad}{haskell.def.FuelMonad}% context document
\hspagedef{FwdFixedPoint}{haskell.def.FwdFixedPoint}% context document
\hsfigdef[forward.sol.sig]{fwd\_iter}{haskell.def.fwd:uniter}% context figure
\hspagedef{getAllFacts}{haskell.def.getAllFacts}% context document
\hspagedef{getFact}{haskell.def.getFact}% context document
\hsfigdef{GF}{haskell.def.GF}% context figure
\hspagedef{GlobalVar}{haskell.def.GlobalVar}% context document
\hsfigdef{Graph}{haskell.def.Graph}% context figure
\hspagedef{GraphClosure}{haskell.def.GraphClosure}% context document
\hsfigdef[GraphFactKont]{GraphFactKont}{haskell.def.GraphFactKont}% context figure
\hsfigdef[GraphKont]{GraphKont}{haskell.def.GraphKont}% context figure
\hspagedef{HavingSuccessors}{haskell.def.HavingSuccessors}% context document
\hsprelude{head}{Prelude}% context prelude
\hspagedef{iar\_OC}{haskell.def.iar:unOC}% context document
\hsfigdef[iar.OO]{iar\_OO}{haskell.def.iar:unOO}% context figure
\hsprelude{id}{Prelude}% context prelude
\hsfigdefll[solve.mid.1]{in'}{haskell.def.in'}% context figure
\hsfigdefll[forward.sol.args]{in\_fact}{haskell.def.in:unfact}% context figure
\hsfigdef{insertLateReloads}{haskell.def.insertLateReloads}% context figure
\hsprelude{Int}{Prelude}% context prelude
\hsfigdef{interAvail}{haskell.def.interAvail}% context figure
\omithspagedef{isEmptyVarSet :: VarSet -> Bool}{haskell.def.isEmptyVarSet}% context document
\omithspagedef{isStackSlot :: CmmExpr -> Bool}{haskell.def.isStackSlot}% context document
\omithspagedef{isStackSlotOf :: CmmExpr -> LocalVar -> Bool}{haskell.def.isStackSlotOf}% context document
\hsfigdef[solve.block.sig]{iter\_block}{haskell.def.iter:unblock}% context figure
\hsfigdef[solve.ex.sig]{iter\_ex}{haskell.def.iter:unex}% context figure
\hsfigdef[solve.first.sig]{iter\_first}{haskell.def.iter:unfirst}% context figure
\hsfigdef[solve.last.sig]{iter\_last}{haskell.def.iter:unlast}% context figure
\hsfigdef[solve.mid.1]{iter\_mid}{haskell.def.iter:unmid}% context figure
\hsfigdef[solve.OO.def]{iter\_OO}{haskell.def.iter:unOO}% context figure
\hsfigdefll{join}{haskell.def.join}% context figure
\hsprelude{Just}{Prelude}% context prelude
\hsfigdef[Kont]{Kont}{haskell.def.Kont}% context figure
\hsfigdefll{l}{haskell.def.l}% context figure
\hsfigdefll[avail.rewrites.last]{last}{haskell.def.last}% context figure
\hsfigdef{lastAvail}{haskell.def.lastAvail}% context figure
\hspagedef{LastBranch}{haskell.def.LastBranch}% context document
\hspagedef{LastCall}{haskell.def.LastCall}% context document
\hspagedef{LastCondBranch}{haskell.def.LastCondBranch}% context document
\hsfigdef[lastLiveness]{lastLiveness}{haskell.def.lastLiveness}% context figure
\hsfigdef[lastLiveOut.1]{lastLiveOut}{haskell.def.lastLiveOut}% context figure
\hsfigdef{LastOuts}{haskell.def.LastOuts}% context figure
\hspagedef{LastSwitch}{haskell.def.LastSwitch}% context document
\hsfigdefll[assign.avail.1]{lhs}{haskell.def.lhs}% context figure
\hsprelude{liftM}{Prelude}% context prelude
\hsfigdef[Live]{Live}{haskell.def.Live}% context figure
\hsfigdefll{live}{haskell.def.live}% context figure
\hsfigdef[liveLattice]{liveLattice}{haskell.def.liveLattice}% context figure
\hsfigdef{liveTransfers}{haskell.def.liveTransfers}% context figure
\hspagedef{LocalVar}{haskell.def.LocalVar}% context document
\hsfigdef{LOFsKont}{haskell.def.LOFsKont}% context figure
\hsfigdefll{m}{haskell.def.m}% context figure
\hsprelude{map}{Prelude}% context prelude
\hsprelude{mapM\_}{Prelude}% context prelude
\hsprelude{Maybe}{Prelude}% context prelude
\hsfigdef[maybe.reload.before.1]{maybe\_reload\_before}{haskell.def.maybe:unreload:unbefore}% context figure
\hspagedef{MidAssign}{haskell.def.MidAssign}% context document
\hsfigdefll{middle}{haskell.def.middle}% context figure
\hsfigdef{middleAvail}{haskell.def.middleAvail}% context figure
\hsfigdef[middleLiveness]{middleLiveness}{haskell.def.middleLiveness}% context figure
\hsfigdef{middleRemoveDeads}{haskell.def.middleRemoveDeads}% context figure
\hspagedef{MidStore}{haskell.def.MidStore}% context document
\hspagedef{mkLabel}{haskell.def.mkLabel}% context document
\hspagedef{mkLast}{haskell.def.mkLast}% context document
\hspagedef{mkMiddle}{haskell.def.mkMiddle}% context document
\hsfigdefll[forward.sol.args]{name}{haskell.def.name}% context figure
\hsfigdefll{new}{haskell.def.new}% context figure
\hsfigdef{NoChange}{haskell.def.NoChange}% context figure
\hsfigdefll[maybe.reload.before.1]{node}{haskell.def.node}% context figure
\hsprelude{not}{Prelude}% context prelude
\hsprelude{Nothing}{Prelude}% context prelude
\hsfigdefll[deadRewrites.1]{nothing}{haskell.def.nothing}% context figure
\hsfigdef{O}{haskell.def.O}% context figure
\hsfigdefll{old}{haskell.def.old}% context figure
\hspagedef{PassName}{haskell.def.PassName}% context document
\hsfigdefll[mkMiddle]{rel}{haskell.def.rel}% context figure
\hspagedef{reload}{haskell.def.reload}% context document
\hsfigdef{reloadTail}{haskell.def.reloadTail}% context figure
\hsfigdef[liveness.remDefd.def]{remDefd}{haskell.def.remDefd}% context figure
\hsfigdef{removeDeadAssignments}{haskell.def.removeDeadAssignments}% context figure
\hsprelude{return}{Prelude}% context prelude
\hsfigdef{Rewrite}{haskell.def.Rewrite}% context figure
\hspagedef{RewriteDeep}{haskell.def.RewriteDeep}% context document
\hsfigdefll[forward.sol.args]{rewrites}{haskell.def.rewrites}% context figure
\hspagedef{RewriteShallow}{haskell.def.RewriteShallow}% context document
\hspagedef{RewritingDepth}{haskell.def.RewritingDepth}% context document
\hspagedef{runDFM}{haskell.def.runDFM}% context document
\hspagedef{setAllFacts}{haskell.def.setAllFacts}% context document
\hspagedef{setFact}{haskell.def.setFact}% context document
\hsfigdef[set.last.*]{set\_last}{haskell.def.set:unlast}% context figure
\omithspagedef{sizeVarSet :: VarSet -> Int}{haskell.def.sizeVarSet}% context document
\hsfigdef[smallerAvail]{smallerAvail}{haskell.def.smallerAvail}% context figure
\hsprelude{snd}{Prelude}% context prelude
\hsfigdef{SomeChange}{haskell.def.SomeChange}% context figure
\hsfigdefll[forward.sol.args]{start\_facts}{haskell.def.start:unfacts}% context figure
\hsprelude{String}{Prelude}% context prelude
\hspagedef{subAnalysis}{haskell.def.subAnalysis}% context document
\hspagedef{succs}{haskell.def.succs}% context document
\hsprelude{tail}{Prelude}% context prelude
\hsfigdefll[live.lastSwitch]{tbl}{haskell.def.tbl}% context figure
\hsfigdefll[forward.sol.args]{transfers}{haskell.def.transfers}% context figure
\hsprelude{True}{Prelude}% context prelude
\hsfigdef{TxRes}{haskell.def.TxRes}% context figure
\hsprelude{uncurry}{Prelude}% context prelude
\hsprelude{undefined}{Prelude}% context prelude
\omithspagedef{unionManyVarSets :: [VarSet] -> VarSet}{haskell.def.unionManyVarSets}% context document
\omithspagedef{unionVarSets :: VarSet -> VarSet -> VarSet}{haskell.def.unionVarSets}% context document
\hsfigdef[AvailVars]{UniverseMinus}{haskell.def.UniverseMinus}% context figure
\hsfigdefll{used}{haskell.def.used}% context figure
\hspagedef{useOneFuel}{haskell.def.useOneFuel}% context document
\hspagedef{UserOfLocalVars}{haskell.def.UserOfLocalVars}% context document
\omithspagedef{varOfSlot :: CmmExpr -> LocalVar}{haskell.def.varOfSlot}% context document
\hsfigdefll{vars}{haskell.def.vars}% context figure
\omithspagedef{VarSet\textrm{~(a~type)}}{haskell.def.VarSet}% context document
\omithsfigdef{varSetToList :: VarSet -> [LocalVar]}{haskell.def.varSetToList}% context figure
\hspagedef{withDuplicateFuel}{haskell.def.withDuplicateFuel}% context document
\hspagedef{with\_fuel}{haskell.def.with:unfuel}% context document
\hspagedef{zdfFpContents}{haskell.def.zdfFpContents}% context document
\hspagedef{zdfFpFacts}{haskell.def.zdfFpFacts}% context document
\hspagedef{zdfRewriteBwd}{haskell.def.zdfRewriteBwd}% context document
\hspagedef{zdfRewriteFwd}{haskell.def.zdfRewriteFwd}% context document
\hspagedef{zdfSolveBwd}{haskell.def.zdfSolveBwd}% context document
\hspagedef{zdfSolveFwd}{haskell.def.zdfSolveFwd}% context document
\hspagedef{ZJust}{haskell.def.ZJust}% context document
\hspagedef{ZMaybe}{haskell.def.ZMaybe}% context document
\hspagedef{ZNothing}{haskell.def.ZNothing}% context document

\ifundefinedsection.\fi



\endgroup


\iffalse

\section{Dataflow-engine functions}


\begin{figure*}
\setcounter{codeline}{0}
\begin{numberedcode}
\end{numberedcode}
\caption{The forward iterator}
\end{figure*}

\begin{figure*}
\setcounter{codeline}{0}
\begin{numberedcode}
\end{numberedcode}
\caption{The forward actualizer}
\end{figure*}


\fi



\end{document}



% Old captions' text:
% The dataflow fact for the available-reload analysis describes
%   the set of registers for which a reload is available.
%   We list the types of the functions that manipuate sets of available registers,
%   as well as the definition of the lattice.
% The standard gen and kill functions for available expressions
% The transfer functions for the available-reloads analysis.
% Running the available-reloads analysis and extracting the results with \texttt{zdfFpFacts}
% The rewrite functions to insert redundant reloads immediately before uses

% Probably no space for the implementations:
% interAvail (UniverseMinus s) (UniverseMinus s') =
%   UniverseMinus (s `plusVarSet`  s')
% interAvail (AvailVars     s) (AvailVars     s') =
%   AvailVars (s `timesVarSet` s')
% interAvail (AvailVars     s) (UniverseMinus s') =
%   AvailVars (s  `minusVarSet` s')
% interAvail (UniverseMinus s) (AvailVars     s') =
%   AvailVars (s' `minusVarSet` s )
% 
% smallerAvail (AvailVars _) (UniverseMinus _) = True
% smallerAvail (UniverseMinus _) (AvailVars _) = False
% smallerAvail (AvailVars     s) (AvailVars    s')  =
%   sizeVarSet s < sizeVarSet s'
% smallerAvail (UniverseMinus s) (UniverseMinus s') =
%   sizeVarSet s > sizeVarSet s'
% 
% extendAvail (UniverseMinus s) r =
%   UniverseMinus (deleteFromVarSet s r)
% extendAvail (AvailVars     s) r =
%   AvailVars (extendVarSet s r)
% 
% delFromAvail (UniverseMinus s) r =
%   UniverseMinus (extendVarSet s r)
% delFromAvail (AvailVars     s) r =
%   AvailVars (deleteFromVarSet s r)
% 
% elemAvail (UniverseMinus s) r =
%   not $ elemVarSet r s
% elemAvail (AvailVars     s) r =
%   elemVarSet r s



THE FUEL PROBLEM:


Here is the problem:

  A graph has an entry sequence, a body, and an exit sequence.
  Correctly computing facts on and flowing out of the body requires
  iteration; computation on the entry and exit sequences do not, since
  each is connected to the body by exactly one flow edge.

  The problem is to provide the correct fuel supply to the combined
  analysis/rewrite (iterator) functions, so that speculative rewriting
  is limited by the fuel supply.

  I will number iterations from 1 and name the fuel supplies as
  follows:

     f_pre      fuel remaining before analysis/rewriting starts
     f_0        fuel remaining after analysis/rewriting of the entry sequence
     f_i, i>0   fuel remaining after iteration i of the body
     f_post     fuel remaining after analysis/rewriting of the exit sequence

  The issue here is that only the last iteration of the body 'counts'.
  To formalize, I will name fuel consumed:

     C_pre      fuel consumed by speculative rewrites in entry sequence
     C_i        fuel consumed by speculative rewrites in iteration i of body
     C_post     fuel consumed by speculative rewrites in exit sequence

  These quantities should be related as follows:

     f_0    = f_pre - C_pref
     f_i    = f_0 - C_i            where i > 0
     f_post = f_n - C_post         where iteration converges after n steps

When the fuel supply is passed explicitly as parameter and result, it
is fairly easy to see how to keep reusing f_0 at every iteration, then
extract f_n for use before the exit sequence.  It is not obvious to me
how to do it cleanly using the fuel monad.


Norman


\newif\ifpagetuning \pagetuningtrue  % adjust page breaks

\newif\ifnoauthornotes \noauthornotesfalse

\newif\iftimestamp\timestamptrue  % show MD5 stamp of paper

% \timestampfalse % it's post-submission time

\IfFileExists{timestamp.tex}{}{\timestampfalse}

\newif\ifnotcutting\notcuttingfalse


\newif\ifgenkill\genkillfalse  % have a section on gen and kill
\genkillfalse


\newif\ifnotesinmargin \notesinmarginfalse
\IfFileExists{notesinmargin.tex}{\notesinmargintrue}{\relax}

\documentclass[blockstyle,natbib,preprint]{sigplanconf}

\newcommand\ourlib{Hoopl}
   % higher-order optimization library
   % ('Hoople' was taken -- see hoople.org)
\let\hoopl\ourlib

\newcommand\ag{\ensuremath{\mathit{ag}}}
\renewcommand\ag{\ensuremath{g}}  % not really seeing that 'ag' is helpful here ---NR
\newcommand\rw{\ensuremath{\mathit{rw}}}

% l2h substitution ourlib Hoopl
% l2h substitution hoopl Hoopl

\newcommand\fs{\ensuremath{\mathit{fs}}} % dataflow facts, possibly plural

\newcommand\vfilbreak[1][\baselineskip]{%
  \vskip 0pt plus #1 \penalty -200 \vskip 0pt plus -#1 }

\usepackage{alltt}
\usepackage{array}
\usepackage{afterpage}
\newcommand\lbr{\char`\{}
\newcommand\rbr{\char`\}}
 
\clubpenalty=10000
\widowpenalty=10000

\usepackage{verbatim} % allows to define \begin{smallcode}
\newenvironment{smallcode}{\par\unskip\small\verbatim}{\endverbatim}
\newenvironment{fuzzcode}[1]{\par\unskip\hfuzz=#1 \verbatim}{\endverbatim}
\newenvironment{smallfuzzcode}[1]{\par\unskip\small\hfuzz=#1 \verbatim}{\endverbatim}
\newenvironment{smallttcode}{\par\unskip\small\alltt}{\endalltt}

\newcommand\smallverbatiminput[1]{{\small\verbatiminput{#1}}}
\newcommand\smallfuzzverbatiminput[2]{{\small\hfuzz=#1 \verbatiminput{#2}}}

\newcommand\lineref[1]{line~\ref{line:#1}}
\newcommand\linepairref[2]{lines \ref{line:#1}~and~\ref{line:#2}}
\newcommand\linerangeref[2]{\mbox{lines~\ref{line:#1}--\ref{line:#2}}}
\newcommand\Lineref[1]{Line~\ref{line:#1}}
\newcommand\Linepairref[2]{Lines \ref{line:#1}~and~\ref{line:#2}}
\newcommand\Linerangeref[2]{\mbox{Lines~\ref{line:#1}--\ref{line:#2}}}

\makeatletter

\def \@copyrightspace {%
 \@float{copyrightbox}[b]%
 \vbox to 0.65in{%  less than 1in, please
   \vfill
   \parbox[b]{20pc}{%
     \scriptsize
     \if \@preprint
       [Copyright notice will appear here
        once 'preprint' option is removed.]\par
     \else
       \@toappear
     \fi
     \if \@reprint
       \noindent Reprinted from \@conferencename,
       \@proceedings,
       \@conferenceinfo,
       pp.~\number\thepage--\pageref{sigplanconf@finalpage}.\par
     \fi}}%
 \end@float}


\let\c@table=
           \c@figure % one counter for tables and figures, please


\newcommand\setlabel[1]{%
  \setlabel@#1!!\@endsetlabel
}
\def\setlabel@#1!#2!#3\@endsetlabel{%
  \ifx*#1*% line begins with label or is empty
     \ifx*#2*% line is empty
        \verbatim@line{}%
     \else
       \@stripbangs#3\@endsetlabel%
       \label{line:#2}%
     \fi
  \else
     \@stripbangs#1!#2!#3\@endsetlabel%
  \fi
}
\def\@stripbangs#1!!\@endsetlabel{%
  \verbatim@line{#1}%
}


\verbatim@line{hello mama}

\newcommand{\numberedcodebackspace}{0.5\baselineskip}

\newcounter{codeline}
\newenvironment{numberedcode}
  {\endgraf
     \def\verbatim@processline{%
        \noindent
        \expandafter\ifx\expandafter+\the\verbatim@line+  % blank line
               %{\small\textit{\def\rmdefault{cmr}\rmfamily\phantom{00}\phantom{: \,}}}%
            \else
               \refstepcounter{codeline}%
               {\small\textit{\def\rmdefault{cmr}\rmfamily\phantom{00}\llap{\arabic{codeline}}: \,}}%
            \fi
        \expandafter\setlabel\expandafter{\the\verbatim@line}%
        \expandafter\ifx\expandafter+\the\verbatim@line+  % blank line
          \vspace*{-\numberedcodebackspace}\par%
        \else
          \the\verbatim@line\par
        \fi}%
   \verbatim
   }
   {\endverbatim}

\makeatother

\newcommand\arrow{\rightarrow}

\newcommand\join{\sqcup}
\newcommand\slotof[1]{\ensuremath{s_{#1}}}
\newcommand\tempof[1]{\ensuremath{t_{#1}}}
\let\tempOf=\tempof
\let\slotOf=\slotof

\makeatletter
\newcommand{\nrmono}[1]{%
  {\@tempdima = \fontdimen2\font\relax
   \texttt{\spaceskip = 1.1\@tempdima #1}}}
\makeatother

\usepackage{times}  % denser fonts
\renewcommand{\ttdefault}{aett} % \texttt that goes better with times fonts
\usepackage{enumerate}
\usepackage{url}
\usepackage{graphicx}
\usepackage{natbib}  % redundant for Simon
\bibpunct();A{},
\let\cite\citep
\let\citeyearnopar=\citeyear
\let\citeyear=\citeyearpar

\usepackage[ps2pdf,bookmarksopen,breaklinks,pdftitle=Hoopl]{hyperref}
\usepackage{breakurl} % enables \burl

\newcommand\naive{na\"\i ve}
\newcommand\naively{na\"\i vely}
\newcommand\Naive{Na\"\i ve}

\usepackage{amsfonts}
\newcommand\naturals{\ensuremath{\mathbb{N}}}
\newcommand\true{\ensuremath{\mathbf{true}}}
\newcommand\implies{\supseteq}  % could use \Rightarrow?

\newcommand\PAL{\mbox{C{\texttt{-{}-}}}}
\newcommand\high[1]{\mbox{\fboxsep=1pt \smash{\fbox{\vrule height 6pt
   depth 0pt width 0pt \leavevmode \kern 1pt #1}}}}

\usepackage{tabularx}

%%
%% 2009/05/10: removed 'float' package because it breaks multiple
%% \caption's per {figure} environment.   ---NR
%%
%%  % Put figures in boxes --- WHY??? --NR
%%  \usepackage{float}
%%  \floatstyle{boxed}
%%  \restylefloat{figure}
%%  \restylefloat{table}



% ON LINE THREE, set \noauthornotestrue to suppress notes (or not)

%\newcommand{\qed}{QED}
\ifnotesinmargin
  \long\def\authornote#1{%
      \ifvmode
         \marginpar{\raggedright\hbadness=10000
         \parindent=8pt \parskip=2pt
         \def\baselinestretch{0.8}\tiny
         \itshape\noindent #1\par}%
      \else
          \unskip\raisebox{-3.5pt}{\rlap{$\scriptstyle\diamond$}}%
          \marginpar{\raggedright\hbadness=10000
         \parindent=8pt \parskip=2pt
         \def\baselinestretch{0.8}\tiny
         \itshape\noindent #1\par}%
      \fi}
\else
  % Simon: please set \notesinmarginfalse on the first line
  \long\def\authornote#1{{\em #1\/}}
\fi
\ifnoauthornotes
  \def\authornote#1{\unskip\relax}
\fi

\long\def\newtext#1{{\mbox{$\clubsuit$}}{\slshape\ignorespaces#1}{\mbox{$\clubsuit$}}}
\newenvironment{ntext}{\mbox{$\clubsuit$}\slshape\ignorespaces}{\mbox{$\clubsuit$}}

\newcommand{\simon}[1]{\authornote{SLPJ: #1}}
\newcommand{\norman}[1]{\authornote{NR: #1}}
\let\remark\norman
\def\finalremark#1{\relax}
\let \finalremark \remark % uncomment after submission
\newcommand{\john}[1]{\authornote{JD: #1}}
\newcommand{\todo}[1]{\textbf{To~do:} \emph{#1}}
\newcommand\delendum[1]{\relax\ifvmode\else\unskip\fi\relax}

\newcommand\secref[1]{Section~\ref{sec:#1}}
\newcommand\secreftwo[2]{Sections \ref{sec:#1}~and~\ref{sec:#2}}
\newcommand\seclabel[1]{\label{sec:#1}}

\newcommand\figref[1]{Figure~\ref{fig:#1}}
\newcommand\figreftwo[2]{Figures \ref{fig:#1}~and~\ref{fig:#2}}
\newcommand\figlabel[1]{\label{fig:#1}}

\newcommand\tabref[1]{Table~\ref{tab:#1}}
\newcommand\tablabel[1]{\label{tab:#1}}


\newcommand{\CPS}{\textbf{StkMan}}    % Not sure what to call it.


\usepackage{code}   % At-sign notation

\IfFileExists{timestamp.tex}{\input{timestamp}}{}
\iftimestamp
\preprintfooter{\mdfivestamp}
\fi

\hyphenation{there-by}

\renewcommand{\floatpagefraction}{0.9} % must be less than \topfraction
\renewcommand{\topfraction}{0.95}
\renewcommand{\textfraction}{0.05}

\begin{document}
%\title{\ourlib: Dataflow Optimization Made Simple}
\title{\ourlib: A Modular, Reusable Library for\\ Dataflow Analysis and Transformation}
%\title{Implementing Dataflow Analysis and Optimization by Lifting Node Functions to Basic Blocks and Control-Flow Graphs}
%\subtitle{Programming pearl}

\ifbanner
\titlebanner{\textsf{\mdseries\itshape
Reprinted from the 2010 ACM Haskell Symposium.
}}
\fi

\conferenceinfo{Haskell'10,} {September 30, 2010, Baltimore, Maryland, USA.}
\CopyrightYear{2010}
\copyrightdata{978-1-4503-0252-4/10/09}

\ifnoauthornotes
\makeatletter
\let\HyPsd@Warning=
                \@gobble
\makeatother
\fi

% Jo�o


\authorinfo{Norman Ramsey}{Tufts University}{nr@cs.tufts.edu}
\authorinfo{Jo\~ao Dias}{Tufts University}{dias@cs.tufts.edu}
\authorinfo{Simon Peyton Jones}{Microsoft Research}{simonpj@microsoft.com}


\maketitle
 
\begin{abstract}
\iffalse % A vote for Simon's abstract
\remark{I have replaced a good abstract of the POPL submission with a
bad abstract of \emph{this} submission.}
We present \ourlib, a Haskell library that makes it easy for a
compiler writer
to implement program transformations based on dataflow analyses.
A~client of \ourlib\ defines a representation of 
logical assertions,
a transfer function that computes outgoing assertions from incoming
assertions, 
and a rewrite function that improves code when improvements are
justified by the assertions.
\ourlib\ does the actual analysis and transformation.

\ourlib\ implements state-of-the art algorithms:
Lerner, Grove, and Chambers's 
\citeyearpar{lerner-grove-chambers:2002}
composition of simple analyses and transformations, which achieves
the same precision as complex, handwritten
``super-analyses;''
and Whalley's \citeyearpar{whalley:isolation} dynamic technique for
isolating bugs in a client's code.
\ourlib's implementation is unique in that unlike previous
implementations,
it carefully separates the tricky
elements of each of these algorithms, so that they can be examined and
understood independently.


\simon{Here is an alternative abstract based on the four-sentence model.}
\remark{Four-sentence model?  You must teach me\ldots}
\fi
Dataflow analysis and transformation of control-flow graphs is
pervasive in optimizing compilers, but it is typically tightly
interwoven with the details of a \emph{particular} compiler.  
We~describe \ourlib{}, a~reusable Haskell library that makes it
unusually easy to define new analyses and
transformations for \emph{any} compiler written in Haskell.
\ourlib's interface is modular and polymorphic,
and it offers unusually strong static guarantees.
The implementation encapsulates 
state-of-the-art algorithms (interleaved analysis and rewriting,
dynamic error isolation), and it cleanly separates their tricky elements
so that they can be understood independently.
%
%\ourlib\ will be the workhorse of a new
%back end for the Glasgow Haskell Compiler (version~6.14, forthcoming).

\emph{Readers:} code examples are indexed at 
\iffalse % one more line on page 1 blows up the paper...
{\url{http://bit.ly/doUJpr}}
\else
{\url{http://www.cs.tufts.edu/~nr/pubs/hoopl10-supplement.pdf}} 
(\url{bit.ly/doUJpr}).
\fi
%%% Source: http://www.cs.tufts.edu/~nr/drop/popl-index.pdf
\end{abstract}

\makeatactive   %  Enable @foo@ notation

\section{Introduction}

A mature optimizing compiler for an imperative language includes many
analyses, the results of which justify the optimizer's
code-improving transformations.
Many analyses and transformations---constant
propagation, live-variable analysis, inlining, sinking of loads, 
and so on---should be regarded as particular cases of
a single general problem: \emph{dataflow analysis and optimization}.
%% \remark{I do not feel compelled to cite Muchnick (or anyone else) here}
Dataflow analysis is over thirty years old,
but a recent, seminal paper by \citet{lerner-grove-chambers:2002} goes further, 
describing a powerful but subtle way to
\emph{interleave} analysis and transformation so that each 
piggybacks on the other.

Because optimizations based on dataflow analysis 
share a common intellectual framework, and because that framework is
subtle, it~is tempting to
try to build a single reusable library that embodies the 
subtle ideas, while
making it easy for clients to instantiate the library for different
situations.
% Tempting, but difficult.
Although such libraries exist, as we discuss 
in \secref{related}, they have complex APIs and implementations,
and none implements the Lerner/Grove/Chambers technique.

In this paper we present \ourlib{} (short for ``higher-order
optimization library''), a new Haskell library for dataflow analysis and
optimization.  It has the following distinctive characteristics:

\begin{itemize}
\item
\ourlib\ is purely functional.  
Although pure functional languages are not obviously suited to writing
standard algorithms that
transform control-flow graphs,
pure functional code is actually easier to write, and far easier
to write correctly, than code that is mostly functional but uses a mutable
representation of graphs
\cite{ramsey-dias:applicative-flow-graph}.
When analysis and transformation
are interleaved, so that graphs must be transformed \emph{speculatively},
without knowing whether
a~transformed graph will be retained or discarded,
pure functional code offers even more benefits.

\item
\ourlib\ is polymorphic. Just as a list library is
polymorphic in the list elements, so is \ourlib{} polymorphic, both in
the nodes that inhabit graphs and in the dataflow facts that 
analyses compute over these graphs (\secref{using-hoopl}).

\item
The paper by Lerner, Grove, and Chambers is inspiring but abstract.
We articulate their ideas in a concrete, simple API,
which hides 
a subtle implementation
(\secreftwo{graph-rep}{using-hoopl}).  
You provide a representation for assertions, 
a transfer function that transforms assertions across a node, 
and a rewrite function that uses an assertion to 
justify rewriting a node.
\ourlib\ ``lifts'' these node-level functions to work over
control-flow graphs, solves recursion equations,
and interleaves rewriting with analysis.
Designing APIs is surprisingly
hard; we have been through over a dozen significantly different
iterations, and we offer our API as a contribution.

\item
Because the client
can perform very local reasoning (``@y@ is live before
@x:=y+2@''),
\seclabel{liveness-mention}
 analyses and transformations built on \ourlib\ 
are small, simple, and easy to get right.
\seclabel{liveness-mention}
Moreover, \ourlib\ helps you write correct optimizations:
it~statically rules out transformations that violate invariants
of the control-flow graph (Sections \ref{sec:graph-rep} and \ref{sec:rewrites}),
and dynamically it can help find the first transformation that introduces a fault
in a test program (\secref{fuel}). 
%%\finalremark{SLPJ: I wanted to write more about open/closed,
%%but I like this sentence with its claim to both static and dynamic assistance,
%%and maybe the open/closed story is hard to understand here.}

% \item \ourlib{} makes use of GADTS and type functions to offer unusually
% strong static guarantees. In particular, nodes, basic blocks, and
% graphs are all statically typed by their open or closedness on entry, and
% their open or closedness on exit (\secref{graph-rep}). For example, an add instruction is
% open on entry and exit, while a branch is open on entry and closed on exit.
% Using these types we can statically guarantee that, say, an add instruction
% is rewritten to a graph that is also open at both entry and exit; and 
% that the user cannot construct a block where an add instruction follows an
% unconditional branch.  We know of no other system that offers 
% static guarantees this strong.

\item \ourlib{} implements subtle algorithms, including 
(a)~interleaved analysis and rewriting, (b)~speculative rewriting,
(c)~computing fixed points, and (d)~dynamic fault isolation.
Previous implementations of these algorithms---including three of our
own---are complicated and hard to understand, because the tricky pieces
are implemented all together, inseparably. 
In~this paper, each tricky piece is handled in just 
one place, separate from all the others (\secref{engine}). 
We~emphasize this implementation as an object of interest in
its own right.
\end{itemize}
Our work bridges the gap between abstract, theoretical presentations
and actual compilers. 
\ourlib{} is available from
\burl{http://ghc.cs.tufts.edu/hoopl} and also from Hackage (version~3.8.3.0).
One of \hoopl's clients
is the Glasgow Haskell Compiler, which uses \hoopl\ to optimize 
imperative code in GHC's back end.

\ourlib's API is made possible by
sophisticated aspects of Haskell's type system, such
as higher-rank polymorphism, GADTs, and type functions.
\ourlib{} may therefore also serve as a case study in the utility
of these features.


% \clearpage % Start section 2 on page 2

\section{Dataflow analysis {\&} transformation by \texorpdfstring{\rlap{example}}{example}}
\seclabel{overview}
\seclabel{constant-propagation}
\seclabel{example:transforms}
\seclabel{example:xforms}

%%  We begin by setting the scene, introducing some vocabulary, and
%%  showing a small motivating example.
A \emph{control-flow graph}, perhaps representing the body of a procedure,
is a collection of \emph{basic blocks}---or just ``blocks.''
Each block is a sequence of instructions,
beginning with a label and ending with a
control-transfer instruction that branches to other blocks.
% Each block has a label at the beginning,
% a sequence of 
%  -- each of which has a label at the 
% beginning.  Each block may branch to other blocks with arbitrarily
% complex control flow.
The goal of dataflow optimization is to compute valid
\emph{assertions} (or \emph{dataflow facts}), 
then use those assertions to justify code-improving
transformations (or \emph{rewrites}) on a {control-flow graph}.  

As~a concrete example, we show constant propagation with constant
folding. 
On the left we show a basic block; in the middle we show
facts that hold between statements (or \emph{nodes}) 
in the block; and on
the right we show the result of transforming the block 
based on the facts:
\begin{verbatim}
      Before        Facts        After
          ------------{}-------------
      x := 3+4                   x := 7
          ----------{x=7}------------
      z := x>5                   z := True
          -------{x=7, z=True}-------
      if z                       goto L1
       then goto L1
       else goto L2
\end{verbatim}
Constant propagation works
from top to bottom.  We start with the empty fact.  
Given the empty fact and the node @x:=3+4@ can we make a (constant-folding)
transformation?
Yes: we can replace the node with @x:=7@.
Now, given this transformed node
and the original fact, what fact flows out of the bottom of
the transformed node?  
The~fact~\mbox{\{@x=7@\}}.  
Given the fact \{@x=7@\} and the node @z:=x>5@, can we make a
transformation?  Yes: constant propagation can replace the node with @z:=7>5@.
Now, can we make another transformation?  Yes: constant folding can 
replace the node with @z:=True@.
The~process continues to the end of the block, where we
can replace the conditional branch with an unconditional one, @goto L1@.

The example above is simple because the program has only straight-line code;
when programs have loops, dataflow analysis gets more complicated.
For example, 
consider the following graph,
where we assume @L1@ is the entry point:
\begin{verbatim}
  L1: x=3; y=4; if z then goto L2 else goto L3
  L2: x=7; goto L3
  L3: ...
\end{verbatim}
Because control flows to @L3@ from two places (@L1@~and~@L2@),
we must \emph{join} the facts coming from those two places.
All paths to @L3@ produce the fact @y=@$4$,
so we can conclude that this fact holds at~@L3@.
But depending on the the path to~@L3@, @x@~may have different
values, so we conclude ``@x=@$\top$'',
meaning that there is no single value held by~@x@ at~@L3@.
%\footnote{
%In this example @x@ really does vary at @L3@, but in general
%the analysis might be conservative.}
The final result of joining the dataflow facts that flow to~@L3@
is the new fact $\mathord{@x=@\top\!} \land \mathord{@y=4@} \land \mathord{@z=@\top}$.

\seclabel{const-prop-example}

\paragraph{Forwards and backwards.}
Constant propagation works \emph{forwards}, and a fact is often an
assertion about the program state, such as ``variable~@x@ holds value~@7@.''
Some useful analyses work \emph{backwards}.
A~prime example is live-variable analysis, where a fact takes the~form
``variable @x@ is live'' and is an assertion about the
\emph{continuation} of a program point.  For example, the fact ``@x@~is
live'' at a program point P is an assertion that @x@ is used on some program
path starting at \mbox{P}.  % TeXbook, exercise 12.6
The accompanying transformation is called dead-code elimination;
if @x@~is not live, this transformation 
replaces the node @x:=e@ with a no-op.

\seclabel{simple-tx}
\paragraph{Interleaved transformation and analysis.}
Our first example \emph{interleaves} analysis and transformation.
% constant propagation is a transformation, not an analysis
Interleaving makes it far easier to write effective analyses.
If, instead, we \emph{first} analyzed the block
and \emph{then} transformed it, the analysis would have to ``predict''
the transformations.
For example, given the incoming fact \{@x=7@\}
and the instruction @z:=x>5@,
a pure analysis could produce the outgoing fact
\{@x=7@, @z=True@\} by simplifying @x>5@ to @True@.
But the subsequent transformation must perform
\emph{exactly the same simplification} when it transforms the instruction to @z:=True@!
If instead we \emph{first} rewrite the node to @z:=True@, 
and \emph{then} apply the transfer function to the new node, 
the transfer function becomes wonderfully simple: it merely has to see if the
right hand side is a constant.
You~can see code in \secref{const-prop-client}.

Another example is the interleaving of liveness analysis and dead-code elimination.
As mentioned in \secref{liveness-mention},
it is sufficient for the analysis to~say ``@y@ is live before @x:=y+2@''.
It is not necessary to have the more complex rule
``if @x@~is live after @x:=y+2@ then @y@ is live before~it,''
because if @x@~is \emph{not} live after @x:=y+2@,
the assignment @x:=y+2@ will be eliminated.
If~there are a number of interacting 
analyses and/or transformations,
the benefit of interleaving them is even more compelling; for more
substantial examples, consult \citet{lerner-grove-chambers:2002}.
%
But these compelling benefits come at a cost.
To~compute valid facts for a program that has loops,
an analysis may require multiple iterations over the program.
Before the final iteration, the analysis may
compute a~fact that is invalid,
and a transformation may use the invalid fact to rewrite the program
(\secref{ckpoint-monad}).
To~avoid unjustified rewrites,
any rewrite based on an invalid fact must be rolled back;
transformations must be \emph{speculative}.
As~described in \secref{checkpoint-monad},
\hoopl~manages speculation with cooperation from the client.


\seclabel{debugging-introduced}

While it is wonderful that we can implement complex optimizations by
composing very simple analyses and transformations,
it~is not so wonderful that very simple analyses and transformations,
when composed, can exhibit complex emergent behavior.
Because such behavior is not easily predicted, it is essential to have good
tools for debugging.
\hoopl's primary debugging tool is an implementation of
Whalley's \citeyearpar{whalley:isolation} search technique for
isolating faults in an optimizer (\secref{whalley-from-s2}).


\section{Representing control-flow graphs} \seclabel{graph-rep}

\ourlib{} is a library that makes it easy to define dataflow analyses,
and transformations driven by these analyses,
on control-flow graphs.
Graphs are composed from smaller units, which we discuss from the
bottom up:
\begin{itemize}
\item A \emph{node} is defined by \ourlib's client;
\ourlib{} knows nothing about the representation of nodes (\secref{nodes}).
\item A basic \emph{block} is a sequence of nodes (\secref{blocks}).
\item A \emph{graph} is an arbitrarily complicated control-flow graph:
basic blocks connected by edges (\secref{graphs}).
\end{itemize}

\subsection{Shapes: Open and closed}

Nodes, blocks, and graphs share important properties in common.
In particular, each is \emph{open or closed on entry}
and \emph{open or closed on exit}.  
An \emph{open} point is one at which control may implicitly ``fall through;''
to transfer control at a \emph{closed} point requires an explicit
control-transfer instruction to a named label.
For example,
\begin{itemize}
\item A shift-left instruction is open on entry (because control can fall into it
from the preceding instruction), and open on exit (because control falls through
to the next instruction).
\item An unconditional branch is open on entry, but closed on exit (because 
control cannot fall through to the next instruction).
\item A label is closed on entry (because in \ourlib{} we do not allow
control to fall through into a branch target), but open on exit.
\item
A~function call could, if the language being compiled is simple
 enough, be open on entry and  exit.
But if a function could return in multiple ways---for example by
 transferring either to a normal return continuation or to an
 exception handler---then a function call will have to be represented
 by a node that is closed at exit.
\end{itemize}
% This taxonomy enables \ourlib{} to enforce invariants:
% only nodes closed on entry can be the targets of branches, and only nodes closed
% on exits can transfer control (see also \secref{nonlocal-class}).
% As~a consequence, all control transfers originate at control-transfer
% instructions and terminated at labels; this invariant dramatically
% simplifies analysis and transformation. 
These examples concern nodes, but the same classification applies
to blocks and graphs.  For example the block
\begin{code}
   x:=7; y:=x+2; goto L
\end{code}
is open on entry and closed on exit.  
This is the block's \emph{shape}, which we often abbreviate
``open/closed;''
we may also refer to an ``open/closed block.''

The shape of a thing determines that thing's control-flow properties.
In particular, whenever E is a node, block, or graph,
% : \simon{Removed the claim about a unique entry point.}
\begin{itemize}
\item
If E is open on entry, it has a unique predecessor; 
if it is closed, it may have arbitrarily many predecessors---or none.
\item
If E is open on exit, it has a unique successor; 
if it is closed, it may have arbitrarily many successors---or none.
\end{itemize}
%%%%    
%%%%    
%%%%    % \item Regardless of whether E is open or closed, 
%%%%    % it has a unique entry point where execution begins.
%%%%    \item If E is closed on exit, control leaves \emph{only}
%%%%    by explicit branches from closed-on-exit nodes.
%%%%    \item If E is open on exit, control \emph{may} leave E
%%%%    by ``falling through'' from a distinguished exit point.
%%%%    \remark{If E is a node or block, control \emph{only} leaves E by
%%%%    falling through, but this isn't so for a graph.  Example: a body of a
%%%%    loop contains a \texttt{break} statement} \simon{I don't understand.
%%%%    A break statement would have to be translated as a branch, no?
%%%%    Can you give a an example? I claim that control only leaves an 
%%%%    open graph by falling through.}
%%%%    \end{itemize}


\subsection{Nodes} \seclabel{nodes}

The primitive constituents of a control-flow graph are
\emph{nodes}.
% , which are defined by the client.     SLPJ: said again very shortly
For~example, a node might represent a machine instruction, such as an
assignment, a call, or a conditional branch.  
But \ourlib{}'s graph representation is \emph{polymorphic in the node type},
so each client can define nodes as it likes.
Because they contain nodes defined by the client,
graphs can include arbitrary client-specified data, including
(say) method calls, C~statements, stack maps, or
whatever.


\begin{figure}
\begin{fuzzcode}{0.98pt}
data `Node e x where
  Label  :: Label                   -> Node C O
  `Assign :: Var   -> Expr           -> Node O O
  `Store  :: Expr  -> Expr           -> Node O O
  `Branch :: Label                   -> Node O C
  `Cond   :: Expr  -> Label -> Label -> Node O C
    ... more constructors ...
\end{fuzzcode}
\caption{A typical node type as it might be defined by a client} 
\figlabel{cmm-node}
\end{figure}

The type of a node specifies \emph{at compile time} whether the node is open or
closed on entry and exit. Concretely,
the type constructor for a node has kind @*->*->*@, where the two type parameters
are type-level flags, one for entry and one for exit.
Such a type parameter may be instantiated only with type @O@~(for
open) or type~@C@ (for closed).

As an example,
\figref{cmm-node} shows a typical node type as it might be defined by
one of \ourlib's {clients}.
The type parameters are written @e@ and @x@, for
entry and exit respectively.  
The type is a generalized algebraic data type;
the syntax gives the type of each constructor.  
%%% \cite{peyton-jones:unification-based-gadts}.
For example, constructor @Label@
takes a @Label@ and returns a node of type @Node C O@, where
the~``@C@'' says ``closed on entry'' and the~``@O@'' says ``open on exit''.  
The types @Label@, @O@, and~@C@ are 
defined by \ourlib{} (\figref{graph}).  
As another example, constructor @Assign@ takes a variable and an expression, and it
returns a @Node@ open on both entry and exit; constructor @Store@ is
similar.
Types @`Var@ and @`Expr@ are private to the client, and
\ourlib{} knows nothing of them.  
Finally, control-transfer nodes @Branch@ and @Cond@  (conditional
branch) are open on entry
and closed on exit.  

Nodes closed on entry are the only targets of control transfers;
nodes open on entry and exit never perform control transfers;
and nodes closed on exit always perform control transfers\footnote{%
To obey these invariants,
a node for
a conditional-branch instruction, which typically either transfers control
\emph{or} falls through, must be represented as a two-target
conditional branch, with the fall-through path in a separate block.  
This representation is standard \cite{appel:modern},
and it costs nothing in practice:
such code is easily sequentialized without superfluous branches.
%a late-stage code-layout pass can readily reconstruct efficient code.
}.
Because of the position each type of node occupies in a
basic block,
we~often call them \emph{first}, \emph{middle}, and \emph{last} nodes
respectively.


\begin{figure}
\begin{fuzzcode}{10.5pt}
data `O   -- Open
data `C   -- Closed

data `Block n e x where
 `BFirst  :: n C O                      -> Block n C O
 `BMiddle :: n O O                      -> Block n O O
 `BLast   :: n O C                      -> Block n O C
 `BCat    :: Block n e O -> Block n O x -> Block n e x

data `Graph n e x where
  `GNil  :: Graph n O O
  `GUnit :: Block n O O -> Graph n O O
  `GMany :: MaybeO e (Block n O C) 
        -> LabelMap (Block n C C)
        -> MaybeO x (Block n C O)
        -> Graph n e x

data `MaybeO ^ex t where
  `JustO    :: t -> MaybeO O t
  `NothingO ::      MaybeO C t

newtype `Label -- abstract
newtype `LabelMap a -- finite map from Label to a
`addBlock   :: NonLocal n 
           => Block n C C
           -> LabelMap (Block n C C)
           -> LabelMap (Block n C C)
`blockUnion :: LabelMap a -> LabelMap a -> LabelMap a

class `NonLocal n where
  `entryLabel :: n C x -> Label
  `successors :: n e C -> [Label]
\end{fuzzcode}
\caption{The block and graph types defined by \ourlib} 
\figlabel{graph} \figlabel{edges}
\end{figure}
% omit MaybeC :: * -> * -> *


\subsection{Blocks} \seclabel{blocks}

\ourlib\ combines the client's nodes into
blocks and graphs, which, unlike the nodes, are defined by \ourlib{}
 (\figref{graph}).
A~@Block@ is parameterized over the node type~@n@
as well as over the same flag types that make it open or closed at
entry and exit.

The @BFirst@, @BMiddle@, and @BLast@ constructors create one-node
blocks.  
Each of these constructors is polymorphic in the node's \emph{representation}
but monomorphic in its \emph{shape}.
Why not use a single constructor 
of type \mbox{@n e x -> Block n e x@}, which would be polymorphic in a
node's representation \emph{and} 
shape?
Because by making the shape known statically, we simplify the
implementation of analysis and transformation in 
\secref{dfengine}. 
%%%  \finalremark{We have no principled answer to the question of what
%%%  parts of the representation of blocks are exposed.
%%%  Do we tell our readers or just ignore the question?
%%%  } 
%%%  \simon{Just ignore!}

The @BCat@ constructor concatenates blocks in sequence. 
It~makes sense to concatenate blocks only when control can fall
through from the first to the second; therefore, 
two blocks may be concatenated {only} if each block is open at
the point of concatenation.
This restriction is enforced by the type of @BCat@, whose first 
argument must be open on exit, and whose second argument must be open on entry.
It~is statically impossible, for example, to concatenate a @Branch@
immediately before an
@Assign@.  
Indeed, the @Block@ type statically guarantees that any
closed/closed @Block@---which compiler writers normally 
call a ``basic block''---consists of exactly one first node
(such as @Label@ in \figref{cmm-node}), 
followed by zero or more middle nodes (@Assign@ or @Store@), 
and terminated with exactly one 
last node (@Branch@ or @Cond@).  
Using GADTs to enforce these invariants is one of 
\ourlib{}'s innovations.


\subsection{Graphs} \seclabel{graphs}

\ourlib{} composes blocks into graphs, which are also defined 
in  \figref{graph}.
Like @Block@, the data type @Graph@ is parameterized over
both nodes @n@ and its open/closed shape (@e@ and @x@).
It has three constructors.  The first two
deal with the base cases of open/open graphs:
an empty graph is represented by @GNil@ while a single-block graph
is represented by @GUnit@.

More general graphs are represented by @GMany@, which has three
fields: an optional entry sequence, a body, and an optional exit
sequence.
\begin{itemize}
\item 
If the graph is open on entry, it contains an \emph{entry sequence} of type 
@Block n O C@.
We could represent this sequence as a value of type
@Maybe (Block n O C)@, but we can do better: 
a~value of @Maybe@ type requires a \emph{dynamic} test,
but we know \emph{statically}, at compile time, that the sequence is present if and only
if the graph is open on entry.
We~express our compile-time knowledge by using the type
@MaybeO e (Block n O C)@, a type-indexed version of @Maybe@
which is also defined in \figref{graph}:
the type @MaybeO O a@ is isomorphic to~@a@, while 
the type @MaybeO C a@ is isomorphic to~@()@.
\item 
The \emph{body} of the graph is a collection of  closed/closed blocks.  
To~facilitate traversal of the graph, we represent the body as a finite
map from label to block. 
\simon{{\tt LabelMap} not defined in Fig 2}
\item 
The \emph{exit sequence} is dual to the entry sequence, and like the entry
sequence, its presence or absence is deducible from the static type of the graph.
\end{itemize}

\seclabel{gSplice}

Graphs can be spliced together nicely; the cost is logarithmic in the
number of closed/closed blocks.
Unlike blocks, two graphs may be spliced together
not only when they are both open
 at splice point but also
when they are both closed---but not in the other two cases:
\begingroup
\def\^{\\[-5pt]}%
\hfuzz=10.8pt
\begin{smallttcode}
`gSplice :: Graph n e a -> Graph n a x -> Graph n e x
gSplice GNil g2 = g2
gSplice g1 GNil = g1\^
gSplice (GUnit ^b1) (GUnit ^b2) = GUnit (b1 `BCat` b2)\^
gSplice (GUnit b) (GMany (JustO e) bs x) 
  = GMany (JustO (b `BCat` e)) bs x\^
gSplice (GMany e ^bs (JustO x)) (GUnit b2) 
  = GMany e bs (JustO (x `BCat` b2))\^
gSplice (GMany e1 ^bs1 (JustO x1)) (GMany (JustO e2) ^bs2 x2)
  = GMany e1 (bs1 `blockUnion` (b `addBlock` bs2)) x2
  where b = x1 `BCat` e2\^
gSplice (GMany e1 bs1 NothingO) (GMany NothingO bs2 x2)
  = GMany e1 (bs1 `blockUnion` bs2) x2
\end{smallttcode}
\endgroup
This definition illustrates the power of GADTs: the
pattern matching is exhaustive, and all the open/closed invariants are
statically checked.  For example, consider the second-last equation for @gSplice@.
Since the exit sequence of the first argument is @JustO x1@,
we know that type parameter~@a@ is~@O@, and hence the entry sequence of the second 
argument must be @JustO e2@.
Moreover, block~@x1@ must be
closed/open, and block~@e2@ must be open/closed.  
We can therefore concatenate 
@x1@~and~@e2@
with @BCat@ to produce a closed/closed block~@b@, which is
added to the body of the result.

\ifnotcutting
We~have carefully crafted the types so that if @BCat@ 
is considered as an associative operator, 
every graph has a unique representation.
\finalremark{This \P\ is a candidate to be cut.}
\simon{Right}
%%  \simon{Well, you were the one who was so keen on a unique representation!
%%  And since we have one, I think tis useful to say so. Lastly, the representation of
%%  singleton blocks is not entirely obvious.}
%%%%    
%%%%    An empty open/open graph is represented
%%%%    by @GNil@, while a closed/closed one is @gNilCC@:
%%%%    \par {\small
%%%%    \begin{code}
%%%%      gNilCC :: Graph C C
%%%%      gNilCC = GMany NothingO BodyEmpty NothingO
%%%%    \end{code}}
%%%%    The representation of a @Graph@ consisting of a single block~@b@ 
%%%%    depends on the shape of~@b@:\remark{Does anyone care?}
%%%%    \par{\small
%%%%    \begin{code}
%%%%      gUnitOO :: Block O O -> Graph O O
%%%%      gUnitOC :: Block O C -> Graph O C
%%%%      gUnitCO :: Block O C -> Graph C O
%%%%      gUnitCC :: Block O C -> Graph C C
%%%%      gUnitOO b = GUnit b
%%%%      gUnitOC b = GMany (JustO b) BodyEmpty   NothingO
%%%%      gUnitCO b = GMany NothingO  BodyEmpty   (JustO b)
%%%%      gUnitCC b = GMany NothingO  (BodyUnit b) NothingO
%%%%    \end{code}}
%%%%    Multi-block graphs are similar.
%%%%    From these definitions
To guarantee uniqueness, @GUnit@ is restricted to open/open
blocks.
If~@GUnit@ were more polymorphic, there would be 
more than one way to represent some graphs, and it wouldn't be obvious
to a client which representation to choose---or if the choice made a difference.
{\hfuzz=1.8pt\par}
\fi

\subsection{Edges, labels and successors} 

\seclabel{nonlocal-class}
\seclabel{edges}




Although \ourlib{} is polymorphic in the type of nodes, 
it~still needs to know how control may be transferred from one node to another.
Within a~block, a control-flow edge is implicit in every application of
the @BCat@ constructor.
An~implicit edge originates in a first node or a middle node and flows
to a middle node or a last node.

Between blocks, a~control-flow edge is represented as chosen by the client.
An~explicit edge originates in a last node and flows to a (labelled)
first node. 
If~\hoopl\ is polymorphic in the node type,
how can it{} follow such edges?
\hoopl\ requires the client to make the node type an instance of 
\ourlib's @NonLocal@ type class, which is defined in \figref{edges}.
The @entryLabel@ method takes a first node (one closed on entry, \secref{nodes})
and returns its @Label@;
the~@successors@ method takes a last node (closed on exit) and returns
the @Label@s to 
which it can transfer control.  
%%A~middle node, which is open on both entry and exit, transfers control
%%only locally, to its successor within a basic block,
%%so no corresponding interrogation function is needed.

%% A node type defined by a client must be an instance of @NonLocal@.
In~\figref{cmm-node},
the client's instance declaration for @Node@ would be
\begin{code}
instance NonLocal Node where
  entryLabel (Label l)  = l
  successors (Branch b) = [b]
  successors (Cond e b1 b2) = [b1, b2]
\end{code}
Again, the pattern matching for both functions is exhaustive, and
the compiler statically checks this fact.  
Here, @entryLabel@ 
cannot be applied to an @Assign@ or @Branch@ node,
and any attempt to define a case for @Assign@ or @Branch@ would result
in a type error.

While the client provides this information about
nodes, it is convenient for \ourlib\ to get the same information
about blocks.
Internally,
\ourlib{} uses this instance declaration for the @Block@ type:
\begin{code}
instance NonLocal n => NonLocal (Block n) where
  entryLabel (BFirst n) = entryLabel n
  entryLabel (BCat b _) = entryLabel b
  successors (BLast  n) = successors n
  successors (BCat _ b) = successors b
\end{code}
Because the functions @entryLabel@ and @successors@ are used to track control
flow \emph{within} a graph, \ourlib\ does not need to ask for the
entry label or successors of a @Graph@ itself.
Indeed, @Graph@ \emph{cannot} be an instance of @NonLocal@, because even
if a @Graph@ is closed on entry, it need not have a unique entry label.
%
% A slight infelicity is that we cannot make @Graph@ an instance of @NonLocal@,
% because a graph closed on entry has no unique label.  Fortunately, we never
% need @Graph@s to be in @NonLocal@.
%
% ``Never apologize.  Never confess to infelicity.'' ---SLPJ



\begin{table}
\centerline{%
\begin{tabular}{@{}>{\raggedright\arraybackslash}p{1.03in}>{\scshape}c>{\scshape
}
      c>{\raggedright\arraybackslash}p{1.29in}@{}}
&\multicolumn1{r}{\llap{\emph{Specified}}\hspace*{-0.3em}}&
\multicolumn1{l}{\hspace*{-0.4em}\rlap{\emph{Implemented}}}&\\
\multicolumn1{c}{\emph{Part of optimizer}}
&\multicolumn1{c}{\emph{by}}&
\multicolumn1{c}{\emph{by}}&
\multicolumn1{c}{\emph{How many}}%
\\[5pt]
Control-flow graphs& Us & Us & One \\
Nodes in a control-flow graph & You & You & One type per intermediate language \\[3pt]
Dataflow fact~$F$    & You & You & One type per logic \\
Lattice operations & Us & You & One set per logic \\[3pt]
Transfer functions & Us & You & One per analysis \\
Rewrite functions & Us & You & One per \rlap{transformation} \\[3pt]
Analyze-and-rewrite functions & Us & Us & Two (forward, backward) \\
\end{tabular}%
}
\caption{Parts of an optimizer built with \ourlib}
\tablabel{parts}
\end{table}




\section {Using \ourlib{} to analyze and transform graphs} \seclabel{using-hoopl}
\seclabel{making-simple}
\seclabel{create-analysis}
\seclabel{api}

Now that we have graphs, how do we optimize them?
\ourlib{} makes it easy to build a new dataflow analysis and optimization.  
A~client must supply the following pieces:
\begin{itemize}
\item A \emph{node type} (\secref{nodes}).  
\ourlib{} supplies the @Block@ and @Graph@ types
that let the client build control-flow graphs out of nodes.
\item A \emph{data type of facts} and some operations over 
those facts (\secref{facts}).
Each analysis uses facts that are specific to that particular analysis,
which \ourlib{} accommodates by being polymorphic in 
the fact type.   
\item A \emph{transfer function} that takes a node and returns a
\emph{fact transformer}, which takes a fact flowing into the node and
returns the transformed fact that flows out of the node (\secref{transfers}).  
\item A \emph{rewrite function} that takes a node and an input fact,
performs a monadic action, 
and returns either @Nothing@
or @Just g@,
where @g@~is a graph that should
replace the node
(\secreftwo{rewrites}{shallow-vs-deep}).
The ability to replace a \emph{node} by a \emph{graph} 
%  that may include internal control flow 
is crucial for many code-improving transformations.
%We discuss the rewrite function
%in Sections \ref{sec:rewrites} and \ref{sec:shallow-vs-deep}.
\end{itemize}
These requirements are summarized in \tabref{parts}.
Because facts, transfer functions, and rewrite functions work together,
we~combine them in a single record of type @FwdPass@ (\figref{api-types}).




Given a node type~@n@ and a @FwdPass@,
a client can ask \ourlib{}\ to analyze and rewrite a 
graph.
\hoopl\ provides a fully polymorphic interface, but for purposes of
exposition, we present a function that is specialized to a
closed/closed graph:
\begin{fuzzcode}{10.5pt}
`analyzeAndRewriteFwdBody
 :: ( CkpointMonad m -- Roll back speculative actions
    , NonLocal n )   -- Extract non-local flow edges
 => FwdPass m n f    -- Lattice, transfer, rewrite
 -> [Label]          -- Entry point(s)
 -> Graph n C C      -- Input graph
 -> FactBase f       -- Input fact(s)
 -> m ( Graph n C C  -- Result graph
      , FactBase f ) -- ... and its facts
\end{fuzzcode}
Given a @FwdPass@ and a list of entry points, the 
analyze-and-rewrite function transforms a graph into
an optimized graph.
As its type shows, this function
is polymorphic in the types of nodes~@n@ and facts~@f@;
these types are chosen by the client.
The type of the monad~@m@ is also chosen by the client.

As well as taking and returning a graph, the
function also takes input facts (the @FactBase@) and produces output facts. 
A~@FactBase@ is a finite mapping from @Label@ to facts (\figref{api-types});
if~a @Label@ is not in the domain of the @FactBase@, its fact is the
bottom element of the lattice.
For example, in our constant-propagation example from \secref{const-prop-example},
if the graph
represents the body of a procedure 
with parameters $x,y,z$, we would map the entry @Label@ to a fact
\mbox{$@x=@\top \land @y=@\top \land @z=@\top$}, to specify that the procedure's
parameters are not known to be constants.
%%  
%%  The
%%  output @FactBase@ maps each @Label@ in the body to its fact

The client's model of @analyzeAndRewriteFwdBody@ is as follows:
\ourlib\ walks forward over each block in the graph.
At~each node, \ourlib\ applies the
rewrite function to the node and the incoming fact.  If the rewrite
function returns @Nothing@, the node is retained as part of the output
graph, the transfer function is used to compute the downstream fact,
and \ourlib\ moves on to the next node.
But if the rewrite function returns @Just g@,
indicating that it wants to rewrite the node to the replacement graph~@g@, then
\ourlib\ recursively analyzes and may further rewrite~@g@
before moving on to the next node. 
A~node following a rewritten node sees 
\emph{up-to-date} facts; that is, its input fact is computed by
analyzing the replacement graph.

A~rewrite function may take any action that is justified by
the incoming fact.
If~further analysis invalidates the fact, \hoopl\ rolls
back the action.
Because graphs cannot be mutated,
rolling back to the original graph is easy.
But~rolling back a rewrite function's {monadic} action
requires cooperation from the client:
the~client must  provide @checkpoint@ and
@restart@ operations, which
make~@m@ an instance of
\hoopl's @CkpointMonad@ class
(\secref{checkpoint-monad}). 


Below we flesh out the
interface to @analyzeAndRewriteFwdBody@, leaving the implementation for
\secref{engine}.  
%{\hfuzz=7.2pt\par}


\begin{figure}
\begin{fuzzcode}{25pt}
data `FwdPass m n f
  = FwdPass { `fp_lattice  :: DataflowLattice f
            , `fp_transfer :: FwdTransfer n f
            , `fp_rewrite  :: FwdRewrite m n f }

------- Lattice ----------
data `DataflowLattice f = DataflowLattice
 { `fact_bot  :: f
 , `fact_join :: JoinFun f }
type `JoinFun f = 
  OldFact f -> NewFact f -> (ChangeFlag, f)
newtype `OldFact f = OldFact f
newtype `NewFact f = NewFact f
data `ChangeFlag = `NoChange | `SomeChange

------- Transfers ----------
newtype `FwdTransfer n f      -- abstract type
`mkFTransfer
 :: (forall e x . n e x -> f -> Fact x f)
 -> FwdTransfer n f

------- Rewrites ----------
newtype `FwdRewrite m n f     -- abstract type
`mkFRewrite :: FuelMonad m 
 => (forall e x . n e x -> f -> m (Maybe (Graph n e x)))
 -> FwdRewrite m n f
`thenFwdRw :: FwdRewrite m n f -> FwdRewrite m n f
 -> FwdRewrite m n f
`iterFwdRw :: FwdRewrite m n f -> FwdRewrite m n f
`noFwdRw   :: Monad m          => FwdRewrite m n f

------- Fact-like things, aka "fact(s)" -----
type family   `Fact x f :: *
type instance Fact O f = f
type instance Fact C f = FactBase f

------- FactBase -------
type `FactBase f = LabelMap f
 -- A finite mapping from Labels to facts f
`mkFactBase 
 :: DataflowLattice f -> [(Label, f)] -> FactBase f

------- Rolling back speculative rewrites ----
class Monad m => CkpointMonad m where
  type Checkpoint m
  checkpoint :: m (Checkpoint m)
  restart    :: Checkpoint m -> m () 

------- Optimization fuel ----
type `Fuel = Int
class Monad m => `FuelMonad m where
  `getFuel :: m Fuel
  `setFuel :: Fuel -> m ()
\end{fuzzcode}
\caption{\ourlib{} API data types}
  \figlabel{api-types}
  \figlabel{lattice-type} \figlabel{lattice}
  \figlabel{transfers}  \figlabel{rewrites}
\end{figure}
%%%%  \simon{We previously renamed @fact\_join@ to @fact\_extend@ because it really
%%%%  is not a symmetrical join; we're extending an old fact with a new one.
%%%%  NR: Yes, but the asymmetry is now explicit in the \emph{type}, so it
%%%%  needn't also be reflected in the \emph{name}.}

\subsection{Dataflow lattices} \seclabel{lattices} \seclabel{facts}

For each analysis or transformation, the client must define a type
of dataflow facts.
A~dataflow fact often represents an assertion
about a program point,
but in general, dataflow analysis establishes properties of \emph{paths}:
\begin{itemize}
\item An assertion about all paths \emph{to} a program point is established
by a \emph{forward analysis}. For example the assertion ``$@x@=@3@$'' at point P 
claims that variable @x@ holds value @3@ at P, regardless of the
path by which P is reached.
\item An assertion about all paths \emph{from} a program point is 
established by a \emph{backward analysis}. For example, the 
assertion ``@x@ is dead'' at point P claims that no path from P uses 
variable @x@.
\end{itemize}

A~set of dataflow facts must form a lattice, and \ourlib{} must know
(a)~the bottom element of the lattice and (b)~how to take 
the least upper bound (join) of two elements.
To ensure that analysis
terminates, it is enough if every fact has a finite number of
distinct facts above it, so that repeated joins
eventually reach a fixed point.
%% \simon{There is no mention here of the @OldFact@ and @NewFact@ types.
%% Shall we nuke them for the purposes of the paper?
%% NR: They should definitely not be nuked; they're needed if the reader
%% wants to understand the asymmetrical behavior of @fact\_join@.
%% I'm~also mindful that this paper will be the primary explanation of
%% \hoopl\ to its users, and I don't want to hide this part of the
%% interface.
%% As for mentions in the text, 
%% if you look carefully, you'll
%% see that I've changed the subscripts in the text to ``old'' and
%% ``new''.
%% Together with the code, I believe that's enough to get the point across.
%% SLPJ -- OK.  I don't agree that it's enough, but we have more important fish to fry.
%% }

In practice, joins are computed at labels.
If~$f_{\mathit{old}}$ is the fact currently associated with a
label~$L$, 
and if a transfer function propagates a new fact~$f_{\mathit{new}}$
into label~$L$, 
\hoopl\ replaces $f_{\mathit{old}}$ with
the join  \mbox{$f_{\mathit{old}} \join f_{\mathit{new}}$}.
And \hoopl\ wants to know 
if \mbox{$f_{\mathit{old}} \join f_{\mathit{new}} = f_{\mathit{old}}$},
because if not, the analysis has not reached a fixed point.

The bottom element and join operation of a lattice of facts of
type~@f@ are stored in a value of type @DataflowLattice f@
(\figref{lattice}). 
%%%% \simon{Can we shorten ``@DataflowLattice@'' to just
%%%% ``@Lattice@''?} % bigger fish ---NR
%%Such a value is part of the  @FwdPass n f@ that is passed to
%%@analyzeAndRewriteFwd@ above.
As noted in the previous paragraph, 
\ourlib{} needs to know when the result of a join is equal
to the old fact.
Because this information can often be computed cheaply together
with the join, \ourlib\ does not
 require a separate equality test on facts, which might be expensive.
Instead, \ourlib\ requires that @fact_join@ return a @ChangeFlag@ as
well as the least upper bound.
The @ChangeFlag@ should be @NoChange@ if
the result is the same as the old fact, and
@SomeChange@ if the result differs.  

\seclabel{WithTop}

To help clients create lattices and join functions,
\hoopl\ includes functions and constructors that can extend a type~@a@
with top and bottom elements.
In this paper, we use only type @`WithTop@, which comes with value
constructors that have these types:
\begin{code}
  `PElem :: f -> WithTop f
  `Top   ::      WithTop f
\end{code}
\hoopl\ provides combinators which make it easier to create join
  functions that use @Top@.
The most useful is @extendJoinDomain@, which uses auxiliary types
defined in \figref{lattice}:
\begin{smallcode}
`extendJoinDomain
 :: (OldFact f -> NewFact f -> (ChangeFlag, WithTop f))
 -> JoinFun (WithTop f)
\end{smallcode}
A~client can write a join function that \emph{consumes} only facts of
type~@f@, but may produce either @Top@ or a fact of type~@f@---as
in the example of \figref{const-prop} below.
Calling @extendJoinDomain@ extends the client's function to a proper
join function on the type @WithTop a@,
guaranteeing that joins
involving @Top@ obey the appropriate algebraic laws.

\hoopl\ also provides a value constructor @`Bot@ and type constructors
@`WithBot@ and @`WithTopAndBot@, along with similar functions.
Constructors @Top@ and @Bot@ are polymorphic, so for example,
@Top@~also has type @WithTopAndBot a@.

It is also very common to use a lattice that takes the form of a
finite~map.
In~such lattices it is typical to join maps pointwise, and \hoopl\
provides a function that makes it convenient to do so:
\begin{smallcode}
 `joinMaps :: Ord k => JoinFun f -> JoinFun (Map.Map k f)
\end{smallcode}


\subsection{The transfer function} \seclabel{transfers}

A~forward transfer function is presented with the dataflow fact
coming 
into a node, and it computes dataflow fact(s) on the node's outgoing edge(s).
In a forward analysis, \hoopl\ starts with the fact at the
beginning of a block and applies the transfer function to successive 
nodes in that block, until eventually the transfer function for the last node
computes the facts that are propagated to the block's successors.
For example, consider doing constant propagation (\secref{constant-propagation}) on 
the following graph, with entry at @L1@:
\begin{code}
  L1: x=3; goto L2
  L2: y=x+4; x=x-1; 
      if x>0 then goto L2 else return
\end{code}
Forward analysis starts with the bottom fact \{\} at every label
except the entry~@L1@.
The initial fact at~@L1@ is \{@x=@$\top$@,y=@$\top$\}.
Analyzing~@L1@ propagates this fact forward, applying the transfer 
function successively to the nodes
of~@L1@, and propagating the new fact \{@x=3,y=@$\top$\} to~@L2@.
This new fact is joined with the existing (bottom) fact at~@L2@.
Now~the analysis propagates @L2@'s fact forward, again applying the transfer
function, and propagating the new fact \mbox{\{@x=2@, @y=7@\}} to~@L2@.
Again the new fact is joined with the existing fact at~@L2@, and the process
repeats until the facts for each label reach a fixed point.

A~transfer function has an unusual sort of type:
not quite a dependent type, but not a bog-standard polymorphic type either.
The result type of the transfer function is \emph{indexed} by the shape (i.e.,
the type) of the node argument:
If the node is open on exit, the transfer function produces a single fact.
But if the node is \emph{closed} on exit,
the transfer function 
produces a collection of (@Label@,fact) pairs: one for each outgoing edge.  
The~collection is represented by a @FactBase@, and
the indexing is expressed by Haskell's (recently added) 
\emph{indexed type families}.
\remark{Perhaps @mkFactBase@ needs explaining?}
The relevant part of \ourlib's interface is given in \figref{transfers}.
A~forward transfer function supplied by a client, 
which would be passed to @mkFTransfer@,
is typically a function polymorphic in @e@ and @x@.  
It~takes a 
node of type \mbox{@n@ @e@ @x@}
and it returns a \emph{fact transformer} of type
@f -> Fact x f@.
Type constructor @Fact@
should be thought of as a type-level function: its signature is given in the
@type family@ declaration, while its definition is given by two @type instance@
declarations.  The first declaration says that a @Fact O f@, which
comes out of a node
\emph{open} on exit, is just a fact~@f@. 
The second declaration says that a @Fact C f@, which comes out of a
node \emph{closed} on exit, is a mapping from @Label@ to facts.

%%  
%%  \begin{code}
%%    `transfer_fn :: forall e x . n e x -> f -> Fact x f
%%    `node        :: n e x
%%  \end{code}
%%  then @(transfer_fn node)@ is a fact transformer:
%%  \begin{code}
%%    transfer_fn node :: f -> Fact x f
%%  \end{code}
%%  


\subsection{The rewrite function and the client's monad} 
 \seclabel{rewrites} 
 \seclabel{example-rewrites}


We compute dataflow facts in order to enable code-improving
transformations.
In our constant-propagation example,
the dataflow facts may enable us
to simplify an expression by performing constant folding, or to 
turn a conditional branch into an unconditional one.
Similarly, a liveness analysis may allow us to 
replace a dead assignment with a no-op.

A @FwdPass@ therefore includes a \emph{rewrite function}, whose
type, @FwdRewrite@, is abstract (\figref{api-types}).
A~programmer creating a rewrite function chooses the type of a node~@n@ and
a dataflow fact~@f@.
A~rewrite function might also want to consume fresh
names (e.g., to label new blocks) or take other actions (e.g., logging
rewrites). 
So~that a rewrite function may take actions, \hoopl\
requires that a programmer creating a rewrite function also choose a
monad~@m@.
So~that \hoopl\ may roll back actions taken by speculative rewrites, the monad
must satisfy the constraint @CkpointMonad m@, as described in
\secref{checkpoint-monad}. 
The programmer may write code that works with any such monad,
may create a monad just for the client,
or may use a monad supplied by \hoopl. 


When these choices are made, the easy way to create a rewrite
function is to call the function @mkFRewrite@ in \figref{api-types}.
The client supplies a function that is specialized to a particular
node, fact, and monad, but is polymorphic in the
\emph{shape} of the node to be rewritten.
The function, which we will call~$r$, takes a node and a fact and returns a monadic
computation, but what is the result of that computation?
One  might
expect that the result should be a new node, but that is not enough:
in~general, it must be possible for rewriting to result in a graph.
For example,
we might want to remove a node by returning the empty graph,
or more ambitiously, we might want to replace a high-level operation
with a tree of conditional branches or a loop, which would entail
returning a graph containing new blocks with internal control flow.


It~must also be possible for a rewrite function to decide to do nothing.
The result of the monadic computation returned by~$r$ may therefore be
@Nothing@, indicating that the node should
not be rewritten,
or $@Just@\;\ag$, indicating that the node should
be replaced with~\ag: the replacement graph. 
%%The additional value $\rw$ tells \hoopl\ whether
%%and how the replacement graph~$\ag$ should be analyzed and rewritten
%%further;
%%we explain~$\rw$ in \secref{shallow-vs-deep}.


The type of @mkFRewrite@ in \figref{api-types} guarantees
that
the replacement graph $\ag$ has
the \emph{same} open/closed shape as the node being rewritten.
For example, a branch instruction can be replaced only by a graph 
closed on exit.
%%  Moreover, because only an open/open graph can be 
%%  empty---look at the type of @GNil@ in \figref{graph}---the type 
%%  of @FwdRewrite@ 
%%  guarantees, at compile time, that no head of a block (closed/open)
%%  or tail of a block (open/closed) can ever be deleted by being
%%  rewritten to an empty graph.

\subsection{Shallow rewriting, deep rewriting, rewriting combinators,
and the meaning of {\textmd\texttt{FwdRewrite}}}

 \seclabel{shallow-vs-deep}

When a node is rewritten, the replacement graph~$g$
must itself be analyzed, and its nodes may be further rewritten.
\hoopl\ can make a recursive call to 
@analyzeAndRewriteFwdBody@---but exactly how should it rewrite the
replacement graph~$g$?
There are two common ways to proceed:
\begin{itemize}
\item
Rewrite~$g$ using the same rewrite function that produced~$g$.
This procedure is
called \emph{deep rewriting}. 
When deep rewriting is used, the client's rewrite function must
ensure that the graphs it produces are not rewritten indefinitely
(\secref{correctness}). 
\item
Analyze~$g$ {without} rewriting it.
This procedure is called \emph{shallow rewriting}.
\end{itemize}
Deep rewriting is essential to achieve the full benefits of
interleaved analysis and transformation
\citep{lerner-grove-chambers:2002}.
But shallow rewriting can be vital as well; 
for example, a backward dataflow pass that inserts
a spill before a call must not rewrite the call again, lest it attempt
to insert infinitely many spills.



\seclabel{rewrite-model}

An~innovation of \hoopl\ is to build the choice of shallow or deep
rewriting into each rewrite function, through the use of the four
combinators
@mkFRewrite@, @thenFwdRw@, @iterFwdRw@, and @noFwdRw@ shown in
\figref{api-types}. 
Every rewrite function is made with these combinators,
and its behavior is characterized by the answers to two questions:
Does the  function rewrite a~node?
If~so, what rewrite function
should be used to analyze the replacement graph recursively?
To~answer these questions, we~present an
algebraic datatype that models @FwdRewrite@:
\begin{smallcode}
data `Rw a = `Mk a | `Then (Rw a) (Rw a) | `Iter (Rw a) | `No
\end{smallcode}
Using this model, we specify how a rewrite function works by
giving a reference implementation.
The code is in continuation-passing style;
when the node is rewritten,
the first continuation~@j@ accepts a pair containing the replacement
graph and the new rewrite function to be used to analyze it.
When the node is not rewritten, 
the second continuation~@n@ is the (lazily evaluated) result.
\begin{smallcode}
rewrite :: Monad m => FwdRewrite m n f -> n e x -> f
        -> m (Maybe (Graph n e x, FwdRewrite m n f))
`rewrite ^rs node f = rew rs (return . Just) (return Nothing)
 where
  `rew (Mk r) j n = do ^mg <- r node f
                      case mg of Nothing -> n
                                 Just g -> j (g, No)
  rew (r1 `Then` r2) j n = rew r1 (j . add r2) (rew r2 j n)
  rew (Iter r)       j n = rew r (j . add (Iter r)) n
  rew No             j n = n
  `add nextrw (g, r) = (g, r `Then` nextrw)
\end{smallcode}
Appealing to this model, we see that
\begin{itemize}
\item
A~function @mkFRewrite rw@ never rewrites a replacement
graph---this behavior is shallow rewriting.
\item
When a~function @r1 `thenFwdRw` r2@ is applied to a node,
if @r1@ replaces the node, then @r2@~is used to transform the
replacement graph.
And~if @r1@~does not replace the node, \hoopl\ tries~@r2@.
\item
When a~function @iterFwdRw r@ rewrites a node, then @iterFwdRw r@ is
used to transform the replacement graph---this behavior is deep
rewriting.
If~@r@~does not rewrite a~node, neither does @iterFwdRw r@.
\item
Finally, @noFwdRw@ never replaces a graph.
\end{itemize}
For~convenience, we~also provide the~function @`deepFwdRw@,
which is the composition of @iterFwdRw@ and~@mkFRewrite@.
\remark{maybe some of the real type signatures ought to be repeated?}

%%  \begin{code}
%%   `iterFwdRw :: FwdRewrite m n f -> FwdRewrite m n f
%%  \end{code}

%%  To try multiple rewrite functions in sequence,
%%  provided they all use the same type of fact, \hoopl\ combines them with
%%  \verbatiminput{comb1}
%%  % defn thenFwdRw
%%  Rewrite function @r1 `thenFwdRw` r2@ first does the rewrites of~@r1@,
%%  then the rewrites of~@r2@.

%%  
%%  
%%  by using
%%  Rewrite functions are potentially more plentiful than transfer
%%  functions, because
%%  a~single dataflow fact might justify more than one kind of rewrite.
%%  \hoopl\ makes it easy for a client to combine multiple rewrite
%%  functions that use the same fact:

Our combinators satisfy the algebraic laws that you would expect;
for~example 
@noFwdRw@ is a left and right identity of @thenFwdRw@.
A~more interesting law is
\begin{code}
  iterFwdRw r = r `thenFwdRw` iterFwdRw r
\end{code}
Unfortunately, this law cannot be used to \emph{define}
@iterFwdRw@:
if~we used this law to define @iterFwdRw@, then when @r@ returned @Nothing@,
@iterFwdRw r@ would diverge.



%%%%  \hoopl\ provides
%%%%  a function that makes a shallow rewrite deep:\finalremark{algebraic
%%%%  law wanted!}
%%%%  \remark{Do we really care about @iterFwdRw@ or can we just drop it?}
%%%%  \verbatiminput{iterf}
%%%%  % defn iterFwdRw


\subsection{When the type of nodes is not known}

We note above (\secref{transfers}) that
the type of a transfer function's result 
depends on the argument's shape on exit.
It~is easy for a client to write a type-indexed transfer function,
because the client defines the constructor and shape for each node.
The client's transfer functions discriminate on the constructor
and so can return a result that is indexed by each node's shape.

What if you want to write a transfer function that
{does \emph{not} know the type of the node}?
%
For example, a dominator analysis need not scrutinize nodes;
it needs to know only about
labels and edges in the graph.
Ideally, a dominator analysis would work
with \emph{any} type of node~@n@, 
provided only that @n@~is an
instance of the @NonLocal@ type class.
But if we don't know
the type of~@n@, 
we can't write a function of type 
@n e x -> f -> Fact x f@,
because the only way to get the result type right is to 
scrutinize the constructors of~@n@.

\seclabel{triple-of-functions}

There is another way;
instead of requiring a single function that is polymorphic in shape,
\hoopl\ also accepts a triple of functions, each of which is
polymorphic in the node's type but monomorphic in its shape:
\begin{code}
`mkFTransfer3 :: (n C O -> f -> Fact O f)
             -> (n O O -> f -> Fact O f)
             -> (n O C -> f -> Fact C f)
             -> FwdTransfer n f
\end{code}
We have used this interface to write a number of functions that are
 polymorphic in the node type~@n@:
\begin{itemize}
\item
A function that takes a @FwdTransfer@ and wraps it in logging code, so
an analysis can be debugged by watching the facts flow through the
nodes
\item
A pairing function that runs two passes interleaved, not sequentially,
potentially producing better results than any sequence:
\verbatiminput{pairf}
% defn pairFwd
\item
 An efficient dominator analysis
in the style of 
\citet{cooper-harvey-kennedy:simple-dominance},
whose transfer function is implemented 
using only the
functions in the \texttt{NonLocal} type class
\end{itemize}


%%  Or, you might want to write a combinator that
%%  computes the pairwise composition of any two analyses
%%  defined on any node type.
%%  
%%  You cannot write either of these examples
%%  using the polymorphic function described in \secref{transfers}
%%  because these examples require the ability
%%  to inspect the shape of the node.
%%  
%%  
%%  \subsection{Stuff cut from 4.2 that should disappear when the previous section is finished}
%%  
%%  So much for the interface.
%%  What about the implementation?
%%  The way GADTs work is that the compiler uses the value constructor for
%%  type \mbox{@n@ @e@ @x@} to determine whether @e@~and~@x@ are open or
%%  closed.
%%  But we want to write functions that are \emph{polymorphic} in the node
%%  type~@n@!
%%  Such functions include
%%  \begin{itemize}
%%  \item
%%  A function that takes a pair of @FwdTransfer@s for facts @f@~and~@f'@,
%%  and returns a single @FwdTransfer@ for the fact @(f, f')@
%%  \end{itemize}
%%  \simon{These bullets are utterly opaque. They belong in 4.5.  The rest of this section 
%%  is also very hard to understand without more explanation. See email.  }
%%  Such functions may also be useful in \hoopl's \emph{clients}:
%%  % may need
%%  %functions that are polymorphic in the node type~@n@:
%%  \begin{itemize}
%%  \item
%%  A dominator analysis in the style of
%%  \citet{cooper-harvey-kennedy:simple-dominance} requires only the
%%  functions in the \texttt{NonLocal} type class; 
%%  we have written such an analysis using transfer functions that are 
%%  polymorphic in~@n@.
%%  \end{itemize}
%%  Because the mapping from
%%  value constructors to shape is different for each node type~@n@, transfer
%%   functions cannot be polymorphic in both
%%  the representation and the shape of nodes.
%%   Our implementation therefore sacrifices polymorphism in shape:
%%  internally, \hoopl\ represents
%%  a~@FwdTransfer n f@ as a \emph{triple} of functions,
%%  each polymorphic in~@n@ but monomorphic in shape:
%%  \begin{code}
%%  newtype FwdTransfer n f 
%%    = FwdTransfers ( n C O -> f -> f
%%                   , n O O -> f -> f
%%                   , n O C -> f -> FactBase f
%%                   )
%%  \end{code}
%%  \simon{I argue strongly that the implementation should be 
%%  polymorphic too, using a shape classifier where necessary.
%%  Regardless of how this debate pans out, though, I think it's 
%%  a bad idea to introduce triples here. They are simply confusing.}
%%  Such triples can easily be composed and wrapped without requiring a
%%  pattern match on the value constructor of an unknown node
%%  type.\remark{Need to refer to this junk in the conclusion}
%%  Working with triples is tedious, but only \hoopl\ itself is forced to
%%  work with triples; each client, which knows the node type, may supply
%%  a triple,
%%  but it typically supplies a single function polymorphic in the shape
%%  of the (known) node.  
%%  
%%  

%%    \subsection{Composing rewrite functions and dataflow passes} \seclabel{combinators}
%%    
%%    Requiring each rewrite to return a new rewrite function has another
%%    advantage, beyond controlling the shallow-vs-deep choice: it
%%    enables a variety of combinators over rewrite functions. 
%%    \remark{This whole subsection needs to be redone in light of the new
%%    (triple-based) representation.  It's not pretty any more.}
%%    For example, here is a function
%%    that combines two rewrite functions in sequence:
%%    \remark{This code must be improved}
%%    What a beautiful type @thenFwdRw@ has! 
%%    \remark{And what an ugly implementation!  Implementations must go.}
%%    It tries @rw1@, and if @rw1@
%%    declines to rewrite, it behaves like @rw2@.  But if
%%    @rw1@ rewrites, returning a new rewriter @rw1a@, then the overall call also
%%    succeeds, returning a new rewrite function obtained by combining @rw1a@
%%    with @rw2@.  (We cannot apply @rw1a@ or @rw2@ 
%%    directly to the replacement graph~@g@, 
%%    because @r1@~returns a graph and @rw2@~expects a node.)
%%    The rewriter @noFwdRw@ is the identity of @thenFwdRw@.
%%    Finally, @thenFwdRw@ can 
%%    combine a deep-rewriting function and a shallow-rewriting function,
%%    to produce a rewrite function that is a combination of deep and shallow.
%%    %%This is something that the Lerner/Grove/Chambers framework could not do,
%%    %%because there was a global shallow/deep flag.
%%    %% Our shortsightedness; Lerner/Grove/Chambers is deep only ---NR
%%    
%%    
%%    A shallow rewrite function can be made deep by iterating
%%    it:\remark{Algebraic law wanted!}
%%    If we have shallow rewrites $A$~and~$B$ then we can build $AB$,
%%    $A^*B$, $(AB)^*$, 
%%    and so on: sequential composition is @thenFwdRw@ and the Kleene star
%%    is @iterFwdRw@.\remark{Do we still believe this claim?}
%%    \remark{We can't define @iterFwdRew@ in terms of @thenFwdRew@ because
%%    the second argument to @thenFwdRew@ would have to be the constant
%%    nothing function when applied but would have to be the original triple
%%    when passed to @thenFwdRew@ as the second argument (in the recursive
%%    call).}
%%    
%%    
%%    The combinators above operate on rewrite functions that share a common
%%    fact type and transfer function.
%%    It~can also be useful to combine entire dataflow passes that use
%%    different facts.
%%    We~invite you to write one such combinator, with type
%%    \begin{code}
%%      `pairFwd :: Monad m
%%              => FwdPass m n f1 
%%              -> FwdPass m n f2
%%              -> FwdPass m n (f1,f2)
%%    \end{code}
%%    The two passes run interleaved, not sequentially, and each may help
%%    the other,
%%    yielding better results than running $A$~and then~$B$ or $B$~and then~$A$
%%    \citep{lerner-grove-chambers:2002}.
%%    %%  call these passes. ``super-analyses;''
%%    %%  in \hoopl, construction of super-analyses is
%%    %%  particularly concrete.

\subsection{Example: Constant propagation and constant folding} 
  \seclabel{const-prop-client}

% omit Binop :: Operator -> Expr -> Expr -> Expr
% omit Add :: Operator

\begin{figure}
\smallfuzzverbatiminput{2.5pt}{cprop}
% local node
% defn ConstFact
% defn constLattice
% defn constFactAdd
% defn varHasLit
% local ft
% defn constProp
% local cp
% local lookup
% defn simplify
% local simp
% local s_node
% local s_exp
% defn constPropPass
% local new
% local old
% local tl
% local fl
\caption{The client for constant propagation and constant folding\break (extracted automatically from code distributed with Hoopl)}
\figlabel{const-prop}
\end{figure}


\figref{const-prop} shows client code for
constant propagation and constant folding.
For each variable, at each program point, the analysis concludes one
of three facts: 
the variable holds a constant value of type~@`Lit@,
the variable might hold a non-constant value,
or what the variable holds is unknown.
We~represent these facts using a finite map from a variable to a
fact of type @WithTop Lit@ (\secref{WithTop}).
% Any one procedure has only
% finitely many variables; only finitely many facts are computed at any
% program point; and in this lattice any one fact can increase at most
% twice.  These properties ensure that the dataflow engine will reach a
% fixed point.
A~variable with a constant value maps to @Just (PElem k)@, where
@k@ is the constant value; 
a variable with a non-constant value maps to @Just Top@;
and a variable with an unknown value maps to @Nothing@ (it is not
in the domain of the finite map).

% \afterpage{\clearpage}

The definition of the lattice (@constLattice@) is straightforward.
The bottom element is an empty map (nothing is known about what
any variable holds). 
%
The join function is implemented with the help of combinators provided
by \hoopl.
The client writes a simple function, @constFactAdd@, which 
compares two values of type @Lit@ and returns a result of type
@WithTop Lit@.
The client uses @extendJoinDomain@ to lift @constFactAdd@ into a join
function on @WithTop Lit@, then uses
@joinMaps@ to lift \emph{that} join function up to the map
containing facts for all variables.

The forward transfer function @varHasLit@ is defined using the
shape-polymorphic auxiliary function~@ft@.
For most nodes~@n@, @ft n@ simply propagates the input fact forward.
But for an assignment node, if a variable~@x@ gets a constant value~@k@,
@ft@ extends the input fact by mapping @x@ to~@PElem k@.
And if a variable~@x@ is assigned a non-constant value,
@ft@ extends the input fact by mapping @x@ to~@Top@.
There is one other interesting case:
a conditional branch where the condition
is a variable.
If the conditional branch flows to the true successor,
the variable holds @True@, and similarly for the false successor,
\emph{mutatis mutandis}.
Function @ft@ updates the fact flowing to each successor accordingly.
Because~@ft@ scrutinizes a GADT, it cannot
use a wildcard to default the uninteresting cases.

The transfer function need not consider complicated cases such as 
an assignment @x:=y@ where @y@ holds a constant value~@k@.
Instead, we rely on the interleaving of transformation
and analysis to first transform the assignment to @x:=k@,
which is exactly what our simple transfer function expects.
As we mention in \secref{simple-tx},
interleaving makes it possible to write
very simple transfer functions without missing
opportunities to improve the code.

\figref{const-prop}'s rewrite function for constant propagation, @constProp@,
rewrites each use of a variable to its constant value.
The client has defined auxiliary functions that may change expressions
or nodes:
\begin{smallcode}
type `MaybeChange a = a -> Maybe a
`mapVE :: (Var  -> Maybe Expr) -> MaybeChange Expr
`mapEE :: MaybeChange Expr     -> MaybeChange Expr
`mapEN :: MaybeChange Expr     -> MaybeChange (Node e x)
`mapVN :: (Var  -> Maybe Expr) -> MaybeChange (Node e x)
`nodeToG :: Node e x -> Graph Node e x
\end{smallcode}
The client composes @map@\emph{XX} functions
to apply @lookup@ to each use of a variable in each kind of node;
@lookup@ substitutes for each variable that has a constant value.
Applying @liftM nodeToG@ lifts the final node, if present, into a~@Graph@.


\figref{const-prop} also gives another, distinct function 
for constant
folding: @simplify@.
This function
rewrites constant expressions to their values,
and it rewrites a conditional branch on a
boolean constant to an unconditional branch.
To~rewrite constant expressions, 
it runs @s_exp@ on every subexpression.
Function @simplify@ does not check whether a variable holds a
constant value; it relies on @constProp@ to have replaced the
variable by the constant.
Indeed, @simplify@ does not consult the
incoming fact, so it is polymorphic in~@f@.


The @FwdRewrite@ functions @constProp@ and @simplify@
are useful independently.
In this case, however, we
want \emph{both} of them,
so we compose them with @thenFwdRw@.
The composition, along with the lattice and the
transfer function,
goes into @constPropPass@ (bottom of \figref{const-prop}).
Given @constPropPass@, we can
improve a graph~@g@ by passing @constPropPass@ and~@g@
to
@analyzeAndRewriteFwdBody@.



%%%%    \subsection{Fixed points and speculative rewrites} \seclabel{fixpoints}
%%%%    
%%%%    Are rewrites sound, especially when there are loops?
%%%%    Many analyses compute a fixed point starting from unsound
%%%%    ``facts''; for example, a live-variable analysis starts from the
%%%%    assumption that all variables are dead.  This means \emph{rewrites
%%%%    performed before a fixed point is reached may be unsound, and their results
%%%%    must be discarded}.  Each iteration of the fixed-point computation must
%%%%    start afresh with the original graph.  
%%%%    
%%%%    
%%%%    Although the rewrites may be unsound, \emph{they must be performed}
%%%%    (speculatively, and possibly recursively), 
%%%%    so that the facts downstream of the replacement graphs are as accurate
%%%%    as possible.
%%%%    For~example, consider this graph, with entry at @L1@:
%%%%    \par{\small
%%%%    \begin{code}
%%%%      L1: x=0; goto L2
%%%%      L2: x=x+1; if x==10 then goto L3 else goto L2
%%%%    \end{code}}
%%%%    The first traversal of block @L2@ starts with the unsound ``fact'' \{x=0\};
%%%%    but analysis of the block propagates the new fact \{x=1\} to @L2@, which joins the
%%%%    existing fact to get \{x=$\top$\}.
%%%%    What if the predicate in the conditional branch were @x<10@ instead
%%%%    of @x==10@?
%%%%    Again the first iteration would begin with the tentative fact \{x=0\}.
%%%%    Using that fact, we would rewrite the conditional branch to an unconditional
%%%%    branch @goto L3@.  No new fact would propagate to @L2@, and we would
%%%%    have successfully (and soundly) eliminated the loop.
%%%%    This example is contrived, but it illustrates that 
%%%%    for best results we should
%%%%    \begin{itemize}
%%%%    \item Perform the rewrites on every iteration.
%%%%    \item Begin each new iteration with the original, virgin graph.
%%%%    \end{itemize}
%%%%    This sort of algorithm is hard to implement in an imperative setting, where rewrites
%%%%    mutate a graph in place.
%%%%    But  with an immutable graph, implementing the algorithm
%%%%    is trivially easy: we simply revert to the original graph at the start
%%%%    of each fixed-point iteration.

\subsection{Checkpointing the client's monad}

\seclabel{ckpoint-monad} \seclabel{checkpoint-monad}

When analyzing a program with loops, a rewrite function could make a change
that later has to be rolled back.
For example, consider constant propagation in this loop, which
computes factorial: 
\begin{code}
     i = 1; `prod = 1;
 L1: if (i >= n) goto L3 else goto L2;
 L2: i = i + 1; prod = prod * i;
     goto L1;
 L3: ...
\end{code}
Function @analyzeAndRewriteFwdBody@ iterates through this graph until
the dataflow facts stop changing.
On~the first iteration, the assignment @i = i + 1@ will 
be analyzed with an incoming fact @i=1@, and the assignment will be
rewritten to the graph @i = 2@.
But~on a later iteration, the incoming fact will increase to @i=@$\top$,
and the~rewrite will no longer be justified.
After each iteration, \hoopl\ starts the next iteration with
\emph{new} facts but with the \emph{original} graph---by~virtue
of using purely functional data structures, rewrites from
previous iterations are automatically rolled back.

But a rewrite function doesn't only produce new graphs; it
can also take a \emph{monadic action}, such as
acquiring a fresh name.
These~actions must also be rolled back, and because the~client chooses
the monad in which the actions take place, the client must provide the
means to roll them back.
\hoopl\ therefore defines a rollback \emph{interface}, which each client must implement;
it is the type class @CkpointMonad@ from
\figref{api-types}:
\begin{code}
class Monad m => `CkpointMonad m where
  type `Checkpoint m
  `checkpoint :: m (Checkpoint m)
  `restart    :: Checkpoint m -> m () 
\end{code}
\hoopl\ calls the @checkpoint@ method at the beginning of an
iteration, then calls the @restart@ method if another iteration is
necessary. 
These operations must obey the following algebraic law:
\begin{code}
 do { s <- checkpoint; m; restart s } == return ()
\end{code}
where @m@~represents any combination of monadic actions that might be
taken by rewrite functions.
(The safest course is to make sure the law holds throughout the entire
monad.)
The~type of the saved checkpoint~@s@ is up to the client;
it~is specified as an associated type of the @CkpointMonad@ class.

\subsection{Correctness} \seclabel{correctness}

% Facts computed by @analyzeAndRewriteFwdBody@ depend on graphs produced by the rewrite
Facts computed by the transfer function depend on graphs produced by the rewrite
function, which in turn depend on facts computed by the transfer function.
How~do we know this algorithm is sound, or if it terminates?
A~proof requires a POPL paper
\cite{lerner-grove-chambers:2002};
here we merely state the conditions for correctness as applied to \hoopl:
\begin{itemize}
\item 
The lattice must have no \emph{infinite ascending chains}; that is,
every sequence of calls to @fact_join@ must eventually return @NoChange@.
\item 
The transfer function must be 
\emph{monotonic}: given a more informative fact in,
it must produce a more informative fact out.
\item 
The rewrite function must be \emph{sound}:
if it replaces a node @n@ by a replacement graph~@g@, then @g@~must be
observationally equivalent to~@n@ under the  
assumptions expressed by the incoming dataflow fact~@f@.
%%\footnote{We do not permit a transformation to change
%%  the @Label@ of a node. We have not found any optimizations
%%  that are prevented (or even affected) by this restriction.}
%
Moreover, analysis of~@g@ must produce output fact(s)
that are at least as informative as the fact(s) produced by applying
the transfer function to~@n@.
%%  \item 
%%  The rewrite function must be \emph{consistent} with the transfer function;
%%  that is, \mbox{@`transfer n f@ $\sqsubseteq$ @transfer g f@}.
For example, if the transfer function says that @x=7@ after the node~@n@,
then after analysis of~@g@,
 @x@~had better still be~@7@.
\item 
%  To ensure termination, 
A transformation that uses deep rewriting
must not return a replacement graph which
contains a node that could be rewritten indefinitely.
\end{itemize}
%%  Without the conditions on monotonicity and consistency,
%%  our algorithm will terminate, 
%%  but there is no guarantee that it will compute
%%  a fixed point of the analysis.  And that in turn threatens the
%%  soundness of rewrites based on possibly bogus ``facts''.
Under these conditions, the algorithm terminates and is
sound.
%%  
%%  \begin{itemize} 
%%  \item
%%  The algorithm terminates.  The fixed-point loop must terminate because the 
%%  lattice has no infinite ascending chains. And the client is responsible
%%  for avoiding infinite recursion when deep rewriting is used.
%%  \item 
%%  The algorithm is sound.  Why? Because if each rewrite is sound (in the sense given above), 
%%  then applying a succession of rewrites is also sound.
%%  Moreover, a~sound analysis of the replacement graph
%%  may generate only dataflow facts that could have been
%%  generated by a more complicated analysis of the original graph.
%%  \end{itemize}
%%  
%%  \finalremark{Doesn't the rewrite have to be have the following property:
%%  for a forward analysis/transform, if (rewrite P s) = Just s',
%%  then (transfer P s $\sqsubseteq$ transfer P s').
%%  For backward: if (rewrite Q s) = Just s', then (transfer Q s' $\sqsubseteq$ transfer Q s).
%%  Works for liveness.
%%  ``It works for liveness, so it must be true'' (NR).
%%  If this is true, it's worth a QuickCheck property!
%%  }%
%%  \finalremark{Version 2, after further rumination.  Let's define
%%  $\scriptstyle \mathit{rt}(f,s) = \mathit{transform}(f, \mathit{rewrite}(f,s))$.
%%   Then $\mathit{rt}$ should
%%  be monotonic in~$f$.  We think this is true of liveness, but we are not sure
%%  whether it's just a generally good idea, or whether it's actually a 
%%  precondition for some (as yet unarticulated) property of \ourlib{} to hold.}%

%%%%    \simon{The rewrite functions must presumably satisfy
%%%%    some monotonicity property.  Something like: given a more informative
%%%%    fact, the rewrite function will rewrite a node to a more informative graph
%%%%    (in the fact lattice.).
%%%%    \textbf{NR}: actually the only obligation of the rewrite function is
%%%%    to preserve observable behavior.  There's no requirement that it be
%%%%    monotonic or indeed that it do anything useful.  It just has to
%%%%    preserve semantics (and be a pure function of course).
%%%%    \textbf{SLPJ} In that case I think I could cook up a program that
%%%%    would never reach a fixed point. Imagine a liveness analysis with a loop;
%%%%    x is initially unused anywhere.
%%%%    At some assignment node inside the loop, the rewriter behaves as follows: 
%%%%    if (and only if) x is dead downstream, 
%%%%    make it alive by rewriting the assignment to mention x.
%%%%    Now in each successive iteration x will go live/dead/live/dead etc.  I
%%%%    maintain my claim that rewrite functions must satisfy some
%%%%    monotonicity property.
%%%%    \textbf{JD}: in the example you cite, monotonicity of facts at labels
%%%%    means x cannot go live/dead/live/dead etc.  The only way we can think
%%%%    of not to terminate is infinite ``deep rewriting.''
%%%%    }




\section{\ourlib's implementation}
\seclabel{implementation}
\seclabel{engine}
\seclabel{dfengine}

\secref{making-simple}
gives a client's-eye view of \hoopl, showing how to 
create analyses and transformations.
\hoopl's interface is simple, but 
the \emph{implementation} of interleaved analysis and rewriting is~not.  
\citet{lerner-grove-chambers:2002} 
do not describe their implementation.  We have written
at least three previous implementations, all of which
were long and hard to understand, and only one of which
provided compile-time guarantees about open and closed shapes.
We are not confident that any of our previous implementations are correct.

In this paper we describe a new implementation.  It is elegant and short
(about a third of the size of our last attempt), and it offers
strong compile-time guarantees about shapes.  
\finalremark{Wanted: enumerate the critical components and give each one's size}
%
We describe the implementation of \emph{forward} 
analysis and transformation.
The implementations of backward analysis and transformation are
exactly analogous and are included in \hoopl.

We~also, starting in \secref{first-debugging-section}, explain how we
isolate errors in faulty optimizers, and how the fault-isolation
machinery is integrated with the rest of the implementation.





\subsection{Overview}


%%  We  on @analyzeAndRewriteFwd@, whose type is more general
%%  than that of  @analyzeAndRewriteFwdBody@:
%%  \begin{smallcode}
%%  `analyzeAndRewriteFwd
%%   :: forall m n f e x. (FuelMonad m, NonLocal n)
%%   => FwdPass m n f    -- lattice, transfers, rewrites
%%   -> MaybeC e [Label] -- entry points for a closed graph
%%   -> Graph n e x      -- the original graph
%%   -> Fact e f         -- fact(s) flowing into the entry/entries
%%   -> m (Graph n e x, FactBase f, MaybeO x f)
%%  \end{smallcode}
%%  We analyze graphs of all shapes; a single @FwdPass@ may be used with
%%  multiple shapes.
%%  If a graph is closed on entry, a list of entry points must be
%%  provided;
%%  if the graph is open on entry, it must be the case that the
%%  implicit entry point is the only entry point.
%%  The fact or set of facts flowing into the entries is similarly
%%  determined by the shape of the entry point.
%%  Finally, the result is a rewritten graph, a @FactBase@ that gives a
%%  fixed point of the analysis on the rewritten graph, and if the graph
%%  is open on exit, an ``exit  fact'' flowing out.

Instead of the interface function @analyzeAndRewriteFwdBody@, we present
the private function @arfGraph@, which is short for ``analyze and rewrite
forward graph:''
\begin{smallfuzzcode}{15.1pt}
`arfGraph
 :: forall m n f e x. (CkpointMonad m, NonLocal n)
 => FwdPass m n f    -- lattice, transfers, rewrites
 -> MaybeC e [Label] -- entry points for a closed graph
 -> Graph n e x      -- the original graph
 -> Fact e f         -- fact(s) flowing into entry/entries
 -> m (DG f n e x, Fact x f)
\end{smallfuzzcode}
Function @arfGraph@ has a more general type than
the function @analyzeAndRewriteFwdBody@ % praying for a decent line break
because @arfGraph@ is used recursively
to analyze graphs of all shapes.
If a graph is closed on entry, a list of entry points must be
provided;
if the graph is open on entry,
the graph's entry sequence must be the only entry point.
The~graph's shape on entry also determines the type of fact or facts
flowing in.
Finally, the result is a ``decorated graph''
@DG f n e x@,
and if the graph
is open on exit, an ``exit  fact'' flowing out.

%% \simon{I suggest (a) putting the paragraph break one sentence earlier,
%%% so that this para is all about DGs.}
%%%  NR: previous para is about the type of arfGraph; I don't want to
%%%  leave the result type dangling.  I hope the opening sentence of
%%%  this para suggests that the para is all about DGs.
A~``decorated graph'' is one in which each block is decorated with the
fact that holds at the start of the block.
@DG@ actually shares a representation with @Graph@,
which is possible because the definition of
@Graph@ in \figref{graph} contains a white lie: @Graph@~is a type
synonym for an underlying type @`Graph'@, which takes the type of block
as an additional parameter.
(Similarly, function @gSplice@ in \secref{gSplice} is actually a
higher-order function that takes a block-concatenation function as a
parameter.) 
The truth about @Graph@ and @DG@ is as follows:
\smallverbatiminput{dg}
% defn DG
% defn DBlock
Type~@DG@ is internal to \hoopl; it is not seen by any client.
To~convert a~@DG@ to the @Graph@ and @FactBase@
that are returned by the API function @analyzeAndRewriteFwdBody@,
we use a 12-line function:
\begin{smallfuzzcode}{2.5pt}
`normalizeGraph
 :: NonLocal n => DG f n e x -> (Graph n e x, FactBase f)
\end{smallfuzzcode}

Function @arfGraph@ is implemented as follows:
\begingroup
\def\^{\\[-6pt]}%
\hfuzz=15.1pt
\begin{smallttcode}
arfGraph ^pass entries = graph
 where\^
 node :: forall e x . (ShapeLifter e x) 
      => n e x       -> f        -> m (DG f n e x, Fact x f)\^
 block:: forall e x . 
         Block n e x -> f        -> m (DG f n e x, Fact x f)\^
 body :: [Label] -> LabelMap (Block n C C)
                     -> Fact C f -> m (DG f n C C, Fact C f)\^
 `graph:: Graph n e x -> Fact e f -> m (DG f n e x, Fact x f)\^
 ... definitions of 'node', 'block', 'body', and 'graph' ...
\end{smallttcode}
The four auxiliary functions help us separate concerns: for example, only 
\endgroup
@node@ knows about rewrite functions,
and only @body@ knows about fixed points.
%%  All four functions have access to the @FwdPass@ and any entry points
%%  that are passed to @arfGraph@.
%%  These functions also have access to type variables bound by
%%  @arfGraph@:
%%  @n@~is the type of nodes; @f@~is the type of facts;
%%  @m@~is the monad used in the rewrite functions of the @FwdPass@;
%%  and
%%  @e@~and~@x@ give the shape of the graph passed to @arfGraph@.
%%  The types of the inner functions are 
%%  \begin{smallcode}
%%  \end{smallcode}
Each auxiliary function works the same way: it~takes a ``thing'' and
returns an \emph{extended fact transformer}.
An~extended fact transformer takes dataflow fact(s) coming into
the ``thing,'' and it returns an output fact.
It~also returns a decorated graph representing the (possibly
rewritten) ``thing''---that's the \emph{extended} part.
Finally, because rewrites are monadic,
every extended fact transformer is monadic.

%%%%    \begin{figure}
%%%%    SIMON HAS ASKED IF TYPE SYNONYMS MIGHT IMPROVE THINGS FOR EXTENDED
%%%%    FACT TRANSFORMERS.  JUDGE FOR YOURSELF.
%%%%    FIRST, SOMETHING THAT IS SOMEWHAT READABLE BUT IS NOT LEGAL HASKELL:
%%%%    \begin{smallcode}
%%%%     type EFFX ipt e x = ipt -> m (DG f n e x, Fact x f) 
%%%%        -- extended forward fact transformer
%%%%    
%%%%     node  :: forall e x . (ShapeLifter e x) 
%%%%           => n e x       -> EFFX f          e x
%%%%     block :: forall e x . 
%%%%              Block n e x -> EFFX f          e x
%%%%     body  :: [Label] -> LabelMap (Block n C C)
%%%%                          -> EFFX (Fact C f) C C
%%%%     graph :: Graph n e x -> EFFX (Fact e f) e x
%%%%    \end{smallcode}
%%%%    IF WE MAKE IT LEGAL HASKELL, IT BECOMES COMPLETELY HOPELESS:
%%%%    \begin{smallcode}
%%%%     type EFFX m n f ipt e x = ipt -> m (DG f n e x, Fact x f) 
%%%%        -- extended forward fact transformer
%%%%    
%%%%     node  :: forall e x . (ShapeLifter e x) 
%%%%           => n e x       -> EFFX m n f f e x
%%%%     block :: forall e x . 
%%%%              Block n e x -> EFFX m n f f e x
%%%%     body  :: [Label] -> LabelMap (Block n C C)
%%%%                          -> EFFX m n f (Fact C f) C C
%%%%     graph :: Graph n e x -> EFFX m n f (Fact e f) e x
%%%%    \end{smallcode}
%%%%    \caption{EXPERIMENTS WITH TYPE SYNONYMS}
%%%%    \end{figure}
%%%%    


The types of the
\ifpagetuning
\else
four
\fi
extended fact transformers are not quite
identical:
\begin{itemize}
\item
Extended fact transformers for nodes and blocks have the same type;
like forward transfer functions,
they expect a fact~@f@ rather than the more general @Fact e f@
required for a graph.
Because a node or a block has
exactly one fact flowing into the entry, it is easiest  simply to pass
that fact.
\item
Extended fact transformers for graphs have the most general type,
as expressed using @Fact@:
if the graph is open on entry, its~fact transformer expects a
single fact;
if the graph is closed on entry, its~fact transformer expects a
@FactBase@.
\item
Extended fact transformers for bodies have the same type as 
extended fact transformers for closed/closed graphs.
\end{itemize}


Function @arfGraph@ and its four auxiliary functions comprise a cycle of
mutual recursion: 
@arfGraph@ calls @graph@;
@graph@ calls @body@ and @block@;
@body@ calls @block@;
@block@ calls @node@;
and 
@node@ calls @arfGraph@.
These five functions do three different kinds of work:
compose extended fact transformers, analyze and rewrite nodes, and compute
fixed points.



\subsection{Analyzing blocks and graphs by composing extended fact transformers}
\seclabel{block-impl}

Extended fact transformers compose nicely.
For example, @block@ is implemented thus:
% we need the foralls
\begin{smallcode}
  `block :: forall e x .
            Block n e x -> f -> m (DG f n e x, Fact x f)
  block (BFirst  n)  = node n
  block (BMiddle n)  = node n
  block (BLast   n)  = node n
  block (BCat b1 b2) = block b1 `cat` block b2
\end{smallcode}
The composition function @cat@ feeds facts from one extended fact
transformer to another, and it splices decorated graphs.
\smallverbatiminput{cat}
% defn cat
% local ft1
% local ft2
(Function @`dgSplice@ is the same splicing function used for an ordinary
@Graph@, but it uses a one-line block-concatenation function suitable
for @DBlock@s.)
The name @cat@ comes from the concatenation of the decorated graphs,
but it is also appropriate because the style in which it is used is
reminiscent of @concatMap@, with the @node@ and @block@ functions
playing the role of @map@.

\seclabel{concat-map-style}

Function @graph@ is much like @block@, but it has more cases.


%%%%    
%%%%    \begin{itemize} 
%%%%    \item 
%%%%    The @arfNode@ function processes nodes (\secref{arf-node}).
%%%%    It handles the subtleties of interleaved analysis and rewriting,
%%%%    and it deals with fuel consumption.  It calls @arfGraph@ to analyze
%%%%    and transform rewritten graphs.
%%%%    \item 
%%%%    Based on @arfNode@ it is extremely easy to write @arfBlock@, which lifts
%%%%    the analysis and rewriting from nodes to blocks (\secref{arf-block}).
%%%%    
%%%%    
%%%%    
%%%%    \item
%%%%    Using @arfBlock@ we define @arfBody@, which analyzes and rewrites a
%%%%    @Body@: a~group of closed/closed blocks linked by arbitrary
%%%%    control flow.
%%%%    Because a @Body@ is
%%%%    always closed/closed and does not take shape parameters, function
%%%%    @arfBody@ is less polymorphic than the others; its type is what
%%%%    would be obtained by expanding and specializing the definition of
%%%%    @ARF@ for a @thing@ which is always closed/closed and is equivalent to
%%%%    a @Body@.
%%%%    
%%%%    Function @arfBody@ takes care of fixed points (\secref{arf-body}).
%%%%    \item 
%%%%    Based on @arfBody@ it is easy to write @arfGraph@ (\secref{arf-graph}).
%%%%    \end{itemize}
%%%%    Given these functions, writing the main analyzer is a simple
%%%%    matter of matching the external API to the internal functions:
%%%%    \begin{code}
%%%%      `analyzeAndRewriteFwdBody
%%%%         :: forall n f. NonLocal n
%%%%         => FwdPass n f -> Body n -> FactBase f
%%%%         -> FuelMonad (Body n, FactBase f)
%%%%    
%%%%      analyzeAndRewriteFwdBody pass ^body facts
%%%%        = do { (^rg, _) <- arfBody pass body facts
%%%%             ; return (normalizeBody rg) }
%%%%    \end{code}
%%%%     
%%%%    \subsection{From nodes to blocks} \seclabel{arf-block}
%%%%    \seclabel{arf-graph}
%%%%    
%%%%    We begin our explanation with the second task:
%%%%    writing @arfBlock@, which analyzes and transforms blocks.
%%%%    \begin{code}
%%%%    `arfBlock :: NonLocal n => ARF (Block n) n
%%%%    arfBlock pass (BUnit node) f 
%%%%      = arfNode pass node f
%%%%    arfBlock pass (BCat b1 b2) f 
%%%%      = do { (g1,f1) <- arfBlock pass b1 f  
%%%%           ; (g2,f2) <- arfBlock pass b2 f1 
%%%%           ; return (g1 `DGCatO` g2, f2) }
%%%%    \end{code}
%%%%    The code is delightfully simple.
%%%%    The @BUnit@ case is implemented by @arfNode@.
%%%%    The @BCat@ case is implemented by recursively applying @arfBlock@ to the two
%%%%    sub-blocks, threading the output fact from the first as the 
%%%%    input to the second.  
%%%%    Each recursive call produces a rewritten graph;
%%%%    we concatenate them with @DGCatO@. 
%%%%    
%%%%    Function @arfGraph@ is equally straightforward:
%%%%    XXXXXXXXXXXXXXX
%%%%    The pattern is the same as for @arfBlock@: thread
%%%%    facts through the sequence, and concatenate the results.
%%%%    Because the constructors of type~@DG@ are more polymorphic than those
%%%%    of @Graph@, type~@DG@ can represent
%%%%    graphs more simply than @Graph@; for example, each element of a
%%%%    @GMany@ becomes a single @DG@ object, and these @DG@ objects are then 
%%%%    concatenated to form a single result of type~@DG@.
%%%%    

\subsection{Analyzing and rewriting nodes} \seclabel{arf-node}

The @node@ function is where we interleave analysis with rewriting:
\smallfuzzverbatiminput{15.1pt}{node}
% defn ShapeLifter
% defn singletonDG
% defn fwdEntryFact
% defn fwdEntryLabel
% defn ftransfer
% defn frewrite
% local pass'
% local grw
%
Function @node@ uses @frewrite@ to extract the rewrite function from
@pass@, 
and applies the rewrite function to the node~@n@ and the incoming fact~@f@.
The result, @grw@, is 
scrutinized by the @case@ expression.

In the @Nothing@ case, no rewrite takes place.
We~return node~@n@ and its incoming fact~@f@
as the decorated graph @singletonDG f n@.
To produce the outgoing fact, we apply the transfer function
@ftransfer pass@ to @n@~and~@f@.

In the @Just@ case, we receive a replacement
graph~@g@ and a new rewrite function~@rw@, as specified by the model
in \secref{rewrite-model}.
We~use @rw@ to recursively analyze and rewrite~@g@ with @arfGraph@.  
This analysis uses @pass'@, which contains the original lattice and transfer
function from @pass@, together with~@rw@.
Function @fwdEntryFact@ converts fact~@f@ from the type~@f@,
which @node@ expects, to the type @Fact e f@, which @arfGraph@ expects.

As you see, several functions called in @node@ are overloaded over a
(private) class @ShapeLifter@.  Their implementations depend
on the open/closed shape of the node.
By design, the shape of a node is known statically everywhere @node@
is called, so
this use of @ShapeLifter@ is specialized
away by the compiler.

%%  And that's it!  If~the client wanted deep rewriting, it is
%%  implemented by the call to @arfGraph@;
%%  if the client wanted
%%  shallow rewriting, the rewrite function will have returned
%%  @noFwdRw@ as~@rw@, which is implanted in @pass'@
%%  (\secref{shallow-vs-deep}).


\subsection{Fixed points} \seclabel{arf-body}

The fourth and final auxiliary function of @arfGraph@ is
@body@, which iterates to a fixed point.
This part of the implementation is the only really tricky part, and it is
cleanly separated from everything else:
\smallfuzzverbatiminput{2.5pt}{bodyfun}
% defn body
% local do_block
% local blocks
% local lattice
% local entryFact
% local entries
% local init_fbase
% local blockmap
% local fb
Function @getFact@ looks up a fact by its label.
If the label is not found,
@getFact@ returns
the bottom element of the lattice: 
\begin{smallcode}
`getFact :: DataflowLattice f -> Label -> FactBase f -> f
\end{smallcode}
Function @forwardBlockList@ takes a list of possible entry points and 
a finite map from labels to blocks.
It returns a list of
blocks, sorted into an order that makes forward dataflow efficient.\footnote
{The order of the blocks does not affect the fixed point or any other
result; it affects only the number of iterations needed to
reach the fixed point.}
\begin{smallcode}
 `forwardBlockList 
   :: NonLocal n 
   => [Label] -> LabelMap (Block n C C) -> [Block n C C]
\end{smallcode}
For
example, if the entry point is at~@L2@, and the block at~@L2@ 
branches to~@L1@, but not vice versa, then \hoopl\ will reach a fixed point
more quickly if we process @L2@ before~@L1@.  
To~find an efficient order, @forwardBlockList@ uses
the methods of the @NonLocal@ class---@entryLabel@ and @successors@---to
perform a reverse postorder depth-first traversal of the control-flow graph.
%%
%%The @NonLocal@ type-class constraint on~@n@ propagates to all the
%%@`arfThing@ functions.
%%  paragraph carrying too much freight
%%

The rest of the work is done by @`fixpoint@, which is shared by
both forward and backward analyses:
\smallfuzzverbatiminput{2.5pt}{fptype}
% defn Direction
% defn Fwd
% defn Bwd
Except for the @Direction@ passed as the first argument,
the type signature tells the story.
The third argument is an extended fact transformer for a single block; 
@fixpoint@ applies that function successively to each block in the list
passed as the fourth argument.
The result is an extended fact transformer for the list.

The extended fact transformer returned by @fixpoint@
 maintains a
 ``current @FactBase@''
which grows monotonically:
as each block is analyzed,
the block's input fact is taken from
the current @FactBase@,
and the current @FactBase@
is
augmented with the facts that flow out of the block.
%
The initial value of the current @FactBase@ is the input @FactBase@, 
and
the extended fact transformer
iterates over the blocks until the current @FactBase@ 
stops changing.



Implementing @fixpoint@ requires about 90 lines,
formatted narrowly for display in one column.
%%  
%%  for completeness, it
%%  appears in Appendix~\ref{app:fixpoint}.  
The~code is mostly straightforward, although we try to be clever
about deciding when a new fact means that another iteration 
will be required.
\finalremark{Rest of this \S\ is a candidate to be cut}
There is one more subtle point worth mentioning, which we highlight by 
considering a forward analysis of this graph, where execution starts at~@L1@:
\begin{code}
  L1: x:=3; goto L4
  L2: x:=4; goto L4
  L4: if x>3 goto L2 else goto L5
\end{code}
Block @L2@ is unreachable. 
But if we \naively\ process all the blocks (say in 
order @L1@, @L4@, @L2@), then we will start with the bottom fact for @L2@, propagate
\{@x=4@\} to @L4@, where it will join with \{@x=3@\} to yield
\{@x=@$\top$\}.  
Given @x=@$\top$, the
conditional in @L4@ cannot be rewritten, and @L2@~seems reachable.  We have
lost a good optimization.

Function @fixpoint@ solves this problem 
by analyzing a block only if the block is
reachable from an entry point.
This trick is safe only for a forward analysis, which
 is why
@fixpoint@ takes a @Direction@ as its first argument.

%%  Although the trick can be implemented in just a couple of lines of
%%  code, the reasoning behind it is quite subtle---exactly the sort of
%%  thing that should be implemented once in \hoopl, so clients don't have
%%  to worry about it.


\subsection{Throttling rewriting using ``optimization fuel''}
\seclabel{vpoiso}
\seclabel{fuel}
\seclabel{whalley-from-s2}
\seclabel{first-debugging-section}

When optimization produces a faulty program,
we use Whalley's \citeyearpar{whalley:isolation} technique to find the fault:
given a program that fails when compiled with optimization,
a binary search on the number of rewrites
finds an~$n$ such that the program works after $n-1$ rewrites
but fails after $n$~rewrites.
The $n$th rewrite is faulty.
As~alluded to at the end of \secref{debugging-introduced}, this
technique enables us to debug complex optimizations by
identifying one single rewrite that is faulty.

This debugging technique requires the~ability to~limit
the number of~rewrites.
We limit rewrites using \emph{optimization fuel}.
Each rewrite consumes one unit of fuel,
and when fuel is exhausted, all rewrite functions return @Nothing@.
To~debug, we do binary search on the amount of fuel.

The supply of fuel is encapsulated in the @FuelMonad@ type class (\figref{api-types}),
which must be implemented by the client's monad @m@.
To~ensure that each rewrite consumes one~unit of~fuel,
@mkFRewrite@ wraps the client's rewrite function, which is oblivious to fuel,
in~another function that satisfies the following contract:
\begin{itemize}
\item 
If the fuel supply is empty, the wrapped function always returns @Nothing@. 
\item
If the wrapped function returns @Just g@, it has the monadic effect of
reducing the fuel supply by one unit.
\end{itemize}

%%%%    \seclabel{fuel-monad}
%%%%    
%%%%    \begin{ntext}
%%%%    \subsection{Rewrite functions}
%%%%    
%%%%    
%%%%    
%%%%    \begin{code}
%%%%    `withFuel :: FuelMonad m => Maybe a -> m (Maybe a)
%%%%    \end{code}
%%%%    
%%%%    
%%%%    as expressed by the
%%%%    @FwdRew@ type returned by a @FwdRewrite@ (\figref{api-types}).
%%%%    The first component of the @FwdRew@ is the replacement graph, as discussed earlier.
%%%%    The second component, $\rw$, is a 
%%%%    \emph{new rewrite function} to use when recursively processing
%%%%    the replacement graph. 
%%%%    For shallow rewriting this new function is
%%%%    the constant @Nothing@ function; for deep rewriting it is the original
%%%%    rewrite function.
%%%%    While @mkFRewrite@ allows for general rewriting, most clients will
%%%%    take advantage of \hoopl's support for these two common cases:
%%%%    \begin{smallcode}
%%%%    `deepFwdRw, `shallowFwdRw
%%%%       :: Monad m 
%%%%       => (forall e x . n e x -> f -> m (Maybe (Graph n e x)) 
%%%%       -> FwdRewrite m n f
%%%%    \end{smallcode}
%%%%    \end{ntext}



\section {Related work} \seclabel{related}

While there is a vast body of literature on
dataflow analysis and optimization,
relatively little can be found on
the \emph{design} of optimizers, which is the topic of this paper.
We therefore focus on the foundations of dataflow analysis
and on the implementations of some comparable dataflow frameworks.

\paragraph{Foundations}

When transfer functions are monotone and lattices are finite in height,
iterative dataflow analysis converges to a fixed point
\cite{kam-ullman:global-iterative-analysis}. 
If~the lattice's join operation distributes over transfer
functions,
this fixed point is equivalent to a join-over-all-paths solution to
the recursive dataflow equations
\cite{kildall:unified-optimization}.\footnote
{Kildall uses meets, not joins.  
Lattice orientation is a matter of convention, and conventions have changed.
We use Dana Scott's
orientation, in which higher elements carry more information.}
\citet{kam-ullman:monotone-flow-analysis} generalize to some
monotone functions.
Each~client of \hoopl\ must guarantee monotonicity.

\citet{cousot:abstract-interpretation:1977,cousot:systematic-analysis-frameworks}
introduce abstract interpretation as a technique for developing
lattices for program analysis.
\citet{steffen:data-flow-analysis-model-checking:1991} shows that
a dataflow analysis can be implemented using model checking;
\citet{schmidt:data-flow-analysis-model-checking}
expands on this~result by showing that
an all-paths dataflow problem can be viewed as model checking an
abstract interpretation.

\citet{marlowe-ryder:properties-data-flow-frameworks} 
present a survey of different methods for performing dataflow analyses,
with emphasis on theoretical results.
\citet{muchnick:compiler-implementation} 
presents many examples of both particular analyses and related
algorithms.


The soundness of interleaving analysis and transformation,
even when not all speculative transformations are performed on later
iterations, is shown by
\citet{lerner-grove-chambers:2002}.


\paragraph{Frameworks}
Most dataflow frameworks support only analysis, not transformation.
The framework computes a fixed point of transfer functions, and it is
up to the client of 
the framework to use that fixed point for transformation.
Omitting transformation makes it much easier to build frameworks,
and one can find a spectrum of designs.
We~describe  two representative
designs, then move on to frameworks that do interleave
analysis and transformation.

The Soot framework is designed for analysis of Java programs
\cite{hendren:soot:2000}.
%%  {This citation is probably the
%%  best for Soot in general, but there doesn't appear 
%%   to be any formal publication that actually details the dataflow
%%   framework part. ---JD}
While Soot's dataflow library supports only analysis, not
 transformation, we found much 
 to admire in its design.
Soot's library is abstracted over the representation of
the control-flow graph and the representation of instructions.
Soot's interface for defining lattice and analysis functions is
like our own, 
although because Soot is implemented in an imperative style, 
additional functions are needed to copy lattice elements.


The CIL toolkit \cite{necula:cil:2002}
%%  \finalremark{No good citation
%%  for same reason as Soot above ---JD}
supports both analysis and rewriting of C~programs,
but rewriting is clearly distinct from analysis:
one runs an analysis to completion and then rewrites based on the
results. 
The framework is limited to one representation of control-flow graphs
and one representation of instructions, both of which are provided by
the framework.
The~API is complicated;
much of the complexity is needed to enable the client to
affect which instructions 
the analysis iterates over.


%%  \finalremark{FYI, LLVM has Pass Managers that try to control the
%%  order of passes, 
%%    but I'll be darned if I can find anything that might be termed a
%%    dataflow framework.} 

The Whirlwind compiler contains the dataflow framework implemented
by \citet{lerner-grove-chambers:2002}, who were the first to 
interleave analysis and transformation.
Their implementation is much like our early efforts:
it is a complicated mix of code that simultaneously manages interleaving,
deep rewriting, and fixed-point computation.
By~separating these tasks, 
our implementation simplifies the problem dramatically.
Whirlwind's implementation also suffers from the difficulty of
maintaining pointer invariants in a mutable representation of
control-flow graphs, a problem we have discussed elsewhere
\cite{ramsey-dias:applicative-flow-graph}. 

Because speculative transformation is difficult in an imperative setting,
Whirlwind's implementation is split into two phases.
The first phase runs the interleaved analyses and transformations
to compute the final dataflow facts and a representation of the transformations
that should be applied to the input graph.
The second phase executes the transformations.
In~\hoopl, because control-flow graphs are immutable, speculative transformations
can be applied immediately, and there is no need
for a phase distinction.

%%% % repetitious...
%%%
%%%   \ourlib\ also improves upon Whirlwind's dataflow framework by providing
%%%   new support for the optimization writer:
%%%   \begin{itemize}
%%%   \item Using static type guarantees, \hoopl\ rules out a whole
%%%     class of possible bugs: transformations that produced malformed
%%%     control-flow graphs.
%%%   \item Using dynamic testing,
%%%     we can isolate the rewrite that transforms a working program
%%%     into a faulty program,
%%%     using Whalley's  \citeyearpar{whalley:isolation} fault-isolation technique.
%%%   \end{itemize}

%% what follows now looks redundant with discussion below ---NR

%%  In previous work \cite{ramsey-dias:applicative-flow-graph}, we
%%  described a zipper-based representation of control-flow 
%%  graphs, stressing the advantages
%%  of immutability.
%%  Our new representation, described in \secref{graph-rep}, is a significant improvement:
%%  \begin{itemize}
%%  \item
%%  We can concatenate nodes, blocks, and graphs in constant time.
%%  %Previously, we had to resort to Hughes's
%%  %\citeyearpar{hughes:lists-representation:article} technique, representing
%%  %a graph as a function.
%%  \item
%%  We can do a backward analysis without having
%%  to ``unzip'' (and allocate a copy of) each block.
%%  \item
%%  Using GADTs, we can represent a flow-graph
%%  node using a single type, instead of the triple of first, middle, and
%%  last types used in our earlier representation.
%%  This change simplifies the interface significantly:
%%  instead of providing three transfer functions and three rewrite
%%  functions per pass---one for 
%%  each type of node---a client of \hoopl\ provides only one transfer
%%  function and one rewrite function per pass.
%%  \item
%%  Errors in concatenation are ruled out at
%%  compile-compile time by Haskell's static
%%  type system.
%%  In~earlier implementations, such errors were not detected until
%%  the compiler~ran, at which point we tried to compensate
%%  for the errors---but
%%  the compensation code harbored subtle faults,
%%  which we discovered while developing a new back end
%%  for the Glasgow Haskell Compiler.
%%  \end{itemize}
%%  
%%  The implementation of \ourlib\ is also much better than
%%  our earlier implementations.
%%  Not only is the code simpler conceptually,
%%  but it is also shorter:
%%  our new implementation is about a third as long
%%  as the previous version, which is part of GHC, version 6.12.




\section{Performance considerations}

Our work on \hoopl\ is too new for us to be able to say much
about performance.
It's~important to know how well \hoopl\ performs, but the
question is comparative, and there isn't another library we can compare
\hoopl\ with.
For example, \hoopl\ is not a drop-in  replacement for an existing
component of GHC; we introduced \hoopl\ to GHC as part of a
major refactoring of GHC's back end.
With \hoopl,  GHC seems about 15\%~slower than
the previous~GHC, and we
don't know what portion of the slowdown can be attributed to the
optimizer.
%
We~can say that the costs of using \hoopl\ seem reasonable;
there is no ``big performance hit.''
And~a somewhat similar library, written in an \emph{impure} functional
language, actually improved performance in an apples-to-apples
comparison with a library using a mutable control-flow graph
\cite{ramsey-dias:applicative-flow-graph}. 

Although thorough evaluation of \hoopl's performance must await
future work, we can identify some design decisions that might affect
performance. 
\begin{itemize}
\item
In \figref{graph}, we show a single concatenation operator for blocks.
Using this representation, a block of $N$~nodes is represented using
$2N-1$ heap objects.
We~have also implemented a representation of blocks that include
``cons-like'' and ``snoc-like'' constructors;
this representation requires only $N+1$ heap objects.
We~don't know how this choice affects performance.
\item
In \secref{engine}, the @body@ function analyzes and (speculatively)
rewrites the body of a control-flow graph, and @fixpoint@ iterates
this analysis until it reaches a fixed point.
Decorated graphs computed on earlier iterations are thrown away.
For~each decorated graph of $N$~nodes, 
at least $2N-1$~thunks are allocated; they correspond to applications of
@singletonDG@ in~@node@ and of @dgSplice@ in~@cat@.
In~an earlier version of \hoopl, this overhead was
eliminated by splitting @arfGraph@ into two functions: one to compute the
fixed point, and the other to produce the rewritten graph.
The single @arfGraph@ is simpler and easier
to maintain; we don't know the extra thunks matter.
\item
The representation of a forward-transfer function is private to
\hoopl.
Two representations are possible:
we may store a triple of functions, one for each shape a node may
have;
or we may store a single, polymorphic function.
If~we use triples throughout, the costs are straightforward, but the
code is complex.
If~we use a single, polymorphic function, we sometimes have to use a
\emph{shape classifier} (supplied by the client) when composing
transfer functions.
Using a shape classifier may introduce extra @case@ discriminations
every time a transfer function or rewrite function is applied to a
node.
We~don't know how these extra discriminations might affect
performance.
\end{itemize}
In summary, \hoopl\ performs well enough for use in~GHC,
but there is much we don't know.
We have no evidence that \emph{any} of the decisions above 
measurably affects performance---systematic investigation 
is indicated.



\section{Discussion}

We built \hoopl\ in order to combine three good ideas (interleaved
analysis and transformation, optimization fuel, and an applicative
control-flow graph) in a way that could easily be reused by many
compiler writers.
To~evaluate how well we succeeded, we examine how \hoopl\ has been
used,
we~examine the API, and we examine the implementation.
We~also sketch some more alternatives.

\paragraph{Using \hoopl}

As~suggested by the constant-propagation example in
\figref{const-prop}, \hoopl\ makes it easy to implement many standard
dataflow analyses.
Students using \hoopl\ in a class at Tufts were able to implement
such optimizations as lazy code motion \cite{knoop:lazy-code-motion} 
and induction-variable elimination
\cite{cocke-kennedy:operator-strength-reduction} in just a few weeks.
Students at Yale and at Portland State have also succeeded in building
a variety of optimizations.

\hoopl's graphs can support optimizations beyond classic
dataflow. 
For example, in~GHC, \hoopl's  graphs are used 
to implement optimizations based on control flow,
such as eliminating branch chains.

\hoopl\ is SSA-neutral:
although we know of no attempt to use
\hoopl\ to establish or enforce SSA~invariants,
\hoopl\ makes it easy to include $\phi$-functions in the
representation of first nodes,
and if a transformation preserves SSA~invariants, it will continue to do
so when implemented in \hoopl.

\paragraph{Examining the API}

We hope that our presentation of the API in \secref{api} speaks for
itself,
but there are a couple of properties we think are worth highlighting.
First, it's a good sign that the API provides many higher-order
combinators that make it easier to write client code. % with simple, expressive types.
We~have had space to mention only a few: 
@extendJoinDomain@, 
@joinMaps@,
@thenFwdRw@, @iterFwdRw@, @deepFwdRw@, and @pairFwd@.

Second,
the static encoding of open and closed shapes at compile time worked
out well.
% especially because it applies equally to nodes, blocks, and graphs.
Shapes may
seem like a small refinement, but they helped eliminate a number of
bugs from GHC, and we expect them to help other clients too.
GADTs are a convenient way to express shapes, and for clients
written in Haskell, they are clearly appropriate.
If~one wished to port \hoopl\ to a language without GADTs,
many of the benefits could be realized by making the shapes phantom
types, but without GADTs, pattern matching would be significantly more
tedious and error-prone.


% An~advantage of our ``shapely'' node API is that a client can
% write a \emph{single} transfer function that is polymorphic in shape.
% To~make this design work, however, we \emph{must} have
% the type-level function 
% @Fact@ (\figref{api-types}), to express how incoming
% and outgoing facts depend on the shape of a node.
% Without type-level functions, we would have had to force clients to
% use \emph{only} the triple-of-functions interface described in
% \secref{triple-of-functions}.

\paragraph{Examining the implementation}

If you are thinking of adopting \hoopl, you should consider not
only whether you like the API, but whether, in case of emergency, you
could maintain the implementation.
We~believe that \secref{dfengine} sketches enough to show that \hoopl's
implementation is a clear improvement over previous implementations
of similar ideas.
% The implementation is more difficult to evaluate than the~API.
% Previous implementations of similar ideas have rolled the problems
% into a big ball of mud.
By~decomposing our implementation into @node@, @block@, @body@,
@graph@, @cat@, @fixpoint@, and @mkFRewrite@, we have clearly separated
multiple concerns:
interleaving analysis with rewriting,
throttling rewriting using optimization fuel,
and 
computing a fixed point using speculative rewriting.
Because of this separation of concerns,
we believe our implementation will be much easier to maintain than
anything that preceded~it.

%%  Another good sign is that we have been able to make substantial
%%  changes in the implementation without changing the API.
%%  For example, in addition to ``@concatMap@ style,'' we have also
%%  implemented @arfGraph@ in ``fold style'' and in continuation-passing
%%  style.
%%  Which style is preferred is a matter of taste, and possibly
%%  a matter of  performance.




\iffalse

(We have also implemented a ``fold style,'' but because the types are
a little less intuitive, we are sticking with @concatMap@ style for now.)


> Some numbers, I have used it nine times, and would need the general fold
 > once to define blockToNodeList (or CB* equivalent suggested by you).
 > (We are using it in GHC to
 >   - computing hash of the blocks from the nodes
 >   - finding the last node of a block
 >   - converting block to the old representation (2x)
 >   - computing interference graph
 >   - counting Area used by a block (2x)
 >   - counting stack high-water mark for a block
 >   - prettyprinting block)


type-parameter hell, newtype hell, typechecking hell, instance hell,
triple hell

\fi


% We have spent six years implementing and reimplementing frameworks for
% dataflow analysis and transformation.
%  This formidable design problem taught us
% two kinds of lessons:
% we learned some very specific lessons about representations and
% algorithms for optimizing compilers,
% and we were forcibly reminded of some very general, old lessons that are well
% known not just to functional programmers, but to programmers
% everywhere.



%%%%    \remark{Orphaned: but for transfer functions that
%%%%    approximate weakest preconditions or strongest postconditions,
%%%%    monotonicity falls out naturally.}
%%%%    
%%%%    
%%%%    In conclusion we offer the following lessons from the experience of designing
%%%%    and implementing \ourlib{}.
%%%%    \begin{itemize}
%%%%    \item 
%%%%    Although we have not stressed this point, there is a close connection
%%%%    between dataflow analyses and program logic:
%%%%    \begin{itemize}
%%%%    \item
%%%%    A forward dataflow analysis is represented by a predicate transformer
%%%%    that is related to \emph{strongest postconditions}
%%%%    \cite{floyd:meaning}.\footnote
%%%%    {In Floyd's paper the output of the predicate transformer is called
%%%%    the \emph{strongest verifiable consequent}, not the ``strongest
%%%%    postcondition.''} 
%%%%    \item
%%%%    A backward dataflow analysis is represented by a predicate transformer
%%%%    that is related to \emph{weakest preconditions} \cite{dijkstra:discipline}.
%%%%    \end{itemize}
%%%%    Logicians write down the predicate transformers for the primitive
%%%%    program fragments, and then use compositional rules to ``lift'' them 
%%%%    to a logic for whole programs.  In the same way \ourlib{} lets the client
%%%%    write simple predicate transformers,
%%%%    and local rewrites based on those assertions, and ``lifts'' them to entire
%%%%    function bodies with arbitrary control flow.

\iffalse


Reuse requires abstraction, and as is well known,
designing good abstractions is challenging. 
\hoopl's data types and the functions over those types have been
through {dozens} of revisions.
\remark{dozens alert}
As~we were refining our design, we~found it invaluable to operate in
two modes:
In the first mode, we designed, built, and used a framework as an
important component of a real compiler (first Quick~{\PAL}, then GHC).
In the second mode, we designed and built a standalone library, then
redesigned and rebuilt it, sometimes going through several significant
changes in a week.
Operating in the first mode---inside a live compiler---forced us to
make sure that no corners were cut, that we were solving a real
problem, and that we did not inadvertently
cripple some other part of the compiler.
Operating in the second mode---as a standalone library---enabled us to
iterate furiously, trying out many more ideas than would have
been possible in the first mode.
 Alternating between these two modes has led to a
better design than operating in either mode alone.

%% We were forcibly reminded of timeless truths:
It is a truth universally acknowledged that
interfaces are more important than implementations and data
is more important than code.
This truth is reflected in this paper, in which
we
have given \hoopl's API three times as much space as \hoopl's implementation.

We have evaluate \hoopl's API through small, focused classroom
projects and by using \hoopl\ in the back end of the Glasgow Haskell
Compiler. 



We were also reminded that Haskell's type system (polymorphism, GADTs,
higher-order functions, type classes, and so on) is a remarkably
effective 
language for thinking about data and code---and that
Haskell lacks a language of interfaces (like ML's signatures) that
would make it equally effective for thinking about APIs at a larger scale.
Still, as usual, the types were a remarkable aid to writing the code:
when we finally agreed on the types presented above, the
code almost wrote itself.  

Types are widely appreciated at ICFP, but  here are three specific
examples of how types helped us:
\begin{itemize}
\item 
Reuse is enabled by representation-independence, which in a functional
language is
expressed through parametric polymorphism.
Making \ourlib{} polymorphic in the nodes 
made the code simpler, easier to understand, and easier to maintain.
In particular, it forced us to make explicit \emph{exactly} what
\ourlib\ must know about nodes, and to embody that knowledge in the
@NonLocal@ type class (\secref{nonlocal-class}). 
\item
\remark{too much? Simon: better?}
%
% this paper is just not about run-time performance ---NR
%
%%%%    Moreover, the implementation is faster than it would otherwise be,
%%%%    because, say, a @(Fact O f)e@ is known to be just an @f@ rather than
%%%%    being a sum type that must be tested (with a statically known outcome!).
%
Giving the \emph{same} shapes
to nodes, blocks, and graphs helped our
thinking and helped to structure the implementation.
\item
\end{itemize}

\fi

\paragraph{More alternative interfaces and implementations}
Why do we allow the client to define the monad~@m@ used in rewrite
functions and in @analyzeAndRewriteFwdBody@?
The~obvious alternative, which we have implemented and explored, is to require
\hoopl's clients to use a monad provided by \hoopl.
This alternative has advantages: 
because \hoopl{} implements
@checkpoint@, @restart@, 
@setFuel@, and @getFuel@, 
we can ensure they are right, and
that the client cannot possibly misuse them. 
The downside is that a rewrite function can \emph{only} use
\hoopl{}-provided monadic actions.  Clearly this monad must be
able to supply fresh labels (for new blocks), but what if 
% In~this alternative API, \hoopl\ also provides a supply of unique names.
% 
% But~we are wary of mandating this abstraction;
% unique names affect many parts of a compiler,
% and
% no~matter how well designed the~API,
% if it does not play well with existing code,
% it won't be used.
%
% Moreover, experience has shown us that for the client, the convenience
% of a custom monad is well worth the cognitive costs of understanding
% the more complex API and the costs of meeting the contracts for
% @FuelMonad@ and @CkpointMonad@.
% As~a very simple example, 
a client wanted one set of
unique names for labels and a different set for registers?
Moreover, a client might want a rewrite to perform other actions
entirely that could not be anticipated by \hoopl{}.
For example, in order to judge the effectiveness of an optimization,
a client might want to log how many rewrites take place, or in what
functions they take place.
As~a more ambitious example, \citet{runciman:increasing-prs} describes
Primitive Redex 
Speculation, a code-improving transformation that can create new
function definitions. 
A~\hoopl\ client implementing this transformation would define a monad
that could accumulate new definitions.
The law governing @checkpoint@ and @restart@
would ensure that a speculative rewrite, if later rolled back, would not
create  a function definition (or a log entry).

Another merit of a user-defined monad~@m@ is that, 
if~a user wants to manage optimization fuel differently,
he or she can make~@m@ an instance of @FuelMonad@ in which the fuel
supply is infinite.
The user is then free to create a new fuel supply in~@m@ and to wrap
rewrite functions---or not---so as to consume fuel in the new supply.
This freedom can used to implement more exotic uses of fuel;
for~example, a~user might find it convenient if it were possible to
instantiate a compiler temporary as a real hardware register without
consuming fuel.

%%  \simon{These next two paras are incomprehensible. Cut?}
%%  Of the many varied implementations we have tried,
%%  we have space only to raise a few questions, with even fewer answers.
%%  %
%%  An~earlier implementation of @fixpoint@ stored the
%%  ``current'' @FactBase@ in a monad; we~find it easier to understand and
%%  maintain the code that passes the current @FactBase@ as an argument.
%%  Among @concatMap@ style, fold style, and continuation-passing style, 
%%  which is best?
%%  No~one of these styles makes all the code easiest to read
%%  and understand: @concatMap@ style is relatively straightforward throughout;
%%  fold~style is similar overall but different in detail;
%%  and continuation-passing style is clear and elegant to those who
%%  like continuations, but baffling to those who don't.

Which value constructors should be\simon{still incomprehensible.  cut?}
polymorphic in the shapes of their arguments, and which should be
monomorphic?
We~experimented with a polymorphic
\begin{code}
  `BNode :: n e x -> Block n e x
\end{code}
but we found that there are significant advantages to knowing the type
of every node statically, using purely local information---so instead
we use
the three monomorphic constructors @BFirst@, @BMiddle@, and @BLast@
(\figref{graph}). 
Similar questions arise about the polymorphic @BCat@ and about the
graph constructors, and even among ourselves, we are divided about how
best to answer them.
Yet another question is whether it is worthwhile to save a level of
indirection by providing a cons-like constructor to concatenate a node
and a block. 
Perhaps some of these questions can be answered by appealing to
performance, but the experiments that will provide the answers have
yet to be devised.




\paragraph{Final remarks}

Dataflow optimization is usually described as a way to improve imperative
programs by mutating control-flow graphs.
Such transformations appear very different from the tree rewriting
that functional languages are so well known for and which makes
\ifhaskellworkshop
Haskell
\else
functional languages 
\fi
so attractive for writing other parts of compilers.
But even though dataflow optimization looks very different from
what we are used to,
writing a dataflow optimizer
in
\ifhaskellworkshop
Haskell
\else
a pure functional language 
\fi
was a win:
%  We could not possibly have conceived \ourlib{} in C++.
we had to make every input and output explicit,
and we had a strong incentive to implement things compositionally.
Using Haskell helped us make real improvements in the implementation of
some very sophisticated ideas.
% %%  
% %%  
% %%  In~a pure functional language, not only do we know that
% %%  no data structure will be unexpectedly mutated,
% %%  but we are forced to be
% %%  explicit about every input and output, 
% %%  and we are encouraged to implement things compositionally.
% %%  This kind of thinking has helped us make
% %%  significant improvements to the already tricky work of Lerner, Grove,
% %%  and Chambers:
% %%  per-function control of shallow vs deep rewriting 
% %%  (\secref{shallow-vs-deep}),
% %%  optimization fuel (\secref{fuel}),
% %%  and transparent management of unreachable blocks (\secref{arf-body}).
% We~trust that the improvements are right only because they are
% implemented in separate 
% parts of the code that cannot interact except through
% explicit function calls.
% %%  %%
% %%  %%An ancestor of \ourlib{} is in the Glasgow Haskell Compiler today,
% %%  %%in version~6.12.
% %%  With this new, improved design in hand, we are now moving back to
% %%  live-compiler mode,  pushing \hoopl\ into version
% %%  6.13 of the Glasgow Haskell Compiler.


\acks

Brian Huffman and Graham Hutton helped with algebraic laws.
Sukyoung Ryu told us about Primitive Redex Speculation.
Several anonymous reviewers helped improve the presentation.
% , especially reviewer~C, who suggested better language with which to
% describe our work.  

The first and second authors were funded 
by a grant from Intel Corporation and
by NSF awards CCF-0838899 and CCF-0311482.
These authors also thank Microsoft Research Ltd, UK, for funding
extended visits to the third author.


\makeatother

\providecommand\includeftpref{\relax} %% total bafflement -- workaround
\IfFileExists{nrbib.tex}{\bibliography{cs,ramsey}}{\bibliography{dfopt}}
\bibliographystyle{plainnatx}


\clearpage

\appendix

% don't omit LabelSet :: *
% don't omit delFromFactBase :: FactBase f -> [(Label,f)] -> FactBase f
% don't omit elemFactBase :: Label -> FactBase f -> Bool
% don't omit elemLabelSet :: Label -> LabelSet -> Bool
% don't omit emptyLabelSet :: LabelSet
% don't omit factBaseLabels :: FactBase f -> [Label]
% don't omit extendFactBase :: FactBase f -> Label -> f -> FactBase f
% don't omit extendLabelSet :: LabelSet -> Label -> LabelSet
% don't omit lookupFact :: FactBase f -> Label -> Maybe f
% don't omit factBaseList :: FactBase f -> [(Label, f)]

%%  \section{Code for \textmd{\texttt{fixpoint}}}
%%  \label{app:fixpoint}
%%  
%%  {\def\baselinestretch{0.95}\hfuzz=20pt
%%  \begin{smallcode}
%%  data `TxFactBase n f
%%    = `TxFB { `tfb_fbase :: FactBase f
%%           , `tfb_rg  :: DG n f C C -- Transformed blocks
%%           , `tfb_cha   :: ChangeFlag
%%           , `tfb_lbls  :: LabelSet }
%%   -- Set the tfb_cha flag iff 
%%   --   (a) the fact in tfb_fbase for or a block L changes
%%   --   (b) L is in tfb_lbls.
%%   -- The tfb_lbls are all Labels of the *original* 
%%   -- (not transformed) blocks
%%  
%%  `updateFact :: DataflowLattice f -> LabelSet -> (Label, f)
%%             -> (ChangeFlag, FactBase f) 
%%             -> (ChangeFlag, FactBase f)
%%  updateFact ^lat ^lbls (lbl, ^new_fact) (^cha, fbase)
%%    | NoChange <- ^cha2        = (cha,        fbase)
%%    | lbl `elemLabelSet` lbls = (SomeChange, new_fbase)
%%    | otherwise               = (cha,        new_fbase)
%%    where
%%      (cha2, ^res_fact) 
%%        = case lookupFact fbase lbl of
%%           Nothing -> (SomeChange, new_fact)
%%           Just ^old_fact -> fact_extend lat old_fact new_fact
%%      ^new_fbase = extendFactBase fbase lbl res_fact
%%  
%%  fixpoint :: forall n f. NonLocal n
%%           => Bool        -- Going forwards?
%%           -> DataflowLattice f
%%           -> (Block n C C -> FactBase f
%%                -> FuelMonad (DG n f C C, FactBase f))
%%           -> FactBase f -> [(Label, Block n C C)]
%%           -> FuelMonad (DG n f C C, FactBase f)
%%  fixpoint ^is_fwd lat ^do_block ^init_fbase ^blocks
%%   = do { ^fuel <- getFuel  
%%        ; ^tx_fb <- loop fuel init_fbase
%%        ; return (tfb_rg tx_fb, 
%%                  tfb_fbase tx_fb `delFromFactBase` blocks) }
%%            -- The outgoing FactBase contains facts only for 
%%            -- Labels *not* in the blocks of the graph
%%   where
%%    `tx_blocks :: [(Label, Block n C C)] 
%%              -> TxFactBase n f -> FuelMonad (TxFactBase n f)
%%    tx_blocks []             tx_fb = return tx_fb
%%    tx_blocks ((lbl,blk):bs) tx_fb = tx_block lbl blk tx_fb
%%                                     >>= tx_blocks bs
%%  
%%    `tx_block :: Label -> Block n C C 
%%             -> TxFactBase n f -> FuelMonad (TxFactBase n f)
%%    tx_block ^lbl ^blk tx_fb@(TxFB { tfb_fbase = fbase
%%                                 , tfb_lbls  = lbls
%%                                 , tfb_rg    = ^blks
%%                                 , tfb_cha   = cha })
%%      | is_fwd && not (lbl `elemFactBase` fbase)
%%      = return tx_fb    -- Note [Unreachable blocks]
%%      | otherwise
%%      = do { (rg, ^out_facts) <- do_block blk fbase
%%           ; let (^cha', ^fbase') 
%%                   = foldr (updateFact lat lbls) (cha,fbase) 
%%                           (factBaseList out_facts)
%%           ; return (TxFB { tfb_lbls = extendLabelSet lbls lbl
%%                          , tfb_rg   = rg `DGCatC` blks
%%                          , tfb_fbase = fbase'
%%                          , tfb_cha = cha' }) }
%%  
%%    loop :: Fuel -> FactBase f -> FuelMonad (TxFactBase n f)
%%    `loop fuel fbase 
%%      = do { let ^init_tx_fb = TxFB { tfb_fbase = fbase
%%                                   , tfb_cha   = NoChange
%%                                   , tfb_rg    = DGNil
%%                                   , tfb_lbls  = emptyLabelSet}
%%           ; tx_fb <- tx_blocks blocks init_tx_fb
%%           ; case tfb_cha tx_fb of
%%               NoChange   -> return tx_fb
%%               SomeChange -> setFuel fuel >>
%%                             loop fuel (tfb_fbase tx_fb) }
%%  \end{smallcode}
%%  \par
%%  } % end \baselinestretch


\section{Index of defined identifiers}

This appendix lists every nontrivial identifier used in the body of
the paper.  
For each identifier, we list the page on which that identifier is
defined or discussed---or when appropriate, the figure (with line
number where possible).
For those few identifiers not defined or discussed in text, we give
the type signature and the page on which the identifier is first
referred to.

Some identifiers used in the text are defined in the Haskell Prelude;
for those readers less familiar with Haskell (possible even at the
Haskell Symposium!), these identifiers are
listed in Appendix~\ref{sec:prelude}.

\newcommand\dropit[3][]{}

\newcommand\hsprelude[2]{\noindent
  \texttt{#1} defined in the Haskell Prelude\\}
\let\hsprelude\dropit

\newcommand\hspagedef[3][]{\noindent
  \texttt{#2} defined on page~\pageref{#3}.\\}
\newcommand\omithspagedef[3][]{\noindent
  \texttt{#2} not shown (but see page~\pageref{#3}).\\}
\newcommand\omithsfigdef[3][]{\noindent
  \texttt{#2} not shown (but see Figure~\ref{#3} on page~\pageref{#3}).\\}
\newcommand\hsfigdef[3][]{%
  \noindent
  \ifx!#1!%
    \texttt{#2} defined in Figure~\ref{#3} on page~\pageref{#3}.\\
  \else
    \texttt{#2} defined on \lineref{#1} of Figure~\ref{#3} on page~\pageref{#3}.\\
  \fi
}    
\newcommand\hstabdef[3][]{%
  \noindent
  \ifx!#1!
    \texttt{#2} defined in Table~\ref{#3} on page~\pageref{#3}.\\
  \else
    \texttt{#2} defined on \lineref{#1} of Table~\ref{#3} on page~\pageref{#3}.\\
  \fi
}    
\newcommand\hspagedefll[3][]{\noindent
  \texttt{#2} {let}- or $\lambda$-bound on page~\pageref{#3}.\\}
\newcommand\hsfigdefll[3][]{%
  \noindent
  \ifx!#1!%
    \texttt{#2} {let}- or $\lambda$-bound in Figure~\ref{#3} on page~\pageref{#3}.\\
  \else
    \texttt{#2} {let}- or $\lambda$-bound on \lineref{#1} of Figure~\ref{#3} on page~\pageref{#3}.\\
  \fi
}    

\newcommand\nothspagedef[3][]{\notdefd\ndpage{#1}{#2}{#3}}
\newcommand\nothsfigdef[3][]{\notdefd\ndfig{#1}{#2}{#3}}
\newcommand\nothslinedef[3][]{\notdefd\ndline{#1}{#2}{#3}}

\newcommand\ndpage[3]{\texttt{#2}~(p\pageref{#3})}
\newcommand\ndfig[3]{\texttt{#2}~(Fig~\ref{#3},~p\pageref{#3})}
\newcommand\ndline[3]{%
  \ifx!#1!%
      \ndfig{#1}{#2}{#3}%
  \else
      \texttt{#2}~(Fig~\ref{#3}, line~\lineref{#1}, p\pageref{#3})%
  \fi
}

\newif\ifundefinedsection\undefinedsectionfalse

\newcommand\notdefd[4]{%
  \ifundefinedsection
    , #1{#2}{#3}{#4}%
  \else
    \undefinedsectiontrue
    \par
    \section{Undefined identifiers}
    #1{#2}{#3}{#4}%
  \fi
}

\begingroup
\raggedright

\hsprelude{!}{Prelude}% context prelude
\hsprelude{\$}{Prelude}% context prelude
\hsprelude{\&}{Prelude}% context prelude
\hsprelude{\&\&}{Prelude}% context prelude
\hsprelude{*}{Prelude}% context prelude
\hsprelude{+}{Prelude}% context prelude
\hsprelude{++}{Prelude}% context prelude
\hsprelude{-}{Prelude}% context prelude
\hsprelude{.}{Prelude}% context prelude
\hsprelude{/}{Prelude}% context prelude
\hspagedef{<*>}{haskell.def.<*>}% context document
\hsprelude{==}{Prelude}% context prelude
\hsprelude{>}{Prelude}% context prelude
\hsprelude{>=}{Prelude}% context prelude
\hsprelude{>>}{Prelude}% context prelude
\hsprelude{>>=}{Prelude}% context prelude
\hsfigdefll{add}{haskell.def.add}% context figure
\hsfigdef{addUsed}{haskell.def.addUsed}% context figure
\hspagedef{anal\_f\_OO}{haskell.def.anal:unf:unOO}% context document
\hsfigdef[rew.first]{ar\_first}{haskell.def.ar:unfirst}% context figure
\hsfigdef[rew.last]{ar\_last}{haskell.def.ar:unlast}% context figure
\hsfigdef[rew.mid.1]{ar\_mid}{haskell.def.ar:unmid}% context figure
\hsfigdefll[reload1]{avail}{haskell.def.avail}% context figure
\hsfigdef{availRewrites}{haskell.def.availRewrites}% context figure
\hsfigdef[avail.first]{availTransfers}{haskell.def.availTransfers}% context figure
\hsfigdef[AvailVars]{AvailVars}{haskell.def.AvailVars}% context figure
\hsfigdef{availVarsLattice}{haskell.def.availVarsLattice}% context figure
\hsfigdef{BackTransfers}{haskell.def.BackTransfers}% context figure
\hsfigdef{BackwardRewrites}{haskell.def.BackwardRewrites}% context figure
\hspagedef{Block}{haskell.def.Block}% context document
\hspagedef{BlockEnv}{haskell.def.BlockEnv}% context document
\hspagedef{BlockId}{haskell.def.BlockId}% context document
\hsprelude{Bool}{Prelude}% context prelude
\hsfigdef{br\_first}{haskell.def.br:unfirst}% context figure
\hsfigdef{br\_last}{haskell.def.br:unlast}% context figure
\hsfigdef{br\_middle}{haskell.def.br:unmiddle}% context figure
\hsfigdef{bt\_first\_in}{haskell.def.bt:unfirst:unin}% context figure
\hsfigdef{bt\_last\_in}{haskell.def.bt:unlast:unin}% context figure
\hsfigdef{bt\_middle\_in}{haskell.def.bt:unmiddle:unin}% context figure
\hsfigdef{C}{haskell.def.C}% context figure
\omithsfigdef{catMaybes :: [Maybe a] -> [a]}{haskell.def.catMaybes}% context figure
\hsfigdef{ChangeFlag}{haskell.def.ChangeFlag}% context figure
\hspagedef{Cmm}{haskell.def.Cmm}% context document
\hsfigdef{cmmAvailableVars}{haskell.def.cmmAvailableVars}% context figure
\hspagedef{CmmExpr}{haskell.def.CmmExpr}% context document
\hspagedef{CmmGlobal}{haskell.def.CmmGlobal}% context document
\hspagedef{CmmLast}{haskell.def.CmmLast}% context document
\hsfigdef{cmmLiveness}{haskell.def.cmmLiveness}% context figure
\hspagedef{CmmLoad}{haskell.def.CmmLoad}% context document
\hspagedef{CmmLocal}{haskell.def.CmmLocal}% context document
\hspagedef{CmmMiddle}{haskell.def.CmmMiddle}% context document
\hspagedef{CmmVar}{haskell.def.CmmVar}% context document
\hsprelude{const}{Prelude}% context prelude
\hsprelude{curry}{Prelude}% context prelude
\hsprelude{Data.Map}{Prelude}% context prelude
\hsfigdef{DataflowLattice}{haskell.def.DataflowLattice}% context figure
\hsfigdef{deadRewrites}{haskell.def.deadRewrites}% context figure
\hspagedef{DefinerOfLocalVars}{haskell.def.DefinerOfLocalVars}% context document
\hsfigdef{delFromAvail}{haskell.def.delFromAvail}% context figure
\omithspagedef{delFromVarSet :: VarSet -> LocalVar -> VarSet}{haskell.def.delFromVarSet}% context document
\hsfigdefll[forward.sol.args]{depth}{haskell.def.depth}% context figure
\hspagedef{DFM}{haskell.def.DFM}% context document
\hsfigdef{elemAvail}{haskell.def.elemAvail}% context figure
\omithspagedef{elemVarSet :: LocalVar -> VarSet -> Bool}{haskell.def.elemVarSet}% context document
\hsfigdefll{empty}{haskell.def.empty}% context figure
\omithspagedef{emptyBlockEnv :: BlockEnv a}{haskell.def.emptyBlockEnv}% context document
\hspagedef{emptyGraph}{haskell.def.emptyGraph}% context document
\omithspagedef{emptyVarSet :: VarSet}{haskell.def.emptyVarSet}% context document
\hsfigdefll{entry}{haskell.def.entry}% context figure
\hsfigdefll[lastLiveOut.1]{env}{haskell.def.env}% context figure
\hspagedefll{ex}{haskell.def.ex}% context document
\hsfigdefll{exit}{haskell.def.exit}% context figure
\hsfigdefll[assign.avail.1]{\_expr}{haskell.def.:unexpr}% context figure
\hsfigdef[extendAvail]{extendAvail}{haskell.def.extendAvail}% context figure
\omithspagedef{extendVarSet :: VarSet -> LocalVar -> VarSet}{haskell.def.extendVarSet}% context document
\hsfigdef{fact\_add\_to}{haskell.def.fact:unadd:unto}% context figure
\hsfigdef{fact\_bot}{haskell.def.fact:unbot}% context figure
\hsfigdef[FactKont]{FactKont}{haskell.def.FactKont}% context figure
\omithspagedef{fact\_name :: DataflowLattice a -> String}{haskell.def.fact:unname}% context document
\hsprelude{False}{Prelude}% context prelude
\omithspagedef{filterVarsUsed :: UserOfLocalVars e => (LocalVar -> Bool) -> e -> VarSet}{haskell.def.filterVarsUsed}% context document
\hsfigdefll[avail.rewrites.first]{first}{haskell.def.first}% context figure
\hsprelude{flip}{Prelude}% context prelude
\hsprelude{foldl}{Prelude}% context prelude
\hsprelude{foldr}{Prelude}% context prelude
\hspagedef{foldVarsDefd}{haskell.def.foldVarsDefd}% context document
\hspagedef{foldVarsUsed}{haskell.def.foldVarsUsed}% context document
\hsfigdef{ForwardRewrites}{haskell.def.ForwardRewrites}% context figure
\hsfigdef{ForwardTransfers}{haskell.def.ForwardTransfers}% context figure
\hsfigdefll[avail.solve.1]{fp}{haskell.def.fp}% context figure
\hsfigdef{fr\_first}{haskell.def.fr:unfirst}% context figure
\hsfigdef{fr\_last}{haskell.def.fr:unlast}% context figure
\hsfigdef{fr\_middle}{haskell.def.fr:unmiddle}% context figure
\hsprelude{fst}{Prelude}% context prelude
\hsfigdef{ft\_first\_out}{haskell.def.ft:unfirst:unout}% context figure
\hsfigdef{ft\_last\_outs}{haskell.def.ft:unlast:unouts}% context figure
\hsfigdef{ft\_middle\_out}{haskell.def.ft:unmiddle:unout}% context figure
\hspagedef{fuelExhausted}{haskell.def.fuelExhausted}% context document
\hspagedef{FuelMonad}{haskell.def.FuelMonad}% context document
\hspagedef{FwdFixedPoint}{haskell.def.FwdFixedPoint}% context document
\hsfigdef[forward.sol.sig]{fwd\_iter}{haskell.def.fwd:uniter}% context figure
\hspagedef{getAllFacts}{haskell.def.getAllFacts}% context document
\hspagedef{getFact}{haskell.def.getFact}% context document
\hsfigdef{GF}{haskell.def.GF}% context figure
\hspagedef{GlobalVar}{haskell.def.GlobalVar}% context document
\hsfigdef{Graph}{haskell.def.Graph}% context figure
\hspagedef{GraphClosure}{haskell.def.GraphClosure}% context document
\hsfigdef[GraphFactKont]{GraphFactKont}{haskell.def.GraphFactKont}% context figure
\hsfigdef[GraphKont]{GraphKont}{haskell.def.GraphKont}% context figure
\hspagedef{HavingSuccessors}{haskell.def.HavingSuccessors}% context document
\hsprelude{head}{Prelude}% context prelude
\hspagedef{iar\_OC}{haskell.def.iar:unOC}% context document
\hsfigdef[iar.OO]{iar\_OO}{haskell.def.iar:unOO}% context figure
\hsprelude{id}{Prelude}% context prelude
\hsfigdefll[solve.mid.1]{in'}{haskell.def.in'}% context figure
\hsfigdefll[forward.sol.args]{in\_fact}{haskell.def.in:unfact}% context figure
\hsfigdef{insertLateReloads}{haskell.def.insertLateReloads}% context figure
\hsprelude{Int}{Prelude}% context prelude
\hsfigdef{interAvail}{haskell.def.interAvail}% context figure
\omithspagedef{isEmptyVarSet :: VarSet -> Bool}{haskell.def.isEmptyVarSet}% context document
\omithspagedef{isStackSlot :: CmmExpr -> Bool}{haskell.def.isStackSlot}% context document
\omithspagedef{isStackSlotOf :: CmmExpr -> LocalVar -> Bool}{haskell.def.isStackSlotOf}% context document
\hsfigdef[solve.block.sig]{iter\_block}{haskell.def.iter:unblock}% context figure
\hsfigdef[solve.ex.sig]{iter\_ex}{haskell.def.iter:unex}% context figure
\hsfigdef[solve.first.sig]{iter\_first}{haskell.def.iter:unfirst}% context figure
\hsfigdef[solve.last.sig]{iter\_last}{haskell.def.iter:unlast}% context figure
\hsfigdef[solve.mid.1]{iter\_mid}{haskell.def.iter:unmid}% context figure
\hsfigdef[solve.OO.def]{iter\_OO}{haskell.def.iter:unOO}% context figure
\hsfigdefll{join}{haskell.def.join}% context figure
\hsprelude{Just}{Prelude}% context prelude
\hsfigdef[Kont]{Kont}{haskell.def.Kont}% context figure
\hsfigdefll{l}{haskell.def.l}% context figure
\hsfigdefll[avail.rewrites.last]{last}{haskell.def.last}% context figure
\hsfigdef{lastAvail}{haskell.def.lastAvail}% context figure
\hspagedef{LastBranch}{haskell.def.LastBranch}% context document
\hspagedef{LastCall}{haskell.def.LastCall}% context document
\hspagedef{LastCondBranch}{haskell.def.LastCondBranch}% context document
\hsfigdef[lastLiveness]{lastLiveness}{haskell.def.lastLiveness}% context figure
\hsfigdef[lastLiveOut.1]{lastLiveOut}{haskell.def.lastLiveOut}% context figure
\hsfigdef{LastOuts}{haskell.def.LastOuts}% context figure
\hspagedef{LastSwitch}{haskell.def.LastSwitch}% context document
\hsfigdefll[assign.avail.1]{lhs}{haskell.def.lhs}% context figure
\hsprelude{liftM}{Prelude}% context prelude
\hsfigdef[Live]{Live}{haskell.def.Live}% context figure
\hsfigdefll{live}{haskell.def.live}% context figure
\hsfigdef[liveLattice]{liveLattice}{haskell.def.liveLattice}% context figure
\hsfigdef{liveTransfers}{haskell.def.liveTransfers}% context figure
\hspagedef{LocalVar}{haskell.def.LocalVar}% context document
\hsfigdef{LOFsKont}{haskell.def.LOFsKont}% context figure
\hsfigdefll{m}{haskell.def.m}% context figure
\hsprelude{map}{Prelude}% context prelude
\hsprelude{mapM\_}{Prelude}% context prelude
\hsprelude{Maybe}{Prelude}% context prelude
\hsfigdef[maybe.reload.before.1]{maybe\_reload\_before}{haskell.def.maybe:unreload:unbefore}% context figure
\hspagedef{MidAssign}{haskell.def.MidAssign}% context document
\hsfigdefll{middle}{haskell.def.middle}% context figure
\hsfigdef{middleAvail}{haskell.def.middleAvail}% context figure
\hsfigdef[middleLiveness]{middleLiveness}{haskell.def.middleLiveness}% context figure
\hsfigdef{middleRemoveDeads}{haskell.def.middleRemoveDeads}% context figure
\hspagedef{MidStore}{haskell.def.MidStore}% context document
\hspagedef{mkLabel}{haskell.def.mkLabel}% context document
\hspagedef{mkLast}{haskell.def.mkLast}% context document
\hspagedef{mkMiddle}{haskell.def.mkMiddle}% context document
\hsfigdefll[forward.sol.args]{name}{haskell.def.name}% context figure
\hsfigdefll{new}{haskell.def.new}% context figure
\hsfigdef{NoChange}{haskell.def.NoChange}% context figure
\hsfigdefll[maybe.reload.before.1]{node}{haskell.def.node}% context figure
\hsprelude{not}{Prelude}% context prelude
\hsprelude{Nothing}{Prelude}% context prelude
\hsfigdefll[deadRewrites.1]{nothing}{haskell.def.nothing}% context figure
\hsfigdef{O}{haskell.def.O}% context figure
\hsfigdefll{old}{haskell.def.old}% context figure
\hspagedef{PassName}{haskell.def.PassName}% context document
\hsfigdefll[mkMiddle]{rel}{haskell.def.rel}% context figure
\hspagedef{reload}{haskell.def.reload}% context document
\hsfigdef{reloadTail}{haskell.def.reloadTail}% context figure
\hsfigdef[liveness.remDefd.def]{remDefd}{haskell.def.remDefd}% context figure
\hsfigdef{removeDeadAssignments}{haskell.def.removeDeadAssignments}% context figure
\hsprelude{return}{Prelude}% context prelude
\hsfigdef{Rewrite}{haskell.def.Rewrite}% context figure
\hspagedef{RewriteDeep}{haskell.def.RewriteDeep}% context document
\hsfigdefll[forward.sol.args]{rewrites}{haskell.def.rewrites}% context figure
\hspagedef{RewriteShallow}{haskell.def.RewriteShallow}% context document
\hspagedef{RewritingDepth}{haskell.def.RewritingDepth}% context document
\hspagedef{runDFM}{haskell.def.runDFM}% context document
\hspagedef{setAllFacts}{haskell.def.setAllFacts}% context document
\hspagedef{setFact}{haskell.def.setFact}% context document
\hsfigdef[set.last.*]{set\_last}{haskell.def.set:unlast}% context figure
\omithspagedef{sizeVarSet :: VarSet -> Int}{haskell.def.sizeVarSet}% context document
\hsfigdef[smallerAvail]{smallerAvail}{haskell.def.smallerAvail}% context figure
\hsprelude{snd}{Prelude}% context prelude
\hsfigdef{SomeChange}{haskell.def.SomeChange}% context figure
\hsfigdefll[forward.sol.args]{start\_facts}{haskell.def.start:unfacts}% context figure
\hsprelude{String}{Prelude}% context prelude
\hspagedef{subAnalysis}{haskell.def.subAnalysis}% context document
\hspagedef{succs}{haskell.def.succs}% context document
\hsprelude{tail}{Prelude}% context prelude
\hsfigdefll[live.lastSwitch]{tbl}{haskell.def.tbl}% context figure
\hsfigdefll[forward.sol.args]{transfers}{haskell.def.transfers}% context figure
\hsprelude{True}{Prelude}% context prelude
\hsfigdef{TxRes}{haskell.def.TxRes}% context figure
\hsprelude{uncurry}{Prelude}% context prelude
\hsprelude{undefined}{Prelude}% context prelude
\omithspagedef{unionManyVarSets :: [VarSet] -> VarSet}{haskell.def.unionManyVarSets}% context document
\omithspagedef{unionVarSets :: VarSet -> VarSet -> VarSet}{haskell.def.unionVarSets}% context document
\hsfigdef[AvailVars]{UniverseMinus}{haskell.def.UniverseMinus}% context figure
\hsfigdefll{used}{haskell.def.used}% context figure
\hspagedef{useOneFuel}{haskell.def.useOneFuel}% context document
\hspagedef{UserOfLocalVars}{haskell.def.UserOfLocalVars}% context document
\omithspagedef{varOfSlot :: CmmExpr -> LocalVar}{haskell.def.varOfSlot}% context document
\hsfigdefll{vars}{haskell.def.vars}% context figure
\omithspagedef{VarSet\textrm{~(a~type)}}{haskell.def.VarSet}% context document
\omithsfigdef{varSetToList :: VarSet -> [LocalVar]}{haskell.def.varSetToList}% context figure
\hspagedef{withDuplicateFuel}{haskell.def.withDuplicateFuel}% context document
\hspagedef{with\_fuel}{haskell.def.with:unfuel}% context document
\hspagedef{zdfFpContents}{haskell.def.zdfFpContents}% context document
\hspagedef{zdfFpFacts}{haskell.def.zdfFpFacts}% context document
\hspagedef{zdfRewriteBwd}{haskell.def.zdfRewriteBwd}% context document
\hspagedef{zdfRewriteFwd}{haskell.def.zdfRewriteFwd}% context document
\hspagedef{zdfSolveBwd}{haskell.def.zdfSolveBwd}% context document
\hspagedef{zdfSolveFwd}{haskell.def.zdfSolveFwd}% context document
\hspagedef{ZJust}{haskell.def.ZJust}% context document
\hspagedef{ZMaybe}{haskell.def.ZMaybe}% context document
\hspagedef{ZNothing}{haskell.def.ZNothing}% context document
%
\ifundefinedsection.\fi

\undefinedsectionfalse


\renewcommand\hsprelude[2]{\noindent
  \ifundefinedsection
    , \texttt{#1}%
  \else
    \undefinedsectiontrue
    \par
    \section{Identifiers defined in Haskell Prelude or a standard library}\label{sec:prelude}
    \texttt{#1}%
  \fi
}
\let\hspagedef\dropit
\let\omithspagedef\dropit
\let\omithsfigdef\dropit
\let\hsfigdef\dropit
\let\hstabdef\dropit
\let\hspagedefll\dropit
\let\hsfigdefll\dropit
\let\nothspagedef\dropit
\let\nothsfigdef\dropit
\let\nothslinedef\dropit

\hsprelude{!}{Prelude}% context prelude
\hsprelude{\$}{Prelude}% context prelude
\hsprelude{\&}{Prelude}% context prelude
\hsprelude{\&\&}{Prelude}% context prelude
\hsprelude{*}{Prelude}% context prelude
\hsprelude{+}{Prelude}% context prelude
\hsprelude{++}{Prelude}% context prelude
\hsprelude{-}{Prelude}% context prelude
\hsprelude{.}{Prelude}% context prelude
\hsprelude{/}{Prelude}% context prelude
\hspagedef{<*>}{haskell.def.<*>}% context document
\hsprelude{==}{Prelude}% context prelude
\hsprelude{>}{Prelude}% context prelude
\hsprelude{>=}{Prelude}% context prelude
\hsprelude{>>}{Prelude}% context prelude
\hsprelude{>>=}{Prelude}% context prelude
\hsfigdefll{add}{haskell.def.add}% context figure
\hsfigdef{addUsed}{haskell.def.addUsed}% context figure
\hspagedef{anal\_f\_OO}{haskell.def.anal:unf:unOO}% context document
\hsfigdef[rew.first]{ar\_first}{haskell.def.ar:unfirst}% context figure
\hsfigdef[rew.last]{ar\_last}{haskell.def.ar:unlast}% context figure
\hsfigdef[rew.mid.1]{ar\_mid}{haskell.def.ar:unmid}% context figure
\hsfigdefll[reload1]{avail}{haskell.def.avail}% context figure
\hsfigdef{availRewrites}{haskell.def.availRewrites}% context figure
\hsfigdef[avail.first]{availTransfers}{haskell.def.availTransfers}% context figure
\hsfigdef[AvailVars]{AvailVars}{haskell.def.AvailVars}% context figure
\hsfigdef{availVarsLattice}{haskell.def.availVarsLattice}% context figure
\hsfigdef{BackTransfers}{haskell.def.BackTransfers}% context figure
\hsfigdef{BackwardRewrites}{haskell.def.BackwardRewrites}% context figure
\hspagedef{Block}{haskell.def.Block}% context document
\hspagedef{BlockEnv}{haskell.def.BlockEnv}% context document
\hspagedef{BlockId}{haskell.def.BlockId}% context document
\hsprelude{Bool}{Prelude}% context prelude
\hsfigdef{br\_first}{haskell.def.br:unfirst}% context figure
\hsfigdef{br\_last}{haskell.def.br:unlast}% context figure
\hsfigdef{br\_middle}{haskell.def.br:unmiddle}% context figure
\hsfigdef{bt\_first\_in}{haskell.def.bt:unfirst:unin}% context figure
\hsfigdef{bt\_last\_in}{haskell.def.bt:unlast:unin}% context figure
\hsfigdef{bt\_middle\_in}{haskell.def.bt:unmiddle:unin}% context figure
\hsfigdef{C}{haskell.def.C}% context figure
\omithsfigdef{catMaybes :: [Maybe a] -> [a]}{haskell.def.catMaybes}% context figure
\hsfigdef{ChangeFlag}{haskell.def.ChangeFlag}% context figure
\hspagedef{Cmm}{haskell.def.Cmm}% context document
\hsfigdef{cmmAvailableVars}{haskell.def.cmmAvailableVars}% context figure
\hspagedef{CmmExpr}{haskell.def.CmmExpr}% context document
\hspagedef{CmmGlobal}{haskell.def.CmmGlobal}% context document
\hspagedef{CmmLast}{haskell.def.CmmLast}% context document
\hsfigdef{cmmLiveness}{haskell.def.cmmLiveness}% context figure
\hspagedef{CmmLoad}{haskell.def.CmmLoad}% context document
\hspagedef{CmmLocal}{haskell.def.CmmLocal}% context document
\hspagedef{CmmMiddle}{haskell.def.CmmMiddle}% context document
\hspagedef{CmmVar}{haskell.def.CmmVar}% context document
\hsprelude{const}{Prelude}% context prelude
\hsprelude{curry}{Prelude}% context prelude
\hsprelude{Data.Map}{Prelude}% context prelude
\hsfigdef{DataflowLattice}{haskell.def.DataflowLattice}% context figure
\hsfigdef{deadRewrites}{haskell.def.deadRewrites}% context figure
\hspagedef{DefinerOfLocalVars}{haskell.def.DefinerOfLocalVars}% context document
\hsfigdef{delFromAvail}{haskell.def.delFromAvail}% context figure
\omithspagedef{delFromVarSet :: VarSet -> LocalVar -> VarSet}{haskell.def.delFromVarSet}% context document
\hsfigdefll[forward.sol.args]{depth}{haskell.def.depth}% context figure
\hspagedef{DFM}{haskell.def.DFM}% context document
\hsfigdef{elemAvail}{haskell.def.elemAvail}% context figure
\omithspagedef{elemVarSet :: LocalVar -> VarSet -> Bool}{haskell.def.elemVarSet}% context document
\hsfigdefll{empty}{haskell.def.empty}% context figure
\omithspagedef{emptyBlockEnv :: BlockEnv a}{haskell.def.emptyBlockEnv}% context document
\hspagedef{emptyGraph}{haskell.def.emptyGraph}% context document
\omithspagedef{emptyVarSet :: VarSet}{haskell.def.emptyVarSet}% context document
\hsfigdefll{entry}{haskell.def.entry}% context figure
\hsfigdefll[lastLiveOut.1]{env}{haskell.def.env}% context figure
\hspagedefll{ex}{haskell.def.ex}% context document
\hsfigdefll{exit}{haskell.def.exit}% context figure
\hsfigdefll[assign.avail.1]{\_expr}{haskell.def.:unexpr}% context figure
\hsfigdef[extendAvail]{extendAvail}{haskell.def.extendAvail}% context figure
\omithspagedef{extendVarSet :: VarSet -> LocalVar -> VarSet}{haskell.def.extendVarSet}% context document
\hsfigdef{fact\_add\_to}{haskell.def.fact:unadd:unto}% context figure
\hsfigdef{fact\_bot}{haskell.def.fact:unbot}% context figure
\hsfigdef[FactKont]{FactKont}{haskell.def.FactKont}% context figure
\omithspagedef{fact\_name :: DataflowLattice a -> String}{haskell.def.fact:unname}% context document
\hsprelude{False}{Prelude}% context prelude
\omithspagedef{filterVarsUsed :: UserOfLocalVars e => (LocalVar -> Bool) -> e -> VarSet}{haskell.def.filterVarsUsed}% context document
\hsfigdefll[avail.rewrites.first]{first}{haskell.def.first}% context figure
\hsprelude{flip}{Prelude}% context prelude
\hsprelude{foldl}{Prelude}% context prelude
\hsprelude{foldr}{Prelude}% context prelude
\hspagedef{foldVarsDefd}{haskell.def.foldVarsDefd}% context document
\hspagedef{foldVarsUsed}{haskell.def.foldVarsUsed}% context document
\hsfigdef{ForwardRewrites}{haskell.def.ForwardRewrites}% context figure
\hsfigdef{ForwardTransfers}{haskell.def.ForwardTransfers}% context figure
\hsfigdefll[avail.solve.1]{fp}{haskell.def.fp}% context figure
\hsfigdef{fr\_first}{haskell.def.fr:unfirst}% context figure
\hsfigdef{fr\_last}{haskell.def.fr:unlast}% context figure
\hsfigdef{fr\_middle}{haskell.def.fr:unmiddle}% context figure
\hsprelude{fst}{Prelude}% context prelude
\hsfigdef{ft\_first\_out}{haskell.def.ft:unfirst:unout}% context figure
\hsfigdef{ft\_last\_outs}{haskell.def.ft:unlast:unouts}% context figure
\hsfigdef{ft\_middle\_out}{haskell.def.ft:unmiddle:unout}% context figure
\hspagedef{fuelExhausted}{haskell.def.fuelExhausted}% context document
\hspagedef{FuelMonad}{haskell.def.FuelMonad}% context document
\hspagedef{FwdFixedPoint}{haskell.def.FwdFixedPoint}% context document
\hsfigdef[forward.sol.sig]{fwd\_iter}{haskell.def.fwd:uniter}% context figure
\hspagedef{getAllFacts}{haskell.def.getAllFacts}% context document
\hspagedef{getFact}{haskell.def.getFact}% context document
\hsfigdef{GF}{haskell.def.GF}% context figure
\hspagedef{GlobalVar}{haskell.def.GlobalVar}% context document
\hsfigdef{Graph}{haskell.def.Graph}% context figure
\hspagedef{GraphClosure}{haskell.def.GraphClosure}% context document
\hsfigdef[GraphFactKont]{GraphFactKont}{haskell.def.GraphFactKont}% context figure
\hsfigdef[GraphKont]{GraphKont}{haskell.def.GraphKont}% context figure
\hspagedef{HavingSuccessors}{haskell.def.HavingSuccessors}% context document
\hsprelude{head}{Prelude}% context prelude
\hspagedef{iar\_OC}{haskell.def.iar:unOC}% context document
\hsfigdef[iar.OO]{iar\_OO}{haskell.def.iar:unOO}% context figure
\hsprelude{id}{Prelude}% context prelude
\hsfigdefll[solve.mid.1]{in'}{haskell.def.in'}% context figure
\hsfigdefll[forward.sol.args]{in\_fact}{haskell.def.in:unfact}% context figure
\hsfigdef{insertLateReloads}{haskell.def.insertLateReloads}% context figure
\hsprelude{Int}{Prelude}% context prelude
\hsfigdef{interAvail}{haskell.def.interAvail}% context figure
\omithspagedef{isEmptyVarSet :: VarSet -> Bool}{haskell.def.isEmptyVarSet}% context document
\omithspagedef{isStackSlot :: CmmExpr -> Bool}{haskell.def.isStackSlot}% context document
\omithspagedef{isStackSlotOf :: CmmExpr -> LocalVar -> Bool}{haskell.def.isStackSlotOf}% context document
\hsfigdef[solve.block.sig]{iter\_block}{haskell.def.iter:unblock}% context figure
\hsfigdef[solve.ex.sig]{iter\_ex}{haskell.def.iter:unex}% context figure
\hsfigdef[solve.first.sig]{iter\_first}{haskell.def.iter:unfirst}% context figure
\hsfigdef[solve.last.sig]{iter\_last}{haskell.def.iter:unlast}% context figure
\hsfigdef[solve.mid.1]{iter\_mid}{haskell.def.iter:unmid}% context figure
\hsfigdef[solve.OO.def]{iter\_OO}{haskell.def.iter:unOO}% context figure
\hsfigdefll{join}{haskell.def.join}% context figure
\hsprelude{Just}{Prelude}% context prelude
\hsfigdef[Kont]{Kont}{haskell.def.Kont}% context figure
\hsfigdefll{l}{haskell.def.l}% context figure
\hsfigdefll[avail.rewrites.last]{last}{haskell.def.last}% context figure
\hsfigdef{lastAvail}{haskell.def.lastAvail}% context figure
\hspagedef{LastBranch}{haskell.def.LastBranch}% context document
\hspagedef{LastCall}{haskell.def.LastCall}% context document
\hspagedef{LastCondBranch}{haskell.def.LastCondBranch}% context document
\hsfigdef[lastLiveness]{lastLiveness}{haskell.def.lastLiveness}% context figure
\hsfigdef[lastLiveOut.1]{lastLiveOut}{haskell.def.lastLiveOut}% context figure
\hsfigdef{LastOuts}{haskell.def.LastOuts}% context figure
\hspagedef{LastSwitch}{haskell.def.LastSwitch}% context document
\hsfigdefll[assign.avail.1]{lhs}{haskell.def.lhs}% context figure
\hsprelude{liftM}{Prelude}% context prelude
\hsfigdef[Live]{Live}{haskell.def.Live}% context figure
\hsfigdefll{live}{haskell.def.live}% context figure
\hsfigdef[liveLattice]{liveLattice}{haskell.def.liveLattice}% context figure
\hsfigdef{liveTransfers}{haskell.def.liveTransfers}% context figure
\hspagedef{LocalVar}{haskell.def.LocalVar}% context document
\hsfigdef{LOFsKont}{haskell.def.LOFsKont}% context figure
\hsfigdefll{m}{haskell.def.m}% context figure
\hsprelude{map}{Prelude}% context prelude
\hsprelude{mapM\_}{Prelude}% context prelude
\hsprelude{Maybe}{Prelude}% context prelude
\hsfigdef[maybe.reload.before.1]{maybe\_reload\_before}{haskell.def.maybe:unreload:unbefore}% context figure
\hspagedef{MidAssign}{haskell.def.MidAssign}% context document
\hsfigdefll{middle}{haskell.def.middle}% context figure
\hsfigdef{middleAvail}{haskell.def.middleAvail}% context figure
\hsfigdef[middleLiveness]{middleLiveness}{haskell.def.middleLiveness}% context figure
\hsfigdef{middleRemoveDeads}{haskell.def.middleRemoveDeads}% context figure
\hspagedef{MidStore}{haskell.def.MidStore}% context document
\hspagedef{mkLabel}{haskell.def.mkLabel}% context document
\hspagedef{mkLast}{haskell.def.mkLast}% context document
\hspagedef{mkMiddle}{haskell.def.mkMiddle}% context document
\hsfigdefll[forward.sol.args]{name}{haskell.def.name}% context figure
\hsfigdefll{new}{haskell.def.new}% context figure
\hsfigdef{NoChange}{haskell.def.NoChange}% context figure
\hsfigdefll[maybe.reload.before.1]{node}{haskell.def.node}% context figure
\hsprelude{not}{Prelude}% context prelude
\hsprelude{Nothing}{Prelude}% context prelude
\hsfigdefll[deadRewrites.1]{nothing}{haskell.def.nothing}% context figure
\hsfigdef{O}{haskell.def.O}% context figure
\hsfigdefll{old}{haskell.def.old}% context figure
\hspagedef{PassName}{haskell.def.PassName}% context document
\hsfigdefll[mkMiddle]{rel}{haskell.def.rel}% context figure
\hspagedef{reload}{haskell.def.reload}% context document
\hsfigdef{reloadTail}{haskell.def.reloadTail}% context figure
\hsfigdef[liveness.remDefd.def]{remDefd}{haskell.def.remDefd}% context figure
\hsfigdef{removeDeadAssignments}{haskell.def.removeDeadAssignments}% context figure
\hsprelude{return}{Prelude}% context prelude
\hsfigdef{Rewrite}{haskell.def.Rewrite}% context figure
\hspagedef{RewriteDeep}{haskell.def.RewriteDeep}% context document
\hsfigdefll[forward.sol.args]{rewrites}{haskell.def.rewrites}% context figure
\hspagedef{RewriteShallow}{haskell.def.RewriteShallow}% context document
\hspagedef{RewritingDepth}{haskell.def.RewritingDepth}% context document
\hspagedef{runDFM}{haskell.def.runDFM}% context document
\hspagedef{setAllFacts}{haskell.def.setAllFacts}% context document
\hspagedef{setFact}{haskell.def.setFact}% context document
\hsfigdef[set.last.*]{set\_last}{haskell.def.set:unlast}% context figure
\omithspagedef{sizeVarSet :: VarSet -> Int}{haskell.def.sizeVarSet}% context document
\hsfigdef[smallerAvail]{smallerAvail}{haskell.def.smallerAvail}% context figure
\hsprelude{snd}{Prelude}% context prelude
\hsfigdef{SomeChange}{haskell.def.SomeChange}% context figure
\hsfigdefll[forward.sol.args]{start\_facts}{haskell.def.start:unfacts}% context figure
\hsprelude{String}{Prelude}% context prelude
\hspagedef{subAnalysis}{haskell.def.subAnalysis}% context document
\hspagedef{succs}{haskell.def.succs}% context document
\hsprelude{tail}{Prelude}% context prelude
\hsfigdefll[live.lastSwitch]{tbl}{haskell.def.tbl}% context figure
\hsfigdefll[forward.sol.args]{transfers}{haskell.def.transfers}% context figure
\hsprelude{True}{Prelude}% context prelude
\hsfigdef{TxRes}{haskell.def.TxRes}% context figure
\hsprelude{uncurry}{Prelude}% context prelude
\hsprelude{undefined}{Prelude}% context prelude
\omithspagedef{unionManyVarSets :: [VarSet] -> VarSet}{haskell.def.unionManyVarSets}% context document
\omithspagedef{unionVarSets :: VarSet -> VarSet -> VarSet}{haskell.def.unionVarSets}% context document
\hsfigdef[AvailVars]{UniverseMinus}{haskell.def.UniverseMinus}% context figure
\hsfigdefll{used}{haskell.def.used}% context figure
\hspagedef{useOneFuel}{haskell.def.useOneFuel}% context document
\hspagedef{UserOfLocalVars}{haskell.def.UserOfLocalVars}% context document
\omithspagedef{varOfSlot :: CmmExpr -> LocalVar}{haskell.def.varOfSlot}% context document
\hsfigdefll{vars}{haskell.def.vars}% context figure
\omithspagedef{VarSet\textrm{~(a~type)}}{haskell.def.VarSet}% context document
\omithsfigdef{varSetToList :: VarSet -> [LocalVar]}{haskell.def.varSetToList}% context figure
\hspagedef{withDuplicateFuel}{haskell.def.withDuplicateFuel}% context document
\hspagedef{with\_fuel}{haskell.def.with:unfuel}% context document
\hspagedef{zdfFpContents}{haskell.def.zdfFpContents}% context document
\hspagedef{zdfFpFacts}{haskell.def.zdfFpFacts}% context document
\hspagedef{zdfRewriteBwd}{haskell.def.zdfRewriteBwd}% context document
\hspagedef{zdfRewriteFwd}{haskell.def.zdfRewriteFwd}% context document
\hspagedef{zdfSolveBwd}{haskell.def.zdfSolveBwd}% context document
\hspagedef{zdfSolveFwd}{haskell.def.zdfSolveFwd}% context document
\hspagedef{ZJust}{haskell.def.ZJust}% context document
\hspagedef{ZMaybe}{haskell.def.ZMaybe}% context document
\hspagedef{ZNothing}{haskell.def.ZNothing}% context document

\ifundefinedsection.\fi



\endgroup


\iffalse

\section{Dataflow-engine functions}


\begin{figure*}
\setcounter{codeline}{0}
\begin{numberedcode}
\end{numberedcode}
\caption{The forward iterator}
\end{figure*}

\begin{figure*}
\setcounter{codeline}{0}
\begin{numberedcode}
\end{numberedcode}
\caption{The forward actualizer}
\end{figure*}


\fi



\end{document}




THE FUEL PROBLEM:


Here is the problem:

  A graph has an entry sequence, a body, and an exit sequence.
  Correctly computing facts on and flowing out of the body requires
  iteration; computation on the entry and exit sequences do not, since
  each is connected to the body by exactly one flow edge.

  The problem is to provide the correct fuel supply to the combined
  analysis/rewrite (iterator) functions, so that speculative rewriting
  is limited by the fuel supply.

  I will number iterations from 1 and name the fuel supplies as
  follows:

     f_pre      fuel remaining before analysis/rewriting starts
     f_0        fuel remaining after analysis/rewriting of the entry sequence
     f_i, i>0   fuel remaining after iteration i of the body
     f_post     fuel remaining after analysis/rewriting of the exit sequence

  The issue here is that only the last iteration of the body 'counts'.
  To formalize, I will name fuel consumed:

     C_pre      fuel consumed by speculative rewrites in entry sequence
     C_i        fuel consumed by speculative rewrites in iteration i of body
     C_post     fuel consumed by speculative rewrites in exit sequence

  These quantities should be related as follows:

     f_0    = f_pre - C_pref
     f_i    = f_0 - C_i            where i > 0
     f_post = f_n - C_post         where iteration converges after n steps

When the fuel supply is passed explicitly as parameter and result, it
is fairly easy to see how to keep reusing f_0 at every iteration, then
extract f_n for use before the exit sequence.  It is not obvious to me
how to do it cleanly using the fuel monad.


Norman
